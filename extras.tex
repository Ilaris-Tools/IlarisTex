% befehle
\newcommand{\trennlinie}{\hr}
\newcommand{\anfang}{\raggedcolumns\frontmatter}
\newcommand{\hauptteil}{\mainmatter}
\newcommand{\schluss}{\backmatter}
\newcommand{\spaltenanfang}{\begin{multicols}{2}}
        \newcommand{\spaltenende}{\end{multicols}}
\newcommand{\skapitel}[1]{\end{multicols}\kapitel{#1}\begin{multicols}{2}}
\newcommand{\kapitel}[1]{{\let\cleardoublepage\clearpage\chapter*{#1}}\addcontentsline{toc}{chapter}{#1}}
\newcommand{\abschnitt}[1]{\section{#1}}
\newcommand{\leereseite}{\newpage\makebox{}\clearpage}
\newcommand{\platz}{\vfill}
\newcommand{\begriff}[1]{\textbf{#1}}
\newcommand{\geschichte}[2]{%
    \vspace{3mm}
    \begin{center}
        \begin{minipage}[b]{0.9\linewidth}{\em #1}
            \begin{flushright}--- #2 \end{flushright}
        \end{minipage}
    \end{center}}
\newcommand{\kasten}[1]{\begin{pergamentbox}#1\end{pergamentbox}}
\newcommand{\inhaltsverzeichnis}{{\let\cleardoublepage\clearpage\tableofcontents}}
\newcommand{\fanprodukt}{\begin{center}\includegraphics[width=5cm]{layout/dsa_logo_fanprodukt}\end{center}}
\newcommand{\titelbild}[1]{
    \tikz[remember picture,overlay] \node[inner sep=0pt] at (current page.center){\includegraphics[width=\paperwidth,height=\paperheight]{#1}};}
\newcommand{\tfarbe}{orange!15}
\newcommand{\tsfarbe}{black}
\newcommand{\titelfarbe}[1]{
    \renewcommand{\tfarbe}{#1}
}
\newcommand{\titelschattenfarbe}[1]{
    \renewcommand{\tsfarbe}{#1}
}
\newcommand{\titel}[1]{
    \title{#1}
    \begin{centering}
        % 26.12.2020 \\
        % \color{orange!15}
        \color{\tfarbe}
        %\vspace{0.55\textheight}
        \Huge
        %\fontfamily{\aniron}
        \fontsize{42}{42}
        \selectfont
        \aniron
        \shadowcolor{\tsfarbe}
        \shadowtext{#1} \\
    \end{centering}
}
\newcommand{\authorin}[1]{\begin{centering}\author{#1}\Large #1\\\end{centering}}
\newcommand{\titelseite}[1]{\thispagestyle{empty}#1}
%\newcommand{\kreatur}[1]{\begin{creature}#1\end{creature}}


\newcommand{\kreaturinfo}[2]{
    \footnotesize\textbf{#1: } #2}

\newcommand{\kreatur}[4]{%
    \begin{creaturebox}
        \begin{minipage}[b]{0.86\linewidth}
            {\color{dunkelrot}\Large\textbf{#1}}\\[1mm]
            \footnotesize \emph{#2}
        \end{minipage}
        \hfill\includegraphics[width=0.1\linewidth]{#3}
        \hr[0.8pt]
        \tcblower
        \vspace{2mm}
        %\tcbline
        #4
        \hr
    \end{creaturebox}
}

\newcommand{\credits}[1]{%
    \begin{center}
        #1
        \vfill
        DAS SCHWARZE AUGE, AVENTURIEN, DERE, MYRANOR, THARUN, UTHURIA, RIESLAND und
        THE DARK EYE sind eingetragene Marken der Ulisses Spiele GmbH, Waldems. Die Verwendung der
        Grafiken erfolgt unter den von Ulisses Spiele erlaubten Richtlinien. Eine Verwendung über diese
        Richtlinien hinaus darf nur nach vorheriger schriftlicher Genehmigung der Ulisses Medien und Spiel
        Distribution GmbH erfolgen.

    \end{center}
}
\newcommand{\absatz}[1]{\subsection*{#1}}
\newcommand{\block}[1]{\subsubsection*{#1}}
\newcommand{\spaltenumbruch}{\columnbreak}
\newcommand{\credit}[2]{%
    \begin{center}
        {\aniron\fontsize{18}{18}\color{dunkelrot}#1}\\
        #2
    \end{center}
}

\newcommand{\lizenz}{
    \vfill
    \begin{center}
        DAS SCHWARZE AUGE, AVENTURIEN, DERE, MYRANOR, THARUN, UTHURIA, RIESLAND und
        THE DARK EYE sind eingetragene Marken der Ulisses Spiele GmbH, Waldems. Die Verwendung der
        Grafiken erfolgt unter den von Ulisses Spiele erlaubten Richtlinien. Eine Verwendung über diese
        Richtlinien hinaus darf nur nach vorheriger schriftlicher Genehmigung der Ulisses Medien und Spiel
        Distribution GmbH erfolgen.
    \end{center}
}
\newcommand{\creditlayout}{
    \credit{Layoutvorlage}{\href{https://github.com/Ilaris-dev/IlarisTex}{IlarisTex}}}
\newcommand{\glossar}{
    \printglossaries
}
\newcommand{\komplettesglossar}{
    \glsaddall
    \printglossaries
}