
\newglossaryentry{achaz_Vorteil}
{
    name={Achaz},
    description={\textbf{Kosten}: 0 \textbf{Nachkauf}: Extrem selten\newline Du kannst mit deinem Schwanz waffenlose Angriffe in Reichweite 1 (siehe auch S. 38) durchführen. Allerdings leidest du unter Kältestarre; für jede Temperaturstufe unter normal (S. 35) sind alle körperlichen Proben um –4 erschwert.}
}


\newglossaryentry{angepasst(Schnee)I_Vorteil}
{
    name={Angepasst (Schnee) I},
    description={\textbf{Kosten}: 20 \textbf{Nachkauf}: Häufig\newline Durch deine Spezies oder langjährige Erfahrung hast du dich an Schnee gewöhnt. Abzüge durch unsicheren oder eisigen Untergrund, insbesondere im Kampf, sinken für dich um 1 Stufe.}
}


\newglossaryentry{angepasst(Dunkelheit)I_Vorteil}
{
    name={Angepasst (Dunkelheit) I},
    description={\textbf{Kosten}: 40 \textbf{Nachkauf}: Häufig\newline Durch deine Spezies oder langjährige Erfahrung hast du dich an Dunkelheit gewöhnt. Abzüge durch schlechte Lichtverhältnisse, insbesondere im Kampf, sinken für dich um 1 Stufe.}
}


\newglossaryentry{angepasst(Wasser)I_Vorteil}
{
    name={Angepasst (Wasser) I},
    description={\textbf{Kosten}: 20 \textbf{Nachkauf}: Häufig\newline Durch deine Spezies oder langjährige Erfahrung hast du dich an Wasser gewöhnt. Abzüge durch duch knietiefes- und hüfttiefes Wasser und unter Wasser, insbesondere im Kampf, sinken für dich um 1 Stufe.}
}


\newglossaryentry{angepasst(Wald)I_Vorteil}
{
    name={Angepasst (Wald) I},
    description={\textbf{Kosten}: 40 \textbf{Nachkauf}: Häufig\newline Durch deine Spezies oder langjährige Erfahrung hast du dich an den Wald gewöhnt. Abzüge durch Wurzeln, Gestrüpp und dichtes Unterholz, insbesondere im Kampf, sinken für dich um 1 Stufe.}
}


\newglossaryentry{angepasstI_Vorteil}
{
    name={Angepasst I},
    description={\textbf{Kosten}: 20 \textbf{Nachkauf}: Häufig\newline Durch deine Spezies oder langjährige Erfahrung hast du dich an eine bestimmte Umgebung oder Umweltbedingung gewöhnt. Abzüge durch diese Umgebung (Beispiele auf S. 38), insbesondere im Kampf, sinken für dich um 1 Stufe.\newline Die Kosten für Angepasst I legt der Spielleiter fest, wobei er sich an der Häufigkeit der Umgebung orientieren sollte. Zu allgemein gefasste Umgebungen wie „unsicherer Untergrund“ sollte er nicht zulassen.\newline Sephrasto: Wähle am besten direkt den separaten Vorteil Angepasst (Dunkelheit, Schnee, Wasser oder Wald) I. Falls du eine andere Umgebung möchtest, dann wähle diesen Vorteil und trage diese in das Kommentarfeld ein.}
}


\newglossaryentry{angepasstII_Vorteil}
{
    name={Angepasst II},
    description={\textbf{Kosten}: 20 \textbf{Voraussetzungen}: Vorteil Angepasst I \textbf{Nachkauf}: Selten\newline Durch deine Spezies oder langjährige Erfahrung hast du dich an eine bestimmte Umgebung oder Umweltbedingung gewöhnt. Abzüge durch diese Umgebung (Beispiele auf S. 38), insbesondere im Kampf, sinken für dich um 1 weitere Stufe.\newline Die Kosten für Angepasst II entsprechen den vom Spielleiter festgelegten Kosten von Angepasst I für diesen Umstand.\newline Sephrasto: Trage die gewählte Umgebung in das Kommentarfeld ein. Entferne diese Umgebung aus dem Kommentar von Angepasst I, damit der Regelanhang korrekt ausgegeben wird.}
}


\newglossaryentry{angepasst(Dunkelheit)II_Vorteil}
{
    name={Angepasst (Dunkelheit) II},
    description={\textbf{Kosten}: 40 \textbf{Voraussetzungen}: Vorteil Angepasst (Dunkelheit) I \textbf{Nachkauf}: Häufig\newline Durch deine Spezies oder langjährige Erfahrung hast du dich an Dunkelheit gewöhnt. Abzüge durch schlechte Lichtverhältnisse, insbesondere im Kampf, sinken für dich um 2 Stufen.}
}


\newglossaryentry{angepasst(Schnee)II_Vorteil}
{
    name={Angepasst (Schnee) II},
    description={\textbf{Kosten}: 20 \textbf{Voraussetzungen}: Vorteil Angepasst (Schnee) I \textbf{Nachkauf}: Häufig\newline Durch deine Spezies oder langjährige Erfahrung hast du dich an Schnee gewöhnt. Abzüge durch unsicheren oder eisigen Untergrund, insbesondere im Kampf, sinken für dich um 2 Stufen.}
}


\newglossaryentry{angepasst(Wald)II_Vorteil}
{
    name={Angepasst (Wald) II},
    description={\textbf{Kosten}: 40 \textbf{Voraussetzungen}: Vorteil Angepasst (Wald) I \textbf{Nachkauf}: Häufig\newline Durch deine Spezies oder langjährige Erfahrung hast du dich an den Wald gewöhnt. Abzüge durch Wurzeln, Gestrüpp und dichtes Unterholz, insbesondere im Kampf, sinken für dich um 2 Stufen.}
}


\newglossaryentry{angepasst(Wasser)II_Vorteil}
{
    name={Angepasst (Wasser) II},
    description={\textbf{Kosten}: 20 \textbf{Voraussetzungen}: Vorteil Angepasst (Wasser) I \textbf{Nachkauf}: Häufig\newline Durch deine Spezies oder langjährige Erfahrung hast du dich an Wasser gewöhnt. Abzüge durch duch knietiefes- und hüfttiefes Wasser und unter Wasser, insbesondere im Kampf, sinken für dich um 2 Stufen.}
}


\newglossaryentry{besondererBesitz_Vorteil}
{
    name={Besonderer Besitz},
    description={\textbf{Kosten}: 40 \textbf{Nachkauf}: Nicht möglich\newline Ein besonderer Gegenstand wie die 33-fach geflämmte Klinge deines ruhmreichen Großvaters oder ein heilendes Diadem befindet sich in deinem Besitz. Der Spielleiter sollte diesem Gegenstand einen gewissen Schutz gewähren; er wird nicht zufällig von einem Taschendieb gestohlen oder geht durch Pech verloren, ohne dass du ihn wieder zurückerlangen kannst. Umgekehrt darf der Charakter den Gegenstand auch nicht verkaufen.\newline Die Kosten für einen Besonderen Besitz legt der Spielleiter fest, wobei er sich am Spielnutzen des Gegenstandes orientieren sollte. Ein Besonderer Besitz für 40 EP bringt einen willkommenen Bonus in manchen Situationen, während ein Besonderer Besitz für 160 EP in vielen Situationen entscheidend ist. Der Wert des Gegenstandes spielt hierbei keine Rolle.\newline Beispiele für besonderen Besitz sind:\newline - Schlachtross/verbesserte Waffe (+2 TP)/verbesserte Rüstung (-1 BE), Ferrara Kutsche mit Pferden (40 EP).\newline - Verbesserte und magische Waffe, aufladbares Schutzartefakt mit einfachem Auslöser (80 EP).\newline - Verbessertes Enduriumschwert, verbesserter Toschkrilpanzer, Reithippogriff (120 EP).\newline - Bewaffnetes Handelsschiff, semipermanentes Universalschutzamulett mit intelligentem Auslöser (160 EP).\newline Sephrasto: Trage den gewählten Besitz in das Kommentarfeld ein.}
}


\newglossaryentry{einkommenI_Vorteil}
{
    name={Einkommen I},
    description={\textbf{Kosten}: 20 \textbf{Nachkauf}: Häufig\newline Deine Familie, Organisation oder ein persönlicher Mäzen stellt dir ein monatliches Einkommen von 4 Dukaten zur Verfügung, die du beispielsweise über eine Filiale der Nordlandbank oder ein Ordenshaus beziehen kannst. Fern der Zivilisation kannst du nicht auf dein Einkommen zugreifen. Das Einkommen reicht aus, um die Lebenserhaltungskosten eines Angehörigen der Unterschicht zu decken.}
}


\newglossaryentry{einkommenII_Vorteil}
{
    name={Einkommen II},
    description={\textbf{Kosten}: 20 \textbf{Voraussetzungen}: Vorteil Einkommen I \textbf{Nachkauf}: Häufig\newline Deine Familie, Organisation oder ein persönlicher Mäzen stellt dir ein monatliches Einkommen von 16 Dukaten zur Verfügung, die du beispielsweise über eine Filiale der Nordlandbank oder ein Ordenshaus beziehen kannst. Fern der Zivilisation kannst du nicht auf dein Einkommen zugreifen. Das Einkommen reicht aus, um die Lebenserhaltungskosten eines Angehörigen der Mittelschicht zu decken.}
}


\newglossaryentry{einkommenIII_Vorteil}
{
    name={Einkommen III},
    description={\textbf{Kosten}: 20 \textbf{Voraussetzungen}: Vorteil Einkommen II \textbf{Nachkauf}: Häufig\newline Deine Familie, Organisation oder ein persönlicher Mäzen stellt dir ein monatliches Einkommen von 64 Dukaten zur Verfügung, die du beispielsweise über eine Filiale der Nordlandbank oder ein Ordenshaus beziehen kannst. Fern der Zivilisation kannst du nicht auf dein Einkommen zugreifen. Das Einkommen reicht aus, um die Lebenserhaltungskosten eines Angehörigen der Oberschicht zu decken.}
}


\newglossaryentry{einkommenIV_Vorteil}
{
    name={Einkommen IV},
    description={\textbf{Kosten}: 20 \textbf{Voraussetzungen}: Vorteil Einkommen III \textbf{Nachkauf}: Häufig\newline Deine Familie, Organisation oder ein persönlicher Mäzen stellt dir ein monatliches Einkommen von 256 Dukaten zur Verfügung, die du beispielsweise über eine Filiale der Nordlandbank oder ein Ordenshaus beziehen kannst. Fern der Zivilisation kannst du nicht auf dein Einkommen zugreifen. Das Einkommen reicht aus, um die Lebenserhaltungskosten eines Angehörigen der Elite zu decken.}
}


\newglossaryentry{eisenaffineAura_Vorteil}
{
    name={Eisenaffine Aura},
    description={\textbf{Kosten}: 40 \textbf{Voraussetzungen}: Vorteil Zauberer I \textbf{Nachkauf}: Selten\newline Erschwernisse durch den Bann des Eisens (S. 78) sinken um 8 Punkte.}
}


\newglossaryentry{gefahreninstinkt_Vorteil}
{
    name={Gefahreninstinkt},
    description={\textbf{Kosten}: 100 \textbf{Nachkauf}: Extrem selten\newline Mit der Gabe Gefahreninstinkt kannst du mit dem Talent Wachsamkeit auch Gefahren erfassen, die mit gewöhnlichen Sinnen nicht wahrzunehmen sind. Dazu gehören etwa magische Fallen oder eine bald losbrechende Lawine. Ist die Gefahr auch mit gewöhnlichen Sinnen wahrnehmbar erhältst du eine Erleichterung von +4.}
}


\newglossaryentry{geweihtI_Vorteil}
{
    name={Geweiht I},
    description={\textbf{Kosten}: 40 \textbf{Voraussetzungen}: Kein Vorteil Paktierer I \textbf{Nachkauf}: Üblich\newline Du verfügst über 8 Karmapunkte und kannst kirchliche Traditionen erlernen.}
}


\newglossaryentry{geweihtII_Vorteil}
{
    name={Geweiht II},
    description={\textbf{Kosten}: 40 \textbf{Voraussetzungen}: Vorteil Geweiht I \textbf{Nachkauf}: Üblich\newline Du verfügst über 16 Karmapunkte und kannst kirchliche Traditionen erlernen.}
}


\newglossaryentry{geweihtIII_Vorteil}
{
    name={Geweiht III},
    description={\textbf{Kosten}: 40 \textbf{Voraussetzungen}: Vorteil Geweiht II \textbf{Nachkauf}: Üblich\newline Du verfügst über 24 Karmapunkte und kannst kirchliche Traditionen erlernen.}
}


\newglossaryentry{geweihtIV_Vorteil}
{
    name={Geweiht IV},
    description={\textbf{Kosten}: 40 \textbf{Voraussetzungen}: Vorteil Geweiht III \textbf{Nachkauf}: Üblich\newline Du verfügst über 32 Karmapunkte und kannst kirchliche Traditionen erlernen.}
}


\newglossaryentry{glückI_Vorteil}
{
    name={Glück I},
    description={\textbf{Kosten}: 40 \textbf{Nachkauf}: Häufig\newline Deine maximalen Schicksalspunkte steigen auf 5.}
}


\newglossaryentry{glückII_Vorteil}
{
    name={Glück II},
    description={\textbf{Kosten}: 40 \textbf{Voraussetzungen}: Vorteil Glück I \textbf{Nachkauf}: Häufig\newline Deine maximalen Schicksalspunkte steigen auf 6.}
}


\newglossaryentry{kreisderVerdammnisI_Vorteil}
{
    name={Kreis der Verdammnis I},
    description={\textbf{Kosten}: -400 \textbf{Nachkauf}: Üblich\newline Der Vorteil hat vier Auswirkungen. Erstens erhältst du beim Erwerb der ersten Stufe 400 EP, für jede weitere Stufe 200 EP. Zweitens werden deine Gunstpunkte sofort aufgefüllt. Drittens erhältst du für jede Stufe eine verdorbene Eigenheit. Viertens entwickelst du ein Dämonenmal und kannst über die Liturgie Seelenprüfung als Paktierer erkannt werden. Auf geweihtem/heiligen Boden musst du jede Minute eine Willenskraft-Probe (20/28) ablegen, um nicht durch starke Schmerzen aufzufallen und eine Wunde zu erleiden.}
}


\newglossaryentry{kreisderVerdammnisII_Vorteil}
{
    name={Kreis der Verdammnis II},
    description={\textbf{Kosten}: -200 \textbf{Voraussetzungen}: Vorteil Kreis der Verdammnis I \textbf{Nachkauf}: Üblich\newline Der Vorteil hat vier Auswirkungen. Erstens erhältst du beim Erwerb der ersten Stufe 400 EP, für jede weitere Stufe 200 EP. Zweitens werden deine GuP sofort aufgefüllt. Drittens erhältst du für jede Stufe eine verdorbene Eigenheit. Viertens entwickelst du ein Dämonenmal und kannst über die Liturgie Seelenprüfung als Paktierer erkannt werden. Auf geweihtem/heiligen Boden musst du jede Minute eine Willenskraft-Probe (20/28) ablegen, um nicht durch starke Schmerzen aufzufallen und eine Wunde zu erleiden.}
}


\newglossaryentry{kreisderVerdammnisIII_Vorteil}
{
    name={Kreis der Verdammnis III},
    description={\textbf{Kosten}: -200 \textbf{Voraussetzungen}: Vorteil Kreis der Verdammnis II \textbf{Nachkauf}: Üblich\newline Der Vorteil hat vier Auswirkungen. Erstens erhältst du beim Erwerb der ersten Stufe 400 EP, für jede weitere Stufe 200 EP. Zweitens werden deine GuP sofort aufgefüllt. Drittens erhältst du für jede Stufe eine verdorbene Eigenheit. Viertens entwickelst du ein Dämonenmal und kannst über die Liturgie Seelenprüfung als Paktierer erkannt werden. Auf geweihtem/heiligen Boden musst du jede Minute eine Willenskraft-Probe (20/28) ablegen, um nicht durch starke Schmerzen aufzufallen und eine Wunde zu erleiden.}
}


\newglossaryentry{kreisderVerdammnisIV_Vorteil}
{
    name={Kreis der Verdammnis IV},
    description={\textbf{Kosten}: -200 \textbf{Voraussetzungen}: Vorteil Kreis der Verdammnis III \textbf{Nachkauf}: Üblich\newline Der Vorteil hat vier Auswirkungen. Erstens erhältst du beim Erwerb der ersten Stufe 400 EP, für jede weitere Stufe 200 EP. Zweitens werden deine GuP sofort aufgefüllt. Drittens erhältst du für jede Stufe eine verdorbene Eigenheit. Viertens entwickelst du ein Dämonenmal und kannst über die Liturgie Seelenprüfung als Paktierer erkannt werden. Auf geweihtem/heiligen Boden musst du jede Minute eine Willenskraft-Probe (20/28) ablegen, um nicht durch starke Schmerzen aufzufallen und eine Wunde zu erleiden.}
}


\newglossaryentry{kreisderVerdammnisV_Vorteil}
{
    name={Kreis der Verdammnis V},
    description={\textbf{Kosten}: -200 \textbf{Voraussetzungen}: Vorteil Kreis der Verdammnis IV \textbf{Nachkauf}: Üblich\newline Der Vorteil hat vier Auswirkungen. Erstens erhältst du beim Erwerb der ersten Stufe 400 EP, für jede weitere Stufe 200 EP. Zweitens werden deine GuP sofort aufgefüllt. Drittens erhältst du für jede Stufe eine verdorbene Eigenheit. Viertens entwickelst du ein Dämonenmal und kannst über die Liturgie Seelenprüfung als Paktierer erkannt werden. Auf geweihtem/heiligen Boden musst du jede Minute eine Willenskraft-Probe (20/28) ablegen, um nicht durch starke Schmerzen aufzufallen und eine Wunde zu erleiden.}
}


\newglossaryentry{kreisderVerdammnisVI_Vorteil}
{
    name={Kreis der Verdammnis VI},
    description={\textbf{Kosten}: -200 \textbf{Voraussetzungen}: Vorteil Kreis der Verdammnis V \textbf{Nachkauf}: Üblich\newline Der Vorteil hat vier Auswirkungen. Erstens erhältst du beim Erwerb der ersten Stufe 400 EP, für jede weitere Stufe 200 EP. Zweitens werden deine GuP sofort aufgefüllt. Drittens erhältst du für jede Stufe eine verdorbene Eigenheit. Viertens entwickelst du ein Dämonenmal und kannst über die Liturgie Seelenprüfung als Paktierer erkannt werden. Auf geweihtem/heiligen Boden musst du jede Minute eine Willenskraft-Probe (20/28) ablegen, um nicht durch starke Schmerzen aufzufallen und eine Wunde zu erleiden.}
}


\newglossaryentry{kreisderVerdammnisVII_Vorteil}
{
    name={Kreis der Verdammnis VII},
    description={\textbf{Kosten}: -200 \textbf{Voraussetzungen}: Vorteil Kreis der Verdammnis VI \textbf{Nachkauf}: Üblich\newline Der Vorteil hat vier Auswirkungen. Erstens erhältst du beim Erwerb der ersten Stufe 400 EP, für jede weitere Stufe 200 EP. Zweitens werden deine GuP sofort aufgefüllt. Drittens erhältst du für jede Stufe eine verdorbene Eigenheit. Viertens entwickelst du ein Dämonenmal und kannst über die Liturgie Seelenprüfung als Paktierer erkannt werden. Auf geweihtem/heiligen Boden musst du jede Minute eine Willenskraft-Probe (20/28) ablegen, um nicht durch starke Schmerzen aufzufallen und eine Wunde zu erleiden.}
}


\newglossaryentry{magieabweisend_Vorteil}
{
    name={Magieabweisend},
    description={\textbf{Kosten}: 40 \textbf{Nachkauf}: Extrem selten\newline Zauber wirken auf dich deutlich schwächer. Du ignorierst bei allen Zaubern eine Stufe der spontanen Modifikation Mächtige Magie. Zauber ohne Mächtige Magie haben auf dich keine Wirkung.}
}


\newglossaryentry{magiegespür_Vorteil}
{
    name={Magiegespür},
    description={\textbf{Kosten}: 60 \textbf{Nachkauf}: Extrem selten\newline In der Nähe astraler Kräfte überfällt dich ein Frösteln, du hörst sphärische Klänge oder ein Farbschleier legt sich über die Umgebung. Mit dem Talent Sinnenschärfe kannst du Intensitätsanalysen (S. 80) von magischen Gegenständen durchführen. Nach dem aktiven Einsatz der Gabe erleidest du 1 Punkt Erschöpfung.}
}


\newglossaryentry{minderpakt_Vorteil}
{
    name={Minderpakt},
    description={\textbf{Kosten}: 0 \textbf{Nachkauf}: Üblich\newline Minderpakte werden deutlich leichter als reguläre Pakte geschlossen - und oft auch unfreiwillig. Dafür bedeutet ein Minderpakt noch keinen Eintritt in die Kreise der Verdammnis. Stattdessen kann ein Minderpaktierer sofort einen Vorteil für maximal 20 EP erwerben, für den er nicht zwingend die Voraussetzungen erfüllen muss - ein klassisches Beispiel ist die Tradition der Borbaradianer I.\newline   Im Gegenzug erhält ein Minderpaktierer die Eigenheit „Von (Erzdämon) berührt“. Mit dieser Eigenheit kann der Erzdämon versuchen, den Minderpaktierer in passenden Situationen zu einem Seelenpakt zu verführen.}
}


\newglossaryentry{natürlicheRüstung_Vorteil}
{
    name={Natürliche Rüstung},
    description={\textbf{Kosten}: 80 \textbf{Nachkauf}: Extrem selten\newline Du verfügst über ein dichtes Fell oder zähe Schuppenhaut, wodurch dein RS um 1 steigt. Die BE verändert sich dadurch nicht.\newline Dieser Vorteil kann üblicherweise nur von Achaz, Orks und ähnlichen Rassen gewählt werden.}
}


\newglossaryentry{paktiererI_Vorteil}
{
    name={Paktierer I},
    description={\textbf{Kosten}: 40 \textbf{Voraussetzungen}: Vorteil Kreis der Verdammnis I, Kein Vorteil Geweiht I \textbf{Nachkauf}: Üblich\newline Du verfügst über 8 Gunstpunkte und kannst die dämonische Tradition deines Erzdämons erlernen.}
}


\newglossaryentry{paktiererII_Vorteil}
{
    name={Paktierer II},
    description={\textbf{Kosten}: 40 \textbf{Voraussetzungen}: Vorteil Paktierer I \textbf{Nachkauf}: Üblich\newline Du verfügst über 16 Gunstpunkte und kannst die dämonische Tradition deines Erzdämons erlernen.}
}


\newglossaryentry{paktiererIII_Vorteil}
{
    name={Paktierer III},
    description={\textbf{Kosten}: 40 \textbf{Voraussetzungen}: Vorteil Paktierer II \textbf{Nachkauf}: Üblich\newline Du verfügst über 24 Gunstpunkte und kannst die dämonische Tradition deines Erzdämons erlernen.}
}


\newglossaryentry{paktiererIV_Vorteil}
{
    name={Paktierer IV},
    description={\textbf{Kosten}: 40 \textbf{Voraussetzungen}: Vorteil Paktierer III \textbf{Nachkauf}: Üblich\newline Du verfügst über 32 Gunstpunkte und kannst die dämonische Tradition deines Erzdämons erlernen.}
}


\newglossaryentry{privilegien(Adel)_Vorteil}
{
    name={Privilegien (Adel)},
    description={\textbf{Kosten}: 40 \textbf{Nachkauf}: Selten\newline Du darfst ein „von“ im Namen tragen, an Turnieren teilnehmen und jederzeit angemessen bewaffnet und gerüstet sein. Vor Gericht kannst du nur von höheren Adeligen gerichtet werden und du darfst bei der Abwesenheit eines Richters über Leibeigene richten, solange dabei kein Blut fließt. Diese Privilegien gelten in ähnlicher Art und Weise im Mittel- und Horasreich, dem Bornland und anderen feudalen Gesellschaften.}
}


\newglossaryentry{privilegien(Krieger)_Vorteil}
{
    name={Privilegien (Krieger)},
    description={\textbf{Kosten}: 20 \textbf{Nachkauf}: Selten\newline Als Krieger/Schwertgeselle/Rondrageweihter darfst du an Turnieren teilnehmen und in Städten Waffen tragen. Diese Privilegien gelten im Mittel- und Horasreich, sowie fast überall, wo es einen anerkannten Kriegerstand gibt.}
}


\newglossaryentry{privilegien(Gildenmagier)_Vorteil}
{
    name={Privilegien (Gildenmagier)},
    description={\textbf{Kosten}: 20 \textbf{Voraussetzungen}: Vorteil Zauberer I, Vorteil Tradition der Gildenmagier I \textbf{Nachkauf}: Selten\newline Du darfst Geld für magische Dienstleistungen verlangen und hast erleichterten Zugang zu Bibliotheken und Lehrmeistern der eigenen Gilde. Von weltlichen Gerichten kannst du nur in deiner Anwesenheit verurteilt werden oder du wirst vor ein Gildengericht gestellt. Im Gegenzug musst du dich den Bedingungen des Codex Albyricus unterwerfen, also stets als Gildenmagier erkenntlich sein. Auch die meisten Waffen und fast alle Rüstungen mit einem RS von mehr als 1 sind verboten. Diese Privilegien gelten im Einflussgebiet des Codex Albyricus, wie im Mittel­ und Horasreich, dem Bornland, im Kalifat und in Aranien, sowie eingeschränkt in Südaventurien.}
}


\newglossaryentry{privilegien_Vorteil}
{
    name={Privilegien},
    description={\textbf{Kosten}: 20 \textbf{Nachkauf}: Selten\newline Unzählige Privilegien können die Leben ihrer Träger erleichtern, weswegen wir auch nicht jedes einzelne Privileg in Werte kleiden möchten.\newline Die Kosten eines Privilegs legt der Spielleiter fest, wobei er sich an der Spielrelevanz und dem Geltungsbereich der Privilegien orientieren sollte. Zum Beispiel bringt ein eigenes Wappen keine spielrelevanten Vorteile - ganz im Gegensatz zu der Erlaubnis, in Städten Waffen zu tragen. Doch auch diese Erlaubnis verliert an Wert, wenn sie nur in Albenhus oder  am Markttag gilt. \newline Sephrasto: Wähle am besten direkt den separaten Vorteil Privilegien (Adel, Krieger oder Gildenmagier). Falls du andere Privilegien möchtest, dann wähle diesen Vorteil und trage diese in das Kommentarfeld ein.}
}


\newglossaryentry{prophezeien_Vorteil}
{
    name={Prophezeien},
    description={\textbf{Kosten}: 40 \textbf{Nachkauf}: Selten\newline Du kannst das Talent Willenskraft nutzen, um mit Spielkarten, Würfeln, Astrologie, Drogen oder prophetischen Träumen einen vagen und meist mehrdeutigen Blick in die Zukunft zu werfen. Durch den Einsatz der Gabe erleidest du 1 Punkt Erschöpfung.}
}


\newglossaryentry{resistenzgegenGifte_Vorteil}
{
    name={Resistenz gegen Gifte},
    description={\textbf{Kosten}: 40 \textbf{Nachkauf}: Selten\newline Resistenz mildert die Auswirkungen von Giften (S. 35).}
}


\newglossaryentry{immunitätgegenGifte_Vorteil}
{
    name={Immunität gegen Gifte},
    description={\textbf{Kosten}: 40 \textbf{Voraussetzungen}: Vorteil Resistenz gegen Gifte \textbf{Nachkauf}: Extrem selten\newline Du bist immun gegen Gifte (S. 35).}
}


\newglossaryentry{resistenzgegenKrankheiten_Vorteil}
{
    name={Resistenz gegen Krankheiten},
    description={\textbf{Kosten}: 20 \textbf{Nachkauf}: Selten\newline Resistenz mildert die Auswirkungen von Krankheiten (S. 35).}
}


\newglossaryentry{immunitätgegenKrankheiten_Vorteil}
{
    name={Immunität gegen Krankheiten},
    description={\textbf{Kosten}: 20 \textbf{Voraussetzungen}: Vorteil Resistenz gegen Krankheiten \textbf{Nachkauf}: Extrem selten\newline Du bist immun gegen Krankheiten (S.35).}
}


\newglossaryentry{resistenzgegenHitze_Vorteil}
{
    name={Resistenz gegen Hitze},
    description={\textbf{Kosten}: 40 \textbf{Nachkauf}: Selten\newline Verschiebt die Temperaturstufe bei hohen Temperaturen um eine Stufe in Richtung normal.}
}


\newglossaryentry{resistenzgegenKälte_Vorteil}
{
    name={Resistenz gegen Kälte},
    description={\textbf{Kosten}: 40 \textbf{Nachkauf}: Selten\newline Verschiebt die Temperaturstufe bei niedrigen Temperaturen um eine Stufe in Richtung normal.}
}


\newglossaryentry{tierempathie_Vorteil}
{
    name={Tierempathie},
    description={\textbf{Kosten}: 60 \textbf{Nachkauf}: Extrem selten\newline Mit der Gabe der Tierempathie kannst du das Talent Überleben verwenden, um die Gedanken von Tieren zu verstehen. Zusätzlich kannst du ihnen sogar einfache Botschaften zukommen lassen. Nach dem aktiven Einsatz der Gabe erleidest du 1 Punkt Erschöpfung.\newline Die Kosten dieses Vorteils hängen von der Größe der Gruppe von Tieren ab, mit denen du kommunizieren kannst, und dem möglichen Spielnutzen dieser Fähigkeit. Beispiele wären:\newline - Fische: Du kannst mit allen Fischen kommunizieren (20 EP).\newline - Vögel: Du kannst mit allen Vögeln kommunizieren (40 EP).\newline - Alle Tiere: Du kannst mit allen Tieren kommunizieren (60 EP).\newline Sephrasto: Trage die gewählte Gruppe von Tieren in das Kommentarfeld ein.}
}


\newglossaryentry{verbindungen_Vorteil}
{
    name={Verbindungen},
    description={\textbf{Kosten}: 20 \textbf{Nachkauf}: Üblich\newline Du hast einen guten Draht zu einer bestimmten Organisation oder Person, die dir üblicherweise freundlich gegenübersteht. Außerdem gelten passende Verbindungen als Werkzeug für Recherchen und viele andere Tätigkeiten. Du kannst Verbindungen zu mehreren Organisationen und Personen haben. Die Kosten deiner Verbindungen legt der Spielleiter fest, wobei er sich an der Macht und dem Einflussbereich der Verbindungen orientieren sollte.\newline Beispiele für Verbindungen sind:\newline - Örtlicher Baron: Starker lokaler Einfluss mit geringem Einflussbereich (20 EP).\newline - Badilakaner: Geringer Einfluss in weiten Teilen des zwölfgöttergläubigen Aventuriens (20 EP).\newline - Efferdbrüder: Ansehnlicher Einfluss in Hafenstädten des zwölfgöttergläubigen Aventuriens (40 EP).\newline - Haus Gareth: Immenser Einfluss im Mittelreich, weitreichende Beziehungen in ganz Aventurien (80 EP).\newline Sephrasto: Trage die gewählten Verbindungen in das Kommentarfeld ein. Wenn du mehrere Verbindungen hast, dann erhöhe die EP-Kosten entsprechend.}
}


\newglossaryentry{zaubererI_Vorteil}
{
    name={Zauberer I},
    description={\textbf{Kosten}: 40 \textbf{Nachkauf}: Extrem selten\newline Du verfügst über 8 Astralpunkte und kannst magische Traditionen erlernen. Dämonen, Elementare oder Hellsichtsmagier können dich magisch wahrnehmen. Zauberer I entspricht einem einfachen Magiedilettanten.}
}


\newglossaryentry{zaubererII_Vorteil}
{
    name={Zauberer II},
    description={\textbf{Kosten}: 40 \textbf{Voraussetzungen}: Vorteil Zauberer I \textbf{Nachkauf}: Üblich\newline Du verfügst über 16 Astralpunkte und kannst magische Traditionen erlernen. Dämonen, Elementare oder Hellsichtsmagier können dich magisch wahrnehmen. Zauberer II entspricht einem wenig begabten Alchemisten. }
}


\newglossaryentry{zaubererIII_Vorteil}
{
    name={Zauberer III},
    description={\textbf{Kosten}: 40 \textbf{Voraussetzungen}: Vorteil Zauberer II \textbf{Nachkauf}: Üblich\newline Du verfügst über 24 Astralpunkte und kannst magische Traditionen erlernen. Dämonen, Elementare oder Hellsichtsmagier können dich magisch wahrnehmen. Zauberer III entspricht einem ausgebildeten Zauberer.}
}


\newglossaryentry{zaubererIV_Vorteil}
{
    name={Zauberer IV},
    description={\textbf{Kosten}: 40 \textbf{Voraussetzungen}: Vorteil Zauberer III \textbf{Nachkauf}: Üblich\newline Du verfügst über 32 Astralpunkte und kannst magische Traditionen erlernen. Dämonen, Elementare oder Hellsichtsmagier können dich magisch wahrnehmen. Zauberer IV entspricht einem sehr kompetenten Zauberer.}
}


\newglossaryentry{zwergennase_Vorteil}
{
    name={Zwergennase},
    description={\textbf{Kosten}: 60 \textbf{Nachkauf}: Extrem selten\newline Mit der Gabe der Zwergennase besitzt du einen übernatürlichen Riecher für Verstecke, Geheimgänge und mechanische Fallen. Du kannst sie mit dem Talent Wachsamkeit wahrnehmen, auch wenn das mit gewöhnlichen Sinnen unmöglich wäre. Ist das Objekt auch mit gewöhnlichen Sinnen wahrnehmbar, erhältst du eine Erleichterung von +4.}
}


\newglossaryentry{eindrucksvollI_Vorteil}
{
    name={Eindrucksvoll I},
    description={\textbf{Kosten}: 20 \textbf{Voraussetzungen}: Attribut CH 4 \textbf{Nachkauf}: Häufig\newline In Rededuellen sind Proben auf Betören und Einschüchtern um +2 erleichtert.}
}


\newglossaryentry{eindrucksvollII_Vorteil}
{
    name={Eindrucksvoll II},
    description={\textbf{Kosten}: 40 \textbf{Voraussetzungen}: Attribut CH 6 \textbf{Nachkauf}: Häufig\newline In Rededuellen sind Proben auf Betören und Einschüchtern um +2 erleichtert.\newline Bei Proben auf Betören, Einschüchtern und Gebräuche gelten Patzer als gewöhnlich misslungen, außer sie entstehen durch eine ungewohnte Umgebung (S. 55).}
}


\newglossaryentry{sozialeAnpassungsfähigkeit_Vorteil}
{
    name={Soziale Anpassungsfähigkeit},
    description={\textbf{Kosten}: 60 \textbf{Voraussetzungen}: Attribut CH 8 \textbf{Nachkauf}: Häufig\newline Du bist immun gegen alle Auswirkungen durch ungewohnte Umgebung (S. 55).}
}


\newglossaryentry{starkeAura_Vorteil}
{
    name={Starke Aura},
    description={\textbf{Kosten}: 80 \textbf{Voraussetzungen}: Attribut CH 10 \textbf{Nachkauf}: Häufig\newline Während eines Rededuells kannst du eine CH-Probe gegen die Willenskraft deines Gegenübers ablegen. Wenn die Probe gelingt, kannst du eine Probe in diesem Rededuell wiederholen.}
}


\newglossaryentry{routiniertI_Vorteil}
{
    name={Routiniert I},
    description={\textbf{Kosten}: 20 \textbf{Voraussetzungen}: Attribut FF 4 \textbf{Nachkauf}: Häufig\newline Die Basiszeit zur Herstellung handwerklicher Produkte, zur Wundversorgung und zum Schlösser knacken verkürzt sich um ein Viertel des unmodifizierten Werts.}
}


\newglossaryentry{routiniertII_Vorteil}
{
    name={Routiniert II},
    description={\textbf{Kosten}: 40 \textbf{Voraussetzungen}: Attribut FF 6 \textbf{Nachkauf}: Häufig\newline Die Basiszeit zur Herstellung handwerklicher Produkte, zur Wundversorgung und zum Schlösser knacken verkürzt sich um ein Viertel des unmodifizierten Werts.\newline Bei Proben auf Alchemie, Handwerk, Heilkunde und Verschlagenheit gelten Patzer als gewöhnlich misslungen.}
}


\newglossaryentry{improvisation_Vorteil}
{
    name={Improvisation},
    description={\textbf{Kosten}: 60 \textbf{Voraussetzungen}: Attribut FF 8 \textbf{Nachkauf}: Häufig\newline Unzureichendes Werkzeug oder minderwertige Verbrauchsmaterialien gelten um eine Stufe höher.}
}


\newglossaryentry{meisterwerk_Vorteil}
{
    name={Meisterwerk},
    description={\textbf{Kosten}: 80 \textbf{Voraussetzungen}: Attribut FF 10 \textbf{Nachkauf}: Häufig\newline Mit hervorragenden Materialien und dem doppelten Zeitaufwand kannst du ein Meisterwerk schaffen. Dieses erhält zusätzlich 2x die Modifikation hohe Qualität und kann verbesserte Eigenschaften nach Meisterentscheid haben. Solche Werke sind extrem selten und Gegenstand von Sagen und Legenden. Der Vorteil kann nur mit Fertigkeiten mit einem unmodifizierten Probenwert von mindestens 16 angewandt werden. }
}


\newglossaryentry{zerstörerischI_Vorteil}
{
    name={Zerstörerisch I},
    description={\textbf{Kosten}: 20 \textbf{Voraussetzungen}: Attribut KK 4 \textbf{Nachkauf}: Häufig\newline Proben zum Zerstören oder Durchbrechen von Gegenständen sind um +4 erleichtert.}
}


\newglossaryentry{zerstörerischII_Vorteil}
{
    name={Zerstörerisch II},
    description={\textbf{Kosten}: 40 \textbf{Voraussetzungen}: Attribut KK 6 \textbf{Nachkauf}: Häufig\newline Proben zum Zerstören oder Durchbrechen von Gegenständen sind um +4 erleichtert.\newline Erlaubt das Manöver Hammerschlag gegen Gegenstände. Bei waffenlosen Angriffen gegen Gegenstände verletzt du dich normalerweise nicht.}
}


\newglossaryentry{muskelprotz_Vorteil}
{
    name={Muskelprotz},
    description={\textbf{Kosten}: 60 \textbf{Voraussetzungen}: Attribut KK 8 \textbf{Nachkauf}: Häufig\newline Einschüchtern-Proben im Kampf sind um +4 erleichtert und du kannst ohne zusätzliche Erschwernis bis zu 4 Gegner einschüchtern.}
}


\newglossaryentry{adrenalinschub_Vorteil}
{
    name={Adrenalinschub},
    description={\textbf{Kosten}: 80 \textbf{Voraussetzungen}: Attribut KK 10 \textbf{Nachkauf}: Häufig\newline Du kannst dir eine Probe bei einer körperlichen Tätigkeit (z.B. Schmieden, Laufen, Nahkampfangriff) um +4 erleichtern. Anschließend erleidest du einen Punkt Erschöpfung.}
}


\newglossaryentry{scharfsinnigI_Vorteil}
{
    name={Scharfsinnig I},
    description={\textbf{Kosten}: 20 \textbf{Voraussetzungen}: Attribut KL 4 \textbf{Nachkauf}: Häufig\newline Proben bei einer Ermittlung oder Recherche sind um +2 erleichtert.}
}


\newglossaryentry{scharfsinnigII_Vorteil}
{
    name={Scharfsinnig II},
    description={\textbf{Kosten}: 40 \textbf{Voraussetzungen}: Attribut KL 6 \textbf{Nachkauf}: Häufig\newline Proben bei einer Ermittlung oder Recherche sind um +2 erleichtert.\newline Bei Recherchen erhältst du einen zusätzlichen Informationsgrad, wenn der gewertete Würfel eine 12 oder höher zeigt.}
}


\newglossaryentry{vorbereitung_Vorteil}
{
    name={Vorbereitung},
    description={\textbf{Kosten}: 60 \textbf{Voraussetzungen}: Attribut KL 8 \textbf{Nachkauf}: Häufig\newline Du kannst Proben auf profane Fertigkeiten besonders sorgfältig vorbereiten (sofern Vorbereitung sinnvoll ist). Wenn du die doppelte notwendige Zeit aufwendest, erhältst du +4 Punkte Erleichterung.}
}


\newglossaryentry{eingebung_Vorteil}
{
    name={Eingebung},
    description={\textbf{Kosten}: 80 \textbf{Voraussetzungen}: Attribut KL 10 \textbf{Nachkauf}: Häufig\newline Pro Abenteuer kannst du den Spielleiter W3 mal um einen Tipp bitten. Diese Tipps sollen nicht das Abenteuer lösen, können dich aber auf Fehler in deinem Plan oder bisher übersehene Aspekte oder Zusammenhänge aufmerksam machen. }
}


\newglossaryentry{flinkI_Vorteil}
{
    name={Flink I},
    description={\textbf{Kosten}: 20 \textbf{Voraussetzungen}: Attribut GE 4 \textbf{Nachkauf}: Häufig\newline Deine GS steigt um einen Punkt.}
}


\newglossaryentry{flinkII_Vorteil}
{
    name={Flink II},
    description={\textbf{Kosten}: 40 \textbf{Voraussetzungen}: Attribut GE 6 \textbf{Nachkauf}: Häufig\newline Deine GS steigt um einen Punkt.\newline Bei Athletikproben gelten Patzer als gewöhnlich misslungen.}
}


\newglossaryentry{katzenhaft_Vorteil}
{
    name={Katzenhaft},
    description={\textbf{Kosten}: 60 \textbf{Voraussetzungen}: Attribut GE 8 \textbf{Nachkauf}: Häufig\newline Erschwernisse durch unsicheren Untergrund sinken um eine Stufe und gegen den Schaden aus Stürzen, Zusammenstößen usw. kannst du deine GE als WS verwenden.}
}


\newglossaryentry{körperbeherrschung_Vorteil}
{
    name={Körperbeherrschung},
    description={\textbf{Kosten}: 80 \textbf{Voraussetzungen}: Attribut GE 10 \textbf{Nachkauf}: Häufig\newline Du kannst Nah- oder Fernkampfangriffen, elementaren Schadenszaubern oder ähnlichen Schadensquellen mit einer GE-Gegenprobe entgehen. Der Einsatz von Körperbeherrschung wird angesagt, nachdem übliche Verteidigungsmöglichkeiten (wie eine Verteidigung) versagt haben, aber bevor der Schaden bestimmt wird. Anschließend erleidest du einen Punkt Erschöpfung.}
}


\newglossaryentry{vorausschauendI_Vorteil}
{
    name={Vorausschauend I},
    description={\textbf{Kosten}: 20 \textbf{Voraussetzungen}: Attribut IN 4 \textbf{Nachkauf}: Häufig\newline In Rededuellen sind Proben auf Rhetorik und Überreden um +2 erleichtert.}
}


\newglossaryentry{vorausschauendII_Vorteil}
{
    name={Vorausschauend II},
    description={\textbf{Kosten}: 40 \textbf{Voraussetzungen}: Attribut IN 6 \textbf{Nachkauf}: Häufig\newline In Rededuellen sind Proben auf Rhetorik und Überreden um +2 erleichtert.\newline Bei Proben auf Überreden, Rhetorik und Menschenkenntnis gelten Patzer als gewöhnlich misslungen, außer sie entstehen durch eine ungewohnte Umgebung (S. 55).}
}


\newglossaryentry{bedächtig_Vorteil}
{
    name={Bedächtig},
    description={\textbf{Kosten}: 60 \textbf{Voraussetzungen}: Attribut IN 8 \textbf{Nachkauf}: Häufig\newline Wenn du in einem Rededuell abwartest, ist deine Probe um +4 erleichtert.}
}


\newglossaryentry{empathie_Vorteil}
{
    name={Empathie},
    description={\textbf{Kosten}: 80 \textbf{Voraussetzungen}: Attribut IN 10 \textbf{Nachkauf}: Häufig\newline Du kannst eine IN-Probe gegen die Willenskraft deines Gegenübers ablegen. Wenn die Probe gelingt, erfährst du eine seiner Schwächen (S. 16). Eine einmal abgelegte Probe, egal ob ge- oder misslungen, kannst du nur wiederholen, wenn du das Gegenüber besser kennengelernt hast.}
}


\newglossaryentry{abgehärtetI_Vorteil}
{
    name={Abgehärtet I},
    description={\textbf{Kosten}: 20 \textbf{Voraussetzungen}: Attribut KO 4 \textbf{Nachkauf}: Häufig\newline Proben zur Abwehr von Giften und Krankheiten sind um +4 erleichtert.}
}


\newglossaryentry{abgehärtetII_Vorteil}
{
    name={Abgehärtet II},
    description={\textbf{Kosten}: 40 \textbf{Voraussetzungen}: Attribut KO 6 \textbf{Nachkauf}: Häufig\newline Proben zur Abwehr von Giften und Krankheiten sind um +4 erleichtert und das Durchhaltevermögen (S. 19) steigt um 2.}
}


\newglossaryentry{schnelleHeilung_Vorteil}
{
    name={Schnelle Heilung},
    description={\textbf{Kosten}: 60 \textbf{Voraussetzungen}: Attribut KO 8 \textbf{Nachkauf}: Häufig\newline Du regenerierst 2 Wunden pro durchschlafener Nacht und immer noch 1 Wunde, wenn die Schlafphase unterbrochen wurde.}
}


\newglossaryentry{unverwüstlich_Vorteil}
{
    name={Unverwüstlich},
    description={\textbf{Kosten}: 80 \textbf{Voraussetzungen}: Attribut KO 10 \textbf{Nachkauf}: Häufig\newline Deine WS steigt um 1 Punkt.\newline Wenn Zähigkeitsproben zum Ignorieren von Kampfunfähigkeit misslingen, erleidest du keine Erschöpfung.}
}


\newglossaryentry{willensstarkI_Vorteil}
{
    name={Willensstark I},
    description={\textbf{Kosten}: 20 \textbf{Voraussetzungen}: Attribut MU 4 \textbf{Nachkauf}: Häufig\newline Deine MR steigt um 4 Punkte.}
}


\newglossaryentry{willensstarkII_Vorteil}
{
    name={Willensstark II},
    description={\textbf{Kosten}: 40 \textbf{Voraussetzungen}: Attribut MU 6 \textbf{Nachkauf}: Häufig\newline Deine MR steigt um 4 Punkte.\newline Auf dich wirkende Furcht-Effekte (S. 46) zählen als eine Stufe niedriger.}
}


\newglossaryentry{geisterpanzer_Vorteil}
{
    name={Geisterpanzer},
    description={\textbf{Kosten}: 60 \textbf{Voraussetzungen}: Attribut MU 8 \textbf{Nachkauf}: Häufig\newline Gegen direkten Schaden aus Zaubern wie Fulminictus, Ignisphaero oder Hexengalle kannst du deinen MU als WS verwenden.}
}


\newglossaryentry{unbeugsamkeit_Vorteil}
{
    name={Unbeugsamkeit},
    description={\textbf{Kosten}: 80 \textbf{Voraussetzungen}: Attribut MU 10 \textbf{Nachkauf}: Häufig\newline Deine MR steigt um MU/2 Punkte.\newline Mit einer Aktion Konflikt und einer Konterprobe (MU, 16) kannst du einen auf dir liegenden Zauber abschütteln. Anschließend erleidest du einen Punkt Erschöpfung.}
}


\newglossaryentry{kommando:HaltetStand!_Vorteil}
{
    name={Kommando: Haltet Stand!},
    description={\textbf{Kosten}: 20 \textbf{Voraussetzungen}: Attribut CH 4 \textbf{Nachkauf}: Häufig\newline Furcht-Effekte für Mitstreiter zählen als eine Stufe niedriger.}
}


\newglossaryentry{kommando:FormiertEuch!_Vorteil}
{
    name={Kommando: Formiert Euch!},
    description={\textbf{Kosten}: 40 \textbf{Voraussetzungen}: Attribut CH 6 \textbf{Nachkauf}: Häufig\newline Alle Mitstreiter erhalten VT +1.}
}


\newglossaryentry{kommando:KeineGefangenen!_Vorteil}
{
    name={Kommando: Keine Gefangenen!},
    description={\textbf{Kosten}: 60 \textbf{Voraussetzungen}: Attribut CH 8 \textbf{Nachkauf}: Häufig\newline Alle Mitstreiter erhalten TP +2.}
}


\newglossaryentry{kommando:KenntKeinenSchmerz!_Vorteil}
{
    name={Kommando: Kennt Keinen Schmerz!},
    description={\textbf{Kosten}: 80 \textbf{Voraussetzungen}: Attribut CH 10 \textbf{Nachkauf}: Häufig\newline Alle Mitstreiter erhalten WS +1.}
}


\newglossaryentry{ruhigeHand_Vorteil}
{
    name={Ruhige Hand},
    description={\textbf{Kosten}: 20 \textbf{Voraussetzungen}: Attribut FF 4 \textbf{Nachkauf}: Häufig\newline Verdoppelt im Fernkampf den Bonus durch Zielen.}
}


\newglossaryentry{reflexschuss_Vorteil}
{
    name={Reflexschuss},
    description={\textbf{Kosten}: 40 \textbf{Voraussetzungen}: Attribut FF 6 \textbf{Nachkauf}: Häufig\newline Im Fernkampf sinken Abzüge durch Bewegung oder Wind um 1 Stufe.}
}


\newglossaryentry{schnellziehen_Vorteil}
{
    name={Schnellziehen},
    description={\textbf{Kosten}: 60 \textbf{Voraussetzungen}: Attribut FF 8 \textbf{Nachkauf}: Häufig\newline Erlaubt das Manöver Schnellschuss (FK -4) im Fernkampf: Verkürzt die Vorbereitungszeit um die Hälfte. Eine Vorbereitungszeit von 1 Aktion wird zu 0 Aktionen. Die Aktion Konflikt ist von dieser Modifikation nicht betroffen. Kann bis zu zweimal pro FK eingesetzt werden.\newline Nahkampf- und Wurfwaffen können in einer freien Aktion gezogen werden.}
}


\newglossaryentry{meisterschuss_Vorteil}
{
    name={Meisterschuss},
    description={\textbf{Kosten}: 80 \textbf{Voraussetzungen}: Attribut FF 10 \textbf{Nachkauf}: Häufig\newline Erlaubt das Manöver Meisterschuss (FK -8) im Fernkampf: Der Angriff richtet zwei zusätzliche Wunden an - auch wenn der Schaden zu gering ist, um eine Wunde anzurichten.}
}


\newglossaryentry{niederwerfen_Vorteil}
{
    name={Niederwerfen},
    description={\textbf{Kosten}: 20 \textbf{Voraussetzungen}: Attribut KK 4 \textbf{Nachkauf}: Häufig\newline Erlaubt das Manöver Niederwerfen (AT -4): Gegenprobe: KK. Dein Ziel stürzt und liegt am Boden.}
}


\newglossaryentry{waffenloserKampf_Vorteil}
{
    name={Waffenloser Kampf},
    description={\textbf{Kosten}: 40 \textbf{Voraussetzungen}: Attribut KK 6 \textbf{Nachkauf}: Häufig\newline Deine unbewaffneten Angriffe erhalten die Waffeneigenschaften Kopflastig und Wendig (S. 47).}
}


\newglossaryentry{hammerschlag_Vorteil}
{
    name={Hammerschlag},
    description={\textbf{Kosten}: 60 \textbf{Voraussetzungen}: Attribut KK 8 \textbf{Nachkauf}: Häufig\newline Erlaubt das Manöver Hammerschlag (AT -8): Der Angriff richtet doppelten Waffenschaden an. Zusätzliche TP aus Kommandos, anderen Manövern usw. werden nicht verdoppelt.}
}


\newglossaryentry{unaufhaltsam_Vorteil}
{
    name={Unaufhaltsam},
    description={\textbf{Kosten}: 80 \textbf{Voraussetzungen}: Attribut KK 10 \textbf{Nachkauf}: Häufig\newline Selbst bei einer gelungenen gegnerischen Verteidigung richtest du halben Schaden an. Ausweichen vermeidet diesen Schaden. Manövereffekte wirken weiterhin nur bei gelungenen Attacken.}
}


\newglossaryentry{standfest_Vorteil}
{
    name={Standfest},
    description={\textbf{Kosten}: 20 \textbf{Voraussetzungen}: Attribut GE 4 \textbf{Nachkauf}: Häufig\newline Nach einem Sturz befindest du dich in kniender statt liegender Position.}
}


\newglossaryentry{sturmangriff_Vorteil}
{
    name={Sturmangriff},
    description={\textbf{Kosten}: 40 \textbf{Voraussetzungen}: Attribut GE 6 \textbf{Nachkauf}: Häufig\newline Erlaubt das Manöver Sturmangriff (Bewegung und AT): Die Kombination der Aktionen Bewegung und Konflikt ist nicht erschwert. Außerdem richtet die nächste Attacke GS Trefferpunkte zusätzlich an.}
}


\newglossaryentry{todesstoß_Vorteil}
{
    name={Todesstoß},
    description={\textbf{Kosten}: 60 \textbf{Voraussetzungen}: Attribut GE 8 \textbf{Nachkauf}: Häufig\newline Erlaubt das Manöver Todesstoß (AT -8): Der Angriff richtet zwei zusätzliche Wunden an, auch wenn der Schaden zu gering ist, um Wunden anzurichten.}
}


\newglossaryentry{präzision_Vorteil}
{
    name={Präzision},
    description={\textbf{Kosten}: 80 \textbf{Voraussetzungen}: Attribut GE 10 \textbf{Nachkauf}: Häufig\newline Zeigt der gewertete Würfel bei einem Nahkampfangriff eine 16 oder höher, richtet der Angriff GE TP zusätzlich an.}
}


\newglossaryentry{ausfall_Vorteil}
{
    name={Ausfall},
    description={\textbf{Kosten}: 20 \textbf{Voraussetzungen}: Attribut MU 4 \textbf{Nachkauf}: Häufig\newline Erlaubt das Manöver Ausfall (AT -2 -BE): Das Ziel muss in einer Freien Reaktion bis zu 2 Schritt zurückweichen, du folgst in einer Freien Aktion. Möchte es in seiner nächsten Initiativephase einen anderen Gegner als dich angreifen, darfst du einen Passierschlag ausführen.}
}


\newglossaryentry{offensiverKampfstil_Vorteil}
{
    name={Offensiver Kampfstil},
    description={\textbf{Kosten}: 40 \textbf{Voraussetzungen}: Attribut MU 6 \textbf{Nachkauf}: Häufig\newline Der Verteidigungs-Malus durch volle Offensive sinkt auf -4.}
}


\newglossaryentry{gegenhalten_Vorteil}
{
    name={Gegenhalten},
    description={\textbf{Kosten}: 60 \textbf{Voraussetzungen}: Attribut MU 8 \textbf{Nachkauf}: Häufig\newline Wenn du eine Attacke abwehrst, die mit einem Manöver verstärkt wurde, darfst du sofort einen Passierschlag gegen den Angreifer ausführen.}
}


\newglossaryentry{kalteWut_Vorteil}
{
    name={Kalte Wut},
    description={\textbf{Kosten}: 80 \textbf{Voraussetzungen}: Attribut MU 10 \textbf{Nachkauf}: Häufig\newline Der Held kann sich im Kampf in kalte Wut versetzen und Wundabzüge ignorieren.}
}


\newglossaryentry{kampfreflexe_Vorteil}
{
    name={Kampfreflexe},
    description={\textbf{Kosten}: 20 \textbf{Voraussetzungen}: Attribut IN 4 \textbf{Nachkauf}: Häufig\newline Deine INI steigt um 4 Punkte.}
}


\newglossaryentry{defensiverKampfstil_Vorteil}
{
    name={Defensiver Kampfstil},
    description={\textbf{Kosten}: 40 \textbf{Voraussetzungen}: Attribut IN 6 \textbf{Nachkauf}: Häufig\newline Volle Defensive (S. 36) ist für dich eine einfache Aktion. Dadurch könntest du beispielsweise zusätzlich die Aktion Konflikt wählen, um eine AT –4 ausführen.}
}


\newglossaryentry{aufmerksamkeit_Vorteil}
{
    name={Aufmerksamkeit},
    description={\textbf{Kosten}: 60 \textbf{Voraussetzungen}: Attribut IN 8 \textbf{Nachkauf}: Häufig\newline Misslingt deinem Gegner eine Verteidigung gegen dich, die mit einem Manöver verstärkt wurde, darfst du sofort einen Passierschlag gegen ihn ausführen. Pro Runde kannst du die erste Verteidigung gegen einen Passierschlag als Freie Reaktion ausführen.}
}


\newglossaryentry{klingentanz_Vorteil}
{
    name={Klingentanz},
    description={\textbf{Kosten}: 80 \textbf{Voraussetzungen}: Attribut IN 10 \textbf{Nachkauf}: Häufig\newline Erlaubt das Manöver Klingentanz (AT -4): Wenn der Angriff gelingt, darfst du sofort noch einmal angreifen. Dieser Folgeangriff ist nicht erschwert.}
}


\newglossaryentry{durchatmen_Vorteil}
{
    name={Durchatmen},
    description={\textbf{Kosten}: 20 \textbf{Voraussetzungen}: Attribut KO 4 \textbf{Nachkauf}: Häufig\newline Einmal zwischen zwei Ruhephasen kannst du eine Aktion Konzentration aufwenden und eine Zähigkeits-Probe (16) ablegen. Gelingt die Probe, regenerierst du sofort 1 Wunde oder 2 Erschöpfung, die du im laufenden Kampf erlitten hast.}
}


\newglossaryentry{rüstungsgewöhnung_Vorteil}
{
    name={Rüstungsgewöhnung},
    description={\textbf{Kosten}: 40 \textbf{Voraussetzungen}: Attribut KO 6 \textbf{Nachkauf}: Häufig\newline Die BE aller Rüstungen ist um 1 gesenkt.}
}


\newglossaryentry{atemtechnik_Vorteil}
{
    name={Atemtechnik},
    description={\textbf{Kosten}: 60 \textbf{Voraussetzungen}: Attribut KO 8 \textbf{Nachkauf}: Häufig\newline Du fügst +1W6 Trefferpunkte zu. Du aktivierst den Effekt in einer Freien Aktion. Der Effekt endet, wenn du auf einen Schlag zwei Wunden erleidest oder nach KO Initiativephasen. Dann erleidest du einen Punkt Erschöpfung.}
}


\newglossaryentry{verbesserteRüstungsgewöhnung_Vorteil}
{
    name={Verbesserte Rüstungsgewöhnung},
    description={\textbf{Kosten}: 80 \textbf{Voraussetzungen}: Attribut KO 10 \textbf{Nachkauf}: Häufig\newline Die BE aller Rüstungen ist um 2 gesenkt.}
}


\newglossaryentry{beidhändigerKampfI_Vorteil}
{
    name={Beidhändiger Kampf I},
    description={\textbf{Kosten}: 20 \textbf{Voraussetzungen}: Attribut GE 4 \textbf{Nachkauf}: Häufig\newline Bedingungen: zwei einhändige Waffen, kein Schild, nicht beritten.\newline Du erhältst AT +1.}
}


\newglossaryentry{beidhändigerKampfII_Vorteil}
{
    name={Beidhändiger Kampf II},
    description={\textbf{Kosten}: 40 \textbf{Voraussetzungen}: Attribut GE 6, Vorteil Beidhändiger Kampf I \textbf{Nachkauf}: Häufig\newline Du erhältst AT +1.\newline Deine zweite Waffe ignoriert die übliche Erschwernis für Nebenwaffen. Dadurch kannst du den vollen Nutzen aus zwei unterschiedlichen Waffen ziehen. Falls du dennoch zwei identische Waffen nutzt, kannst du den ersten (bzw. mit Wendigen Waffen zweiten) Passierschlag zwischen zwei INI-Phasen als Freie Reaktion ausführen.}
}


\newglossaryentry{beidhändigerKampfIII_Vorteil}
{
    name={Beidhändiger Kampf III},
    description={\textbf{Kosten}: 60 \textbf{Voraussetzungen}: Attribut GE 8, Vorteil Beidhändiger Kampf II \textbf{Nachkauf}: Häufig\newline Du erhältst AT +1.\newline Erlaubt das Manöver Doppelangriff (2x AT -4): Du führst mit beiden Händen je eine Attacke –4 aus. Beide Angriffe sind unabhängig voneinander; sie werden separat ausgewürfelt, können sich gegen verschiedene Ziele richten und mit unterschiedlichen Manövern versehen werden.}
}


\newglossaryentry{beidhändigerKampfIV_Vorteil}
{
    name={Beidhändiger Kampf IV},
    description={\textbf{Kosten}: 80 \textbf{Voraussetzungen}: Attribut GE 10, Vorteil Beidhändiger Kampf III \textbf{Nachkauf}: Häufig\newline Bedingung: zwei weitere Attribute auf insgesamt 16.\newline 8 Punkte können zur Verbesserung des Kampfstils verwendet werden. Sephrasto: Trage die Verbesserungen in das Kommentarfeld ein.}
}


\newglossaryentry{kraftvollerKampfI_Vorteil}
{
    name={Kraftvoller Kampf I},
    description={\textbf{Kosten}: 20 \textbf{Voraussetzungen}: Attribut KK 4 \textbf{Nachkauf}: Häufig\newline Bedingungen: einzelne Nahkampfwaffe, nicht beritten.\newline Dein Waffenschaden steigt um +1.}
}


\newglossaryentry{kraftvollerKampfII_Vorteil}
{
    name={Kraftvoller Kampf II},
    description={\textbf{Kosten}: 40 \textbf{Voraussetzungen}: Attribut KK 6, Vorteil Kraftvoller Kampf I \textbf{Nachkauf}: Häufig\newline Dein Waffenschaden steigt um +1.\newline Du kannst in eine Halbschwertstellung wechseln, um deine Waffe in einer um eins kleineren Reichweite zu führen. Der Wechsel in die Stellung oder zurück geschieht in einer Freien Aktion.}
}


\newglossaryentry{kraftvollerKampfIII_Vorteil}
{
    name={Kraftvoller Kampf III},
    description={\textbf{Kosten}: 60 \textbf{Voraussetzungen}: Attribut KK 8, Vorteil Kraftvoller Kampf II \textbf{Nachkauf}: Häufig\newline Dein Waffenschaden steigt um +1.\newline Erlaubt das Manöver Befreiungsschlag (AT -4): Du führst eine Attacke aus, die sich (samt anderen Manövern) gegen alle Ziele in einem Winkel von 180° vor dir richtet. Jedem Ziel steht eine eigene Verteidigung zu.}
}


\newglossaryentry{kraftvollerKampfIV_Vorteil}
{
    name={Kraftvoller Kampf IV},
    description={\textbf{Kosten}: 80 \textbf{Voraussetzungen}: Attribut KK 10, Vorteil Kraftvoller Kampf III \textbf{Nachkauf}: Häufig\newline Bedingung: zwei weitere Attribute auf insgesamt 16.\newline 8 Punkte können zur Verbesserung des Kampfstils verwendet werden. Sephrasto: Trage die Verbesserungen in das Kommentarfeld ein.}
}


\newglossaryentry{schildkampfI_Vorteil}
{
    name={Schildkampf I},
    description={\textbf{Kosten}: 20 \textbf{Voraussetzungen}: Attribut KK 4 \textbf{Nachkauf}: Häufig\newline Bedingung: Schild.\newline Du erhältst VT +1.}
}


\newglossaryentry{schildkampfII_Vorteil}
{
    name={Schildkampf II},
    description={\textbf{Kosten}: 40 \textbf{Voraussetzungen}: Attribut KK 6, Vorteil Schildkampf I \textbf{Nachkauf}: Häufig\newline Du erhältst VT +1.\newline Dein Schild ignoriert die übliche Erschwernis für Nebenwaffen (S. 39). Außerdem kannst du die erste VT zwischen zwei Initiativphasen als Freie Reaktion ausführen.}
}


\newglossaryentry{schildkampfIII_Vorteil}
{
    name={Schildkampf III},
    description={\textbf{Kosten}: 60 \textbf{Voraussetzungen}: Attribut KK 8, Vorteil Schildkampf II \textbf{Nachkauf}: Häufig\newline Du erhältst VT +1.\newline Erlaubt das Manöver Schildwall (VT -4): Du kannst einen Angriff auf einen benachbarten Verbündeten abwehren. Das Manöver erfolgt vor der Verteidigung des Verbündeten.}
}


\newglossaryentry{schildkampfIV_Vorteil}
{
    name={Schildkampf IV},
    description={\textbf{Kosten}: 80 \textbf{Voraussetzungen}: Attribut KK 10, Vorteil Schildkampf III \textbf{Nachkauf}: Häufig\newline Bedingung: zwei weitere Attribute auf insgesamt 16.\newline 8 Punkte können zur Verbesserung des Kampfstils verwendet werden. Sephrasto: Trage die Verbesserungen in das Kommentarfeld ein.}
}


\newglossaryentry{schnellerKampfI_Vorteil}
{
    name={Schneller Kampf I},
    description={\textbf{Kosten}: 20 \textbf{Voraussetzungen}: Attribut GE 4 \textbf{Nachkauf}: Häufig\newline Bedingungen: einzelne Nahkampfwaffe, nicht beritten.\newline Du erhältst AT +1.}
}


\newglossaryentry{schnellerKampfII_Vorteil}
{
    name={Schneller Kampf II},
    description={\textbf{Kosten}: 40 \textbf{Voraussetzungen}: Attribut GE 6, Vorteil Schneller Kampf I \textbf{Nachkauf}: Häufig\newline Du erhältst AT +1.\newline Du kannst in eine Halbschwertstellung wechseln, um deine Waffe in einer um 1 kleineren Reichweite zu führen. Der Wechsel in die Stellung oder zurück geschieht in einer Freien Aktion.}
}


\newglossaryentry{schnellerKampfIII_Vorteil}
{
    name={Schneller Kampf III},
    description={\textbf{Kosten}: 60 \textbf{Voraussetzungen}: Attribut GE 8, Vorteil Schneller Kampf II \textbf{Nachkauf}: Häufig\newline Du erhältst AT +1.\newline Erlaubt das Manöver Unterlaufen (VT -4): Wenn die Verteidigung gelingt, darfst du in der nächsten eigenen Initiativephase eine Freie Aktion nutzen, um deinen Gegner (ein weiteres Mal) anzugreifen.}
}


\newglossaryentry{schnellerKampfIV_Vorteil}
{
    name={Schneller Kampf IV},
    description={\textbf{Kosten}: 80 \textbf{Voraussetzungen}: Attribut GE 10, Vorteil Schneller Kampf III \textbf{Nachkauf}: Häufig\newline Bedingung: zwei weitere Attribute auf insgesamt 16.\newline 8 Punkte können zur Verbesserung des Kampfstils verwendet werden. Sephrasto: Trage die Verbesserungen in das Kommentarfeld ein.}
}


\newglossaryentry{parierwaffenkampfI_Vorteil}
{
    name={Parierwaffenkampf I},
    description={\textbf{Kosten}: 20 \textbf{Voraussetzungen}: Attribut GE 4 \textbf{Nachkauf}: Häufig\newline Bedingungen: Parierwaffe, nicht beritten.\newline Du kannst gegen humanoide Gegner –1 Erschwernis aus Manövern ignorieren.}
}


\newglossaryentry{parierwaffenkampfII_Vorteil}
{
    name={Parierwaffenkampf II},
    description={\textbf{Kosten}: 40 \textbf{Voraussetzungen}: Attribut GE 6, Vorteil Parierwaffenkampf I \textbf{Nachkauf}: Häufig\newline Du kannst gegen humanoide Gegner –1 Erschwernis aus Manövern ignorieren.\newline Deine Parierwaffe ignoriert die üblichen Erschwernisse für Nebenwaffen (S. 39). Wenn ideale Kampfbedingungen herrschen – du also keine Erschwernisse durch Position, Untergrund und Licht erleidest – verbessert sich deine Position um eine Stufe (S. 38).}
}


\newglossaryentry{parierwaffenkampfIII_Vorteil}
{
    name={Parierwaffenkampf III},
    description={\textbf{Kosten}: 60 \textbf{Voraussetzungen}: Attribut GE 8, Vorteil Parierwaffenkampf II \textbf{Nachkauf}: Häufig\newline Du kannst gegen humanoide Gegner –1 Erschwernis aus Manövern ignorieren.\newline Erlaubt das Manöver Riposte (VT -4): Du fügst dem Angreifer den Waffenschaden deiner Waffe zu. Die Riposte ist kombinierbar mit Attackemanövern, die in diesem Fall die Verteidigung zusätzlich erschweren. Zum Beispiel wäre eine Kombination aus Riposte und Hammerschlag um –12 erschwert, würde aber doppelten Schaden anrichten.}
}


\newglossaryentry{parierwaffenkampfIV_Vorteil}
{
    name={Parierwaffenkampf IV},
    description={\textbf{Kosten}: 80 \textbf{Voraussetzungen}: Attribut GE 10, Vorteil Parierwaffenkampf III \textbf{Nachkauf}: Häufig\newline Bedingung: zwei weitere Attribute auf insgesamt 16.\newline 8 Punkte können zur Verbesserung des Kampfstils verwendet werden. Sephrasto: Trage die Verbesserungen in das Kommentarfeld ein.}
}


\newglossaryentry{reiterkampfI_Vorteil}
{
    name={Reiterkampf I},
    description={\textbf{Kosten}: 20 \textbf{Voraussetzungen}: Attribut GE 4 ODER Attribut KK 4 \textbf{Nachkauf}: Häufig\newline Bedingung: beritten.\newline Du erhältst Waffenschaden +1, AT +1 und VT +1.\newline Erlaubt das Manöver Sturmangriff (Bewegung und AT) mit dem Pferd: Die Kombination der Aktionen Bewegung und Konflikt ist nicht erschwert. Außerdem richtet die nächste Attacke GS Trefferpunkte zusätzlich an.}
}


\newglossaryentry{reiterkampfII_Vorteil}
{
    name={Reiterkampf II},
    description={\textbf{Kosten}: 40 \textbf{Voraussetzungen}: Attribut GE 6 ODER Attribut KK 6, Vorteil Reiterkampf I \textbf{Nachkauf}: Häufig\newline Du erhältst Waffenschaden +1, AT +1 und VT +1.\newline Dein Reittier ignoriert die üblichen Erschwernisse für Nebenwaffen (S. 39) und im Reiterkampf ist die BE durch Rüstungen um 1 gesenkt. Außerdem ignorierst du im Fernkampf du den Malus für berittene Schützen (S. 46).}
}


\newglossaryentry{reiterkampfIII_Vorteil}
{
    name={Reiterkampf III},
    description={\textbf{Kosten}: 60 \textbf{Voraussetzungen}: Attribut GE 8 ODER Attribut KK 8, Vorteil Reiterkampf II \textbf{Nachkauf}: Häufig\newline Du erhältst Waffenschaden +1, AT +1 und VT +1.\newline Erlaubt das Manöver Überrennen (Bewegung und AT) mit dem Pferd: Das Reittier stürmt in die gegnerische Formation. Seine Attacke trifft (samt anderen Manövern) alle Gegner in seiner Bahn, bis es durch mindestens 2 erfolgreiche Verteidigungen gestoppt wird. Außerdem richtet die Attacke GS Trefferpunkte zusätzlich an. Die Kombination der Aktionen Bewegung und Konflikt ist nicht erschwert.}
}


\newglossaryentry{reiterkampfIV_Vorteil}
{
    name={Reiterkampf IV},
    description={\textbf{Kosten}: 80 \textbf{Voraussetzungen}: Attribut GE 10 ODER Attribut KK 10, Vorteil Reiterkampf III \textbf{Nachkauf}: Häufig\newline Bedingung: zwei weitere Attribute auf insgesamt 16.\newline 8 Punkte können zur Verbesserung des Kampfstils verwendet werden. Sephrasto: Trage die Verbesserungen in das Kommentarfeld ein.}
}


\newglossaryentry{bändigerderElemente_Vorteil}
{
    name={Bändiger der Elemente},
    description={\textbf{Kosten}: 20 \textbf{Voraussetzungen}: Vorteil Zauberer I ODER Vorteil Tradition der Borbaradianer I, Attribut CH 4 \textbf{Nachkauf}: Häufig\newline Bei Elementaren sinkt die Erschwernis für schwierige und anmaßende Dienste (S. 81) um eine Stufe.}
}


\newglossaryentry{meisterderWünsche_Vorteil}
{
    name={Meister der Wünsche},
    description={\textbf{Kosten}: 40 \textbf{Voraussetzungen}: Vorteil Zauberer I ODER Vorteil Tradition der Borbaradianer I, Attribut CH 6 \textbf{Nachkauf}: Häufig\newline Nachdem ein gebundenes Elementar (S. 81) einen Dienst erfüllt hat, kannst du für die Hälfte der Basisbeschwörungskosten einen zusätzlichen Dienst erbitten.}
}


\newglossaryentry{gefäßderSterne_Vorteil}
{
    name={Gefäß der Sterne},
    description={\textbf{Kosten}: 60 \textbf{Voraussetzungen}: Vorteil Zauberer I ODER Vorteil Tradition der Borbaradianer I, Attribut CH 8 \textbf{Nachkauf}: Häufig\newline Deine maximale Astralenergie steigt um 4 + CH Punkte. Solltest du nach dem Erwerb dieses Vorteils dein CH erhöhen, erhöht das auch deine Astralenergie.}
}


\newglossaryentry{gebieterderUrgewalten_Vorteil}
{
    name={Gebieter der Urgewalten},
    description={\textbf{Kosten}: 80 \textbf{Voraussetzungen}: Vorteil Zauberer I ODER Vorteil Tradition der Borbaradianer I, Attribut CH 10 \textbf{Nachkauf}: Häufig\newline Gebundene Elementare kosten nur noch die halben gAsP und eine misslungene Beherrschungsprobe kann einmalig für die Hälfte der Beschwörungskosten wiederholt werden.}
}


\newglossaryentry{reaktivierung_Vorteil}
{
    name={Reaktivierung},
    description={\textbf{Kosten}: 20 \textbf{Voraussetzungen}: Vorteil Zauberer I ODER Vorteil Tradition der Borbaradianer I, Attribut FF 4 \textbf{Nachkauf}: Häufig\newline Du kannst wieder aufladbare Artefakte erschaffen und reaktivieren (S. 78).}
}


\newglossaryentry{matrixverständnis_Vorteil}
{
    name={Matrixverständnis},
    description={\textbf{Kosten}: 40 \textbf{Voraussetzungen}: Vorteil Zauberer I ODER Vorteil Tradition der Borbaradianer I, Attribut FF 6 \textbf{Nachkauf}: Häufig\newline Der erste wirkende Zauber, der bei der Artefakterschaffung misslingt, darf einmal wiederholt werden.\newline Bei Strukturanalysen (S. 80) erhältst du einen zusätzlichen Analysegrad, wenn der gewertete Würfel eine 12 oder höher zeigt.}
}


\newglossaryentry{semipermanenz_Vorteil}
{
    name={Semipermanenz},
    description={\textbf{Kosten}: 60 \textbf{Voraussetzungen}: Vorteil Zauberer I ODER Vorteil Tradition der Borbaradianer I, Attribut FF 8 \textbf{Nachkauf}: Häufig\newline Du kannst semipermanente Artefakte erschaffen (S. 78).}
}


\newglossaryentry{thaumaturg_Vorteil}
{
    name={Thaumaturg},
    description={\textbf{Kosten}: 80 \textbf{Voraussetzungen}: Vorteil Zauberer I ODER Vorteil Tradition der Borbaradianer I, Attribut FF 10 \textbf{Nachkauf}: Häufig\newline Artefakte kosten nur die halben gAsP. Außerdem können Thaumaturgen weitere Vorteile zur Artefaktherstellung nutzen: Mit Kraftlinienmagie kannst du ortsgebundene Artefakte erschaffen, ohne gAsP aufzuwenden. Mit Meister der Wünsche/Meister der Seelenlosen kannst du Elementare/Dämonen in Artefakte binden. Beim Auslösen solcher Artefakte ist keine Beherrschungsprobe nötig, solange diese nicht schwieriger wäre als die vom Erschaffer abgelegte Beherrschungsprobe (z.B. durch einen schwierigeren Dienst). }
}


\newglossaryentry{astraleRegeneration_Vorteil}
{
    name={Astrale Regeneration},
    description={\textbf{Kosten}: 20 \textbf{Voraussetzungen}: Vorteil Zauberer I ODER Vorteil Tradition der Borbaradianer I, Attribut KO 4 \textbf{Nachkauf}: Häufig\newline Du regenerierst 1 zusätzlichen AsP pro Nacht.}
}


\newglossaryentry{verbotenePforten_Vorteil}
{
    name={Verbotene Pforten},
    description={\textbf{Kosten}: 40 \textbf{Voraussetzungen}: Vorteil Zauberer I ODER Vorteil Tradition der Borbaradianer I, Attribut KO 6 \textbf{Nachkauf}: Häufig\newline Mit diesem Vorteil kannst du deine Zauber mit deiner Lebenskraft speisen. Wenn du nicht über ausreichend AsP für einen Zauber verfügst, kannst du dir selbst Wunden zufügen. Jede dieser Wunden stellt WS+4 AsP für deinen Zauber zur Verfügung, überschüssige AsP verfallen. Verwendest du zudem die Tradition der Borbaradianer, kannst du sogar WS+8 AsP pro Wunde nutzen. Die eigene Lebenskraft kann jedoch niemals genutzt werden, um sich selbst zu heilen. Die Verbotenen Pforten können übrigens keine gAsP zur Verfügung stellen – gAsP müssen immer vom Zaubernden selbst stammen.}
}


\newglossaryentry{verbesserteastraleRegeneration_Vorteil}
{
    name={Verbesserte astrale Regeneration},
    description={\textbf{Kosten}: 60 \textbf{Voraussetzungen}: Vorteil Zauberer I ODER Vorteil Tradition der Borbaradianer I, Attribut KO 8 \textbf{Nachkauf}: Häufig\newline Du regenerierst 2 weitere, zusätzliche AsP pro Nacht.}
}


\newglossaryentry{meisterlicheastraleRegeneration_Vorteil}
{
    name={Meisterliche astrale Regeneration},
    description={\textbf{Kosten}: 80 \textbf{Voraussetzungen}: Vorteil Zauberer I ODER Vorteil Tradition der Borbaradianer I, Attribut KO 10 \textbf{Nachkauf}: Häufig\newline Du regenerierst 3 weitere, zusätzliche AsP pro Nacht.}
}


\newglossaryentry{bändigerderKreaturen_Vorteil}
{
    name={Bändiger der Kreaturen},
    description={\textbf{Kosten}: 20 \textbf{Voraussetzungen}: Vorteil Zauberer I ODER Vorteil Tradition der Borbaradianer I, Attribut MU 4 \textbf{Nachkauf}: Häufig\newline Du kannst von dir beschworenen unheiligen Wesenheiten (S. 81) zusätzliche Fähigkeiten im Wert von 4 Punkten verleihen.}
}


\newglossaryentry{meisterderSeelenlosen_Vorteil}
{
    name={Meister der Seelenlosen},
    description={\textbf{Kosten}: 40 \textbf{Voraussetzungen}: Vorteil Zauberer I ODER Vorteil Tradition der Borbaradianer I, Attribut MU 6 \textbf{Nachkauf}: Häufig\newline Wann immer ein gebundenes unheiliges Wesen (S. 81) seinen Dienst erfüllt hat, kannst du für die Hälfte der Basisbeschwörungskosten einen zusätzlichen Dienst erbitten.}
}


\newglossaryentry{blutmagie_Vorteil}
{
    name={Blutmagie},
    description={\textbf{Kosten}: 60 \textbf{Voraussetzungen}: Vorteil Zauberer I ODER Vorteil Tradition der Borbaradianer I, Attribut MU 8 \textbf{Nachkauf}: Häufig\newline Mit finsterer und streng verbotener Blutmagie nutzt du fremde Lebenskraft für deine Zauber. Das Opfer wird dabei völlig ausgezehrt und stirbt. Ein zauberkundiges Opfer stellt 8 x WS AsP, ein humanoides Opfer 4 x WS AsP und ein profanes Tier WS AsP zur Verfügung. Blutmagie erleichtert die Beschwörung von Dämonen um +4, aber erschwert die Kontrollprobe von Dämonen und Elementaren um –4. Blutmagie kann übrigens keine gAsP zur Verfügung stellen – gAsP müssen immer vom Zaubernden selbst stammen.}
}


\newglossaryentry{gebieterderFinsternis_Vorteil}
{
    name={Gebieter der Finsternis},
    description={\textbf{Kosten}: 80 \textbf{Voraussetzungen}: Vorteil Zauberer I ODER Vorteil Tradition der Borbaradianer I, Attribut MU 10 \textbf{Nachkauf}: Häufig\newline Gebundene unheilige Wesenheiten kosten nur noch die halben gAsP und eine misslungene Beschwörungsprobe kann einmalig für die Hälfte der Beschwörungskosten wiederholt werden.}
}


\newglossaryentry{unitatio_Vorteil}
{
    name={Unitatio},
    description={\textbf{Kosten}: 20 \textbf{Voraussetzungen}: Vorteil Zauberer I ODER Vorteil Tradition der Borbaradianer I, Attribut IN 4 \textbf{Nachkauf}: Häufig\newline Der Vorteil ermöglicht es dir, mit anderen Kennern dieses Vorteils einen Zauberzirkel zu bilden. Alle Teilnehmer des Zirkels müssen sich berühren und willens sein, ihre Astralenergie zu teilen.\newline Solange der Zirkel besteht, können die Teilnehmer die Regeln zur Zusammenarbeit (S. 8) nutzen, um sich gegenseitig beim Zaubern zu unterstützen. Üblicherweise sind dabei maximal vier Helfer möglich, welche die Probe des Zaubernden um jeweils +2 erleichtern können. Allerdings werden Elfensippen und manchen Hexenzirkeln noch viel mächtigere Unitatio-Zauberei nachgesagt.\newline Zusätzlich werden die Zauberkosten aller Teilnehmer gleichmäßig auf den gesamten Zirkel aufgeteilt. Sollte ein Teilungsrest übrigbleiben oder ein Teilnehmer über zu wenige AsP verfügen, muss der Zaubernde diese Kosten tragen. Nur gAsP muss der Zaubernde stets selbst tragen.}
}


\newglossaryentry{flexibleMagie_Vorteil}
{
    name={Flexible Magie},
    description={\textbf{Kosten}: 40 \textbf{Voraussetzungen}: Vorteil Zauberer I ODER Vorteil Tradition der Borbaradianer I, Attribut IN 6 \textbf{Nachkauf}: Häufig\newline Während du einen Zauber vorbereitest, kannst du dich mit einer Erschwernis von -4 verteidigen.}
}


\newglossaryentry{müheloseMagie_Vorteil}
{
    name={Mühelose Magie},
    description={\textbf{Kosten}: 60 \textbf{Voraussetzungen}: Vorteil Zauberer I ODER Vorteil Tradition der Borbaradianer I, Attribut IN 8 \textbf{Nachkauf}: Häufig\newline Zeigt der gewertete Würfel bei einem Zauber eine 16 oder höher, sinken die Kosten um die Hälfte der Basiskosten und der Zauber erhält zusätzlich 1 Stufe Mächtige Magie.}
}


\newglossaryentry{flinkeMagie_Vorteil}
{
    name={Flinke Magie},
    description={\textbf{Kosten}: 80 \textbf{Voraussetzungen}: Vorteil Zauberer I ODER Vorteil Tradition der Borbaradianer I, Attribut IN 10 \textbf{Nachkauf}: Häufig\newline Einmal pro Initiativphase kannst du eine Zauberprobe in einer freien Aktion statt einer Aktion Konflikt durchführen.}
}


\newglossaryentry{kontrolliertesZaubern_Vorteil}
{
    name={Kontrolliertes Zaubern},
    description={\textbf{Kosten}: 20 \textbf{Voraussetzungen}: Vorteil Zauberer I ODER Vorteil Tradition der Borbaradianer I, Attribut KL 4 \textbf{Nachkauf}: Häufig\newline Du kannst Zauber während ihrer Wirkungsdauer jederzeit beenden.}
}


\newglossaryentry{kraftlinienmagie_Vorteil}
{
    name={Kraftlinienmagie},
    description={\textbf{Kosten}: 40 \textbf{Voraussetzungen}: Vorteil Zauberer I ODER Vorteil Tradition der Borbaradianer I, Attribut KL 6 \textbf{Nachkauf}: Häufig\newline An Kraftlinien oder -knoten sind Zauber mit passenden Fertigkeiten je nach Linienstärke um +2 bis +4 erleichtert und die Regeneration ist um 2-4 AsP erhöht.}
}


\newglossaryentry{effizientesZaubern_Vorteil}
{
    name={Effizientes Zaubern},
    description={\textbf{Kosten}: 60 \textbf{Voraussetzungen}: Vorteil Zauberer I ODER Vorteil Tradition der Borbaradianer I, Attribut KL 8 \textbf{Nachkauf}: Häufig\newline Ermöglicht die spontane Modifikation Kosten sparen (Zauber -4): Die AsP-Kosten des Zaubers sinken um ein Viertel der Basiskosten. Die Kosten können dadurch nicht unter die Hälfte der Basiskosten sinken.}
}


\newglossaryentry{vorbereitendesZaubern_Vorteil}
{
    name={Vorbereitendes Zaubern},
    description={\textbf{Kosten}: 80 \textbf{Voraussetzungen}: Vorteil Zauberer I ODER Vorteil Tradition der Borbaradianer I, Attribut KL 10 \textbf{Nachkauf}: Häufig\newline Zwischen dem Vorbereiten des Zaubers und der Aktion Konflikt dürfen bis zu KL Minuten liegen. Danach gilt der Zauber als fehlgeschlagen. Ein Ziel in Reichweite wird erst in der Aktion Konflikt ausgewählt. Du kannst nur einen Zauber in Vorbereitung halten.}
}


\newglossaryentry{tiergeist(Affe)_Vorteil}
{
    name={Tiergeist (Affe)},
    description={\textbf{Kosten}: 0 \textbf{Voraussetzungen}: Vorteil Tradition der Anach-Nurim I ODER Vorteil Tradition der Durro-dun I ODER Vorteil Tradition der Schamanen I, Kein Vorteil Tiergeist (Bär) ODER Vorteil Tradition der Schamanen I, Kein Vorteil Tiergeist (Elefant) ODER Vorteil Tradition der Schamanen I, Kein Vorteil Tiergeist (Eule) ODER Vorteil Tradition der Schamanen I, Kein Vorteil Tiergeist (Falke) ODER Vorteil Tradition der Schamanen I, Kein Vorteil Tiergeist (Fischotter) ODER Vorteil Tradition der Schamanen I, Kein Vorteil Tiergeist (Fuchs/Mungo) ODER Vorteil Tradition der Schamanen I, Kein Vorteil Tiergeist (Gebirgsbock) ODER Vorteil Tradition der Schamanen I, Kein Vorteil Tiergeist (Löwe) ODER Vorteil Tradition der Schamanen I, Kein Vorteil Tiergeist (Mammut) ODER Vorteil Tradition der Schamanen I, Kein Vorteil Tiergeist (Rabe) ODER Vorteil Tradition der Schamanen I, Kein Vorteil Tiergeist (Schlange) ODER Vorteil Tradition der Schamanen I, Kein Vorteil Tiergeist (Stier) ODER Vorteil Tradition der Schamanen I, Kein Vorteil Tiergeist (Wildkatze/Panther) ODER Vorteil Tradition der Schamanen I, Kein Vorteil Tiergeist (Wildschwein) ODER Vorteil Tradition der Schamanen I, Kein Vorteil Tiergeist (Wolf/Khoramsbestie) ODER Vorteil Tradition der Schamanen I \textbf{Nachkauf}: Häufig\newline Archetyp: Akrobat\newline Wertebonus: Akrobatik, Klettern, Pirschen, Wurfwaffen, GE\newline Verwandlung: Schwanz (Akrobatik und Klettern +4), Kopf (Sinnenschärfe und Wachsamkeit +4)\newline Zauber: Attributo, Axxeleratus, Motoricus, Wipfellauf\newline Sephrasto: Die Rituale der Anach-nûrim und der Durro-dûn, sowie die Schutzgeister und Flüche der Schamanen beziehen sich immer auf eine bestimmte Tierart. Die Verbindung mit einem solchen Tiergeist wird als kostenloser Vorteil abgebildet. Dabei ist zu beachten, dass nur Schamanen mehrere Tiergeister wählen können. Durro-dûn müssen sich für ein Tier entscheiden und können dann die Zauber des Tiergeists unter der Fertigkeit Gaben des Odun erlernen. Anach-nûrim können das Tier wechseln, indem sie einen neuen Blutgeist aufnehmen. Sie beherrschen dann automatisch alle Zauber des Tiergeists: setze einen Haken bei allen kostenlosen Talenten der Fertigkeit Gaben des Blutgeists.}
}


\newglossaryentry{tiergeist(Bär)_Vorteil}
{
    name={Tiergeist (Bär)},
    description={\textbf{Kosten}: 0 \textbf{Voraussetzungen}: Vorteil Tradition der Anach-Nurim I ODER Vorteil Tradition der Durro-dun I ODER Vorteil Tradition der Schamanen I, Kein Vorteil Tiergeist (Affe) ODER Vorteil Tradition der Schamanen I, Kein Vorteil Tiergeist (Elefant) ODER Vorteil Tradition der Schamanen I, Kein Vorteil Tiergeist (Eule) ODER Vorteil Tradition der Schamanen I, Kein Vorteil Tiergeist (Falke) ODER Vorteil Tradition der Schamanen I, Kein Vorteil Tiergeist (Fischotter) ODER Vorteil Tradition der Schamanen I, Kein Vorteil Tiergeist (Fuchs/Mungo) ODER Vorteil Tradition der Schamanen I, Kein Vorteil Tiergeist (Gebirgsbock) ODER Vorteil Tradition der Schamanen I, Kein Vorteil Tiergeist (Löwe) ODER Vorteil Tradition der Schamanen I, Kein Vorteil Tiergeist (Mammut) ODER Vorteil Tradition der Schamanen I, Kein Vorteil Tiergeist (Rabe) ODER Vorteil Tradition der Schamanen I, Kein Vorteil Tiergeist (Schlange) ODER Vorteil Tradition der Schamanen I, Kein Vorteil Tiergeist (Stier) ODER Vorteil Tradition der Schamanen I, Kein Vorteil Tiergeist (Wildkatze/Panther) ODER Vorteil Tradition der Schamanen I, Kein Vorteil Tiergeist (Wildschwein) ODER Vorteil Tradition der Schamanen I, Kein Vorteil Tiergeist (Wolf/Khoramsbestie) ODER Vorteil Tradition der Schamanen I \textbf{Nachkauf}: Häufig\newline Archetyp: Nahkämpfer\newline Wertebonus: TP im Nahkampf, Zähigkeit, KK\newline Verwandlung: Pranken (2W6+2 TP waffenlos, nicht zerbrechlich, RW 1), Pelz (+2 RS, Kälteschutz wie dicke Winterkleidung (S. 35))\newline Zauber: Bärenruhe, Eiseskälte, Ruhe Körper, Sanftmut, Standfest, Zaubernahrung\newline Sephrasto: Die Rituale der Anach-nûrim und der Durro-dûn, sowie die Schutzgeister und Flüche der Schamanen beziehen sich immer auf eine bestimmte Tierart. Die Verbindung mit einem solchen Tiergeist wird als kostenloser Vorteil abgebildet. Dabei ist zu beachten, dass nur Schamanen mehrere Tiergeister wählen können. Durro-dûn müssen sich für ein Tier entscheiden und können dann die Zauber des Tiergeists unter der Fertigkeit Gaben des Odun erlernen. Anach-nûrim können das Tier wechseln, indem sie einen neuen Blutgeist aufnehmen. Sie beherrschen dann automatisch alle Zauber des Tiergeists: setze einen Haken bei allen kostenlosen Talenten der Fertigkeit Gaben des Blutgeists.}
}


\newglossaryentry{tiergeist(Elefant)_Vorteil}
{
    name={Tiergeist (Elefant)},
    description={\textbf{Kosten}: 0 \textbf{Voraussetzungen}: Vorteil Tradition der Anach-Nurim I ODER Vorteil Tradition der Durro-dun I ODER Vorteil Tradition der Schamanen I, Kein Vorteil Tiergeist (Affe) ODER Vorteil Tradition der Schamanen I, Kein Vorteil Tiergeist (Bär) ODER Vorteil Tradition der Schamanen I, Kein Vorteil Tiergeist (Eule) ODER Vorteil Tradition der Schamanen I, Kein Vorteil Tiergeist (Falke) ODER Vorteil Tradition der Schamanen I, Kein Vorteil Tiergeist (Fischotter) ODER Vorteil Tradition der Schamanen I, Kein Vorteil Tiergeist (Fuchs/Mungo) ODER Vorteil Tradition der Schamanen I, Kein Vorteil Tiergeist (Gebirgsbock) ODER Vorteil Tradition der Schamanen I, Kein Vorteil Tiergeist (Löwe) ODER Vorteil Tradition der Schamanen I, Kein Vorteil Tiergeist (Mammut) ODER Vorteil Tradition der Schamanen I, Kein Vorteil Tiergeist (Rabe) ODER Vorteil Tradition der Schamanen I, Kein Vorteil Tiergeist (Schlange) ODER Vorteil Tradition der Schamanen I, Kein Vorteil Tiergeist (Stier) ODER Vorteil Tradition der Schamanen I, Kein Vorteil Tiergeist (Wildkatze/Panther) ODER Vorteil Tradition der Schamanen I, Kein Vorteil Tiergeist (Wildschwein) ODER Vorteil Tradition der Schamanen I, Kein Vorteil Tiergeist (Wolf/Khoramsbestie) ODER Vorteil Tradition der Schamanen I \textbf{Nachkauf}: Häufig\newline Archetyp: Weiser\newline Wertebonus: Heilkunde, Geographie, Mythenkunde, KL\newline Verwandlung: Haut (+1 RS und Willenskraft +4), Stoßzähne (2W6+2 TP waffenlos, nicht zerbrechlich, RW 1)\newline Zauber: Memorans, Psychostabilis, Seelentier erkennen, Xenographus\newline Sephrasto: Die Rituale der Anach-nûrim und der Durro-dûn, sowie die Schutzgeister und Flüche der Schamanen beziehen sich immer auf eine bestimmte Tierart. Die Verbindung mit einem solchen Tiergeist wird als kostenloser Vorteil abgebildet. Dabei ist zu beachten, dass nur Schamanen mehrere Tiergeister wählen können. Durro-dûn müssen sich für ein Tier entscheiden und können dann die Zauber des Tiergeists unter der Fertigkeit Gaben des Odun erlernen. Anach-nûrim können das Tier wechseln, indem sie einen neuen Blutgeist aufnehmen. Sie beherrschen dann automatisch alle Zauber des Tiergeists: setze einen Haken bei allen kostenlosen Talenten der Fertigkeit Gaben des Blutgeists.}
}


\newglossaryentry{tiergeist(Eule)_Vorteil}
{
    name={Tiergeist (Eule)},
    description={\textbf{Kosten}: 0 \textbf{Voraussetzungen}: Vorteil Tradition der Anach-Nurim I ODER Vorteil Tradition der Durro-dun I ODER Vorteil Tradition der Schamanen I, Kein Vorteil Tiergeist (Affe) ODER Vorteil Tradition der Schamanen I, Kein Vorteil Tiergeist (Bär) ODER Vorteil Tradition der Schamanen I, Kein Vorteil Tiergeist (Elefant) ODER Vorteil Tradition der Schamanen I, Kein Vorteil Tiergeist (Falke) ODER Vorteil Tradition der Schamanen I, Kein Vorteil Tiergeist (Fischotter) ODER Vorteil Tradition der Schamanen I, Kein Vorteil Tiergeist (Fuchs/Mungo) ODER Vorteil Tradition der Schamanen I, Kein Vorteil Tiergeist (Gebirgsbock) ODER Vorteil Tradition der Schamanen I, Kein Vorteil Tiergeist (Löwe) ODER Vorteil Tradition der Schamanen I, Kein Vorteil Tiergeist (Mammut) ODER Vorteil Tradition der Schamanen I, Kein Vorteil Tiergeist (Rabe) ODER Vorteil Tradition der Schamanen I, Kein Vorteil Tiergeist (Schlange) ODER Vorteil Tradition der Schamanen I, Kein Vorteil Tiergeist (Stier) ODER Vorteil Tradition der Schamanen I, Kein Vorteil Tiergeist (Wildkatze/Panther) ODER Vorteil Tradition der Schamanen I, Kein Vorteil Tiergeist (Wildschwein) ODER Vorteil Tradition der Schamanen I, Kein Vorteil Tiergeist (Wolf/Khoramsbestie) ODER Vorteil Tradition der Schamanen I \textbf{Nachkauf}: Häufig\newline Archetyp: Kundschafter\newline Wertebonus: Pirschen, Sinnenschärfe, Überleben, IN\newline Verwandlung: Flügel (bis zu 8 Schritt weite und 4 Schritt hohe Sprünge, Gleitflüge von bis zu 4xHöhe Schritt), Kopf (Sinnenschärfe und Wachsamkeit +4, wenn Sicht gefragt ist, Angepasst (Dunkelheit) steigt um 1 Stufe)\newline Zauber: Blick aufs Wesen, Exposami, Hexenkrallen, Senisbar, Silentium\newline Sephrasto: Die Rituale der Anach-nûrim und der Durro-dûn, sowie die Schutzgeister und Flüche der Schamanen beziehen sich immer auf eine bestimmte Tierart. Die Verbindung mit einem solchen Tiergeist wird als kostenloser Vorteil abgebildet. Dabei ist zu beachten, dass nur Schamanen mehrere Tiergeister wählen können. Durro-dûn müssen sich für ein Tier entscheiden und können dann die Zauber des Tiergeists unter der Fertigkeit Gaben des Odun erlernen. Anach-nûrim können das Tier wechseln, indem sie einen neuen Blutgeist aufnehmen. Sie beherrschen dann automatisch alle Zauber des Tiergeists: setze einen Haken bei allen kostenlosen Talenten der Fertigkeit Gaben des Blutgeists.}
}


\newglossaryentry{tiergeist(Falke)_Vorteil}
{
    name={Tiergeist (Falke)},
    description={\textbf{Kosten}: 0 \textbf{Voraussetzungen}: Vorteil Tradition der Anach-Nurim I ODER Vorteil Tradition der Durro-dun I ODER Vorteil Tradition der Schamanen I, Kein Vorteil Tiergeist (Affe) ODER Vorteil Tradition der Schamanen I, Kein Vorteil Tiergeist (Bär) ODER Vorteil Tradition der Schamanen I, Kein Vorteil Tiergeist (Elefant) ODER Vorteil Tradition der Schamanen I, Kein Vorteil Tiergeist (Eule) ODER Vorteil Tradition der Schamanen I, Kein Vorteil Tiergeist (Fischotter) ODER Vorteil Tradition der Schamanen I, Kein Vorteil Tiergeist (Fuchs/Mungo) ODER Vorteil Tradition der Schamanen I, Kein Vorteil Tiergeist (Gebirgsbock) ODER Vorteil Tradition der Schamanen I, Kein Vorteil Tiergeist (Löwe) ODER Vorteil Tradition der Schamanen I, Kein Vorteil Tiergeist (Mammut) ODER Vorteil Tradition der Schamanen I, Kein Vorteil Tiergeist (Rabe) ODER Vorteil Tradition der Schamanen I, Kein Vorteil Tiergeist (Schlange) ODER Vorteil Tradition der Schamanen I, Kein Vorteil Tiergeist (Stier) ODER Vorteil Tradition der Schamanen I, Kein Vorteil Tiergeist (Wildkatze/Panther) ODER Vorteil Tradition der Schamanen I, Kein Vorteil Tiergeist (Wildschwein) ODER Vorteil Tradition der Schamanen I, Kein Vorteil Tiergeist (Wolf/Khoramsbestie) ODER Vorteil Tradition der Schamanen I \textbf{Nachkauf}: Häufig\newline Archetyp: Fernkämpfer\newline Wertebonus: TP im Fernkampf, Sinnenschärfe, Wachsamkeit, FF\newline Verwandlung: Flügel (bis zu 8 Schritt weite und 4 Schritt hohe Sprünge, Gleitflüge von bis zu 4xHöhe Schritt), Kopf (Sinnenschärfe +8 und Wachsamkeit +4, wenn Sicht gefragt ist)\newline Zauber: Adlerauge, Axxeleratus, Falkenauge, Pfeil der Luft, Aeropulvis\newline Sephrasto: Die Rituale der Anach-nûrim und der Durro-dûn, sowie die Schutzgeister und Flüche der Schamanen beziehen sich immer auf eine bestimmte Tierart. Die Verbindung mit einem solchen Tiergeist wird als kostenloser Vorteil abgebildet. Dabei ist zu beachten, dass nur Schamanen mehrere Tiergeister wählen können. Durro-dûn müssen sich für ein Tier entscheiden und können dann die Zauber des Tiergeists unter der Fertigkeit Gaben des Odun erlernen. Anach-nûrim können das Tier wechseln, indem sie einen neuen Blutgeist aufnehmen. Sie beherrschen dann automatisch alle Zauber des Tiergeists: setze einen Haken bei allen kostenlosen Talenten der Fertigkeit Gaben des Blutgeists.}
}


\newglossaryentry{tiergeist(Fischotter)_Vorteil}
{
    name={Tiergeist (Fischotter)},
    description={\textbf{Kosten}: 0 \textbf{Voraussetzungen}: Vorteil Tradition der Anach-Nurim I ODER Vorteil Tradition der Durro-dun I ODER Vorteil Tradition der Schamanen I, Kein Vorteil Tiergeist (Affe) ODER Vorteil Tradition der Schamanen I, Kein Vorteil Tiergeist (Bär) ODER Vorteil Tradition der Schamanen I, Kein Vorteil Tiergeist (Elefant) ODER Vorteil Tradition der Schamanen I, Kein Vorteil Tiergeist (Eule) ODER Vorteil Tradition der Schamanen I, Kein Vorteil Tiergeist (Falke) ODER Vorteil Tradition der Schamanen I, Kein Vorteil Tiergeist (Fuchs/Mungo) ODER Vorteil Tradition der Schamanen I, Kein Vorteil Tiergeist (Gebirgsbock) ODER Vorteil Tradition der Schamanen I, Kein Vorteil Tiergeist (Löwe) ODER Vorteil Tradition der Schamanen I, Kein Vorteil Tiergeist (Mammut) ODER Vorteil Tradition der Schamanen I, Kein Vorteil Tiergeist (Rabe) ODER Vorteil Tradition der Schamanen I, Kein Vorteil Tiergeist (Schlange) ODER Vorteil Tradition der Schamanen I, Kein Vorteil Tiergeist (Stier) ODER Vorteil Tradition der Schamanen I, Kein Vorteil Tiergeist (Wildkatze/Panther) ODER Vorteil Tradition der Schamanen I, Kein Vorteil Tiergeist (Wildschwein) ODER Vorteil Tradition der Schamanen I, Kein Vorteil Tiergeist (Wolf/Khoramsbestie) ODER Vorteil Tradition der Schamanen I \textbf{Nachkauf}: Häufig\newline Archetyp: Akrobat\newline Wertebonus: Akrobatik, Klettern, Pirschen, Schwimmen, GE\newline Verwandlung: Pfoten (Schwimmen +4, Angepasst (Wasser) steigt um 1 Stufe), Pelz (+2 RS, Kälteschutz wie dicke Winterkleidung (S. 35))\newline Zauber: Eins mit der Natur, Katzenaugen, Foramen, Wasseratem, Wellenlauf, Hilfreiche Tatze\newline Sephrasto: Die Rituale der Anach-nûrim und der Durro-dûn, sowie die Schutzgeister und Flüche der Schamanen beziehen sich immer auf eine bestimmte Tierart. Die Verbindung mit einem solchen Tiergeist wird als kostenloser Vorteil abgebildet. Dabei ist zu beachten, dass nur Schamanen mehrere Tiergeister wählen können. Durro-dûn müssen sich für ein Tier entscheiden und können dann die Zauber des Tiergeists unter der Fertigkeit Gaben des Odun erlernen. Anach-nûrim können das Tier wechseln, indem sie einen neuen Blutgeist aufnehmen. Sie beherrschen dann automatisch alle Zauber des Tiergeists: setze einen Haken bei allen kostenlosen Talenten der Fertigkeit Gaben des Blutgeists.}
}


\newglossaryentry{tiergeist(Fuchs/Mungo)_Vorteil}
{
    name={Tiergeist (Fuchs/Mungo)},
    description={\textbf{Kosten}: 0 \textbf{Voraussetzungen}: Vorteil Tradition der Anach-Nurim I ODER Vorteil Tradition der Durro-dun I ODER Vorteil Tradition der Schamanen I, Kein Vorteil Tiergeist (Affe) ODER Vorteil Tradition der Schamanen I, Kein Vorteil Tiergeist (Bär) ODER Vorteil Tradition der Schamanen I, Kein Vorteil Tiergeist (Elefant) ODER Vorteil Tradition der Schamanen I, Kein Vorteil Tiergeist (Eule) ODER Vorteil Tradition der Schamanen I, Kein Vorteil Tiergeist (Falke) ODER Vorteil Tradition der Schamanen I, Kein Vorteil Tiergeist (Fischotter) ODER Vorteil Tradition der Schamanen I, Kein Vorteil Tiergeist (Gebirgsbock) ODER Vorteil Tradition der Schamanen I, Kein Vorteil Tiergeist (Löwe) ODER Vorteil Tradition der Schamanen I, Kein Vorteil Tiergeist (Mammut) ODER Vorteil Tradition der Schamanen I, Kein Vorteil Tiergeist (Rabe) ODER Vorteil Tradition der Schamanen I, Kein Vorteil Tiergeist (Schlange) ODER Vorteil Tradition der Schamanen I, Kein Vorteil Tiergeist (Stier) ODER Vorteil Tradition der Schamanen I, Kein Vorteil Tiergeist (Wildkatze/Panther) ODER Vorteil Tradition der Schamanen I, Kein Vorteil Tiergeist (Wildschwein) ODER Vorteil Tradition der Schamanen I, Kein Vorteil Tiergeist (Wolf/Khoramsbestie) ODER Vorteil Tradition der Schamanen I \textbf{Nachkauf}: Häufig\newline Archetyp: Gesellschafter\newline Wertebonus: Gebräuche, Überreden, Menschenkenntnis, CH\newline Verwandlung: Fell (+1 RS, Pirschen +4), Kopf (Sinnenschärfe und Wachsamkeit +4)\newline Zauber: Attributo, Harmlose Gestalt, Sensibar, Seidenzunge\newline Sephrasto: Die Rituale der Anach-nûrim und der Durro-dûn, sowie die Schutzgeister und Flüche der Schamanen beziehen sich immer auf eine bestimmte Tierart. Die Verbindung mit einem solchen Tiergeist wird als kostenloser Vorteil abgebildet. Dabei ist zu beachten, dass nur Schamanen mehrere Tiergeister wählen können. Durro-dûn müssen sich für ein Tier entscheiden und können dann die Zauber des Tiergeists unter der Fertigkeit Gaben des Odun erlernen. Anach-nûrim können das Tier wechseln, indem sie einen neuen Blutgeist aufnehmen. Sie beherrschen dann automatisch alle Zauber des Tiergeists: setze einen Haken bei allen kostenlosen Talenten der Fertigkeit Gaben des Blutgeists.}
}


\newglossaryentry{tiergeist(Gebirgsbock)_Vorteil}
{
    name={Tiergeist (Gebirgsbock)},
    description={\textbf{Kosten}: 0 \textbf{Voraussetzungen}: Vorteil Tradition der Anach-Nurim I ODER Vorteil Tradition der Durro-dun I ODER Vorteil Tradition der Schamanen I, Kein Vorteil Tiergeist (Affe) ODER Vorteil Tradition der Schamanen I, Kein Vorteil Tiergeist (Bär) ODER Vorteil Tradition der Schamanen I, Kein Vorteil Tiergeist (Elefant) ODER Vorteil Tradition der Schamanen I, Kein Vorteil Tiergeist (Eule) ODER Vorteil Tradition der Schamanen I, Kein Vorteil Tiergeist (Falke) ODER Vorteil Tradition der Schamanen I, Kein Vorteil Tiergeist (Fischotter) ODER Vorteil Tradition der Schamanen I, Kein Vorteil Tiergeist (Fuchs/Mungo) ODER Vorteil Tradition der Schamanen I, Kein Vorteil Tiergeist (Löwe) ODER Vorteil Tradition der Schamanen I, Kein Vorteil Tiergeist (Mammut) ODER Vorteil Tradition der Schamanen I, Kein Vorteil Tiergeist (Rabe) ODER Vorteil Tradition der Schamanen I, Kein Vorteil Tiergeist (Schlange) ODER Vorteil Tradition der Schamanen I, Kein Vorteil Tiergeist (Stier) ODER Vorteil Tradition der Schamanen I, Kein Vorteil Tiergeist (Wildkatze/Panther) ODER Vorteil Tradition der Schamanen I, Kein Vorteil Tiergeist (Wildschwein) ODER Vorteil Tradition der Schamanen I, Kein Vorteil Tiergeist (Wolf/Khoramsbestie) ODER Vorteil Tradition der Schamanen I \textbf{Nachkauf}: Häufig\newline Archetyp: Sammler\newline Wertebonus: Pflanzenkunde, Tierkunde, Überleben, Wachsamkeit, KO\newline Verwandlung: Hörner (2W6+2 TP waffenlos, nicht zerbrechlich, RW 1), Pelz (+2 RS, Kälteschutz wie dicke Winterkleidung (S. 35))\newline Zauber: Axxeleratus, Eins mit der Natur, Firnlauf, Spinnenlauf, Standfest\newline Sephrasto: Die Rituale der Anach-nûrim und der Durro-dûn, sowie die Schutzgeister und Flüche der Schamanen beziehen sich immer auf eine bestimmte Tierart. Die Verbindung mit einem solchen Tiergeist wird als kostenloser Vorteil abgebildet. Dabei ist zu beachten, dass nur Schamanen mehrere Tiergeister wählen können. Durro-dûn müssen sich für ein Tier entscheiden und können dann die Zauber des Tiergeists unter der Fertigkeit Gaben des Odun erlernen. Anach-nûrim können das Tier wechseln, indem sie einen neuen Blutgeist aufnehmen. Sie beherrschen dann automatisch alle Zauber des Tiergeists: setze einen Haken bei allen kostenlosen Talenten der Fertigkeit Gaben des Blutgeists.}
}


\newglossaryentry{tiergeist(Löwe)_Vorteil}
{
    name={Tiergeist (Löwe)},
    description={\textbf{Kosten}: 0 \textbf{Voraussetzungen}: Vorteil Tradition der Anach-Nurim I ODER Vorteil Tradition der Durro-dun I ODER Vorteil Tradition der Schamanen I, Kein Vorteil Tiergeist (Affe) ODER Vorteil Tradition der Schamanen I, Kein Vorteil Tiergeist (Bär) ODER Vorteil Tradition der Schamanen I, Kein Vorteil Tiergeist (Elefant) ODER Vorteil Tradition der Schamanen I, Kein Vorteil Tiergeist (Eule) ODER Vorteil Tradition der Schamanen I, Kein Vorteil Tiergeist (Falke) ODER Vorteil Tradition der Schamanen I, Kein Vorteil Tiergeist (Fischotter) ODER Vorteil Tradition der Schamanen I, Kein Vorteil Tiergeist (Fuchs/Mungo) ODER Vorteil Tradition der Schamanen I, Kein Vorteil Tiergeist (Gebirgsbock) ODER Vorteil Tradition der Schamanen I, Kein Vorteil Tiergeist (Mammut) ODER Vorteil Tradition der Schamanen I, Kein Vorteil Tiergeist (Rabe) ODER Vorteil Tradition der Schamanen I, Kein Vorteil Tiergeist (Schlange) ODER Vorteil Tradition der Schamanen I, Kein Vorteil Tiergeist (Stier) ODER Vorteil Tradition der Schamanen I, Kein Vorteil Tiergeist (Wildkatze/Panther) ODER Vorteil Tradition der Schamanen I, Kein Vorteil Tiergeist (Wildschwein) ODER Vorteil Tradition der Schamanen I, Kein Vorteil Tiergeist (Wolf/Khoramsbestie) ODER Vorteil Tradition der Schamanen I \textbf{Nachkauf}: Häufig\newline Archetyp: Anführer\newline Wertebonus: Autorität, Willenskraft, MU\newline Verwandlung: Maul (2W6+2 TP waffenlos, Einschüchtern +4), Mähne (Anführen +4, Furcht-Effekte sinken um 1 Stufe)\newline Zauber: Ängste lindern, Armatrutz, Katzenaugen, Kusch!, Standfest\newline Sephrasto: Die Rituale der Anach-nûrim und der Durro-dûn, sowie die Schutzgeister und Flüche der Schamanen beziehen sich immer auf eine bestimmte Tierart. Die Verbindung mit einem solchen Tiergeist wird als kostenloser Vorteil abgebildet. Dabei ist zu beachten, dass nur Schamanen mehrere Tiergeister wählen können. Durro-dûn müssen sich für ein Tier entscheiden und können dann die Zauber des Tiergeists unter der Fertigkeit Gaben des Odun erlernen. Anach-nûrim können das Tier wechseln, indem sie einen neuen Blutgeist aufnehmen. Sie beherrschen dann automatisch alle Zauber des Tiergeists: setze einen Haken bei allen kostenlosen Talenten der Fertigkeit Gaben des Blutgeists.}
}


\newglossaryentry{tiergeist(Mammut)_Vorteil}
{
    name={Tiergeist (Mammut)},
    description={\textbf{Kosten}: 0 \textbf{Voraussetzungen}: Vorteil Tradition der Anach-Nurim I ODER Vorteil Tradition der Durro-dun I ODER Vorteil Tradition der Schamanen I, Kein Vorteil Tiergeist (Affe) ODER Vorteil Tradition der Schamanen I, Kein Vorteil Tiergeist (Bär) ODER Vorteil Tradition der Schamanen I, Kein Vorteil Tiergeist (Elefant) ODER Vorteil Tradition der Schamanen I, Kein Vorteil Tiergeist (Eule) ODER Vorteil Tradition der Schamanen I, Kein Vorteil Tiergeist (Falke) ODER Vorteil Tradition der Schamanen I, Kein Vorteil Tiergeist (Fischotter) ODER Vorteil Tradition der Schamanen I, Kein Vorteil Tiergeist (Fuchs/Mungo) ODER Vorteil Tradition der Schamanen I, Kein Vorteil Tiergeist (Gebirgsbock) ODER Vorteil Tradition der Schamanen I, Kein Vorteil Tiergeist (Löwe) ODER Vorteil Tradition der Schamanen I, Kein Vorteil Tiergeist (Rabe) ODER Vorteil Tradition der Schamanen I, Kein Vorteil Tiergeist (Schlange) ODER Vorteil Tradition der Schamanen I, Kein Vorteil Tiergeist (Stier) ODER Vorteil Tradition der Schamanen I, Kein Vorteil Tiergeist (Wildkatze/Panther) ODER Vorteil Tradition der Schamanen I, Kein Vorteil Tiergeist (Wildschwein) ODER Vorteil Tradition der Schamanen I, Kein Vorteil Tiergeist (Wolf/Khoramsbestie) ODER Vorteil Tradition der Schamanen I \textbf{Nachkauf}: Häufig\newline Archetyp: Anführer\newline Wertebonus: Autorität, Willenskraft, MU\newline Verwandlung: Pelz (+2 RS, Kälteschutz wie dicke Winterkleidung (S. 35)), Stoßzähne (2W6+2 TP waffenlos, RW 1)\newline Zauber: Ängste lindern, Armatrutz, Kusch!, Psychostabilis, Zaubernahrung\newline Sephrasto: Die Rituale der Anach-nûrim und der Durro-dûn, sowie die Schutzgeister und Flüche der Schamanen beziehen sich immer auf eine bestimmte Tierart. Die Verbindung mit einem solchen Tiergeist wird als kostenloser Vorteil abgebildet. Dabei ist zu beachten, dass nur Schamanen mehrere Tiergeister wählen können. Durro-dûn müssen sich für ein Tier entscheiden und können dann die Zauber des Tiergeists unter der Fertigkeit Gaben des Odun erlernen. Anach-nûrim können das Tier wechseln, indem sie einen neuen Blutgeist aufnehmen. Sie beherrschen dann automatisch alle Zauber des Tiergeists: setze einen Haken bei allen kostenlosen Talenten der Fertigkeit Gaben des Blutgeists.}
}


\newglossaryentry{tiergeist(Rabe)_Vorteil}
{
    name={Tiergeist (Rabe)},
    description={\textbf{Kosten}: 0 \textbf{Voraussetzungen}: Vorteil Tradition der Anach-Nurim I ODER Vorteil Tradition der Durro-dun I ODER Vorteil Tradition der Schamanen I, Kein Vorteil Tiergeist (Affe) ODER Vorteil Tradition der Schamanen I, Kein Vorteil Tiergeist (Bär) ODER Vorteil Tradition der Schamanen I, Kein Vorteil Tiergeist (Elefant) ODER Vorteil Tradition der Schamanen I, Kein Vorteil Tiergeist (Eule) ODER Vorteil Tradition der Schamanen I, Kein Vorteil Tiergeist (Falke) ODER Vorteil Tradition der Schamanen I, Kein Vorteil Tiergeist (Fischotter) ODER Vorteil Tradition der Schamanen I, Kein Vorteil Tiergeist (Fuchs/Mungo) ODER Vorteil Tradition der Schamanen I, Kein Vorteil Tiergeist (Gebirgsbock) ODER Vorteil Tradition der Schamanen I, Kein Vorteil Tiergeist (Löwe) ODER Vorteil Tradition der Schamanen I, Kein Vorteil Tiergeist (Mammut) ODER Vorteil Tradition der Schamanen I, Kein Vorteil Tiergeist (Schlange) ODER Vorteil Tradition der Schamanen I, Kein Vorteil Tiergeist (Stier) ODER Vorteil Tradition der Schamanen I, Kein Vorteil Tiergeist (Wildkatze/Panther) ODER Vorteil Tradition der Schamanen I, Kein Vorteil Tiergeist (Wildschwein) ODER Vorteil Tradition der Schamanen I, Kein Vorteil Tiergeist (Wolf/Khoramsbestie) ODER Vorteil Tradition der Schamanen I \textbf{Nachkauf}: Häufig\newline Archetyp: Weiser\newline Wertebonus: Heilkunde, Geographie, Mythenkunde, KL\newline Verwandlung: Flügel (bis zu 8 Schritt weite und 4 Schritt hohe Sprünge, Gleitflüge von bis zu 4xHöhe Schritt), Kopf (Sinnenschärfe und Wachsamkeit +4)\newline Zauber: Krähenruf, Sensibar, Nekropathia, Memorans\newline Sephrasto: Die Rituale der Anach-nûrim und der Durro-dûn, sowie die Schutzgeister und Flüche der Schamanen beziehen sich immer auf eine bestimmte Tierart. Die Verbindung mit einem solchen Tiergeist wird als kostenloser Vorteil abgebildet. Dabei ist zu beachten, dass nur Schamanen mehrere Tiergeister wählen können. Durro-dûn müssen sich für ein Tier entscheiden und können dann die Zauber des Tiergeists unter der Fertigkeit Gaben des Odun erlernen. Anach-nûrim können das Tier wechseln, indem sie einen neuen Blutgeist aufnehmen. Sie beherrschen dann automatisch alle Zauber des Tiergeists: setze einen Haken bei allen kostenlosen Talenten der Fertigkeit Gaben des Blutgeists.}
}


\newglossaryentry{tiergeist(Schlange)_Vorteil}
{
    name={Tiergeist (Schlange)},
    description={\textbf{Kosten}: 0 \textbf{Voraussetzungen}: Vorteil Tradition der Anach-Nurim I ODER Vorteil Tradition der Durro-dun I ODER Vorteil Tradition der Schamanen I, Kein Vorteil Tiergeist (Affe) ODER Vorteil Tradition der Schamanen I, Kein Vorteil Tiergeist (Bär) ODER Vorteil Tradition der Schamanen I, Kein Vorteil Tiergeist (Elefant) ODER Vorteil Tradition der Schamanen I, Kein Vorteil Tiergeist (Eule) ODER Vorteil Tradition der Schamanen I, Kein Vorteil Tiergeist (Falke) ODER Vorteil Tradition der Schamanen I, Kein Vorteil Tiergeist (Fischotter) ODER Vorteil Tradition der Schamanen I, Kein Vorteil Tiergeist (Fuchs/Mungo) ODER Vorteil Tradition der Schamanen I, Kein Vorteil Tiergeist (Gebirgsbock) ODER Vorteil Tradition der Schamanen I, Kein Vorteil Tiergeist (Löwe) ODER Vorteil Tradition der Schamanen I, Kein Vorteil Tiergeist (Mammut) ODER Vorteil Tradition der Schamanen I, Kein Vorteil Tiergeist (Rabe) ODER Vorteil Tradition der Schamanen I, Kein Vorteil Tiergeist (Stier) ODER Vorteil Tradition der Schamanen I, Kein Vorteil Tiergeist (Wildkatze/Panther) ODER Vorteil Tradition der Schamanen I, Kein Vorteil Tiergeist (Wildschwein) ODER Vorteil Tradition der Schamanen I, Kein Vorteil Tiergeist (Wolf/Khoramsbestie) ODER Vorteil Tradition der Schamanen I \textbf{Nachkauf}: Häufig\newline Archetyp: Weiser\newline Wertebonus: Alchemie, Magietheorie, Mythenkunde, KL\newline Verwandlung: Giftzähne (2W6 TP waffenlos, Vergiftung mit Waffengift (Stufe 20, ohne Verzögerung, Intervall und Dauer 2W6 INI-Phasen 2W6 SP)), Schuppen (+1 RS, Pirschen +4)\newline Zauber: Atemnot, Psychostabilis, Seperentialis, Vipernblick, Warmes Blut\newline Sephrasto: Die Rituale der Anach-nûrim und der Durro-dûn, sowie die Schutzgeister und Flüche der Schamanen beziehen sich immer auf eine bestimmte Tierart. Die Verbindung mit einem solchen Tiergeist wird als kostenloser Vorteil abgebildet. Dabei ist zu beachten, dass nur Schamanen mehrere Tiergeister wählen können. Durro-dûn müssen sich für ein Tier entscheiden und können dann die Zauber des Tiergeists unter der Fertigkeit Gaben des Odun erlernen. Anach-nûrim können das Tier wechseln, indem sie einen neuen Blutgeist aufnehmen. Sie beherrschen dann automatisch alle Zauber des Tiergeists: setze einen Haken bei allen kostenlosen Talenten der Fertigkeit Gaben des Blutgeists.}
}


\newglossaryentry{tiergeist(Stier)_Vorteil}
{
    name={Tiergeist (Stier)},
    description={\textbf{Kosten}: 0 \textbf{Voraussetzungen}: Vorteil Tradition der Anach-Nurim I ODER Vorteil Tradition der Durro-dun I ODER Vorteil Tradition der Schamanen I, Kein Vorteil Tiergeist (Affe) ODER Vorteil Tradition der Schamanen I, Kein Vorteil Tiergeist (Bär) ODER Vorteil Tradition der Schamanen I, Kein Vorteil Tiergeist (Elefant) ODER Vorteil Tradition der Schamanen I, Kein Vorteil Tiergeist (Eule) ODER Vorteil Tradition der Schamanen I, Kein Vorteil Tiergeist (Falke) ODER Vorteil Tradition der Schamanen I, Kein Vorteil Tiergeist (Fischotter) ODER Vorteil Tradition der Schamanen I, Kein Vorteil Tiergeist (Fuchs/Mungo) ODER Vorteil Tradition der Schamanen I, Kein Vorteil Tiergeist (Gebirgsbock) ODER Vorteil Tradition der Schamanen I, Kein Vorteil Tiergeist (Löwe) ODER Vorteil Tradition der Schamanen I, Kein Vorteil Tiergeist (Mammut) ODER Vorteil Tradition der Schamanen I, Kein Vorteil Tiergeist (Rabe) ODER Vorteil Tradition der Schamanen I, Kein Vorteil Tiergeist (Schlange) ODER Vorteil Tradition der Schamanen I, Kein Vorteil Tiergeist (Wildkatze/Panther) ODER Vorteil Tradition der Schamanen I, Kein Vorteil Tiergeist (Wildschwein) ODER Vorteil Tradition der Schamanen I, Kein Vorteil Tiergeist (Wolf/Khoramsbestie) ODER Vorteil Tradition der Schamanen I \textbf{Nachkauf}: Häufig\newline Archetyp: Nahkämpfer\newline Wertebonus: TP im Nahkampf, Zähigkeit, KK\newline Verwandlung: Hörner (2W6+2 TP waffenlos, nicht zerbrechlich, RW 1), Haut (+1 RS, Zähigkeit +4)\newline Zauber: Attributo, Horriphobus, Sensattaco, Standfest\newline Sephrasto: Die Rituale der Anach-nûrim und der Durro-dûn, sowie die Schutzgeister und Flüche der Schamanen beziehen sich immer auf eine bestimmte Tierart. Die Verbindung mit einem solchen Tiergeist wird als kostenloser Vorteil abgebildet. Dabei ist zu beachten, dass nur Schamanen mehrere Tiergeister wählen können. Durro-dûn müssen sich für ein Tier entscheiden und können dann die Zauber des Tiergeists unter der Fertigkeit Gaben des Odun erlernen. Anach-nûrim können das Tier wechseln, indem sie einen neuen Blutgeist aufnehmen. Sie beherrschen dann automatisch alle Zauber des Tiergeists: setze einen Haken bei allen kostenlosen Talenten der Fertigkeit Gaben des Blutgeists.}
}


\newglossaryentry{tiergeist(Wildkatze/Panther)_Vorteil}
{
    name={Tiergeist (Wildkatze/Panther)},
    description={\textbf{Kosten}: 0 \textbf{Voraussetzungen}: Vorteil Tradition der Anach-Nurim I ODER Vorteil Tradition der Durro-dun I ODER Vorteil Tradition der Schamanen I, Kein Vorteil Tiergeist (Affe) ODER Vorteil Tradition der Schamanen I, Kein Vorteil Tiergeist (Bär) ODER Vorteil Tradition der Schamanen I, Kein Vorteil Tiergeist (Elefant) ODER Vorteil Tradition der Schamanen I, Kein Vorteil Tiergeist (Eule) ODER Vorteil Tradition der Schamanen I, Kein Vorteil Tiergeist (Falke) ODER Vorteil Tradition der Schamanen I, Kein Vorteil Tiergeist (Fischotter) ODER Vorteil Tradition der Schamanen I, Kein Vorteil Tiergeist (Fuchs/Mungo) ODER Vorteil Tradition der Schamanen I, Kein Vorteil Tiergeist (Gebirgsbock) ODER Vorteil Tradition der Schamanen I, Kein Vorteil Tiergeist (Löwe) ODER Vorteil Tradition der Schamanen I, Kein Vorteil Tiergeist (Mammut) ODER Vorteil Tradition der Schamanen I, Kein Vorteil Tiergeist (Rabe) ODER Vorteil Tradition der Schamanen I, Kein Vorteil Tiergeist (Schlange) ODER Vorteil Tradition der Schamanen I, Kein Vorteil Tiergeist (Stier) ODER Vorteil Tradition der Schamanen I, Kein Vorteil Tiergeist (Wildschwein) ODER Vorteil Tradition der Schamanen I, Kein Vorteil Tiergeist (Wolf/Khoramsbestie) ODER Vorteil Tradition der Schamanen I \textbf{Nachkauf}: Häufig\newline Archetyp: Akrobat\newline Wertebonus: Akrobatik, Klettern, Laufen, Pirschen, GE\newline Verwandlung: Krallen (2W6+2 TP waffenlos, Klettern +4), Fell (+1 RS, Pirschen +4)\newline Zauber: Eins mit der Natur, Katzenaugen, Krötensprung, Spurlos, Standfest, Wipfellauf\newline Sephrasto: Die Rituale der Anach-nûrim und der Durro-dûn, sowie die Schutzgeister und Flüche der Schamanen beziehen sich immer auf eine bestimmte Tierart. Die Verbindung mit einem solchen Tiergeist wird als kostenloser Vorteil abgebildet. Dabei ist zu beachten, dass nur Schamanen mehrere Tiergeister wählen können. Durro-dûn müssen sich für ein Tier entscheiden und können dann die Zauber des Tiergeists unter der Fertigkeit Gaben des Odun erlernen. Anach-nûrim können das Tier wechseln, indem sie einen neuen Blutgeist aufnehmen. Sie beherrschen dann automatisch alle Zauber des Tiergeists: setze einen Haken bei allen kostenlosen Talenten der Fertigkeit Gaben des Blutgeists.}
}


\newglossaryentry{tiergeist(Wildschwein)_Vorteil}
{
    name={Tiergeist (Wildschwein)},
    description={\textbf{Kosten}: 0 \textbf{Voraussetzungen}: Vorteil Tradition der Anach-Nurim I ODER Vorteil Tradition der Durro-dun I ODER Vorteil Tradition der Schamanen I, Kein Vorteil Tiergeist (Affe) ODER Vorteil Tradition der Schamanen I, Kein Vorteil Tiergeist (Bär) ODER Vorteil Tradition der Schamanen I, Kein Vorteil Tiergeist (Elefant) ODER Vorteil Tradition der Schamanen I, Kein Vorteil Tiergeist (Eule) ODER Vorteil Tradition der Schamanen I, Kein Vorteil Tiergeist (Falke) ODER Vorteil Tradition der Schamanen I, Kein Vorteil Tiergeist (Fischotter) ODER Vorteil Tradition der Schamanen I, Kein Vorteil Tiergeist (Fuchs/Mungo) ODER Vorteil Tradition der Schamanen I, Kein Vorteil Tiergeist (Gebirgsbock) ODER Vorteil Tradition der Schamanen I, Kein Vorteil Tiergeist (Löwe) ODER Vorteil Tradition der Schamanen I, Kein Vorteil Tiergeist (Mammut) ODER Vorteil Tradition der Schamanen I, Kein Vorteil Tiergeist (Rabe) ODER Vorteil Tradition der Schamanen I, Kein Vorteil Tiergeist (Schlange) ODER Vorteil Tradition der Schamanen I, Kein Vorteil Tiergeist (Stier) ODER Vorteil Tradition der Schamanen I, Kein Vorteil Tiergeist (Wildkatze/Panther) ODER Vorteil Tradition der Schamanen I, Kein Vorteil Tiergeist (Wolf/Khoramsbestie) ODER Vorteil Tradition der Schamanen I \textbf{Nachkauf}: Häufig\newline Archetyp: Sammler\newline Wertebonus: Pflanzenkunde, Tierkunde, Überleben, Wachsamkeit, KO\newline Verwandlung: Schnauze (2W6 TP waffenlos, Sinnenschärfe und Wachsamkeit +4, wenn der Geruchssinn gefragt ist), Fell (+1 RS, Pirschen +4)\newline Zauber: Abvenenum, Eins mit der Natur, Kusch!, Standfest, Zaubernahrung\newline Sephrasto: Die Rituale der Anach-nûrim und der Durro-dûn, sowie die Schutzgeister und Flüche der Schamanen beziehen sich immer auf eine bestimmte Tierart. Die Verbindung mit einem solchen Tiergeist wird als kostenloser Vorteil abgebildet. Dabei ist zu beachten, dass nur Schamanen mehrere Tiergeister wählen können. Durro-dûn müssen sich für ein Tier entscheiden und können dann die Zauber des Tiergeists unter der Fertigkeit Gaben des Odun erlernen. Anach-nûrim können das Tier wechseln, indem sie einen neuen Blutgeist aufnehmen. Sie beherrschen dann automatisch alle Zauber des Tiergeists: setze einen Haken bei allen kostenlosen Talenten der Fertigkeit Gaben des Blutgeists.}
}


\newglossaryentry{tiergeist(Wolf/Khoramsbestie)_Vorteil}
{
    name={Tiergeist (Wolf/Khoramsbestie)},
    description={\textbf{Kosten}: 0 \textbf{Voraussetzungen}: Vorteil Tradition der Anach-Nurim I ODER Vorteil Tradition der Durro-dun I ODER Vorteil Tradition der Schamanen I, Kein Vorteil Tiergeist (Affe) ODER Vorteil Tradition der Schamanen I, Kein Vorteil Tiergeist (Bär) ODER Vorteil Tradition der Schamanen I, Kein Vorteil Tiergeist (Elefant) ODER Vorteil Tradition der Schamanen I, Kein Vorteil Tiergeist (Eule) ODER Vorteil Tradition der Schamanen I, Kein Vorteil Tiergeist (Falke) ODER Vorteil Tradition der Schamanen I, Kein Vorteil Tiergeist (Fischotter) ODER Vorteil Tradition der Schamanen I, Kein Vorteil Tiergeist (Fuchs/Mungo) ODER Vorteil Tradition der Schamanen I, Kein Vorteil Tiergeist (Gebirgsbock) ODER Vorteil Tradition der Schamanen I, Kein Vorteil Tiergeist (Löwe) ODER Vorteil Tradition der Schamanen I, Kein Vorteil Tiergeist (Mammut) ODER Vorteil Tradition der Schamanen I, Kein Vorteil Tiergeist (Rabe) ODER Vorteil Tradition der Schamanen I, Kein Vorteil Tiergeist (Schlange) ODER Vorteil Tradition der Schamanen I, Kein Vorteil Tiergeist (Stier) ODER Vorteil Tradition der Schamanen I, Kein Vorteil Tiergeist (Wildkatze/Panther) ODER Vorteil Tradition der Schamanen I, Kein Vorteil Tiergeist (Wildschwein) ODER Vorteil Tradition der Schamanen I \textbf{Nachkauf}: Häufig\newline Archetyp: Kundschafter\newline Wertebonus: Pirschen, Sinnenschärfe, Überleben, IN\newline Verwandlung: Schnauze (2W6 TP waffenlos, Sinnenschärfe und Wachsamkeit +4, wenn der Geruchssinn gefragt ist), Pelz (+2 RS, Kälteschutz wie dicke Winterkleidung (S. 35))\newline Zauber: Adlerauge, Axxeleratus, Eins mit der Natur, Kusch!, Movimento, Spurlos\newline Sephrasto: Die Rituale der Anach-nûrim und der Durro-dûn, sowie die Schutzgeister und Flüche der Schamanen beziehen sich immer auf eine bestimmte Tierart. Die Verbindung mit einem solchen Tiergeist wird als kostenloser Vorteil abgebildet. Dabei ist zu beachten, dass nur Schamanen mehrere Tiergeister wählen können. Durro-dûn müssen sich für ein Tier entscheiden und können dann die Zauber des Tiergeists unter der Fertigkeit Gaben des Odun erlernen. Anach-nûrim können das Tier wechseln, indem sie einen neuen Blutgeist aufnehmen. Sie beherrschen dann automatisch alle Zauber des Tiergeists: setze einen Haken bei allen kostenlosen Talenten der Fertigkeit Gaben des Blutgeists.}
}


\newglossaryentry{traditionderAlchemistenI_Vorteil}
{
    name={Tradition der Alchemisten I},
    description={\textbf{Kosten}: 20 \textbf{Voraussetzungen}: Attribut FF 4, Vorteil Zauberer I \textbf{Nachkauf}: Häufig\newline Bedingungen: Sichtkontakt, Geste, Laut gesprochene Zauberformel.\newline Du kannst Zauber in der Tradition der Alchemisten erlernen und benutzen. Solche Zauber findest du unter den allgemeinen Zaubern und unter Schalenzauber (S. 171).\newline Zauber der Fertigkeiten Einfluss, Verwandlung und Eigenschaften können nur in Form von Artefakten oder als „Zutat“ für Tränke verwendet werden..}
}


\newglossaryentry{traditionderAlchemistenII_Vorteil}
{
    name={Tradition der Alchemisten II},
    description={\textbf{Kosten}: 40 \textbf{Voraussetzungen}: Attribut FF 6, Vorteil Tradition der Alchemisten I \textbf{Nachkauf}: Häufig\newline Alchemistische Zauber sind fest in der stofflichen Welt verankert. Die Wirkungsdauer aller Zauber ist verdoppelt.}
}


\newglossaryentry{traditionderAlchemistenIII_Vorteil}
{
    name={Tradition der Alchemisten III},
    description={\textbf{Kosten}: 60 \textbf{Voraussetzungen}: Attribut FF 8, Vorteil Tradition der Alchemisten II \textbf{Nachkauf}: Häufig\newline Erlaubt die spontane Modifikation Zeit lassen (Zauber +2): Verdoppelt die Vorbereitungszeit, aber erleichtert den Zauber um +2. Eine Vorbereitungszeit von 0 Aktionen wird zu 1 Aktion. Kann einmal pro Zauber eingesetzt werden und wirkt nicht auf Zauber mit frei wählbarer Vorbereitungszeit.}
}


\newglossaryentry{traditionderAlchemistenIV_Vorteil}
{
    name={Tradition der Alchemisten IV},
    description={\textbf{Kosten}: 80 \textbf{Voraussetzungen}: Attribut FF 10, Vorteil Tradition der Alchemisten III \textbf{Nachkauf}: Häufig\newline Bedingung: 2 weitere Attribute auf insgesamt 16.\newline 8 Punkte können zur Verbesserung der Tradition verwendet werden. Sephrasto: Trage die Verbesserungen in das Kommentarfeld ein.}
}


\newglossaryentry{traditionderBorbaradianerI_Vorteil}
{
    name={Tradition der Borbaradianer I},
    description={\textbf{Kosten}: 20 \textbf{Voraussetzungen}: Attribut MU 4, Vorteil Zauberer I \textbf{Nachkauf}: Häufig\newline Bedingungen: Sichtkontakt, Geste, Laut gesprochene Zauberformel.\newline Du kannst Zauber in der Tradition der Borbaradianer erlernen und benutzen. Solche Zauber findest du unter den allgemeinen Zaubern.\newline Du kannst den Vorteil auch mit einem Minderpakt (S. 92) erwerben, ohne Zauberer zu sein.}
}


\newglossaryentry{traditionderBorbaradianerII_Vorteil}
{
    name={Tradition der Borbaradianer II},
    description={\textbf{Kosten}: 40 \textbf{Voraussetzungen}: Attribut MU 6, Vorteil Tradition der Borbaradianer I \textbf{Nachkauf}: Häufig\newline Borbaradianische Zauber sind chaotisch und instabil. Du kannst Zauberproben mit 1W20 statt mit 3W20 ablegen. Dadurch verbessern sich deine Chancen bei besonders gewagten Zaubern.}
}


\newglossaryentry{traditionderBorbaradianerIII_Vorteil}
{
    name={Tradition der Borbaradianer III},
    description={\textbf{Kosten}: 60 \textbf{Voraussetzungen}: Attribut MU 8, Vorteil Tradition der Borbaradianer II \textbf{Nachkauf}: Häufig\newline Erlaubt die spontane Modifikation Erzwingen (Zauber +4): Die Kosten des Zaubers steigen um die Hälfte der Basiskosten, dafür ist der Zauber um 4 Punkte erleichtert. Kann nur einmal pro Zauber eingesetzt werden.}
}


\newglossaryentry{traditionderBorbaradianerIV_Vorteil}
{
    name={Tradition der Borbaradianer IV},
    description={\textbf{Kosten}: 80 \textbf{Voraussetzungen}: Attribut MU 10, Vorteil Tradition der Borbaradianer III \textbf{Nachkauf}: Häufig\newline Bedingung: 2 weitere Attribute auf insgesamt 16.\newline 8 Punkte können zur Verbesserung der Tradition verwendet werden. Sephrasto: Trage die Verbesserungen in das Kommentarfeld ein. Du kannst 4 Punkte davon verwenden, um den Minderpakt zu brechen, ohne die borbaradianische Tradition zu verlieren.}
}


\newglossaryentry{traditionderGeodenI_Vorteil}
{
    name={Tradition der Geoden I},
    description={\textbf{Kosten}: 20 \textbf{Voraussetzungen}: Attribut IN 4, Vorteil Zauberer I \textbf{Nachkauf}: Häufig\newline Bedingungen: Sichtkontakt, Geste, kein Kontakt mit verhüttetem Material.\newline Du kannst Zauber in der Tradition der Geoden erlernen und benutzen. Solche Zauber findest du unter den allgemeinen Zaubern, unter Dolchzauber (S. 156), Ringrituale (S. 170) und Vertrautenmagie (S. 173).\newline Du musst dich beim Erlernen für ein bevorzugtes Element entscheiden. Geodische Zauber können nicht schriftlich weitergegeben werden.}
}


\newglossaryentry{traditionderGeodenII_Vorteil}
{
    name={Tradition der Geoden II},
    description={\textbf{Kosten}: 40 \textbf{Voraussetzungen}: Attribut IN 6, Vorteil Tradition der Geoden I \textbf{Nachkauf}: Häufig\newline Geoden gelten als die fähigsten Elementaristen Aventuriens. Wenn du einen Zauber mit deinem bevorzugten Element wirkst, kannst du eine Basismodifikation (z.B. 1x Mächtige Magie) ohne Erschwernis ausführen.}
}


\newglossaryentry{traditionderGeodenIII_Vorteil}
{
    name={Tradition der Geoden III},
    description={\textbf{Kosten}: 60 \textbf{Voraussetzungen}: Attribut IN 8, Vorteil Tradition der Geoden II \textbf{Nachkauf}: Häufig\newline Erlaubt die spontane Modifikation Zeit lassen (Zauber +2): Verdoppelt die Vorbereitungszeit, aber erleichtert den Zauber um +2. Eine Vorbereitungszeit von 0 Aktionen wird zu 1 Aktion. Kann einmal pro Zauber eingesetzt werden und wirkt nicht auf Zauber mit frei wählbarer Vorbereitungszeit.}
}


\newglossaryentry{traditionderGeodenIV_Vorteil}
{
    name={Tradition der Geoden IV},
    description={\textbf{Kosten}: 80 \textbf{Voraussetzungen}: Attribut IN 10, Vorteil Tradition der Geoden III \textbf{Nachkauf}: Häufig\newline Bedingung: 2 weitere Attribute auf insgesamt 16.\newline 8 Punkte können zur Verbesserung der Tradition verwendet werden. Sephrasto: Trage die Verbesserungen in das Kommentarfeld ein.}
}


\newglossaryentry{traditionderGildenmagierI_Vorteil}
{
    name={Tradition der Gildenmagier I},
    description={\textbf{Kosten}: 20 \textbf{Voraussetzungen}: Attribut KL 4, Vorteil Zauberer I \textbf{Nachkauf}: Häufig\newline Bedingungen: Sichtkontakt, Geste, laut gesprochene Zauberformel.\newline Du kannst Zauber in der Tradition der Gildenmagier erlernen und benutzen. Solche Zauber findest du unter den allgemeinen Zaubern, unter Kugelzauber (S. 169), Schalenzauber (S. 171) und Stabzauber (S. 172).\newline Wenn du beim Vorbereiten eines Zaubers gestört wirst, sind entsprechende Willenskraftproben um zusätzliche –4 erschwert.}
}


\newglossaryentry{traditionderGildenmagierII_Vorteil}
{
    name={Tradition der Gildenmagier II},
    description={\textbf{Kosten}: 40 \textbf{Voraussetzungen}: Attribut KL 6, Vorteil Tradition der Gildenmagier I \textbf{Nachkauf}: Häufig\newline Gildenmagier verstehen die Matrix eines Zaubers und können ihn leichter verändern. Wenn du einen Zauber mit mindestens zwei unterschiedlichen Basismodifikationen wirkst, ist er zusätzlich um +2 erleichtert.}
}


\newglossaryentry{traditionderGildenmagierIII_Vorteil}
{
    name={Tradition der Gildenmagier III},
    description={\textbf{Kosten}: 60 \textbf{Voraussetzungen}: Attribut KL 8, Vorteil Tradition der Gildenmagier II \textbf{Nachkauf}: Häufig\newline Erlaubt die spontane Modifikation Zeit lassen (Zauber +2): Verdoppelt die Vorbereitungszeit, aber erleichtert den Zauber um +2. Eine Vorbereitungszeit von 0 Aktionen wird zu 1 Aktion. Kann einmal pro Zauber eingesetzt werden und wirkt nicht auf Zauber mit frei wählbarer Vorbereitungszeit.}
}


\newglossaryentry{traditionderGildenmagierIV_Vorteil}
{
    name={Tradition der Gildenmagier IV},
    description={\textbf{Kosten}: 80 \textbf{Voraussetzungen}: Attribut KL 10, Vorteil Tradition der Gildenmagier III \textbf{Nachkauf}: Häufig\newline Bedingung: 2 weitere Attribute auf insgesamt 16.\newline 8 Punkte können zur Verbesserung der Tradition verwendet werden. Sephrasto: Trage die Verbesserungen in das Kommentarfeld ein.}
}


\newglossaryentry{traditionderDruidenI_Vorteil}
{
    name={Tradition der Druiden I},
    description={\textbf{Kosten}: 20 \textbf{Voraussetzungen}: Attribut KL 4, Vorteil Zauberer I \textbf{Nachkauf}: Häufig\newline Bedingungen: Sichtkontakt, Geste, kein Kontakt mit verhüttetem Material.\newline Du kannst Zauber in der Tradition der Druiden erlernen und benutzen. Solche Zauber findest du unter den allgemeinen Zaubern und unter Dolchzauber (S. 156).\newline Druidische Zauber können nicht schriftlich weitergegeben werden.}
}


\newglossaryentry{traditionderDruidenII_Vorteil}
{
    name={Tradition der Druiden II},
    description={\textbf{Kosten}: 40 \textbf{Voraussetzungen}: Attribut KL 6, Vorteil Tradition der Druiden I \textbf{Nachkauf}: Häufig\newline Druiden wissen um Orte von besonderer Kraft. Auf Kraftlinien und -knoten, die sich auch in vielen Städten finden, sind alle druidische Zauber und Rituale um 2–4 Punkte erleichtert.}
}


\newglossaryentry{traditionderDruidenIII_Vorteil}
{
    name={Tradition der Druiden III},
    description={\textbf{Kosten}: 60 \textbf{Voraussetzungen}: Attribut KL 8, Vorteil Tradition der Druiden II \textbf{Nachkauf}: Häufig\newline Erlaubt die spontane Modifikation Erzwingen (Zauber +4): Die Kosten des Zaubers steigen um die Hälfte der Basiskosten, dafür ist der Zauber um 4 Punkte erleichtert. Kann nur einmal pro Zauber eingesetzt werden.}
}


\newglossaryentry{traditionderDruidenIV_Vorteil}
{
    name={Tradition der Druiden IV},
    description={\textbf{Kosten}: 80 \textbf{Voraussetzungen}: Attribut KL 10, Vorteil Tradition der Druiden III \textbf{Nachkauf}: Häufig\newline Bedingung: 2 weitere Attribute auf insgesamt 16.\newline 8 Punkte können zur Verbesserung der Tradition verwendet werden. Sephrasto: Trage die Verbesserungen in das Kommentarfeld ein.}
}


\newglossaryentry{traditionderElfenI_Vorteil}
{
    name={Tradition der Elfen I},
    description={\textbf{Kosten}: 20 \textbf{Voraussetzungen}: Attribut IN 4, Vorteil Zauberer I \textbf{Nachkauf}: Häufig\newline Bedingungen: Sichtkontakt, Geste, Gesungene oder gespielte Formel.\newline Du kannst Zauber in der Tradition der Elfen erlernen und benutzen. Solche Zauber findest du unter den allgemeinen Zaubern und unter Elfenlieder (S. 158).\newline In Umgebungen mit gestörter Harmonie (z.B. dämonische Verseuchung, Geisterscheinungen) sind Zauber um 2–4 Punkte erschwert.}
}


\newglossaryentry{traditionderElfenII_Vorteil}
{
    name={Tradition der Elfen II},
    description={\textbf{Kosten}: 40 \textbf{Voraussetzungen}: Attribut IN 6, Vorteil Tradition der Elfen I \textbf{Nachkauf}: Häufig\newline Elfen begreifen Magie als alltägliche Unterstützung ihrer Fähigkeiten. Die Wirkungsdauer deiner Zauber, die auf dich selbst wirken, ist verdoppelt.}
}


\newglossaryentry{traditionderElfenIII_Vorteil}
{
    name={Tradition der Elfen III},
    description={\textbf{Kosten}: 60 \textbf{Voraussetzungen}: Attribut IN 8, Vorteil Tradition der Elfen II \textbf{Nachkauf}: Häufig\newline Erlaubt die spontane Modifikation Zeit lassen (Zauber +2): Verdoppelt die Vorbereitungszeit, aber erleichtert den Zauber um +2. Eine Vorbereitungszeit von 0 Aktionen wird zu 1 Aktion. Kann einmal pro Zauber eingesetzt werden und wirkt nicht auf Zauber mit frei wählbarer Vorbereitungszeit.}
}


\newglossaryentry{traditionderElfenIV_Vorteil}
{
    name={Tradition der Elfen IV},
    description={\textbf{Kosten}: 80 \textbf{Voraussetzungen}: Attribut IN 10, Vorteil Tradition der Elfen III \textbf{Nachkauf}: Häufig\newline Bedingung: 2 weitere Attribute auf insgesamt 16.\newline 8 Punkte können zur Verbesserung der Tradition verwendet werden. Sephrasto: Trage die Verbesserungen in das Kommentarfeld ein.}
}


\newglossaryentry{traditionderHexenI_Vorteil}
{
    name={Tradition der Hexen I},
    description={\textbf{Kosten}: 20 \textbf{Voraussetzungen}: Attribut IN 4, Vorteil Zauberer I \textbf{Nachkauf}: Häufig\newline Bedingungen: Sichtkontakt, Geste, Bodenkontakt.\newline Du kannst Zauber in der Tradition der Hexen erlernen und benutzen. Solche Zauber findest du unter den allgemeinen Zaubern, unter Hexenflüche (S. 166), Schalenzauber (S. 171) und Vertrautenmagie (S. 173).\newline Die Komponente Bodenkontakt kann nur ignoriert werden, wenn du zumindest über indirekten Bodenkontakt verfügst (z.B. in einem mehrstöckigen Haus).}
}


\newglossaryentry{traditionderHexenII_Vorteil}
{
    name={Tradition der Hexen II},
    description={\textbf{Kosten}: 40 \textbf{Voraussetzungen}: Attribut IN 6, Vorteil Tradition der Hexen I \textbf{Nachkauf}: Häufig\newline Für Hexen gehören Gefühle und Magie untrennbar zusammen. Im Zustand besonders starker, zum Zauber passender Emotionen sind Zauber und Rituale um 2–4 Punkte erleichtert.}
}


\newglossaryentry{traditionderHexenIII_Vorteil}
{
    name={Tradition der Hexen III},
    description={\textbf{Kosten}: 60 \textbf{Voraussetzungen}: Attribut IN 8, Vorteil Tradition der Hexen II \textbf{Nachkauf}: Häufig\newline Erlaubt die spontane Modifikation Erzwingen (Zauber +4): Die Kosten des Zaubers steigen um die Hälfte der Basiskosten, dafür ist der Zauber um 4 Punkte erleichtert. Kann nur einmal pro Zauber eingesetzt werden.}
}


\newglossaryentry{traditionderHexenIV_Vorteil}
{
    name={Tradition der Hexen IV},
    description={\textbf{Kosten}: 80 \textbf{Voraussetzungen}: Attribut IN 10, Vorteil Tradition der Hexen III \textbf{Nachkauf}: Häufig\newline Bedingung: 2 weitere Attribute auf insgesamt 16.\newline 8 Punkte können zur Verbesserung der Tradition verwendet werden. Sephrasto: Trage die Verbesserungen in das Kommentarfeld ein.}
}


\newglossaryentry{traditionderKristallomantenI_Vorteil}
{
    name={Tradition der Kristallomanten I},
    description={\textbf{Kosten}: 20 \textbf{Voraussetzungen}: Attribut KL 4, Vorteil Zauberer I \textbf{Nachkauf}: Häufig\newline Bedingungen: Sichtkontakt, Geste, passender gebundender Kristall.\newline Du kannst Zauber in der Tradition der Kristallomanten erlernen und benutzen. Solche Zauber findest du unter den allgemeinen Zaubern und unter Kristallmagie (S. 168).\newline Wenn du einen Zauber mit der Modifikation Vorbereitung verkürzen wirkst, ist er zusätzlich um –2 erschwert.}
}


\newglossaryentry{traditionderKristallomantenII_Vorteil}
{
    name={Tradition der Kristallomanten II},
    description={\textbf{Kosten}: 40 \textbf{Voraussetzungen}: Attribut KL 6, Vorteil Tradition der Kristallomanten I \textbf{Nachkauf}: Häufig\newline Durch das Studium der Edelsteine gelten Kristallomanten als herausragende Artefaktmagier. Zauber zur Artefakterschaffung und zur magischen Analyse sind um +2 erleichtert.}
}


\newglossaryentry{traditionderKristallomantenIII_Vorteil}
{
    name={Tradition der Kristallomanten III},
    description={\textbf{Kosten}: 60 \textbf{Voraussetzungen}: Attribut KL 8, Vorteil Tradition der Kristallomanten II \textbf{Nachkauf}: Häufig\newline Erlaubt die spontane Modifikation Zeit lassen (Zauber +2): Verdoppelt die Vorbereitungszeit, aber erleichtert den Zauber um +2. Eine Vorbereitungszeit von 0 Aktionen wird zu 1 Aktion. Kann einmal pro Zauber eingesetzt werden und wirkt nicht auf Zauber mit frei wählbarer Vorbereitungszeit.}
}


\newglossaryentry{traditionderKristallomantenIV_Vorteil}
{
    name={Tradition der Kristallomanten IV},
    description={\textbf{Kosten}: 80 \textbf{Voraussetzungen}: Attribut KL 10, Vorteil Tradition der Kristallomanten III \textbf{Nachkauf}: Häufig\newline Bedingung: 2 weitere Attribute auf insgesamt 16.\newline 8 Punkte können zur Verbesserung der Tradition verwendet werden. Sephrasto: Trage die Verbesserungen in das Kommentarfeld ein.}
}


\newglossaryentry{traditionderSchelmeI_Vorteil}
{
    name={Tradition der Schelme I},
    description={\textbf{Kosten}: 20 \textbf{Voraussetzungen}: Attribut IN 4, Vorteil Zauberer I \textbf{Nachkauf}: Häufig\newline Bedingungen: Sichtkontakt, Geste, Harmlosigkeit.\newline Du kannst Zauber in der Tradition der Schelme erlernen und benutzen. Solche Zauber findest du unter den allgemeinen Zaubern.\newline Schelmische Zauber können nicht schriftlich weitergegeben werden.}
}


\newglossaryentry{traditionderSchelmeII_Vorteil}
{
    name={Tradition der Schelme II},
    description={\textbf{Kosten}: 40 \textbf{Voraussetzungen}: Attribut IN 6, Vorteil Tradition der Schelme I \textbf{Nachkauf}: Häufig\newline Schelmenmagie umgeht den geistigen Widerstand ihrer Ziele. Wenn ein Zauber gegen die Magieresistenz wirkt, ist er um +2 erleichtert.}
}


\newglossaryentry{traditionderSchelmeIII_Vorteil}
{
    name={Tradition der Schelme III},
    description={\textbf{Kosten}: 60 \textbf{Voraussetzungen}: Attribut IN 8, Vorteil Tradition der Schelme II \textbf{Nachkauf}: Häufig\newline Erlaubt die spontane Modifikation Erzwingen (Zauber +4): Die Kosten des Zaubers steigen um die Hälfte der Basiskosten, dafür ist der Zauber um 4 Punkte erleichtert. Kann nur einmal pro Zauber eingesetzt werden.}
}


\newglossaryentry{traditionderSchelmeIV_Vorteil}
{
    name={Tradition der Schelme IV},
    description={\textbf{Kosten}: 80 \textbf{Voraussetzungen}: Attribut IN 10, Vorteil Tradition der Schelme III \textbf{Nachkauf}: Häufig\newline Bedingung: 2 weitere Attribute auf insgesamt 16.\newline 8 Punkte können zur Verbesserung der Tradition verwendet werden. Sephrasto: Trage die Verbesserungen in das Kommentarfeld ein.}
}


\newglossaryentry{traditionderScharlataneI_Vorteil}
{
    name={Tradition der Scharlatane I},
    description={\textbf{Kosten}: 20 \textbf{Voraussetzungen}: Attribut CH 4, Vorteil Zauberer I \textbf{Nachkauf}: Häufig\newline Bedingungen: Sichtkontakt, Geste, laut gesprochene Zauberformel.\newline Du kannst Zauber in der Tradition der Scharlatane erlernen und benutzen. Solche Zauber findest du unter den allgemeinen Zaubern und unter Kugelzauber (S. 169).\newline Zauber mit Basiskosten von 16 AsP oder mehr sind um –2 erschwert.}
}


\newglossaryentry{traditionderScharlataneII_Vorteil}
{
    name={Tradition der Scharlatane II},
    description={\textbf{Kosten}: 40 \textbf{Voraussetzungen}: Attribut CH 6, Vorteil Tradition der Scharlatane I \textbf{Nachkauf}: Häufig\newline Scharlatane sind Meister der Illusionsmagie. Wenn du einen Zauber mit der Fertigkeit Illusion wirkst, kannst du eine Basismodifikation (z.B. 1x Mächtige Magie) ohne Erschwernis ausführen.}
}


\newglossaryentry{traditionderScharlataneIII_Vorteil}
{
    name={Tradition der Scharlatane III},
    description={\textbf{Kosten}: 60 \textbf{Voraussetzungen}: Attribut CH 8, Vorteil Tradition der Scharlatane II \textbf{Nachkauf}: Häufig\newline Erlaubt die spontane Modifikation Zeit lassen (Zauber +2): Verdoppelt die Vorbereitungszeit, aber erleichtert den Zauber um +2. Eine Vorbereitungszeit von 0 Aktionen wird zu 1 Aktion. Kann einmal pro Zauber eingesetzt werden und wirkt nicht auf Zauber mit frei wählbarer Vorbereitungszeit.}
}


\newglossaryentry{traditionderScharlataneIV_Vorteil}
{
    name={Tradition der Scharlatane IV},
    description={\textbf{Kosten}: 80 \textbf{Voraussetzungen}: Attribut CH 10, Vorteil Tradition der Scharlatane III \textbf{Nachkauf}: Häufig\newline Bedingung: 2 weitere Attribute auf insgesamt 16.\newline 8 Punkte können zur Verbesserung der Tradition verwendet werden. Sephrasto: Trage die Verbesserungen in das Kommentarfeld ein.}
}


\newglossaryentry{magiedilettant_Vorteil}
{
    name={Magiedilettant},
    description={\textbf{Kosten}: 20 \textbf{Voraussetzungen}: Vorteil Zauberer I \textbf{Nachkauf}: Häufig\newline Bedingungen: Sichtkontakt, Geste.\newline Du kannst alle allgemeinen Zauber erlernen und benutzen, die weniger als 60 EP kosten.\newline Du kannst pro Zauber maximal zwei spontane Modifikationen (oder eine Modifikation zweimal) einsetzen.}
}


\newglossaryentry{traditionderAnach-NurimI_Vorteil}
{
    name={Tradition der Anach-Nurim I},
    description={\textbf{Kosten}: 20 \textbf{Voraussetzungen}: Attribut KO 4, Vorteil Zauberer I \textbf{Nachkauf}: Häufig\newline Bedingungen: Sichtkontakt, Geste.\newline Du kannst Zauber in der Tradition der Anach-nûrim erlernen und benutzen. Solche Zauber findest du unter Gaben des Blutgeists (S. 159).\newline Wenn du einen Zauber ohne Verbotene Pforten oder Blutmagie wirkst, ist er um -2 erschwert.}
}


\newglossaryentry{traditionderAnach-NurimII_Vorteil}
{
    name={Tradition der Anach-Nurim II},
    description={\textbf{Kosten}: 40 \textbf{Voraussetzungen}: Attribut KO 6, Vorteil Tradition der Anach-Nurim I \textbf{Nachkauf}: Häufig\newline Wenn du einen Zauber mit verbotenen Pforten oder Blutmagie wirkst, kannst du eine Basismodifikation (z.B. 1x Mächtige Magie) ohne Erschwernis ausführen.}
}


\newglossaryentry{traditionderAnach-NurimIII_Vorteil}
{
    name={Tradition der Anach-Nurim III},
    description={\textbf{Kosten}: 60 \textbf{Voraussetzungen}: Attribut KO 8, Vorteil Tradition der Anach-Nurim II \textbf{Nachkauf}: Häufig\newline Erlaubt die spontane Modifikation Erzwingen (Zauber +4): Die Kosten des Zaubers steigen um die Hälfte der Basiskosten, dafür ist der Zauber um 4 Punkte erleichtert. Kann nur einmal pro Zauber eingesetzt werden.}
}


\newglossaryentry{traditionderAnach-NurimIV_Vorteil}
{
    name={Tradition der Anach-Nurim IV},
    description={\textbf{Kosten}: 80 \textbf{Voraussetzungen}: Attribut KO 10, Vorteil Tradition der Anach-Nurim III \textbf{Nachkauf}: Häufig\newline Bedingung: 2 weitere Attribute auf insgesamt 16.\newline 8 Punkte können zur Verbesserung der Tradition verwendet werden. Sephrasto: Trage die Verbesserungen in das Kommentarfeld ein.}
}


\newglossaryentry{traditionderDerwischeI_Vorteil}
{
    name={Tradition der Derwische I},
    description={\textbf{Kosten}: 20 \textbf{Voraussetzungen}: Attribut FF 4, Vorteil Zauberer I \textbf{Nachkauf}: Häufig\newline Bedingungen: Sichtkontakt, Trommelspiel, nur Rashtullahgläubige profitieren von Zaubern.\newline Du kannst Zauber in der Tradition der Derwische erlernen und benutzen. Solche Zauber findest du unter Trommelrituale (S. 172).\newline Die Bedingung Trommelspiel kann nicht ignoriert werden.}
}


\newglossaryentry{traditionderDerwischeII_Vorteil}
{
    name={Tradition der Derwische II},
    description={\textbf{Kosten}: 40 \textbf{Voraussetzungen}: Attribut FF 6, Vorteil Tradition der Derwische I \textbf{Nachkauf}: Häufig\newline Deine Rhythmen sind auf größere Entfernung hörbar. Alle deine Zauber mit dem Ziel Zone wirken auf alle Ziele im Umkreis von 8 Schritt (statt 4 Schritt).}
}


\newglossaryentry{traditionderDerwischeIII_Vorteil}
{
    name={Tradition der Derwische III},
    description={\textbf{Kosten}: 60 \textbf{Voraussetzungen}: Attribut FF 8, Vorteil Tradition der Derwische II \textbf{Nachkauf}: Häufig\newline Erlaubt die spontane Modifikation Opferung (Zauber +4): Ein Opfer erleichtert den Zauber um +4. Derwische legen ihre ganze Kraft in den Zauber und erleiden danach 2 Punkte Erschöpfung.}
}


\newglossaryentry{traditionderDerwischeIV_Vorteil}
{
    name={Tradition der Derwische IV},
    description={\textbf{Kosten}: 80 \textbf{Voraussetzungen}: Attribut FF 10, Vorteil Tradition der Derwische III \textbf{Nachkauf}: Häufig\newline Bedingung: 2 weitere Attribute auf insgesamt 16.\newline 8 Punkte können zur Verbesserung der Tradition verwendet werden. Sephrasto: Trage die Verbesserungen in das Kommentarfeld ein.}
}


\newglossaryentry{traditionderDurro-dunI_Vorteil}
{
    name={Tradition der Durro-dun I},
    description={\textbf{Kosten}: 20 \textbf{Voraussetzungen}: Attribut MU 4, Vorteil Zauberer I \textbf{Nachkauf}: Häufig\newline Bedingungen: Sichtkontakt, Geste.\newline Du kannst Zauber in der Tradition der Durro-dûn erlernen und benutzen. Solche Zauber findest du unter Gaben des Odun (S. 159).\newline Du musst dich beim Erlernen für ein Tier entscheiden (siehe S. 160). Alle deine Zauber beziehen sich auf dieses Tier.\newline Sephrasto: Die Tiertabelle ist in Sephrasto über die kostenlosen Tiergeist-Vorteile abgebildet, wähle einen davon aus. Dies schaltet alle Zauber des Tiergeists über die Fertigkeit Gaben des Odun frei.}
}


\newglossaryentry{traditionderDurro-dunII_Vorteil}
{
    name={Tradition der Durro-dun II},
    description={\textbf{Kosten}: 40 \textbf{Voraussetzungen}: Attribut MU 6, Vorteil Tradition der Durro-dun I \textbf{Nachkauf}: Häufig\newline Erfahrene Durro-dûn rufen die Tiergeister scheinbar mühelos in ihren Körper. Die Kosten aller Zauber, die du auf dich selbst wirkst, sinken um ein Viertel der Basiskosten.}
}


\newglossaryentry{traditionderDurro-dunIII_Vorteil}
{
    name={Tradition der Durro-dun III},
    description={\textbf{Kosten}: 60 \textbf{Voraussetzungen}: Attribut MU 8, Vorteil Tradition der Durro-dun II \textbf{Nachkauf}: Häufig\newline Erlaubt die spontane Modifikation Opferung (Zauber +4): Ein Opfer erleichtert den Zauber um +4. Die Tierkrieger opfern den Geistern ihr Blut und fügen sich dabei eine Wunde zu, deren Auswirkungen bei der Probe für den Zauber ignoriert werden.}
}


\newglossaryentry{traditionderDurro-dunIV_Vorteil}
{
    name={Tradition der Durro-dun IV},
    description={\textbf{Kosten}: 80 \textbf{Voraussetzungen}: Attribut MU 10, Vorteil Tradition der Durro-dun III \textbf{Nachkauf}: Häufig\newline Bedingung: 2 weitere Attribute auf insgesamt 16.\newline 8 Punkte können zur Verbesserung der Tradition verwendet werden. Sephrasto: Trage die Verbesserungen in das Kommentarfeld ein.}
}


\newglossaryentry{traditionderSchamanenI_Vorteil}
{
    name={Tradition der Schamanen I},
    description={\textbf{Kosten}: 20 \textbf{Voraussetzungen}: Attribut IN 4, Vorteil Zauberer I \textbf{Nachkauf}: Häufig\newline Bedingungen: Sichtkontakt, Geste oder Tanz, gesprochene oder gesungene Formel.\newline Du kannst Zauber in der Tradition der Schamanen erlernen und benutzen. Solche Zauber findest du unter Geister d. Stärkung (S. 162), Geister d. Zorns, Geister rufen, Geister vertreiben und Keulenrituale (S. 167).\newline Verärgert der Nutznießer eines positiven schamanischen Zaubers den entsprechenden Geist, endet der Zauber sofort (Spielleiterentscheid).}
}


\newglossaryentry{traditionderSchamanenII_Vorteil}
{
    name={Tradition der Schamanen II},
    description={\textbf{Kosten}: 40 \textbf{Voraussetzungen}: Attribut IN 6, Vorteil Tradition der Schamanen I \textbf{Nachkauf}: Häufig\newline Schamanen binden oft einen beträchtlichen Teil ihrer Astralenergie in ihre Zauber. Wenn du mindestens 2/4/8/16 gAsP gebunden hast, sind deine Zauber um +1/+2/+3/+4 erleichtert.}
}


\newglossaryentry{traditionderSchamanenIII_Vorteil}
{
    name={Tradition der Schamanen III},
    description={\textbf{Kosten}: 60 \textbf{Voraussetzungen}: Attribut IN 8, Vorteil Tradition der Schamanen II \textbf{Nachkauf}: Häufig\newline Erlaubt die spontane Modifikation Zeremonie (Zauber +X): Du kannst die Vorbereitungszeit freiwillig um 1 Minute/Stunde/Tag/Woche/Monat/Jahr erhöhen, wodurch der Zauber um +4/6/8/10/12/14 erleichtert ist. Die Vorbereitungszeit muss dadurch mindestens verdoppelt werden.}
}


\newglossaryentry{traditionderSchamanenIV_Vorteil}
{
    name={Tradition der Schamanen IV},
    description={\textbf{Kosten}: 80 \textbf{Voraussetzungen}: Attribut IN 10, Vorteil Tradition der Schamanen III \textbf{Nachkauf}: Häufig\newline Bedingung: 2 weitere Attribute auf insgesamt 16.\newline 8 Punkte können zur Verbesserung der Tradition verwendet werden. Sephrasto: Trage die Verbesserungen in das Kommentarfeld ein. Für 4 Punkte davon kannst du die Fähigkeit erlangen, mittels freier Seelenfahrt ins Geisterreich zu reisen und dort verschiedenste Zauberwirkungen hervorzurufen. Solche Reisen sind jedoch stets gefährlich, Details sind Spielleiterentscheid.}
}


\newglossaryentry{traditionderZauberbardenI_Vorteil}
{
    name={Tradition der Zauberbarden I},
    description={\textbf{Kosten}: 20 \textbf{Voraussetzungen}: Attribut CH 4, Vorteil Zauberer I \textbf{Nachkauf}: Häufig\newline Bedingungen: Sichtkontakt, Instrumentenspiel, Ziel kann dich hören.\newline Du kannst Zauber in der Tradition der Zauberbarden erlernen und benutzen. Solche Zauber findest du unter Zaubermelodien (S. 174).\newline Die Bedingung Instrumentenspiel kann nicht ignoriert werden.}
}


\newglossaryentry{traditionderZauberbardenII_Vorteil}
{
    name={Tradition der Zauberbarden II},
    description={\textbf{Kosten}: 40 \textbf{Voraussetzungen}: Attribut CH 6, Vorteil Tradition der Zauberbarden I \textbf{Nachkauf}: Häufig\newline Zauberbarden können ihre Melodie flexibel verändern. Während du einen Zauber mit der Eigenschaft Konzentration (S. 124) aufrechterhältst, kannst du weitere Zauber vorbereiten. (Die Aktion Konflikt ist dennoch nötig und beendet den vorherigen Zauber.)}
}


\newglossaryentry{traditionderZauberbardenIII_Vorteil}
{
    name={Tradition der Zauberbarden III},
    description={\textbf{Kosten}: 60 \textbf{Voraussetzungen}: Attribut CH 8, Vorteil Tradition der Zauberbarden II \textbf{Nachkauf}: Häufig\newline Erlaubt die spontane Modifikation Zeit lassen (Zauber +2): Verdoppelt die Vorbereitungszeit, aber erleichtert den Zauber um +2. Eine Vorbereitungszeit von 0 Aktionen wird zu 1 Aktion. Kann einmal pro Zauber eingesetzt werden und wirkt nicht auf Zauber mit frei wählbarer Vorbereitungszeit.}
}


\newglossaryentry{traditionderZauberbardenIV_Vorteil}
{
    name={Tradition der Zauberbarden IV},
    description={\textbf{Kosten}: 80 \textbf{Voraussetzungen}: Attribut CH 10, Vorteil Tradition der Zauberbarden III \textbf{Nachkauf}: Häufig\newline Bedingung: 2 weitere Attribute auf insgesamt 16.\newline 8 Punkte können zur Verbesserung der Tradition verwendet werden. Sephrasto: Trage die Verbesserungen in das Kommentarfeld ein.}
}


\newglossaryentry{traditionderZaubertänzerI_Vorteil}
{
    name={Tradition der Zaubertänzer I},
    description={\textbf{Kosten}: 20 \textbf{Voraussetzungen}: Attribut GE 4, Vorteil Zauberer I \textbf{Nachkauf}: Häufig\newline Bedingungen: Sichtkontakt, Tanz, Ziele können dich sehen.\newline Du kannst Zauber in der Tradition der Zaubertänzer erlernen und benutzen. Solche Zauber findest du unter Zaubertänze (S. 176).\newline Die Bedingung Tanz kann nicht ignoriert werden.}
}


\newglossaryentry{traditionderZaubertänzerII_Vorteil}
{
    name={Tradition der Zaubertänzer II},
    description={\textbf{Kosten}: 40 \textbf{Voraussetzungen}: Attribut GE 6, Vorteil Tradition der Zaubertänzer I \textbf{Nachkauf}: Häufig\newline Die Darbietung eines erfahrenen Zaubertänzers hinterlässt einen bleibenden Eindruck. Die Wirkungsdauer aller deiner Zauber ist verdoppelt.}
}


\newglossaryentry{traditionderZaubertänzerIII_Vorteil}
{
    name={Tradition der Zaubertänzer III},
    description={\textbf{Kosten}: 60 \textbf{Voraussetzungen}: Attribut GE 8, Vorteil Tradition der Zaubertänzer II \textbf{Nachkauf}: Häufig\newline Erlaubt die spontane Modifikation Opferung (Zauber +4): Ein Opfer erleichtert den Zauber um +4. Zaubertänzer legen ihre ganze Kraft in den Zauber und erleiden danach 2 Punkte Erschöpfung.}
}


\newglossaryentry{traditionderZaubertänzerIV_Vorteil}
{
    name={Tradition der Zaubertänzer IV},
    description={\textbf{Kosten}: 80 \textbf{Voraussetzungen}: Attribut GE 10, Vorteil Tradition der Zaubertänzer III \textbf{Nachkauf}: Häufig\newline Bedingung: 2 weitere Attribute auf insgesamt 16.\newline 8 Punkte können zur Verbesserung der Tradition verwendet werden. Sephrasto: Trage die Verbesserungen in das Kommentarfeld ein.}
}


\newglossaryentry{gemeinschaftderGläubigen_Vorteil}
{
    name={Gemeinschaft der Gläubigen},
    description={\textbf{Kosten}: 20 \textbf{Voraussetzungen}: Attribut CH 4, Vorteil Geweiht I \textbf{Nachkauf}: Häufig\newline Für mindestens 4/8/16/32 Mitbetende ist die Liturgieprobe um +1/2/3/4 erleichtert.}
}


\newglossaryentry{unterstützungderGläubigen_Vorteil}
{
    name={Unterstützung der Gläubigen},
    description={\textbf{Kosten}: 40 \textbf{Voraussetzungen}: Attribut CH 6, Vorteil Geweiht I \textbf{Nachkauf}: Häufig\newline Mirakel können auch auf andere Gläubige gewirkt werden. Die Wirkung solcher Mirakel ist halbiert.}
}


\newglossaryentry{auraderHeiligkeit_Vorteil}
{
    name={Aura der Heiligkeit},
    description={\textbf{Kosten}: 60 \textbf{Voraussetzungen}: Attribut CH 8, Vorteil Geweiht I \textbf{Nachkauf}: Häufig\newline Gläubige Mitstreiter in einem Radius von 16 Schritt erhalten eine Erleichterung von +1 auf alle Mirakel-Fertigkeiten deiner Gottheit, sowie alle Liturgien. Du selbst profitierst nicht von diesem Bonus.}
}


\newglossaryentry{segnungderGläubigen_Vorteil}
{
    name={Segnung der Gläubigen},
    description={\textbf{Kosten}: 80 \textbf{Voraussetzungen}: Attribut CH 10, Vorteil Geweiht I \textbf{Nachkauf}: Häufig\newline Liturgien, die ausschließlich gläubige Mitstreiter (und nicht dich selbst) verstärken, kosten nur die halben Basiskosten.}
}


\newglossaryentry{gemeinsamesWunder_Vorteil}
{
    name={Gemeinsames Wunder},
    description={\textbf{Kosten}: 20 \textbf{Voraussetzungen}: Attribut IN 4, Vorteil Geweiht I \textbf{Nachkauf}: Häufig\newline Der Vorteil ermöglicht es dir, mit anderen Kennern dieses Vorteils einen Liturgiezirkel zu bilden. Alle Teilnehmer des Zirkels müssen sich berühren und willens sein, ihre Karmaenergie zu teilen.\newline Solange der Zirkel besteht, können die Teilnehmer die Regeln zur Zusammenarbeit (S. 8) nutzen, um sich gegenseitig bei Liturgien zu unterstützen. Üblicherweise sind dabei maximal vier Helfer möglich, welche die Probe des Wirkenden um jeweils +2 erleichtern können.\newline Zusätzlich werden die Liturgiekosten aller Teilnehmer gleichmäßig auf den gesamten Zirkel aufgeteilt. Sollte ein Teilungsrest übrigbleiben oder ein Teilnehmer über zu wenige KaP verfügen, muss der Wirkende diese Kosten tragen. Nur gKaP muss der Wirkende stets selbst tragen.}
}


\newglossaryentry{zuverlässigesWunder_Vorteil}
{
    name={Zuverlässiges Wunder},
    description={\textbf{Kosten}: 40 \textbf{Voraussetzungen}: Attribut IN 6, Vorteil Geweiht I \textbf{Nachkauf}: Häufig\newline Schlägt eine durch ein Mirakel unterstützte Probe fehl, bleibt die Wirkung des Mirakels aufrecht (1x pro Mirakel). Patzer bei solchen Proben gelten nur als gewöhnlich misslungen.}
}


\newglossaryentry{beeindruckendesWunder_Vorteil}
{
    name={Beeindruckendes Wunder},
    description={\textbf{Kosten}: 60 \textbf{Voraussetzungen}: Attribut IN 8, Vorteil Geweiht I \textbf{Nachkauf}: Häufig\newline Bei Mirakeln verleiht die Modifikation Mächtige Liturgie einen Bonus von +4 statt +2.}
}


\newglossaryentry{göttlicheNähe_Vorteil}
{
    name={Göttliche Nähe},
    description={\textbf{Kosten}: 80 \textbf{Voraussetzungen}: Attribut IN 10, Vorteil Geweiht I \textbf{Nachkauf}: Häufig\newline Du regenerierst 1 zusätzlichen KaP pro Nacht.}
}


\newglossaryentry{liturgischeRoutine_Vorteil}
{
    name={Liturgische Routine},
    description={\textbf{Kosten}: 20 \textbf{Voraussetzungen}: Attribut KL 4, Vorteil Geweiht I \textbf{Nachkauf}: Häufig\newline Bei Liturgien gewürfelte Patzer gelten nur als gewöhnlich misslungen.}
}


\newglossaryentry{liturgischeSorgfalt_Vorteil}
{
    name={Liturgische Sorgfalt},
    description={\textbf{Kosten}: 40 \textbf{Voraussetzungen}: Attribut KL 6, Vorteil Geweiht I \textbf{Nachkauf}: Häufig\newline Misslungene Liturgien kosten nur 1/4 der Basiskosten.}
}


\newglossaryentry{liturgischeDisziplin_Vorteil}
{
    name={Liturgische Disziplin},
    description={\textbf{Kosten}: 60 \textbf{Voraussetzungen}: Attribut KL 8, Vorteil Geweiht I \textbf{Nachkauf}: Häufig\newline Ermöglicht die spontane Modifikation Kosten sparen (Liturgie -4): Die Kosten der Liturgie sinken um ein Viertel der Basiskosten. Die Kosten können dadurch nicht unter die Hälfte der Basiskosten sinken.}
}


\newglossaryentry{lieblingderGottheit_Vorteil}
{
    name={Liebling der Gottheit},
    description={\textbf{Kosten}: 80 \textbf{Voraussetzungen}: Attribut KL 10, Vorteil Geweiht I \textbf{Nachkauf}: Häufig\newline Zeigt der gewertete Würfel bei einer Liturgie eine 16 oder höher, kostet diese keine KaP und erhält eine zusätzliche Stufe Mächtige Liturgie.}
}


\newglossaryentry{verseuchungerspüren_Vorteil}
{
    name={Verseuchung erspüren},
    description={\textbf{Kosten}: 20 \textbf{Voraussetzungen}: Attribut MU 4, Vorteil Geweiht I \textbf{Nachkauf}: Häufig\newline Bei Verwendung spürst du die Anwesenheit und die grobe Stärke - nicht aber den genauen Ort - von dämonischen Präsenzen oder Verseuchung in unmittelbarer Nähe. Kostet 1 KaP/Minute.}
}


\newglossaryentry{gesegneteWaffe_Vorteil}
{
    name={Gesegnete Waffe},
    description={\textbf{Kosten}: 40 \textbf{Voraussetzungen}: Attribut MU 6, Vorteil Geweiht I \textbf{Nachkauf}: Häufig\newline Von dir genutzte Waffen gelten immer als geweiht. Bereits geweihte Waffen verursachen +1W6 TP gegen unheilige Wesen (S. 98).}
}


\newglossaryentry{stärkedesGlaubens_Vorteil}
{
    name={Stärke des Glaubens},
    description={\textbf{Kosten}: 60 \textbf{Voraussetzungen}: Attribut MU 8, Vorteil Geweiht I \textbf{Nachkauf}: Häufig\newline Gegen Angriffe von unheiligen Wesenheiten steigt deine WS um 2.}
}


\newglossaryentry{streiterderSchöpfung_Vorteil}
{
    name={Streiter der Schöpfung},
    description={\textbf{Kosten}: 80 \textbf{Voraussetzungen}: Attribut MU 10, Vorteil Geweiht I \textbf{Nachkauf}: Häufig\newline Alle Proben im Kampf gegen unheilige Wesen (S. 98) sind um +2 erleichtert, Liturgien sogar um +4.}
}


\newglossaryentry{traditionderAvesgeweihtenI_Vorteil}
{
    name={Tradition der Avesgeweihten I},
    description={\textbf{Kosten}: 20 \textbf{Voraussetzungen}: Attribut MU 4, Vorteil Geweiht I, Kein Vorteil Tradition der Praiosgeweihten I, Kein Vorteil Tradition der Borongeweihten I, Kein Vorteil Tradition der Phexgeweihten I, Kein Vorteil Tradition der Efferdgeweihten I, Kein Vorteil Tradition der Hesindegeweihten I, Kein Vorteil Tradition der Ingerimmgeweihten I, Kein Vorteil Tradition der Angroschgeweihten I, Kein Vorteil Tradition der Perainegeweihten I, Kein Vorteil Tradition der Firungeweihten I, Kein Vorteil Tradition der Swafnirgeweihten I, Kein Vorteil Tradition der Tsageweihten I, Kein Vorteil Tradition der Traviageweihten I, Kein Vorteil Tradition der Ifirngeweihten I, Kein Vorteil Tradition der Nandusgeweihten I, Kein Vorteil Tradition der Rondrageweihten I, Kein Vorteil Tradition der Rahjageweihten I, Kein Vorteil Tradition der Korgeweihten I \textbf{Nachkauf}: Häufig\newline Bedingungen: Sichtkontakt, Geste, Bewegung (min. 1 Schritt pro Aktion).\newline Du kannst Liturgien in der Tradition der Avesgeweihten erlernen und benutzen. Solche Liturgien findest du unter den allgemeinen Liturgien, unter Fröhlicher Wanderer und Stiller Wanderer (ab S. 201).\newline Liturgien sind um 2-8 Punkte erschwert, wenn ihr Einsatz gegen die Gebote der Kirche verstößt.}
}


\newglossaryentry{traditionderAvesgeweihtenII_Vorteil}
{
    name={Tradition der Avesgeweihten II},
    description={\textbf{Kosten}: 40 \textbf{Voraussetzungen}: Attribut MU 6, Vorteil Tradition der Avesgeweihten I \textbf{Nachkauf}: Häufig\newline Avesgeweihte sind ständig in Bewegung. Wenn du eine Liturgie vorbereitest, kannst du zusätzlich zur Aktion Konzentration die Aktion Bewegung nutzen.}
}


\newglossaryentry{traditionderAvesgeweihtenIII_Vorteil}
{
    name={Tradition der Avesgeweihten III},
    description={\textbf{Kosten}: 60 \textbf{Voraussetzungen}: Attribut MU 8, Vorteil Tradition der Avesgeweihten II \textbf{Nachkauf}: Häufig\newline Erlaubt die spontane Modifikation Opferung (Liturgie +4): Ein rituelles Opfer erleichtert die Liturgie um +4. Der Opfergegenstand wird dabei verbraucht, zerstört oder verschwindet. Avesgeweihte opfern die Feder eines Zugvogels.}
}


\newglossaryentry{traditionderAvesgeweihtenIV_Vorteil}
{
    name={Tradition der Avesgeweihten IV},
    description={\textbf{Kosten}: 80 \textbf{Voraussetzungen}: Attribut MU 10, Vorteil Tradition der Avesgeweihten III \textbf{Nachkauf}: Häufig\newline Bedingung: 2 weitere Attribute auf insgesamt 16.\newline 8 Punkte können zur Verbesserung der Tradition verwendet werden. Sephrasto: Trage die Verbesserungen in das Kommentarfeld ein.}
}


\newglossaryentry{traditionderBorongeweihtenI_Vorteil}
{
    name={Tradition der Borongeweihten I},
    description={\textbf{Kosten}: 20 \textbf{Voraussetzungen}: Attribut KL 4, Vorteil Geweiht I, Kein Vorteil Tradition der Praiosgeweihten I, Kein Vorteil Tradition der Rondrageweihten I, Kein Vorteil Tradition der Phexgeweihten I, Kein Vorteil Tradition der Efferdgeweihten I, Kein Vorteil Tradition der Hesindegeweihten I, Kein Vorteil Tradition der Ingerimmgeweihten I, Kein Vorteil Tradition der Angroschgeweihten I, Kein Vorteil Tradition der Perainegeweihten I, Kein Vorteil Tradition der Firungeweihten I, Kein Vorteil Tradition der Rahjageweihten I, Kein Vorteil Tradition der Tsageweihten I, Kein Vorteil Tradition der Traviageweihten I, Kein Vorteil Tradition der Swafnirgeweihten I, Kein Vorteil Tradition der Ifirngeweihten I, Kein Vorteil Tradition der Nandusgeweihten I, Kein Vorteil Tradition der Korgeweihten I, Kein Vorteil Tradition der Avesgeweihten I \textbf{Nachkauf}: Häufig\newline Bedingungen: Sichtkontakt, Geste, innere Ruhe (unbeeinflusst von aufwühlenden Emotionen wie Leidenschaft, Zorn oder Panik).\newline Du kannst Liturgien in der Tradition der Borongeweihten erlernen und benutzen. Solche Liturgien findest du unter den allgemeinen Liturgien, unter Schlaf, Tod und Vergessen (ab S. 179).\newline Liturgien sind um 2–8 Punkte erschwert, wenn ihr Einsatz gegen die Prinzipien des Geweihten verstößt.}
}


\newglossaryentry{traditionderBorongeweihtenII_Vorteil}
{
    name={Tradition der Borongeweihten II},
    description={\textbf{Kosten}: 40 \textbf{Voraussetzungen}: Attribut KL 6, Vorteil Tradition der Borongeweihten I \textbf{Nachkauf}: Häufig\newline Borongeweihte würdigen die Ruhe und das Schweigen. In der Stille sind Liturgien um 2–4 Punkte erleichtert.}
}


\newglossaryentry{traditionderBorongeweihtenIII_Vorteil}
{
    name={Tradition der Borongeweihten III},
    description={\textbf{Kosten}: 60 \textbf{Voraussetzungen}: Attribut KL 8, Vorteil Tradition der Borongeweihten II \textbf{Nachkauf}: Häufig\newline Erlaubt die spontane Modifikation Zeremonie (Liturgie +X): Du kannst die Vorbereitungszeit freiwillig um 1 Minute/Stunde/Tag/Woche/Monat/Jahr erhöhen, wodurch die Liturgie um +4/6/8/10/12/14 erleichtert ist. Die Vorbereitungszeit muss dadurch mindestens verdoppelt werden.}
}


\newglossaryentry{traditionderBorongeweihtenIV_Vorteil}
{
    name={Tradition der Borongeweihten IV},
    description={\textbf{Kosten}: 80 \textbf{Voraussetzungen}: Attribut KL 10, Vorteil Tradition der Borongeweihten III \textbf{Nachkauf}: Häufig\newline Bedingung: 2 weitere Attribute auf insgesamt 16.\newline 8 Punkte können zur Verbesserung der Tradition verwendet werden. Sephrasto: Trage die Verbesserungen in das Kommentarfeld ein.}
}


\newglossaryentry{traditionderEfferdgeweihtenI_Vorteil}
{
    name={Tradition der Efferdgeweihten I},
    description={\textbf{Kosten}: 20 \textbf{Voraussetzungen}: Attribut IN 4, Vorteil Geweiht I, Kein Vorteil Tradition der Praiosgeweihten I, Kein Vorteil Tradition der Rondrageweihten I, Kein Vorteil Tradition der Borongeweihten I, Kein Vorteil Tradition der Phexgeweihten I, Kein Vorteil Tradition der Hesindegeweihten I, Kein Vorteil Tradition der Ingerimmgeweihten I, Kein Vorteil Tradition der Angroschgeweihten I, Kein Vorteil Tradition der Perainegeweihten I, Kein Vorteil Tradition der Firungeweihten I, Kein Vorteil Tradition der Rahjageweihten I, Kein Vorteil Tradition der Tsageweihten I, Kein Vorteil Tradition der Traviageweihten I, Kein Vorteil Tradition der Swafnirgeweihten I, Kein Vorteil Tradition der Ifirngeweihten I, Kein Vorteil Tradition der Nandusgeweihten I, Kein Vorteil Tradition der Korgeweihten I, Kein Vorteil Tradition der Avesgeweihten I \textbf{Nachkauf}: Häufig\newline Bedingungen: Sichtkontakt, Geste, gesprochenes oder gesungenes Gebet.\newline Du kannst Liturgien in der Tradition der Efferdgeweihten erlernen und benutzen. Solche Liturgien findest du unter den allgemeinen Liturgien, unter Flüsse und Quellen, Seefahrt sowie Wind und Wogen (ab S. 181).\newline Liturgien sind um 2-8 Punkte erschwert, wenn ihr Einsatz gegen die Prinzipien des Geweihten verstößt.}
}


\newglossaryentry{traditionderEfferdgeweihtenII_Vorteil}
{
    name={Tradition der Efferdgeweihten II},
    description={\textbf{Kosten}: 40 \textbf{Voraussetzungen}: Attribut IN 6, Vorteil Tradition der Efferdgeweihten I \textbf{Nachkauf}: Häufig\newline Efferdgeweihte sind so wechselhaft wie das Meer. Im Zustand besonders starker Emotionen sind Liturgien um 2–4 Punkte erleichtert.}
}


\newglossaryentry{traditionderEfferdgeweihtenIII_Vorteil}
{
    name={Tradition der Efferdgeweihten III},
    description={\textbf{Kosten}: 60 \textbf{Voraussetzungen}: Attribut IN 8, Vorteil Tradition der Efferdgeweihten II \textbf{Nachkauf}: Häufig\newline Erlaubt die spontane Modifikation Opferung (Liturgie +4): Ein rituelles Opfer erleichtert die Liturgie um +4. Der Opfergegenstand wird dabei verbraucht, zerstört oder verschwindet. Efferdgeweihte opfern Perlen oder Gwen-­Petryl-­Steine.}
}


\newglossaryentry{traditionderEfferdgeweihtenIV_Vorteil}
{
    name={Tradition der Efferdgeweihten IV},
    description={\textbf{Kosten}: 80 \textbf{Voraussetzungen}: Attribut IN 10, Vorteil Tradition der Efferdgeweihten III \textbf{Nachkauf}: Häufig\newline Bedingung: 2 weitere Attribute auf insgesamt 16.\newline 8 Punkte können zur Verbesserung der Tradition verwendet werden. Sephrasto: Trage die Verbesserungen in das Kommentarfeld ein.}
}


\newglossaryentry{traditionderFirungeweihtenI_Vorteil}
{
    name={Tradition der Firungeweihten I},
    description={\textbf{Kosten}: 20 \textbf{Voraussetzungen}: Attribut MU 4, Vorteil Geweiht I, Kein Vorteil Tradition der Praiosgeweihten I, Kein Vorteil Tradition der Rondrageweihten I, Kein Vorteil Tradition der Borongeweihten I, Kein Vorteil Tradition der Phexgeweihten I, Kein Vorteil Tradition der Efferdgeweihten I, Kein Vorteil Tradition der Hesindegeweihten I, Kein Vorteil Tradition der Ingerimmgeweihten I, Kein Vorteil Tradition der Angroschgeweihten I, Kein Vorteil Tradition der Perainegeweihten I, Kein Vorteil Tradition der Rahjageweihten I, Kein Vorteil Tradition der Tsageweihten I, Kein Vorteil Tradition der Traviageweihten I, Kein Vorteil Tradition der Swafnirgeweihten I, Kein Vorteil Tradition der Ifirngeweihten I, Kein Vorteil Tradition der Nandusgeweihten I, Kein Vorteil Tradition der Korgeweihten I, Kein Vorteil Tradition der Avesgeweihten I \textbf{Nachkauf}: Häufig\newline Bedingungen: Sichtkontakt, Geste, innere Stärke (unbeeinflusst von Charakterschwächen wie Angst, Jähzorn oder Gier).\newline Du kannst Liturgien in der Tradition der Firungeweihten erlernen und benutzen. Solche Liturgien findest du unter den allgemeinen Liturgien, unter Jagd, Wildnis und Winter (ab S. 184).\newline Liturgien sind um 2-8 Punkte erschwert, wenn ihr Einsatz gegen die Prinzipien des Geweihten verstößt.}
}


\newglossaryentry{traditionderFirungeweihtenII_Vorteil}
{
    name={Tradition der Firungeweihten II},
    description={\textbf{Kosten}: 40 \textbf{Voraussetzungen}: Attribut MU 6, Vorteil Tradition der Firungeweihten I \textbf{Nachkauf}: Häufig\newline Firungeweihte sind Einzelgänger. Liturgien, von denen nur sie profitieren, sind um +2 erleichtert.}
}


\newglossaryentry{traditionderFirungeweihtenIII_Vorteil}
{
    name={Tradition der Firungeweihten III},
    description={\textbf{Kosten}: 60 \textbf{Voraussetzungen}: Attribut MU 8, Vorteil Tradition der Firungeweihten II \textbf{Nachkauf}: Häufig\newline Erlaubt die spontane Modifikation Opferung (Liturgie +4): Ein rituelles Opfer erleichtert die Liturgie um +4. Der Opfergegenstand wird dabei verbraucht, zerstört oder verschwindet. Firungeweihte opfern besondere Beute oder Trophäen.}
}


\newglossaryentry{traditionderFirungeweihtenIV_Vorteil}
{
    name={Tradition der Firungeweihten IV},
    description={\textbf{Kosten}: 80 \textbf{Voraussetzungen}: Attribut MU 10, Vorteil Tradition der Firungeweihten III \textbf{Nachkauf}: Häufig\newline Bedingung: 2 weitere Attribute auf insgesamt 16.\newline 8 Punkte können zur Verbesserung der Tradition verwendet werden. Sephrasto: Trage die Verbesserungen in das Kommentarfeld ein.}
}


\newglossaryentry{traditionderHesindegeweihtenI_Vorteil}
{
    name={Tradition der Hesindegeweihten I},
    description={\textbf{Kosten}: 20 \textbf{Voraussetzungen}: Attribut KL 4, Vorteil Geweiht I, Kein Vorteil Tradition der Praiosgeweihten I, Kein Vorteil Tradition der Rondrageweihten I, Kein Vorteil Tradition der Borongeweihten I, Kein Vorteil Tradition der Phexgeweihten I, Kein Vorteil Tradition der Efferdgeweihten I, Kein Vorteil Tradition der Ingerimmgeweihten I, Kein Vorteil Tradition der Angroschgeweihten I, Kein Vorteil Tradition der Perainegeweihten I, Kein Vorteil Tradition der Firungeweihten I, Kein Vorteil Tradition der Rahjageweihten I, Kein Vorteil Tradition der Tsageweihten I, Kein Vorteil Tradition der Traviageweihten I, Kein Vorteil Tradition der Swafnirgeweihten I, Kein Vorteil Tradition der Ifirngeweihten I, Kein Vorteil Tradition der Nandusgeweihten I, Kein Vorteil Tradition der Korgeweihten I, Kein Vorteil Tradition der Avesgeweihten I \textbf{Nachkauf}: Häufig\newline Bedingungen: Sichtkontakt, Geste, gesprochenes oder gesungenes Gebet.\newline Du kannst Liturgien in der Tradition der Hesindegeweihten erlernen und benutzen. Solche Liturgien findest du unter den allgemeinen Liturgien, unter Magie und Wissen (ab S. 187).\newline Liturgien sind um 2-8 Punkte erschwert, wenn ihr Einsatz gegen die Prinzipien des Geweihten verstößt.}
}


\newglossaryentry{traditionderHesindegeweihtenII_Vorteil}
{
    name={Tradition der Hesindegeweihten II},
    description={\textbf{Kosten}: 40 \textbf{Voraussetzungen}: Attribut KL 6, Vorteil Tradition der Hesindegeweihten I \textbf{Nachkauf}: Häufig\newline Hesindegeweihte können auch mit mehrfach veränderten Liturgien gut umgehen. Wird eine Liturgie mit mindestens zwei unterschiedlichen Basismodifikationen gewirkt, ist sie zusätzlich um +2 erleichtert.}
}


\newglossaryentry{traditionderHesindegeweihtenIII_Vorteil}
{
    name={Tradition der Hesindegeweihten III},
    description={\textbf{Kosten}: 60 \textbf{Voraussetzungen}: Attribut KL 8, Vorteil Tradition der Hesindegeweihten II \textbf{Nachkauf}: Häufig\newline Erlaubt die spontane Modifikation Zeremonie (Liturgie +X): Du kannst die Vorbereitungszeit freiwillig um 1 Minute/Stunde/Tag/Woche/Monat/Jahr erhöhen, wodurch die Liturgie um +4/6/8/10/12/14 erleichtert ist. Die Vorbereitungszeit muss dadurch mindestens verdoppelt werden.}
}


\newglossaryentry{traditionderHesindegeweihtenIV_Vorteil}
{
    name={Tradition der Hesindegeweihten IV},
    description={\textbf{Kosten}: 80 \textbf{Voraussetzungen}: Attribut KL 10, Vorteil Tradition der Hesindegeweihten III \textbf{Nachkauf}: Häufig\newline Bedingung: 2 weitere Attribute auf insgesamt 16.\newline 8 Punkte können zur Verbesserung der Tradition verwendet werden. Sephrasto: Trage die Verbesserungen in das Kommentarfeld ein.}
}


\newglossaryentry{traditionderIfirngeweihtenI_Vorteil}
{
    name={Tradition der Ifirngeweihten I},
    description={\textbf{Kosten}: 20 \textbf{Voraussetzungen}: Attribut IN 4, Vorteil Geweiht I, Kein Vorteil Tradition der Praiosgeweihten I, Kein Vorteil Tradition der Rondrageweihten I, Kein Vorteil Tradition der Borongeweihten I, Kein Vorteil Tradition der Phexgeweihten I, Kein Vorteil Tradition der Efferdgeweihten I, Kein Vorteil Tradition der Hesindegeweihten I, Kein Vorteil Tradition der Ingerimmgeweihten I, Kein Vorteil Tradition der Angroschgeweihten I, Kein Vorteil Tradition der Perainegeweihten I, Kein Vorteil Tradition der Firungeweihten I, Kein Vorteil Tradition der Rahjageweihten I, Kein Vorteil Tradition der Tsageweihten I, Kein Vorteil Tradition der Traviageweihten I, Kein Vorteil Tradition der Swafnirgeweihten I, Kein Vorteil Tradition der Nandusgeweihten I, Kein Vorteil Tradition der Korgeweihten I, Kein Vorteil Tradition der Avesgeweihten I \textbf{Nachkauf}: Häufig\newline Bedingungen: Sichtkontakt, Geste, gesprochenes oder gesungenes Gebet.\newline Du kannst Liturgien in der Tradition der Ifirngeweihten erlernen und benutzen. Solche Liturgien findest du unter den allgemeinen Liturgien, unter Jagd und Wildnis (ab S. 184).\newline Liturgien sind um 2-8 Punkte erschwert, wenn ihr Einsatz gegen die Prinzipien des Geweihten verstößt.}
}


\newglossaryentry{traditionderIfirngeweihtenII_Vorteil}
{
    name={Tradition der Ifirngeweihten II},
    description={\textbf{Kosten}: 40 \textbf{Voraussetzungen}: Attribut IN 6, Vorteil Tradition der Ifirngeweihten I \textbf{Nachkauf}: Häufig\newline Ifirngeweihte durchstreifen die Lande und helfen denen, die in Not geraten sind. Liturgien, von denen nur andere (aber sie selbst nicht) profitieren, sind um +2 erleichtert.}
}


\newglossaryentry{traditionderIfirngeweihtenIII_Vorteil}
{
    name={Tradition der Ifirngeweihten III},
    description={\textbf{Kosten}: 60 \textbf{Voraussetzungen}: Attribut IN 8, Vorteil Tradition der Ifirngeweihten II \textbf{Nachkauf}: Häufig\newline Erlaubt die spontane Modifikation Zeremonie (Liturgie +X): Du kannst die Vorbereitungszeit freiwillig um 1 Minute/Stunde/Tag/Woche/Monat/Jahr erhöhen, wodurch die Liturgie um +4/6/8/10/12/14 erleichtert ist. Die Vorbereitungszeit muss dadurch mindestens verdoppelt werden.}
}


\newglossaryentry{traditionderIfirngeweihtenIV_Vorteil}
{
    name={Tradition der Ifirngeweihten IV},
    description={\textbf{Kosten}: 80 \textbf{Voraussetzungen}: Attribut IN 10, Vorteil Tradition der Ifirngeweihten III \textbf{Nachkauf}: Häufig\newline Bedingung: 2 weitere Attribute auf insgesamt 16.\newline 8 Punkte können zur Verbesserung der Tradition verwendet werden. Sephrasto: Trage die Verbesserungen in das Kommentarfeld ein.}
}


\newglossaryentry{traditionderIngerimmgeweihtenI_Vorteil}
{
    name={Tradition der Ingerimmgeweihten I},
    description={\textbf{Kosten}: 20 \textbf{Voraussetzungen}: Attribut FF 4, Vorteil Geweiht I, Kein Vorteil Tradition der Praiosgeweihten I, Kein Vorteil Tradition der Rondrageweihten I, Kein Vorteil Tradition der Borongeweihten I, Kein Vorteil Tradition der Phexgeweihten I, Kein Vorteil Tradition der Efferdgeweihten I, Kein Vorteil Tradition der Hesindegeweihten I, Kein Vorteil Tradition der Angroschgeweihten I, Kein Vorteil Tradition der Perainegeweihten I, Kein Vorteil Tradition der Firungeweihten I, Kein Vorteil Tradition der Rahjageweihten I, Kein Vorteil Tradition der Tsageweihten I, Kein Vorteil Tradition der Traviageweihten I, Kein Vorteil Tradition der Swafnirgeweihten I, Kein Vorteil Tradition der Ifirngeweihten I, Kein Vorteil Tradition der Nandusgeweihten I, Kein Vorteil Tradition der Korgeweihten I, Kein Vorteil Tradition der Avesgeweihten I \textbf{Nachkauf}: Häufig\newline Bedingungen: Sichtkontakt, Geste, gesprochenes oder gesungenes Gebet.\newline Du kannst Liturgien in der Tradition der Ingerimmgeweihten erlernen und benutzen. Solche Liturgien findest du unter den allgemeinen Liturgien, unter Heiliges Erz, Hl. Feuer und Hl. Handwerk (ab S. 188).\newline Liturgien sind um 2-8 Punkte erschwert, wenn ihr Einsatz gegen die Prinzipien des Geweihten verstößt.}
}


\newglossaryentry{traditionderIngerimmgeweihtenII_Vorteil}
{
    name={Tradition der Ingerimmgeweihten II},
    description={\textbf{Kosten}: 40 \textbf{Voraussetzungen}: Attribut FF 6, Vorteil Tradition der Ingerimmgeweihten I \textbf{Nachkauf}: Häufig\newline Ingerimmgeweihte dienen dem Herren des Feuers und des Erzes. In Gebäuden sind alle Liturgien um 2–4 Punkte erleichtert.}
}


\newglossaryentry{traditionderIngerimmgeweihtenIII_Vorteil}
{
    name={Tradition der Ingerimmgeweihten III},
    description={\textbf{Kosten}: 60 \textbf{Voraussetzungen}: Attribut FF 8, Vorteil Tradition der Ingerimmgeweihten II \textbf{Nachkauf}: Häufig\newline Erlaubt die spontane Modifikation Opferung (Liturgie +4): Ein rituelles Opfer erleichtert die Liturgie um +4. Der Opfergegenstand wird dabei verbraucht, zerstört oder verschwindet. Ingerimmgeweihte opfern Handwerksgegenstände.}
}


\newglossaryentry{traditionderIngerimmgeweihtenIV_Vorteil}
{
    name={Tradition der Ingerimmgeweihten IV},
    description={\textbf{Kosten}: 80 \textbf{Voraussetzungen}: Attribut FF 10, Vorteil Tradition der Ingerimmgeweihten III \textbf{Nachkauf}: Häufig\newline Bedingung: 2 weitere Attribute auf insgesamt 16.\newline 8 Punkte können zur Verbesserung der Tradition verwendet werden. Sephrasto: Trage die Verbesserungen in das Kommentarfeld ein.}
}


\newglossaryentry{traditionderAngroschgeweihtenI_Vorteil}
{
    name={Tradition der Angroschgeweihten I},
    description={\textbf{Kosten}: 20 \textbf{Voraussetzungen}: Attribut FF 4, Vorteil Geweiht I, Kein Vorteil Tradition der Praiosgeweihten I, Kein Vorteil Tradition der Rondrageweihten I, Kein Vorteil Tradition der Borongeweihten I, Kein Vorteil Tradition der Phexgeweihten I, Kein Vorteil Tradition der Efferdgeweihten I, Kein Vorteil Tradition der Hesindegeweihten I, Kein Vorteil Tradition der Ingerimmgeweihten I, Kein Vorteil Tradition der Perainegeweihten I, Kein Vorteil Tradition der Firungeweihten I, Kein Vorteil Tradition der Rahjageweihten I, Kein Vorteil Tradition der Tsageweihten I, Kein Vorteil Tradition der Traviageweihten I, Kein Vorteil Tradition der Swafnirgeweihten I, Kein Vorteil Tradition der Ifirngeweihten I, Kein Vorteil Tradition der Nandusgeweihten I, Kein Vorteil Tradition der Korgeweihten I, Kein Vorteil Tradition der Avesgeweihten I \textbf{Nachkauf}: Häufig\newline Bedingungen: Sichtkontakt, Geste, gesprochenes oder gesungenes Gebet.\newline Du kannst Liturgien in der Tradition der Angroschgeweihten erlernen und benutzen. Solche Liturgien findest du unter den allgemeinen Liturgien, unter Heiliges Erz, Hl. Feuer und Hl. Handwerk (ab S. 188).\newline Liturgien sind um 2-8 Punkte erschwert, wenn ihr Einsatz gegen die Prinzipien des Geweihten verstößt.}
}


\newglossaryentry{traditionderAngroschgeweihtenII_Vorteil}
{
    name={Tradition der Angroschgeweihten II},
    description={\textbf{Kosten}: 40 \textbf{Voraussetzungen}: Attribut FF 6, Vorteil Tradition der Angroschgeweihten I \textbf{Nachkauf}: Häufig\newline Angroschgeweihte dienen dem Herren des Feuers und des Erzes. Unterirdisch sind alle Liturgien um 2–4 Punkte erleichtert.}
}


\newglossaryentry{traditionderAngroschgeweihtenIII_Vorteil}
{
    name={Tradition der Angroschgeweihten III},
    description={\textbf{Kosten}: 60 \textbf{Voraussetzungen}: Attribut FF 8, Vorteil Tradition der Angroschgeweihten II \textbf{Nachkauf}: Häufig\newline Erlaubt die spontane Modifikation Opferung (Liturgie +4): Ein rituelles Opfer erleichtert die Liturgie um +4. Der Opfergegenstand wird dabei verbraucht, zerstört oder verschwindet. Angroschgeweihte opfern Handwerksgegenstände.}
}


\newglossaryentry{traditionderAngroschgeweihtenIV_Vorteil}
{
    name={Tradition der Angroschgeweihten IV},
    description={\textbf{Kosten}: 80 \textbf{Voraussetzungen}: Attribut FF 10, Vorteil Tradition der Angroschgeweihten III \textbf{Nachkauf}: Häufig\newline Bedingung: 2 weitere Attribute auf insgesamt 16.\newline 8 Punkte können zur Verbesserung der Tradition verwendet werden. Sephrasto: Trage die Verbesserungen in das Kommentarfeld ein.}
}


\newglossaryentry{traditionderKorgeweihtenI_Vorteil}
{
    name={Tradition der Korgeweihten I},
    description={\textbf{Kosten}: 20 \textbf{Voraussetzungen}: Attribut MU 4, Vorteil Geweiht I, Kein Vorteil Tradition der Praiosgeweihten I, Kein Vorteil Tradition der Borongeweihten I, Kein Vorteil Tradition der Phexgeweihten I, Kein Vorteil Tradition der Efferdgeweihten I, Kein Vorteil Tradition der Hesindegeweihten I, Kein Vorteil Tradition der Ingerimmgeweihten I, Kein Vorteil Tradition der Angroschgeweihten I, Kein Vorteil Tradition der Perainegeweihten I, Kein Vorteil Tradition der Firungeweihten I, Kein Vorteil Tradition der Swafnirgeweihten I, Kein Vorteil Tradition der Tsageweihten I, Kein Vorteil Tradition der Traviageweihten I, Kein Vorteil Tradition der Ifirngeweihten I, Kein Vorteil Tradition der Nandusgeweihten I, Kein Vorteil Tradition der Rondrageweihten I, Kein Vorteil Tradition der Rahjageweihten I, Kein Vorteil Tradition der Avesgeweihten I \textbf{Nachkauf}: Häufig\newline Bedingungen: Sichtkontakt, Geste, gesprochenes oder gesungenes Gebet.\newline Du kannst Liturgien in der Tradition der Korgeweihten erlernen und benutzen. Solche Liturgien findest du unter den allgemeinen Liturgien, unter Guter Kampf und Gutes Gold (ab S. 202).\newline Liturgien sind um 2-8 Punkte erschwert, wenn ihr Einsatz gegen die Gebote der Kirche verstößt.}
}


\newglossaryentry{traditionderKorgeweihtenII_Vorteil}
{
    name={Tradition der Korgeweihten II},
    description={\textbf{Kosten}: 40 \textbf{Voraussetzungen}: Attribut MU 6, Vorteil Tradition der Korgeweihten I \textbf{Nachkauf}: Häufig\newline Korgeweihte werden durch erlittene Wunden eher noch gefährlicher. Der Wundmalus schränkt dich beim Wirken von Liturgien nicht ein. Kannst du den Wundmalus ohnehin ignorieren, sind Liturgien um den halben Wundmalus erleichtert.}
}


\newglossaryentry{traditionderKorgeweihtenIII_Vorteil}
{
    name={Tradition der Korgeweihten III},
    description={\textbf{Kosten}: 60 \textbf{Voraussetzungen}: Attribut MU 8, Vorteil Tradition der Korgeweihten II \textbf{Nachkauf}: Häufig\newline Erlaubt die spontane Modifikation Opferung (Liturgie +4): Ein rituelles Opfer erleichtert die Liturgie um +4. Der Opfergegenstand wird dabei verbraucht, zerstört oder verschwindet. Korgeweihte opfern ihr Blut (verursacht 1 Wunde).}
}


\newglossaryentry{traditionderKorgeweihtenIV_Vorteil}
{
    name={Tradition der Korgeweihten IV},
    description={\textbf{Kosten}: 80 \textbf{Voraussetzungen}: Attribut MU 10, Vorteil Tradition der Korgeweihten III \textbf{Nachkauf}: Häufig\newline Bedingung: 2 weitere Attribute auf insgesamt 16.\newline 8 Punkte können zur Verbesserung der Tradition verwendet werden. Sephrasto: Trage die Verbesserungen in das Kommentarfeld ein.}
}


\newglossaryentry{traditionderNandusgeweihtenI_Vorteil}
{
    name={Tradition der Nandusgeweihten I},
    description={\textbf{Kosten}: 20 \textbf{Voraussetzungen}: Attribut KL 4, Vorteil Geweiht I, Kein Vorteil Tradition der Praiosgeweihten I, Kein Vorteil Tradition der Rondrageweihten I, Kein Vorteil Tradition der Borongeweihten I, Kein Vorteil Tradition der Phexgeweihten I, Kein Vorteil Tradition der Efferdgeweihten I, Kein Vorteil Tradition der Hesindegeweihten I, Kein Vorteil Tradition der Ingerimmgeweihten I, Kein Vorteil Tradition der Angroschgeweihten I, Kein Vorteil Tradition der Perainegeweihten I, Kein Vorteil Tradition der Firungeweihten I, Kein Vorteil Tradition der Rahjageweihten I, Kein Vorteil Tradition der Tsageweihten I, Kein Vorteil Tradition der Traviageweihten I, Kein Vorteil Tradition der Swafnirgeweihten I, Kein Vorteil Tradition der Ifirngeweihten I, Kein Vorteil Tradition der Korgeweihten I, Kein Vorteil Tradition der Avesgeweihten I \textbf{Nachkauf}: Häufig\newline Bedingungen: Sichtkontakt, Geste, gesprochenes oder gesungenes Gebet.\newline Du kannst Liturgien in der Tradition der Nandusgeweihten erlernen und benutzen. Solche Liturgien findest du unter den allgemeinen Liturgien und unter Wissen (ab S. 187).\newline Liturgien sind um 2-8 Punkte erschwert, wenn ihr Einsatz gegen die Prinzipien des Geweihten verstößt.}
}


\newglossaryentry{traditionderNandusgeweihtenII_Vorteil}
{
    name={Tradition der Nandusgeweihten II},
    description={\textbf{Kosten}: 40 \textbf{Voraussetzungen}: Attribut KL 6, Vorteil Tradition der Nandusgeweihten I \textbf{Nachkauf}: Häufig\newline Nandugeweihte wissen, dass Gesten und Worte nur menschliches Beiwerk einer Liturgie sind. Wird eine Liturgie mit der Modifikation Liturgische Technik ignorieren gewirkt, ist sie zusätzlich um +2 erleichtert.}
}


\newglossaryentry{traditionderNandusgeweihtenIII_Vorteil}
{
    name={Tradition der Nandusgeweihten III},
    description={\textbf{Kosten}: 60 \textbf{Voraussetzungen}: Attribut KL 8, Vorteil Tradition der Nandusgeweihten II \textbf{Nachkauf}: Häufig\newline Erlaubt die spontane Modifikation Opferung (Liturgie +4): Ein rituelles Opfer erleichtert die Liturgie um +4. Der Opfergegenstand wird dabei verbraucht, zerstört oder verschwindet. Nandusgeweihte opfern falsches oder veraltetes Wissen.}
}


\newglossaryentry{traditionderNandusgeweihtenIV_Vorteil}
{
    name={Tradition der Nandusgeweihten IV},
    description={\textbf{Kosten}: 80 \textbf{Voraussetzungen}: Attribut KL 10, Vorteil Tradition der Nandusgeweihten III \textbf{Nachkauf}: Häufig\newline Bedingung: 2 weitere Attribute auf insgesamt 16.\newline 8 Punkte können zur Verbesserung der Tradition verwendet werden. Sephrasto: Trage die Verbesserungen in das Kommentarfeld ein.}
}


\newglossaryentry{traditionderPerainegeweihtenI_Vorteil}
{
    name={Tradition der Perainegeweihten I},
    description={\textbf{Kosten}: 20 \textbf{Voraussetzungen}: Attribut CH 4, Vorteil Geweiht I, Kein Vorteil Tradition der Praiosgeweihten I, Kein Vorteil Tradition der Rondrageweihten I, Kein Vorteil Tradition der Borongeweihten I, Kein Vorteil Tradition der Phexgeweihten I, Kein Vorteil Tradition der Efferdgeweihten I, Kein Vorteil Tradition der Hesindegeweihten I, Kein Vorteil Tradition der Ingerimmgeweihten I, Kein Vorteil Tradition der Angroschgeweihten I, Kein Vorteil Tradition der Firungeweihten I, Kein Vorteil Tradition der Rahjageweihten I, Kein Vorteil Tradition der Tsageweihten I, Kein Vorteil Tradition der Traviageweihten I, Kein Vorteil Tradition der Swafnirgeweihten I, Kein Vorteil Tradition der Ifirngeweihten I, Kein Vorteil Tradition der Nandusgeweihten I, Kein Vorteil Tradition der Korgeweihten I, Kein Vorteil Tradition der Avesgeweihten I \textbf{Nachkauf}: Häufig\newline Bedingungen: Sichtkontakt, Geste, gesprochenes oder gesungenes Gebet.\newline Du kannst Liturgien in der Tradition der Perainegeweihten erlernen und benutzen. Solche Liturgien findest du unter den allgemeinen Liturgien, unter Heilung und Wachstum (ab S. 191).\newline Liturgien sind um 2-8 Punkte erschwert, wenn ihr Einsatz gegen die Prinzipien des Geweihten verstößt.}
}


\newglossaryentry{traditionderPerainegeweihtenII_Vorteil}
{
    name={Tradition der Perainegeweihten II},
    description={\textbf{Kosten}: 40 \textbf{Voraussetzungen}: Attribut CH 6, Vorteil Tradition der Perainegeweihten I \textbf{Nachkauf}: Häufig\newline Perainegeweihte sind oft das Herz der Gemeinschaft. Wird eine Liturgie mit der Modifikation Mehrere Ziele gesprochen, ist sie zusätzlich um +2 erleichtert.}
}


\newglossaryentry{traditionderPerainegeweihtenIII_Vorteil}
{
    name={Tradition der Perainegeweihten III},
    description={\textbf{Kosten}: 60 \textbf{Voraussetzungen}: Attribut CH 8, Vorteil Tradition der Perainegeweihten II \textbf{Nachkauf}: Häufig\newline Erlaubt die spontane Modifikation Zeremonie (Liturgie +X): Du kannst die Vorbereitungszeit freiwillig um 1 Minute/Stunde/Tag/Woche/Monat/Jahr erhöhen, wodurch die Liturgie um +4/6/8/10/12/14 erleichtert ist. Die Vorbereitungszeit muss dadurch mindestens verdoppelt werden.}
}


\newglossaryentry{traditionderPerainegeweihtenIV_Vorteil}
{
    name={Tradition der Perainegeweihten IV},
    description={\textbf{Kosten}: 80 \textbf{Voraussetzungen}: Attribut CH 10, Vorteil Tradition der Perainegeweihten III \textbf{Nachkauf}: Häufig\newline Bedingung: 2 weitere Attribute auf insgesamt 16.\newline 8 Punkte können zur Verbesserung der Tradition verwendet werden. Sephrasto: Trage die Verbesserungen in das Kommentarfeld ein.}
}


\newglossaryentry{traditionderPhexgeweihtenI_Vorteil}
{
    name={Tradition der Phexgeweihten I},
    description={\textbf{Kosten}: 20 \textbf{Voraussetzungen}: Attribut IN 4, Vorteil Geweiht I, Kein Vorteil Tradition der Praiosgeweihten I, Kein Vorteil Tradition der Rondrageweihten I, Kein Vorteil Tradition der Borongeweihten I, Kein Vorteil Tradition der Efferdgeweihten I, Kein Vorteil Tradition der Hesindegeweihten I, Kein Vorteil Tradition der Ingerimmgeweihten I, Kein Vorteil Tradition der Angroschgeweihten I, Kein Vorteil Tradition der Perainegeweihten I, Kein Vorteil Tradition der Firungeweihten I, Kein Vorteil Tradition der Rahjageweihten I, Kein Vorteil Tradition der Tsageweihten I, Kein Vorteil Tradition der Traviageweihten I, Kein Vorteil Tradition der Swafnirgeweihten I, Kein Vorteil Tradition der Ifirngeweihten I, Kein Vorteil Tradition der Nandusgeweihten I, Kein Vorteil Tradition der Korgeweihten I, Kein Vorteil Tradition der Avesgeweihten I \textbf{Nachkauf}: Häufig\newline Bedingungen: Sichtkontakt, Geste, (Symbolische) Gegenleistung an Phex.\newline Du kannst Liturgien in der Tradition der Phexgeweihten erlernen und benutzen. Solche Liturgien findest du unter den allgemeinen Liturgien, unter Abu al‘Mada, List und Nächstlicher Schatten (ab S. 192).\newline Liturgien sind um 2-8 Punkte erschwert, wenn ihr Einsatz gegen die Prinzipien des Geweihten verstößt.}
}


\newglossaryentry{traditionderPhexgeweihtenII_Vorteil}
{
    name={Tradition der Phexgeweihten II},
    description={\textbf{Kosten}: 40 \textbf{Voraussetzungen}: Attribut IN 6, Vorteil Tradition der Phexgeweihten I \textbf{Nachkauf}: Häufig\newline Phexgeweihte vertrauen auf ihr Können, aber manchmal auch auf ihr Glück. Statt mit 3W20 kannst du Proben beim Wirken von Liturgien auch mit 1W20 ablegen. Dadurch verbessern sich deine Chancen bei besonders gewagten Liturgien.}
}


\newglossaryentry{traditionderPhexgeweihtenIII_Vorteil}
{
    name={Tradition der Phexgeweihten III},
    description={\textbf{Kosten}: 60 \textbf{Voraussetzungen}: Attribut IN 8, Vorteil Tradition der Phexgeweihten II \textbf{Nachkauf}: Häufig\newline Erlaubt die spontane Modifikation Opferung (Liturgie +4): Ein rituelles Opfer erleichtert die Liturgie um +4. Der Opfergegenstand wird dabei verbraucht, zerstört oder verschwindet. Phexgeweihte opfern einen wertvollen Besitz.}
}


\newglossaryentry{traditionderPhexgeweihtenIV_Vorteil}
{
    name={Tradition der Phexgeweihten IV},
    description={\textbf{Kosten}: 80 \textbf{Voraussetzungen}: Attribut IN 10, Vorteil Tradition der Phexgeweihten III \textbf{Nachkauf}: Häufig\newline Bedingung: 2 weitere Attribute auf insgesamt 16.\newline 8 Punkte können zur Verbesserung der Tradition verwendet werden. Sephrasto: Trage die Verbesserungen in das Kommentarfeld ein.}
}


\newglossaryentry{traditionderPraiosgeweihtenI_Vorteil}
{
    name={Tradition der Praiosgeweihten I},
    description={\textbf{Kosten}: 20 \textbf{Voraussetzungen}: Attribut CH 4, Vorteil Geweiht I, Kein Vorteil Tradition der Rondrageweihten I, Kein Vorteil Tradition der Borongeweihten I, Kein Vorteil Tradition der Phexgeweihten I, Kein Vorteil Tradition der Efferdgeweihten I, Kein Vorteil Tradition der Hesindegeweihten I, Kein Vorteil Tradition der Ingerimmgeweihten I, Kein Vorteil Tradition der Angroschgeweihten I, Kein Vorteil Tradition der Perainegeweihten I, Kein Vorteil Tradition der Firungeweihten I, Kein Vorteil Tradition der Rahjageweihten I, Kein Vorteil Tradition der Tsageweihten I, Kein Vorteil Tradition der Traviageweihten I, Kein Vorteil Tradition der Swafnirgeweihten I, Kein Vorteil Tradition der Ifirngeweihten I, Kein Vorteil Tradition der Nandusgeweihten I, Kein Vorteil Tradition der Korgeweihten I, Kein Vorteil Tradition der Avesgeweihten I \textbf{Nachkauf}: Häufig\newline Bedingungen: Sichtkontakt, Geste, gesprochenes oder gesungenes Gebet.\newline Du kannst Liturgien in der Tradition der Praiosgeweihten erlernen und benutzen. Solche Liturgien findest du unter den allgemeinen Liturgien, unter Licht, Magiebann und Ordnung (ab S. 193).\newline Liturgien sind um 2-8 Punkte erschwert, wenn ihr Einsatz gegen die Prinzipien des Geweihten verstößt.}
}


\newglossaryentry{traditionderPraiosgeweihtenII_Vorteil}
{
    name={Tradition der Praiosgeweihten II},
    description={\textbf{Kosten}: 40 \textbf{Voraussetzungen}: Attribut CH 6, Vorteil Tradition der Praiosgeweihten I \textbf{Nachkauf}: Häufig\newline Praiosgeweihte erkennen in der Sonne ihren Gott. Im Sonnenlicht sind Liturgien um 2–4 Punkte erleichtert.}
}


\newglossaryentry{traditionderPraiosgeweihtenIII_Vorteil}
{
    name={Tradition der Praiosgeweihten III},
    description={\textbf{Kosten}: 60 \textbf{Voraussetzungen}: Attribut CH 8, Vorteil Tradition der Praiosgeweihten II \textbf{Nachkauf}: Häufig\newline Erlaubt die spontane Modifikation Zeremonie (Liturgie +X): Du kannst die Vorbereitungszeit freiwillig um 1 Minute/Stunde/Tag/Woche/Monat/Jahr erhöhen, wodurch die Liturgie um +4/6/8/10/12/14 erleichtert ist. Die Vorbereitungszeit muss dadurch mindestens verdoppelt werden.}
}


\newglossaryentry{traditionderPraiosgeweihtenIV_Vorteil}
{
    name={Tradition der Praiosgeweihten IV},
    description={\textbf{Kosten}: 80 \textbf{Voraussetzungen}: Attribut CH 10, Vorteil Tradition der Praiosgeweihten III \textbf{Nachkauf}: Häufig\newline Bedingung: 2 weitere Attribute auf insgesamt 16.\newline 8 Punkte können zur Verbesserung der Tradition verwendet werden. Sephrasto: Trage die Verbesserungen in das Kommentarfeld ein.}
}


\newglossaryentry{traditionderRahjageweihtenI_Vorteil}
{
    name={Tradition der Rahjageweihten I},
    description={\textbf{Kosten}: 20 \textbf{Voraussetzungen}: Attribut CH 4, Vorteil Geweiht I, Kein Vorteil Tradition der Praiosgeweihten I, Kein Vorteil Tradition der Borongeweihten I, Kein Vorteil Tradition der Phexgeweihten I, Kein Vorteil Tradition der Efferdgeweihten I, Kein Vorteil Tradition der Hesindegeweihten I, Kein Vorteil Tradition der Ingerimmgeweihten I, Kein Vorteil Tradition der Angroschgeweihten I, Kein Vorteil Tradition der Perainegeweihten I, Kein Vorteil Tradition der Firungeweihten I, Kein Vorteil Tradition der Swafnirgeweihten I, Kein Vorteil Tradition der Tsageweihten I, Kein Vorteil Tradition der Traviageweihten I, Kein Vorteil Tradition der Ifirngeweihten I, Kein Vorteil Tradition der Nandusgeweihten I, Kein Vorteil Tradition der Rondrageweihten I, Kein Vorteil Tradition der Korgeweihten I, Kein Vorteil Tradition der Avesgeweihten I \textbf{Nachkauf}: Häufig\newline Bedingungen: Sichtkontakt, Geste, Gebet oder Zärtlichkeit dem Ziel gegenüber.\newline Du kannst Liturgien in der Tradition der Rahjageweihten erlernen und benutzen. Solche Liturgien findest du unter den allgemeinen Liturgien, unter Harmonie und Rausch (ab S. 195).\newline Liturgien sind um 2-8 Punkte erschwert, wenn ihr Einsatz gegen die Gebote der Kirche verstößt.}
}


\newglossaryentry{traditionderRahjageweihtenII_Vorteil}
{
    name={Tradition der Rahjageweihten II},
    description={\textbf{Kosten}: 40 \textbf{Voraussetzungen}: Attribut CH 6, Vorteil Tradition der Rahjageweihten I \textbf{Nachkauf}: Häufig\newline Rahjgeweihte schätzen den Rausch, egal ob durch Alkohol, Sex, Tanz, einen Ausritt oder schieren Nervenkitzel. In berauschtem Zustand sind Liturgien um 2­4 Punkte erleichtert.}
}


\newglossaryentry{traditionderRahjageweihtenIII_Vorteil}
{
    name={Tradition der Rahjageweihten III},
    description={\textbf{Kosten}: 60 \textbf{Voraussetzungen}: Attribut CH 8, Vorteil Tradition der Rahjageweihten II \textbf{Nachkauf}: Häufig\newline Erlaubt die spontane Modifikation Opferung (Liturgie +4): Ein rituelles Opfer erleichtert die Liturgie um +4. Der Opfergegenstand wird dabei verbraucht, zerstört oder verschwindet. Rahjgeweihte opfern den heiligen Wein Tharf.}
}


\newglossaryentry{traditionderRahjageweihtenIV_Vorteil}
{
    name={Tradition der Rahjageweihten IV},
    description={\textbf{Kosten}: 80 \textbf{Voraussetzungen}: Attribut CH 10, Vorteil Tradition der Rahjageweihten III \textbf{Nachkauf}: Häufig\newline Bedingung: 2 weitere Attribute auf insgesamt 16.\newline 8 Punkte können zur Verbesserung der Tradition verwendet werden. Sephrasto: Trage die Verbesserungen in das Kommentarfeld ein.}
}


\newglossaryentry{traditionderRondrageweihtenI_Vorteil}
{
    name={Tradition der Rondrageweihten I},
    description={\textbf{Kosten}: 20 \textbf{Voraussetzungen}: Attribut MU 4, Vorteil Geweiht I, Kein Vorteil Tradition der Praiosgeweihten I, Kein Vorteil Tradition der Borongeweihten I, Kein Vorteil Tradition der Phexgeweihten I, Kein Vorteil Tradition der Efferdgeweihten I, Kein Vorteil Tradition der Hesindegeweihten I, Kein Vorteil Tradition der Ingerimmgeweihten I, Kein Vorteil Tradition der Angroschgeweihten I, Kein Vorteil Tradition der Perainegeweihten I, Kein Vorteil Tradition der Firungeweihten I, Kein Vorteil Tradition der Rahjageweihten I, Kein Vorteil Tradition der Tsageweihten I, Kein Vorteil Tradition der Traviageweihten I, Kein Vorteil Tradition der Swafnirgeweihten I, Kein Vorteil Tradition der Ifirngeweihten I, Kein Vorteil Tradition der Nandusgeweihten I, Kein Vorteil Tradition der Korgeweihten I, Kein Vorteil Tradition der Avesgeweihten I \textbf{Nachkauf}: Häufig\newline Bedingungen: Sichtkontakt, Geste, gesprochenes oder gesungenes Gebet.\newline Du kannst Liturgien in der Tradition der Rondrageweihten erlernen und benutzen. Solche Liturgien findest du unter den allgemeinen Liturgien, unter Ehre, Heerführung und Schutz der Gläubigen (ab S. 197).\newline Liturgien sind um 2-8 Punkte erschwert, wenn ihr Einsatz gegen die Prinzipien des Geweihten verstößt.}
}


\newglossaryentry{traditionderRondrageweihtenII_Vorteil}
{
    name={Tradition der Rondrageweihten II},
    description={\textbf{Kosten}: 40 \textbf{Voraussetzungen}: Attribut MU 6, Vorteil Tradition der Rondrageweihten I \textbf{Nachkauf}: Häufig\newline Rondrageweihte geben auch in der düstersten Stunde nicht auf. Der Wundmalus schränkt dich beim Wirken von Liturgien nicht ein. Kannst du den Wundmalus ohnehin ignorieren, sind Liturgien um den halben Wundmalus erleichtert.}
}


\newglossaryentry{traditionderRondrageweihtenIII_Vorteil}
{
    name={Tradition der Rondrageweihten III},
    description={\textbf{Kosten}: 60 \textbf{Voraussetzungen}: Attribut MU 8, Vorteil Tradition der Rondrageweihten II \textbf{Nachkauf}: Häufig\newline Erlaubt die spontane Modifikation Opferung (Liturgie +4): Ein rituelles Opfer erleichtert die Liturgie um +4. Der Opfergegenstand wird dabei verbraucht, zerstört oder verschwindet. Rondrageweihte opfern ihr Blut (verursacht 1 Wunde).}
}


\newglossaryentry{traditionderRondrageweihtenIV_Vorteil}
{
    name={Tradition der Rondrageweihten IV},
    description={\textbf{Kosten}: 80 \textbf{Voraussetzungen}: Attribut MU 10, Vorteil Tradition der Rondrageweihten III \textbf{Nachkauf}: Häufig\newline Bedingung: 2 weitere Attribute auf insgesamt 16.\newline 8 Punkte können zur Verbesserung der Tradition verwendet werden. Sephrasto: Trage die Verbesserungen in das Kommentarfeld ein.}
}


\newglossaryentry{traditionderSwafnirgeweihtenI_Vorteil}
{
    name={Tradition der Swafnirgeweihten I},
    description={\textbf{Kosten}: 20 \textbf{Voraussetzungen}: Attribut MU 4, Vorteil Geweiht I, Kein Vorteil Tradition der Praiosgeweihten I, Kein Vorteil Tradition der Borongeweihten I, Kein Vorteil Tradition der Phexgeweihten I, Kein Vorteil Tradition der Efferdgeweihten I, Kein Vorteil Tradition der Hesindegeweihten I, Kein Vorteil Tradition der Ingerimmgeweihten I, Kein Vorteil Tradition der Angroschgeweihten I, Kein Vorteil Tradition der Perainegeweihten I, Kein Vorteil Tradition der Firungeweihten I, Kein Vorteil Tradition der Rahjageweihten I, Kein Vorteil Tradition der Tsageweihten I, Kein Vorteil Tradition der Traviageweihten I, Kein Vorteil Tradition der Ifirngeweihten I, Kein Vorteil Tradition der Nandusgeweihten I, Kein Vorteil Tradition der Rondrageweihten I, Kein Vorteil Tradition der Korgeweihten I, Kein Vorteil Tradition der Avesgeweihten I \textbf{Nachkauf}: Häufig\newline Bedingungen: Sichtkontakt, Geste, gesprochenes oder gesungenes Gebet.\newline Du kannst Liturgien in der Tradition der Swafnirgeweihten erlernen und benutzen. Solche Liturgien findest du unter den allgemeinen Liturgien, unter Seefahrt sowie Wind und Wogen (S. 182).\newline Liturgien sind um 2-8 Punkte erschwert, wenn ihr Einsatz gegen die Prinzipien des Geweihten verstößt.}
}


\newglossaryentry{traditionderSwafnirgeweihtenII_Vorteil}
{
    name={Tradition der Swafnirgeweihten II},
    description={\textbf{Kosten}: 40 \textbf{Voraussetzungen}: Attribut MU 6, Vorteil Tradition der Swafnirgeweihten I \textbf{Nachkauf}: Häufig\newline Swafnirgeweihte begleiten und unterstützen ihre Gefährten auf ihren Fahrten. Auf See sind Liturgien um 2–4 Punkte erleichtert.}
}


\newglossaryentry{traditionderSwafnirgeweihtenIII_Vorteil}
{
    name={Tradition der Swafnirgeweihten III},
    description={\textbf{Kosten}: 60 \textbf{Voraussetzungen}: Attribut MU 8, Vorteil Tradition der Swafnirgeweihten II \textbf{Nachkauf}: Häufig\newline Erlaubt die spontane Modifikation Opferung (Liturgie +4): Ein rituelles Opfer erleichtert die Liturgie um +4. Der Opfergegenstand wird dabei verbraucht, zerstört oder verschwindet. Swafnirgeweihte opfern ihr Blut (verursacht 1 Wunde).}
}


\newglossaryentry{traditionderSwafnirgeweihtenIV_Vorteil}
{
    name={Tradition der Swafnirgeweihten IV},
    description={\textbf{Kosten}: 80 \textbf{Voraussetzungen}: Attribut MU 10, Vorteil Tradition der Swafnirgeweihten III \textbf{Nachkauf}: Häufig\newline Bedingung: 2 weitere Attribute auf insgesamt 16.\newline 8 Punkte können zur Verbesserung der Tradition verwendet werden. Sephrasto: Trage die Verbesserungen in das Kommentarfeld ein.}
}


\newglossaryentry{traditionderTraviageweihtenI_Vorteil}
{
    name={Tradition der Traviageweihten I},
    description={\textbf{Kosten}: 20 \textbf{Voraussetzungen}: Attribut CH 4, Vorteil Geweiht I, Kein Vorteil Tradition der Praiosgeweihten I, Kein Vorteil Tradition der Borongeweihten I, Kein Vorteil Tradition der Phexgeweihten I, Kein Vorteil Tradition der Efferdgeweihten I, Kein Vorteil Tradition der Hesindegeweihten I, Kein Vorteil Tradition der Ingerimmgeweihten I, Kein Vorteil Tradition der Angroschgeweihten I, Kein Vorteil Tradition der Perainegeweihten I, Kein Vorteil Tradition der Firungeweihten I, Kein Vorteil Tradition der Swafnirgeweihten I, Kein Vorteil Tradition der Tsageweihten I, Kein Vorteil Tradition der Rahjageweihten I, Kein Vorteil Tradition der Ifirngeweihten I, Kein Vorteil Tradition der Nandusgeweihten I, Kein Vorteil Tradition der Rondrageweihten I, Kein Vorteil Tradition der Korgeweihten I, Kein Vorteil Tradition der Avesgeweihten I \textbf{Nachkauf}: Häufig\newline Bedingungen: Sichtkontakt, Geste, gesprochenes oder gesungenes Gebet.\newline Du kannst Liturgien in der Tradition der Traviageweihten erlernen und benutzen. Solche Liturgien findest du unter den allgemeinen Liturgien, unter Heim und Herd sowie Sichere Heimkehr (ab S. 199).\newline Liturgien sind um 2-8 Punkte erschwert, wenn ihr Einsatz gegen die Gebote der Kirche verstößt.}
}


\newglossaryentry{traditionderTraviageweihtenII_Vorteil}
{
    name={Tradition der Traviageweihten II},
    description={\textbf{Kosten}: 40 \textbf{Voraussetzungen}: Attribut CH 6, Vorteil Tradition der Traviageweihten I \textbf{Nachkauf}: Häufig\newline Traviageweihte schätzen Treue und Beständigkeit. Wird eine Liturgie mit der Modifikation Wirkungsdauer verlängern gewirkt, ist sie zusätzlich um +2 erleichtert.}
}


\newglossaryentry{traditionderTraviageweihtenIII_Vorteil}
{
    name={Tradition der Traviageweihten III},
    description={\textbf{Kosten}: 60 \textbf{Voraussetzungen}: Attribut CH 8, Vorteil Tradition der Traviageweihten II \textbf{Nachkauf}: Häufig\newline Erlaubt die spontane Modifikation Zeremonie (Liturgie +X): Du kannst die Vorbereitungszeit freiwillig um 1 Minute/Stunde/Tag/Woche/Monat/Jahr erhöhen, wodurch die Liturgie um +4/6/8/10/12/14 erleichtert ist. Die Vorbereitungszeit muss dadurch mindestens verdoppelt werden.}
}


\newglossaryentry{traditionderTraviageweihtenIV_Vorteil}
{
    name={Tradition der Traviageweihten IV},
    description={\textbf{Kosten}: 80 \textbf{Voraussetzungen}: Attribut CH 10, Vorteil Tradition der Traviageweihten III \textbf{Nachkauf}: Häufig\newline Bedingung: 2 weitere Attribute auf insgesamt 16.\newline 8 Punkte können zur Verbesserung der Tradition verwendet werden. Sephrasto: Trage die Verbesserungen in das Kommentarfeld ein.}
}


\newglossaryentry{traditionderTsageweihtenI_Vorteil}
{
    name={Tradition der Tsageweihten I},
    description={\textbf{Kosten}: 20 \textbf{Voraussetzungen}: Attribut IN 4, Vorteil Geweiht I, Kein Vorteil Tradition der Praiosgeweihten I, Kein Vorteil Tradition der Borongeweihten I, Kein Vorteil Tradition der Phexgeweihten I, Kein Vorteil Tradition der Efferdgeweihten I, Kein Vorteil Tradition der Hesindegeweihten I, Kein Vorteil Tradition der Ingerimmgeweihten I, Kein Vorteil Tradition der Angroschgeweihten I, Kein Vorteil Tradition der Perainegeweihten I, Kein Vorteil Tradition der Firungeweihten I, Kein Vorteil Tradition der Swafnirgeweihten I, Kein Vorteil Tradition der Rahjageweihten I, Kein Vorteil Tradition der Traviageweihten I, Kein Vorteil Tradition der Ifirngeweihten I, Kein Vorteil Tradition der Nandusgeweihten I, Kein Vorteil Tradition der Rondrageweihten I, Kein Vorteil Tradition der Korgeweihten I, Kein Vorteil Tradition der Avesgeweihten I \textbf{Nachkauf}: Häufig\newline Bedingungen: Sichtkontakt, Geste, gesprochenes oder gesungenes Gebet.\newline Du kannst Liturgien in der Tradition der Tsageweihten erlernen und benutzen. Solche Liturgien findest du unter den allgemeinen Liturgien, unter Friede und Neubeginn (ab S. 200).\newline Liturgien sind um 2-8 Punkte erschwert, wenn ihr Einsatz gegen die Gebote der Kirche verstößt.}
}


\newglossaryentry{traditionderTsageweihtenII_Vorteil}
{
    name={Tradition der Tsageweihten II},
    description={\textbf{Kosten}: 40 \textbf{Voraussetzungen}: Attribut IN 6, Vorteil Tradition der Tsageweihten I \textbf{Nachkauf}: Häufig\newline Tsageweihte sind schnell entschlossen und spontan. Wird eine Liturgie mit der Modifikation Vorbereitungszeit verkürzen gewirkt, ist sie zusätzlich um +2 erleichtert.}
}


\newglossaryentry{traditionderTsageweihtenIII_Vorteil}
{
    name={Tradition der Tsageweihten III},
    description={\textbf{Kosten}: 60 \textbf{Voraussetzungen}: Attribut IN 8, Vorteil Tradition der Tsageweihten II \textbf{Nachkauf}: Häufig\newline Erlaubt die spontane Modifikation Zeremonie (Liturgie +X): Du kannst die Vorbereitungszeit freiwillig um 1 Minute/Stunde/Tag/Woche/Monat/Jahr erhöhen, wodurch die Liturgie um +4/6/8/10/12/14 erleichtert ist. Die Vorbereitungszeit muss dadurch mindestens verdoppelt werden.}
}


\newglossaryentry{traditionderTsageweihtenIV_Vorteil}
{
    name={Tradition der Tsageweihten IV},
    description={\textbf{Kosten}: 80 \textbf{Voraussetzungen}: Attribut IN 10, Vorteil Tradition der Tsageweihten III \textbf{Nachkauf}: Häufig\newline Bedingung: 2 weitere Attribute auf insgesamt 16.\newline 8 Punkte können zur Verbesserung der Tradition verwendet werden. Sephrasto: Trage die Verbesserungen in das Kommentarfeld ein.}
}


\newglossaryentry{traditionderPaktiererI_Vorteil}
{
    name={Tradition der Paktierer I},
    description={\textbf{Kosten}: 20 \textbf{Voraussetzungen}: Attribut MU 4, Vorteil Paktierer I \textbf{Nachkauf}: Häufig\newline Die Fertigkeit Dämonische Hilfe (Erzdämon) kann erlernt und benutzt werden. Proben auf Dämonische Hilfe (Erzdämon) werden mit 1W20 gewürfelt.}
}


\newglossaryentry{traditionderPaktiererII_Vorteil}
{
    name={Tradition der Paktierer II},
    description={\textbf{Kosten}: 40 \textbf{Voraussetzungen}: Attribut MU 6, Vorteil Tradition der Paktierer I \textbf{Nachkauf}: Häufig\newline Wird eine Anrufung mit der Modifikation Mächtige Anrufung gewirkt, ist sie zusätzlich um +2 erleichtert.}
}


\newglossaryentry{traditionderPaktiererIII_Vorteil}
{
    name={Tradition der Paktierer III},
    description={\textbf{Kosten}: 60 \textbf{Voraussetzungen}: Attribut MU 8, Vorteil Tradition der Paktierer II \textbf{Nachkauf}: Häufig\newline Erlaubt die spontane Modifikation Opferung (Anrufung +4): Das rituelle Opfer eines intelligenten Wesens erleichtert die Anrufung um +4, ein Geweihter der Gegengottheit verdoppelt den Bonus sogar.}
}


\newglossaryentry{traditionderPaktiererIV_Vorteil}
{
    name={Tradition der Paktierer IV},
    description={\textbf{Kosten}: 80 \textbf{Voraussetzungen}: Attribut MU 10, Vorteil Tradition der Paktierer III \textbf{Nachkauf}: Häufig\newline Bedingung: 2 weitere Attribute auf insgesamt 16.\newline 8 Punkte können zur Verbesserung der Tradition verwendet werden. Sephrasto: Trage die Verbesserungen in das Kommentarfeld ein.}
}


\newglossaryentry{unbewaffnet_Talent}
{
    name={Unbewaffnet},
    description={Verbessert den waffenlosen Nahkampf.}
}


\newglossaryentry{handgemengewaffen_Talent}
{
    name={Handgemengewaffen},
    description={Verbessert den Einsatz von Messern, Dolchen, Schlagstöcken und Kettenstäben.}
}


\newglossaryentry{schilde_Talent}
{
    name={Schilde},
    description={Verbessert den Einsatz von Schilden.}
}


\newglossaryentry{einhandklingenwaffen_Talent}
{
    name={Einhandklingenwaffen},
    description={Verbessert den Einsatz einhändig geführter Klingenwaffen wie Schwertern, Säbeln und Fechtwaffen.}
}


\newglossaryentry{einhandhiebwaffen_Talent}
{
    name={Einhandhiebwaffen},
    description={Verbessert den Einsatz einhändig geführter Hiebwaffen wie Keulen, Äxten, Beilen und kleineren Hämmern.}
}


\newglossaryentry{zweihandhiebwaffen_Talent}
{
    name={Zweihandhiebwaffen},
    description={Verbessert den Einsatz beidhändig geführter Hiebwaffen wie großer Äxte und Hämmern.}
}


\newglossaryentry{zweihandklingenwaffen_Talent}
{
    name={Zweihandklingenwaffen},
    description={Verbessert den Einsatz von zweihändig geführten Klingenwaffen wie Anderthalbhändern, Bastardschwertern und Zweihandschwertern.}
}


\newglossaryentry{infanteriewaffenundSpeere_Talent}
{
    name={Infanteriewaffen und Speere},
    description={Verbessert den Einsatz von langen Infanteriewaffen wie Speeren, Hellebarden und Kampfstäben.}
}


\newglossaryentry{lanzenreiten_Talent}
{
    name={Lanzenreiten},
    description={Verbessert den Einsatz von Lanzen im berittenen Kampf.}
}


\newglossaryentry{bögen_Talent}
{
    name={Bögen},
    description={Verbessert den Einsatz von Bögen.}
}


\newglossaryentry{armbrüste_Talent}
{
    name={Armbrüste},
    description={Verbessert den Einsatz von Armbrüsten und Torsionswaffen.}
}


\newglossaryentry{schleudern_Talent}
{
    name={Schleudern},
    description={Verbessert den Einsatz von Schleudern.}
}


\newglossaryentry{blasrohre_Talent}
{
    name={Blasrohre},
    description={Verbessert den Einsatz von Blasrohren.}
}


\newglossaryentry{kurzeWurfwaffen_Talent}
{
    name={Kurze Wurfwaffen},
    description={Verbessert den Einsatz von Wurfäxten, Wurfbeilen, Wurfmessern und Wurfsternen.}
}


\newglossaryentry{wurfspeere_Talent}
{
    name={Wurfspeere},
    description={Verbessert den Einsatz von Wurfspeeren.}
}


\newglossaryentry{diskusse_Talent}
{
    name={Diskusse},
    description={Verbessert den Einsatz von Diskussen.}
}


\newglossaryentry{laufen_Talent}
{
    name={Laufen},
    description={Laufen kommt in Verfolgungsjagden zu Fuß zum Einsatz, wenn du einen Verbrecher stellen oder einem Raubtier entkommen möchtest.}
}


\newglossaryentry{klettern_Talent}
{
    name={Klettern},
    description={Mit Klettern überwindest du alle Arten von Hindernissen.}
}


\newglossaryentry{schwimmen_Talent}
{
    name={Schwimmen},
    description={Schwimmen erlaubt eine schnellere Fortbewegung im Wasser und längere Tauchgänge.}
}


\newglossaryentry{reiten_Talent}
{
    name={Reiten},
    description={Reiten ist die Fähigkeit, ein Pferd, Kamel oder einen Hippogriff zu kontrollieren und als schnelles Reisemittel oder im Kampf einzusetzen.}
}


\newglossaryentry{akrobatik_Talent}
{
    name={Akrobatik},
    description={Akrobatik wird für gewagte Kunststücke und Balanceakte verwendet und kann auch den Fallschaden reduzieren.}
}


\newglossaryentry{pirschen_Talent}
{
    name={Pirschen},
    description={Pirschen ermöglicht das Schleichen, Verstecken und Lauern in der freien Natur. Du pirschst dich auf der Jagd an einen wilden Hirsch heran, legst einen Hinterhalt an einer Reichsstraße oder beobachtest ungesehen das Lager einer Orkbande.}
}


\newglossaryentry{untertauchen_Talent}
{
    name={Untertauchen},
    description={Untertauchen erlaubt dir das Schleichen, Verstecken und Beschatten in der Zivilisation. Du verschwindest in der Menschenmenge am Marktplatz, steigst ungehört in eine Villa eines Ratsherren ein oder beschattest unauffällig einen Hehler.}
}


\newglossaryentry{willenskraft_Talent}
{
    name={Willenskraft},
    description={Willenskraft erlaubt dir, Beeinflussungen und Versuchungen zu widerstehen und dich nicht ablenken zu lassen. Du kannst dich im Chaos eines Gefechts auf einen Zauber konzentrieren oder einem Heiligen Befehl widerstehen.}
}


\newglossaryentry{zähigkeit_Talent}
{
    name={Zähigkeit},
    description={Zähigkeit hilft dir, Schmerzen und Strapazen zu widerstehen. So kannst du trotz zahlreicher Wunden noch handlungsfähig bleiben und einen Kampf zu euren Gunsten wenden oder eine ganze Nacht hindurch wachen.}
}


\newglossaryentry{menschenkenntnis_Talent}
{
    name={Menschenkenntnis},
    description={Menschenkenntnis ist deine Fähigkeit, die Absichten deines Gegenüber zu durchschauen. Damit enttarnst du Lügen und falsche Annäherungsversuche.}
}


\newglossaryentry{sinnenschärfe_Talent}
{
    name={Sinnenschärfe},
    description={Sinnenschärfe ist die aktive Verwendung deiner Sinne, um einen Kollaborateur in einem geschäftigen Wirtshaus zu belauschen, die Flagge eines nahenden Schiffes zu erkennen oder die Nadel im Heuhaufen zu finden.}
}


\newglossaryentry{wachsamkeit_Talent}
{
    name={Wachsamkeit},
    description={Wachsamkeit fasst den passiven Einsatz deiner Sinne zusammen. Mit ihr entdeckst du Hinterhalte, bemerkst eine Unstimmigkeit an einem Tatort oder eine verborgene Notiz am Wegesrand.}
}


\newglossaryentry{anführen_Talent}
{
    name={Anführen},
    description={Mit Anführen leitest und motivierst du Untergebene. Du kannst mit Anführen Löscharbeiten oder eine Truppe von Kämpfern koordinieren, um ihnen so Vorteile zu verschaffen (S. 46). Gerade in großen Schlachten entscheidet der Heerführer über Sieg oder Niederlage.}
}


\newglossaryentry{einschüchtern_Talent}
{
    name={Einschüchtern},
    description={Einschüchtern jagt dem Gegenüber Angst ein und bringt ihn so zu einer gewünschten Handlung. Die Probe wird oft vergleichend gegen den MU oder die Menschenkenntnis des Gegenübers abgelegt.}
}


\newglossaryentry{rhetorik_Talent}
{
    name={Rhetorik},
    description={Rhetorik beinhaltet zahlreiche Fähigkeiten und Kniffe, um die eigenen Argumente wirkungsvoll einzusetzen und die des Gegenüber zu entkräften. Du kannst Rhetorik nur einsetzen, wenn dein Charakter von seinem Standpunkt überzeugt ist - dreiste Lügen fallen unter Überreden.}
}


\newglossaryentry{betören_Talent}
{
    name={Betören},
    description={Beim Betören nutzt du deine persönliche Ausstrahlung, um zu bekommen, was du willst. Das reicht von Kleinigkeiten wie einem Freibier bis zum Verrat geheimer Staatsinformationen.}
}


\newglossaryentry{überreden_Talent}
{
    name={Überreden},
    description={Überreden bedeutet den geschickten Einsatz von Übertreibungen, Unwahrheiten oder Lügen, um das Gegenüber zumindest kurzfristig zu beeinflussen. Mit Überreden feilschst du am Marktplatz, bestichst eine Stadtwache oder infiltrierst ein Borbaradianerkloster.}
}


\newglossaryentry{geographie_Talent}
{
    name={Geographie},
    description={Geographie befasst sich nicht nur mit Reiserouten und fernen Ländern, sondern auch mit dem Sternenhimmel, der Kartographie und Navigation. Ein Geograph kann Schatzkarten lesen und anfertigen, die Position eines Schiffes bestimmen und Sternenkonstellationen deuten.}
}


\newglossaryentry{pflanzenkunde_Talent}
{
    name={Pflanzenkunde},
    description={Pflanzenkundige erforschen die vielen nützlichen, gefährlichen oder wundersamen Pflanzen Aventuriens. Sie sammeln die Kräuter für eine Heilsalbe, kennen die Gefahr von Jagdgras und finden essbare Früchte oder Wurzeln.}
}


\newglossaryentry{tierkunde_Talent}
{
    name={Tierkunde},
    description={Tierkunde bedeutet die Kenntnis der aventurischen Fauna. Dein Charakter kann einen Schleimfleck als Spur einer Riesenamöbe erkennen, einen gereizten Bären beruhigen und kennt die Schwachstelle eines Tatzelwurms.}
}


\newglossaryentry{gebräuche:Mittelreich_Talent}
{
    name={Gebräuche: Mittelreich},
    description={}
}


\newglossaryentry{gebräuche:Tulamidenlande_Talent}
{
    name={Gebräuche: Tulamidenlande},
    description={}
}


\newglossaryentry{gebräuche:Südaventurien_Talent}
{
    name={Gebräuche: Südaventurien},
    description={}
}


\newglossaryentry{gebräuche:Bornland_Talent}
{
    name={Gebräuche: Bornland},
    description={}
}


\newglossaryentry{gebräuche:Thorwal_Talent}
{
    name={Gebräuche: Thorwal},
    description={}
}


\newglossaryentry{gebräuche:Maraskan_Talent}
{
    name={Gebräuche: Maraskan},
    description={}
}


\newglossaryentry{gebräuche:Elfen_Talent}
{
    name={Gebräuche: Elfen},
    description={}
}


\newglossaryentry{gebräuche:Zwerge_Talent}
{
    name={Gebräuche: Zwerge},
    description={}
}


\newglossaryentry{gebräuche:Horasreich_Talent}
{
    name={Gebräuche: Horasreich},
    description={}
}


\newglossaryentry{götterundKulte_Talent}
{
    name={Götter und Kulte},
    description={Götter und Kulte befasst sich mit dem Wesen der Götter, ihrer Schöpfung und ihren sterblichen Dienern. Dein Charakter kennt den Aufbau und die Ziele der Kirchen und weiß, wie man ihren Angehörigen gegenübertritt. Genauso kann er eine alte Kultstätte einer Gottheit oder vielleicht einem Erzdämonen zuordnen.}
}


\newglossaryentry{geschichtenundLegenden_Talent}
{
    name={Geschichten und Legenden},
    description={Geschichten und Legenden ist das Wissen um alte Überlieferungen. Damit kannst du das Alter von Grabmälern und Artefakten bestimmen oder aus einer Sage die Vorlieben und Schwachpunkte eines Riesen ableiten.}
}


\newglossaryentry{dämonenkunde_Talent}
{
    name={Dämonenkunde},
    description={Dämonenkunde beschäftigt sich mit den Erzdämonen, ihren Dienern und dämonischen Zaubern. Du kannst Dämonen bennenen und kennst ihre Schwachstellen.}
}


\newglossaryentry{elementarkunde_Talent}
{
    name={Elementarkunde},
    description={Elementarkunde befasst sich mit den sechs Elementen, der Elementarbeschwörung und den elementaren Zaubern.}
}


\newglossaryentry{magietheorie_Talent}
{
    name={Magietheorie},
    description={Magietheorie hilft dir bei der Einschätzung und der Analyse von magischen Phänomenen.}
}


\newglossaryentry{zauberpraxis_Talent}
{
    name={Zauberpraxis},
    description={Zauberpraxis umfasst die Kenntnis verbreiteter Zauber und Rituale, ihrer Wirkung und möglicher Gegenmaßnahmen.}
}


\newglossaryentry{überleben:HoherNorden_Talent}
{
    name={Überleben: Hoher Norden},
    description={Gilt für alle Regionen nördlich der Salamandersteine.}
}


\newglossaryentry{überleben:Nordaventurien_Talent}
{
    name={Überleben: Nordaventurien},
    description={Gilt für die Region zwischen den Salamandersteinen im Norden und dem Steineichenwald und Ysilisee im Süden.}
}


\newglossaryentry{überleben:Mittelaventurien_Talent}
{
    name={Überleben: Mittelaventurien},
    description={Gilt für die Region zwischen Steineichenwald und Ysilisee im Norden und Eisenwald, Phecanowald und Raschtulswall im Süden.}
}


\newglossaryentry{überleben:Südaventurien_Talent}
{
    name={Überleben: Südaventurien},
    description={Gilt für die Region zwischen Eisenwald, Phecanowald und Raschtulswall im Norden und Drôl und Thalusien im Süden.}
}


\newglossaryentry{überleben:TieferSüden_Talent}
{
    name={Überleben: Tiefer Süden},
    description={Gilt für alle Regionen südlich von Drôl und Thalusien.}
}


\newglossaryentry{überleben:Meer_Talent}
{
    name={Überleben: Meer},
    description={Gilt für den Einsatz auf großen Gewässern wie zum Beispiel dem Perlenmeer.}
}


\newglossaryentry{überleben:Gebirge_Talent}
{
    name={Überleben: Gebirge},
    description={Gilt für den Einsatz im Gebirge, egal ob Raschtulswall oder ehernes Schwert.}
}


\newglossaryentry{überleben:Wüste_Talent}
{
    name={Überleben: Wüste},
    description={Gilt für Wüstenregionen wie die Khomwüste und die Wüste Gor.}
}


\newglossaryentry{überleben:Maraskan_Talent}
{
    name={Überleben: Maraskan},
    description={Gilt für die Insel Maraskan.}
}


\newglossaryentry{falschspiel_Talent}
{
    name={Falschspiel},
    description={Falschsspieler helfen dem Glück etwas nach, zum Beispiel beim Hütchenspiel oder Boltan. Proben auf Falschspielen werden oft vergleichend gegen die Wachsamkeit der Mitspieler abgelegt.}
}


\newglossaryentry{schlösserknacken_Talent}
{
    name={Schlösser knacken},
    description={Schlösser knacken öffnet mit Hilfe eines Dietrichs oder einer Haarnadel Schatzkisten und Tresorräume. Außerdem kann dein Charakter Fallen entschärfen.}
}


\newglossaryentry{stehlen_Talent}
{
    name={Stehlen},
    description={Mit Stehlen kannst du zur richtigen Zeit am richtigen Ort sein, um die richtige Person unauffällig um ihren Besitz zu erleichtern. Außerdem können Diebe den Wert ihrer Beute einschätzen und Kontakt zu einem Hehler aufnehmen.}
}


\newglossaryentry{alchemistischeAnalyse_Talent}
{
    name={Alchemistische Analyse},
    description={Mit einer alchemistischen Analyse kannst du feststellen, ob eine verstaubte Phiole einen Heiltrank oder ein tödliches Gift enthält und ob es sich bei dem schwarzen Erz tatsächlich um Endurium handelt.}
}


\newglossaryentry{magischeElixiere_Talent}
{
    name={Magische Elixiere},
    description={Magische Elixiere beinhaltet Heiltränke, Stärkungsmittel und Verwandlungselixiere, kurz die hohe Kunst der Alchemie. Rezepte findest du auf S. 64.}
}


\newglossaryentry{profaneAlchemika_Talent}
{
    name={Profane Alchemika},
    description={Profane Alchemika ist das weit bodenständigere Handwerk der Meuchler, Waldläufer und Kräuterfrauen, mit dem du Gifte, Heilsalben oder Rauchbomben herstellen kannst, siehe auch S. 64.}
}


\newglossaryentry{gifteundKrankheiten_Talent}
{
    name={Gifte und Krankheiten},
    description={Mit dem Talent Gifte und Krankheiten stoppst du eine Tulmadron-Vergiftung, erkennst die ersten Anzeichen für Zorganpocken und kannst einen Ghulbiss behandeln.}
}


\newglossaryentry{wundheilung_Talent}
{
    name={Wundheilung},
    description={Wundheilung ermöglicht es dir, im Kampf entstandene Blutungen zu stoppen und die Heilung von Verletzungen zu fördern. Zusätzlich kannst du von Wunden Rückschlüsse auf die Tatwaffe und den Tathergang ziehen.}
}


\newglossaryentry{holzbearbeitung_Talent}
{
    name={Holzbearbeitung},
    description={Mit Holzbearbeitung fertigst du Schilde, Bögen oder ein improvisiertes Floß an, reparierst eine Zugbrücke oder flickst ein Leck in einem Schiff.}
}


\newglossaryentry{mechanik_Talent}
{
    name={Mechanik},
    description={Mit Mechanik stellst du Apparaturen wie Armbrüste, Flaschenzüge oder Südweiser her und wartest sie.}
}


\newglossaryentry{schmieden_Talent}
{
    name={Schmieden},
    description={Schmieden ist die Herstellung und Reparatur von Schwertern und Hufeisen bis hin zu Plattenpanzern.}
}


\newglossaryentry{bannschwert_Talent}
{
    name={Bannschwert},
    description={Du legst bannende Zauber auf eine Waffe mit Reichweite 0. Die Waffe gilt als magisch und erleichtert die Bannung von beschworenen Wesenheiten um +2. Gilt als Objektritual.\newline Mächtige Magie: Für 2/4 Stufen kannst du eine Waffe mit Reichweite 1/2 verzaubern.\newline Probenschwierigkeit: 12\newline Vorbereitungszeit: 1 Stunde\newline Ziel: Einzelobjekt\newline Reichweite: Berührung\newline Wirkungsdauer: bis die Bindung gelöst wird\newline Kosten: 16 AsP, davon 4 gAsP\newline Fertigkeiten: Antimagie, Dämonisch\newline Erlernen: Bor, Mag 18; Ach, Alch, Dru, Geo, Elf 20; 40 EP}
}


\newglossaryentry{dämonenbann_Talent}
{
    name={Dämonenbann},
    description={Wirkt gegen Zauber der Fertigkeit Dämonisch sowie gegen Dämonen und Untote, in die keine gAsP geflossen sind. Spezielle Antimagie-Modifikationen siehe S. 126.\newline Fertigkeiten: Antimagie, Dämonisch\newline Erlernen: Dru, Elf, Mag 14; Hex 16; Geo 18; 60 EP}
}


\newglossaryentry{destructiboArcanitas_Talent}
{
    name={Destructibo Arcanitas},
    description={Du bannst einen Zauber, in den gAsP geflossen sind (wie ein Artefakt oder ein permanenter Fluch), oder ein Wesen, in das gAsP geflossen sind (wie eine Chimäre oder einen gebundenen Dämon).\newline Probenschwierigkeit: Probenschwierigkeit bzw. Beschwörungsschwierigkeit\newline Modifikationen: Astralkörper abbauen (Probenschwierigkeit MR, Einzelperson, 8 AsP; das Ziel verliert 16 AsP.)\newline Vorbereitungszeit: 1 Stunde\newline Ziel: Zauber oder Einzelwesen\newline Reichweite: 2 Schritt\newline Wirkungsdauer: augenblicklich\newline Kosten: halbe Basiskosten des Zaubers oder der Beschwörung\newline Fertigkeiten: Antimagie, Kraft\newline Erlernen: Mag 18; Ach, Elf, Dru 20; 40 EP}
}


\newglossaryentry{eigenschaftwiederherstellen_Talent}
{
    name={Eigenschaft wiederherstellen},
    description={Wirkt gegen Zauber der Fertigkeit Eigenschaften, spezielle Antimagie-Modifikationen siehe S. 126.\newline Fertigkeiten: Antimagie, Eigenschaften\newline Erlernen: Alch, Hex 12; Ach, Dru, Geo, Mag, Elf 14; 40 EP}
}


\newglossaryentry{einflussbannen_Talent}
{
    name={Einfluss bannen},
    description={Wirkt gegen Zauber der Fertigkeit Einfluss, spezielle Antimagie-Modifikationen siehe S. 126.\newline Fertigkeiten: Antimagie, Einfluss\newline Erlernen: Dru 8; Geo, Hex 12; Ach, Elf, Mag 14; Alch, Sch, Srl 16; 40 EP}
}


\newglossaryentry{elementarbann_Talent}
{
    name={Elementarbann},
    description={Wirkt gegen Zauber der Fertigkeiten Eis, Erz, Feuer, Humus, Luft und Wasser, sowie gegen Elementare, in die keine gAsP geflossen sind. Spezielle Antimagie-Modifikationen siehe S. 126.\newline Fertigkeiten: Antimagie, Eis, Erz, Feuer, Humus, Luft, Wasser\newline Erlernen: Dru, Elf, Geo, Mag 18; 40 EP}
}


\newglossaryentry{gardianumZauberschild_Talent}
{
    name={Gardianum Zauberschild},
    description={Du erschaffst eine unsichtbare und immaterielle Schutzkuppel mit dir als Zentrum und 2 Schritt Radius. Schadenszauber werden von der Kuppel abgefangen, die nach 8 Beschädigungen zusammenbricht (Härte 2).\newline Mächtige Magie: Die Härte steigt um +1.\newline Probenschwierigkeit: 12\newline Modifikationen: Schild gegen Dämonen (–4; Dämonen mit einer Beschwörungsschwierigkeit von maximal 20 können die Kuppel nicht durchschreiten. Mächtige Magie erhöht die maximale Beschwörungsschwierigkeit um 4. Angriffe des Dämonen werden vom Gardianum abgefangen, der dabei als Koloss II gilt (S. 96). Du kannst einen Dämon mit dem Gardianum zurückdrängen, wenn dir eine vergleichende KK-Probe gelingt. Umgekehrt ist das nicht möglich.),\newline Persönlicher Schild (–4, Wirkungsdauer 1 Stunde; der Zauber legt sich als zweite Haut um den Zauberer.)\newline Vorbereitungszeit: 0 Aktionen\newline Ziel: Zone\newline Reichweite: Berührung\newline Wirkungsdauer: 4 Minuten\newline Kosten: 8 AsP\newline Fertigkeiten: Antimagie, Kraft\newline Erlernen: Elf, Mag 12; Ach 14; Dru, Geo, Hex 16; 40 EP}
}


\newglossaryentry{geisteraustreiben_Talent}
{
    name={Geister austreiben},
    description={Du zeichnest ein Pentagramm, durch das ein Geist deiner Wahl verschwindet. Er kann 1 Woche lang nicht mehr umherspuken.\newline Mächtige Magie: Der Geist kann 1 Monat/1 Jahr/10 Jahre/100 Jahre nicht mehr umherspuken.\newline Probenschwierigkeit: 16 bis 28 (nach Mächtigkeit des Geistes)\newline Vorbereitungszeit: 16 Aktionen\newline Ziel: einzelner Geist\newline Reichweite: 32 Schritt\newline Wirkungsdauer: augenblicklich\newline Kosten: 16 AsP\newline Fertigkeiten: Antimagie, Verständigung\newline Erlernen: Dru 8; Mag 14; Geo, Hex 16; 20 EP}
}


\newglossaryentry{hellsichttrüben_Talent}
{
    name={Hellsicht trüben},
    description={Wirkt gegen Zauber der Fertigkeit Hellsicht, spezielle Antimagie-Modifikationen siehe S. 126.\newline Fertigkeiten: Antimagie, Hellsicht\newline Erlernen: Dru, Geo 12; Mag, Sch, Srl 14; Ach, Elf, Hex 16; Alch 18; 20 EP}
}


\newglossaryentry{illusionauflösen_Talent}
{
    name={Illusion auflösen},
    description={Wirkt gegen Zauber der Fertigkeit Illusion, spezielle Antimagie-Modifikationen siehe S. 126.\newline Fertigkeiten: Antimagie, Illusion\newline Erlernen: Mag, Srl 14; Hex, Sch 18; 20 EP}
}


\newglossaryentry{kraftmagieneutralisieren_Talent}
{
    name={Kraftmagie neutralisieren},
    description={Wirkt gegen Zauber der Fertigkeit Kraft, spezielle Antimagie-Modifikationen siehe S. 126.\newline Fertigkeiten: Antimagie, Kraft\newline Erlernen: Mag 18; 20 EP}
}


\newglossaryentry{protectionisKontrabann_Talent}
{
    name={Protectionis Kontrabann},
    description={Du schützt deinen nächsten, während der Wirkungsdauer begonnenen Zauber vor Antimagie. Jede Antimagie gegen diesen Zauber ist um –4 erschwert.\newline Mächtige Magie: Der Malus steigt um –2.\newline Probenschwierigkeit: 12\newline Modifikationen: Austreibung verhindern (–8; erschwert Antimagie gegen das nächste beschworene Wesen für 1 Tag)\newline Vorbereitungszeit: 4 Aktionen\newline Ziel: selbst\newline Reichweite: Berührung\newline Wirkungsdauer: 16 Initiativphasen\newline Kosten: 8 AsP\newline Fertigkeiten: Antimagie, Kraft\newline Erlernen: Bor, Mag 18; 20 EP}
}


\newglossaryentry{psychostabilis_Talent}
{
    name={Psychostabilis},
    description={Magieresistenz-Proben des Ziels sind um +4 erleichtert.\newline Mächtige Magie: Erhöht den Bonus um +2.\newline Probenschwierigkeit: 12\newline Modifikationen: Schnellsteigerung (–4; Wirkungsdauer 16 Initiativphasen; verdoppelt die Erleichterung.)\newline Stabilisierung (–8; das Ziel darf sofort eine MR-Probe gegen einen auf es wirkenden Zauber wiederholen. Gelingt sie, wird dieser Zauber für die Wirkungsdauer des Psychostabilis unterdrückt.)\newline Vorbereitungszeit: 2 Aktionen\newline Ziel: Einzelperson\newline Reichweite: Berührung\newline Wirkungsdauer: 1 Stunde\newline Kosten: 8 AsP\newline Fertigkeiten: Antimagie, Eigenschaften\newline Erlernen: Alch, Mag 12; Dru, Geo, Hex 14; Ach, Srl, Elf 18; 40 EP}
}


\newglossaryentry{schleierderUnwissenheit_Talent}
{
    name={Schleier der Unwissenheit},
    description={Du schützt die magische Aura des Ziels vor der Entdeckung. Der Analysegrad jeder Analyse auf das Ziel sinkt um eins.\newline Mächtige Magie: Der Analysegrad sinkt um einen weiteren Punkt.\newline Probenschwierigkeit: 12\newline Vorbereitungszeit: 16 Aktionen\newline Ziel: Einzelwesen, Einzelobjekt\newline Reichweite: Berührung\newline Wirkungsdauer: 4 Stunden\newline Kosten: 8 AsP\newline Fertigkeiten: Antimagie, Kraft\newline Erlernen: Hex 12; Mag 16; Dru 18; 20 EP}
}


\newglossaryentry{schutzkreisgegenDaimonide_Talent}
{
    name={Schutzkreis gegen Daimonide},
    description={Du ziehst einen Kreis von maximal 4 Schritt Radius. Liegt die Beschwörungsschwierigkeit eines Daimoniden bei maximal 20, kann er die Zone nicht betreten.\newline Mächtige Magie: Erhöht die Beschwörungsschwierigkeit um 4.\newline Probenschwierigkeit: 12\newline Modifikationen: Bannkreis (8 AsP; statt der Basiswirkung wird ein Daimonid mit einer Beschwörungsschwierigkeit von maximal 20 aus einer Entfernung von bis zu 8 Schritt in den Bannkreis gezogen. Alle zwei Stunden darf er versuchen, sich mit einer Konterprobe (MR, 16) zu befreien. Danach ist er immun gegen den Bannkreis. Es kann stets nur ein Wesen im Bannkreis gefangen sein.)\newline Vorbereitungszeit: 4 Minuten\newline Ziel: Zone\newline Reichweite: Berührung\newline Wirkungsdauer: 1 Woche\newline Kosten: 4 AsP\newline Fertigkeiten: Antimagie, Dämonisch, Verwandlung\newline Erlernen: Ach, Mag 18; 20 EP}
}


\newglossaryentry{schutzkreisgegenDämonen_Talent}
{
    name={Schutzkreis gegen Dämonen},
    description={Du ziehst einen Kreis von maximal 4 Schritt Radius. Liegt die Beschwörungsschwierigkeit eines Dämonen bei maximal 20, kann er die Zone nicht betreten.\newline Mächtige Magie: Erhöht die Beschwörungsschwierigkeit um 4.\newline Probenschwierigkeit: 12\newline Modifikationen: Bannkreis (8 AsP; statt der Basiswirkung wird ein Dämon mit einer Beschwörungsschwierigkeit von maximal 20 aus einer Entfernung von bis zu 8 Schritt in den Bannkreis gezogen. Alle zwei Stunden darf er versuchen, sich mit einer Konterprobe (MR, 16) zu befreien. Danach ist er immun gegen den Bannkreis. Es kann stets nur ein Wesen im Bannkreis gefangen sein.)\newline Vorbereitungszeit: 4 Minuten\newline Ziel: Zone\newline Reichweite: Berührung\newline Wirkungsdauer: 1 Woche\newline Kosten: 4 AsP\newline Fertigkeiten: Antimagie, Dämonisch\newline Erlernen: Bor, Mag 16; Ach, Alch, Hex 18; Dru 20; 40 EP}
}


\newglossaryentry{schutzkreisgegenElementare_Talent}
{
    name={Schutzkreis gegen Elementare},
    description={Du ziehst einen Kreis von maximal 4 Schritt Radius. Liegt die Beschwörungsschwierigkeit eines Elementars bei maximal 20, kann er die Zone nicht betreten.\newline Mächtige Magie: Erhöht die Beschwörungsschwierigkeit um 4.\newline Probenschwierigkeit: 12\newline Modifikationen: Bannkreis (8 AsP; statt der Basiswirkung wird ein Elementar mit einer Beschwörungsschwierigkeit von maximal 20 aus einer Entfernung von bis zu 8 Schritt in den Bannkreis gezogen. Alle zwei Stunden darf er versuchen, sich mit einer Konterprobe (MR, 16) zu befreien. Danach ist er immun gegen den Bannkreis. Es kann stets nur ein Wesen im Bannkreis gefangen sein.)\newline Vorbereitungszeit: 4 Minuten\newline Ziel: Zone\newline Reichweite: Berührung\newline Wirkungsdauer: 1 Woche\newline Kosten: 4 AsP\newline Fertigkeiten: Antimagie, Eis, Erz, Feuer, Humus, Luft, Wasser\newline Erlernen: Ach, Alch, Geo 18; Dru, Mag 20; 20 EP}
}


\newglossaryentry{schutzkreisgegenUntote_Talent}
{
    name={Schutzkreis gegen Untote},
    description={Du ziehst einen Kreis von maximal 4 Schritt Radius. Liegt die Beschwörungsschwierigkeit eines Untoten bei maximal 20, kann er die Zone nicht betreten. Wirkt auch gegen Geister, wobei der Spielleiter entscheidet, ob die Stärke des Kreises ausreicht.\newline Mächtige Magie: Erhöht die Beschwörungsschwierigkeit um 4.\newline Probenschwierigkeit: 12\newline Modifikationen: Bannkreis (8 AsP; statt der Basiswirkung wird ein Untoter mit einer Beschwörungsschwierigkeit von maximal 20 aus einer Entfernung von bis zu 8 Schritt in den Bannkreis gezogen. Alle zwei Stunden darf er versuchen, sich mit einer Konterprobe (MR, 16) zu befreien. Danach ist er immun gegen den Bannkreis. Es kann stets nur ein Wesen im Bannkreis gefangen sein.)\newline Vorbereitungszeit: 4 Minuten\newline Ziel: Zone\newline Reichweite: Berührung\newline Wirkungsdauer: 1 Woche\newline Kosten: 4 AsP\newline Fertigkeiten: Antimagie, Dämonisch\newline Erlernen: Ach 18; Alch, Bor, Mag 20; 20 EP}
}


\newglossaryentry{sinesigillUnerkannt_Talent}
{
    name={Sinesigill Unerkannt},
    description={Während der Wirkungsdauer ist dein Gildensiegel von einer Illusion (Sicht) verdeckt und auch vor ungezielter Hellsicht verborgen.\newline Probenschwierigkeit: 12\newline Modifikationen: Fremde Hand (–4, Einzelperson)\newline Vorbereitungszeit: 16 Aktionen\newline Ziel: selbst\newline Reichweite: Berührung\newline Wirkungsdauer: 8 Stunden\newline Kosten: 4 AsP\newline Fertigkeiten: Antimagie, Illusion\newline Erlernen: Mag 20; 20 EP}
}


\newglossaryentry{veränderungaufheben_Talent}
{
    name={Veränderung aufheben},
    description={Wirkt gegen Zauber der Fertigkeit Umwelt, spezielle Antimagie-Modifikationen siehe S. 126.\newline Fertigkeiten: Antimagie, Umwelt\newline Erlernen: Ach, Geo, Mag 14; Dru, Elf, Sch 16; Alch 18; 20 EP}
}


\newglossaryentry{verwandlungbeenden_Talent}
{
    name={Verwandlung beenden},
    description={Wirkt gegen Zauber der Fertigkeit Verwandlung und Chimären, spezielle Antimagie-Modifikationen siehe S. 126.\newline Fertigkeiten: Antimagie, Verwandlung\newline Erlernen: Ach, Dru, Hex, Mag 14; Elf, Geo, Srl 16; 40 EP}
}


\newglossaryentry{verständigungstören_Talent}
{
    name={Verständigung stören},
    description={Wirkt gegen Zauber der Fertigkeit Verständigung, spezielle Antimagie-Modifikationen siehe S. 126.\newline Fertigkeiten: Antimagie, Verständigung\newline Erlernen: Elf 12; Ach, Dru, Mag 14; Geo 16; Hex 18; 20 EP}
}


\newglossaryentry{brennetoterStoff!_Talent}
{
    name={Brenne toter Stoff!},
    description={Du entzündest ein magisches Feuer, das 2W6 SP pro Initiativephase auf einem unbelebten Ziel anrichtet. Das Feuer brennt auf brennbaren Materialien auch nach dem Ende der Wirkungsdauer weiter.\newline Mächtige Magie: Je zwei Stufen erhöhen die SP um 1W6.\newline Probenschwierigkeit: 12\newline Modifikationen: Flammeninferno (–8, 32 AsP; das Feuer breitet sich mit etwa 1 Schritt pro Initiativphase aus.)\newline Vorbereitungszeit: 1 Aktion\newline Ziel: Einzelobjekt\newline Reichweite: Berührung\newline Wirkungsdauer: 8 Initiativphasen\newline Kosten: 8 AsP\newline Fertigkeiten: Dämonisch (nicht in Geo, Mag, Srl), Feuer, Verwandlung\newline Erlernen: Bor 12; Alch, Geo 14; Ach, Mag 18; Srl 20; 40 EP}
}


\newglossaryentry{chimaeroformHybridgestalt_Talent}
{
    name={Chimaeroform Hybridgestalt},
    description={Du erschaffst eine Chimäre aus zwei oder mehr Lebewesen (mehr zu Beschwörungen siehe S. 81). Eines dieser Lebewesen kann durch einen Dämon ersetzt werden. Die Chimäre hat sich nach 1 Stunde an ihre Existenz gewöhnt und ist einsatzfähig.\newline Probenschwierigkeit: nach Chimäre\newline Vorbereitungszeit: frei wählbar\newline Wirkungsdauer: bis die Bindung gelöst wird\newline Kosten: nach Chimäre, 1/4 der Basiskosten als gAsP\newline Fertigkeiten: Dämonisch, Verwandlung\newline Erlernen: Mag 18; Hex 20; 60 EP}
}


\newglossaryentry{ecliptifactusSchattenkraft_Talent}
{
    name={Ecliptifactus Schattenkraft},
    description={Dein Schatten kämpft an deiner Seite (WS 5, INI 6, GS 6, RW 1, VT 6, AT 6, TP 2W6). Zusätzlich verfügt der Schatten über die Vorteile Schreckgestalt II, Körperlosigkeit, Paraphysikalität und Schmerzimmun II (S. 98). Stirbt der Schatten, verlierst du deine komplette Astralenergie und regenerierst keine AsP, bis der Schatten nach 7 Wochen nachgewachsen ist.\newline Mächtige Magie: AT, VT, TP steigen um +2.\newline Probenschwierigkeit: 12\newline Vorbereitungszeit: 1 Aktion\newline Ziel: selbst\newline Reichweite: Berührung\newline Wirkungsdauer: 16 Initiativphasen\newline Kosten: 8 AsP\newline Fertigkeiten: Dämonisch, Verwandlung\newline Erlernen: Mag 18; Bor 20; 40 EP}
}


\newglossaryentry{eigeneÄngstequälendich!_Talent}
{
    name={Eigene Ängste quälen dich!},
    description={Du raubst deinem Ziel einen Sinn deiner Wahl und lieferst es völlig seinen inneren Ängsten aus. Es erleidet einen Furcht-Effekt Stufe 1. Nach Meisterentscheid sind auch langfristige Nachwirkungen möglich.\newline Mächtige Magie: Du raubst deinem Opfer einen weiteren Sinn und der Furcht-Effekt steigt um eine Stufe.\newline Probenschwierigkeit: Magieresistenz\newline Vorbereitungszeit: 1 Aktion\newline Ziel: Einzelperson\newline Reichweite: Berührung\newline Wirkungsdauer: 1 Stunde\newline Kosten: 16 AsP\newline Fertigkeiten: Dämonisch, Eigenschaften, Einfluss\newline Erlernen: Bor 12; Mag 20; 60 EP}
}


\newglossaryentry{erinnerungverlassedich!_Talent}
{
    name={Erinnerung verlasse dich!},
    description={Dein Ziel verliert sämtliche Erinnerung an sein voriges Leben und kann sich später nicht an Ereignisse während der Wirkungsdauer erinnern.\newline Probenschwierigkeit: Magieresistenz\newline Vorbereitungszeit: 1 Aktion\newline Ziel: Einzelperson\newline Reichweite: Berührung\newline Wirkungsdauer: 4 Stunden\newline Kosten: 16 AsP\newline Fertigkeiten: Dämonisch, Einfluss\newline Erlernen: Bor 16; Mag 18; 40 EP}
}


\newglossaryentry{fluchderPestilenz_Talent}
{
    name={Fluch der Pestilenz},
    description={Du infizierst dein Opfer mit einer dir bekannten Krankheit deiner Wahl, die dann ihren natürlichen Krankheitsverlauf nimmt. Die maximale Krankheitsstufe ist 16.\newline Mächtige Magie: Erhöht die maximale Krankheitsstufe um +4.\newline Probenschwierigkeit: Magieresistenz\newline Modifikationen: Einzelfall (–4; die Krankheit ist nicht ansteckend.)\newline Vorbereitungszeit: 4 Aktionen\newline Ziel: Einzelperson\newline Reichweite: 4 Schritt\newline Wirkungsdauer: augenblicklich\newline Kosten: 16 AsP\newline Fertigkeiten: Dämonisch\newline Erlernen: Dru 14; Bor, Hex, Mag 20; 40 EP}
}


\newglossaryentry{granitundMarmor_Talent}
{
    name={Granit und Marmor},
    description={Dein Opfer verwandelt sich im Verlauf einer Stunde in eine Statue aus natürlichem Gestein. Es gilt nicht mehr als Lebewesen und nimmt die Zeit als Statue nicht wahr. Schäden an der Statue werden direkt in Wunden und Verstümmelungen am Opfer übersetzt.\newline Probenschwierigkeit: Magieresistenz\newline Modifikationen: Statuenträume (–4; das Opfer behält sein Bewusstsein.)\newline Permanenz (–4, Wirkungsdauer bis die Bindung gelöst wird, 16 AsP, davon 2 gAsP)\newline Vorbereitungszeit: 4 Aktionen\newline Ziel: Einzelwesen\newline Reichweite: 4 Schritt\newline Wirkungsdauer: 1 Monat\newline Kosten: 16 AsP\newline Fertigkeiten: Dämonisch, Erz, Verwandlung\newline Erlernen: Bor, Mag 20; 40 EP}
}


\newglossaryentry{hartesschmelze!_Talent}
{
    name={Hartes schmelze!},
    description={Du verzauberst eine kopfgroße Menge an hartem Material, sodass es formbar wie Wachs wird. Nach dem Ende der Wirkungsdauer erstarrt das Material in seiner aktuellen Form.\newline Mächtige Magie: Verdoppelt die Menge.\newline Probenschwierigkeit: 12\newline Vorbereitungszeit: 1 Aktion\newline Ziel: Einzelobjekt oder Teilobjekt\newline Reichweite: Berührung\newline Wirkungsdauer: 16 Minuten\newline Kosten: 8 AsP\newline Fertigkeiten: Dämonisch (nicht Geo, Mag), Verwandlung, Wasser\newline Erlernen: Bor 14; Ach, Dru, Geo 16; Alch, Mag 18; 60 EP}
}


\newglossaryentry{herzschlagruhe!_Talent}
{
    name={Herzschlag ruhe!},
    description={Das Herz deines Opfers bleibt schlagartig stehen. Nach 4 und 8 Initiativephasen erleidet es eine Wunde.\newline Mächtige Magie: Das Ziel erleidet nach 4 weiteren Initiativephasen eine weitere Wunde.\newline Probenschwierigkeit: Magieresistenz\newline Vorbereitungszeit: 1 Aktion\newline Ziel: Einzelwesen\newline Reichweite: Berührung\newline Wirkungsdauer: augenblicklich\newline Kosten: 16 AsP\newline Fertigkeiten: Dämonisch, Eigenschaften, Einfluss\newline Erlernen: Bor 18; 40 EP}
}


\newglossaryentry{höllenpeinzerreißedich!_Talent}
{
    name={Höllenpein zerreiße dich!},
    description={Dein Opfer windet sich in schrecklichen Schmerzen und ist bis zum Ende der Wirkungsdauer handlungsunfähig. Am Ende der Wirkungsdauer erleidet es 2 Punkte Erschöpfung. Die Wirkung endet vorzeitig, wenn das Opfer Schaden nimmt.\newline Mächtige Magie: Das Opfer erleidet einen weiteren Punkt Erschöpfung.\newline Probenschwierigkeit: Magieresistenz\newline Vorbereitungszeit: 2 Aktionen\newline Ziel: Einzelwesen\newline Reichweite: 8 Schritt\newline Wirkungsdauer: 16 Initiativphasen\newline Kosten: 8 AsP\newline Fertigkeiten: Dämonisch, Einfluss\newline Erlernen: Bor 8; Alch, Mag 18; 60 EP}
}


\newglossaryentry{invocatio_Talent}
{
    name={Invocatio},
    description={Ruft einen Dämon herbei (mehr zur Dämonenbeschwörung siehe S. 81), der in deiner unmittelbaren Nähe erscheint.\newline Probenschwierigkeit: nach Dämon\newline Vorbereitungszeit: frei wählbar\newline Ziel: einzelner Dämon\newline Wirkungsdauer: augenblicklich\newline Kosten: nach Dämon\newline Fertigkeiten: Dämonisch\newline Erlernen: Mag 14; Bor, Hex 16; Ach, Alch, Dru 18; 120 EP}
}


\newglossaryentry{karnifiloRaserei_Talent}
{
    name={Karnifilo Raserei},
    description={Dein Ziel verfällt in einen Kampfrausch. Es erhält die Vorteile Kalte Wut und Offensiver Kampfstil (S. 43) und nutzt in jeder Aktion volle Offensive. Sind keine Gegner mehr da, greift es auch Verbündete an.\newline Mächtige Magie: Die AT steigt um +1.\newline Probenschwierigkeit: Magieresistenz\newline Vorbereitungszeit: 1 Aktion\newline Ziel: Einzelperson\newline Reichweite: Berührung\newline Wirkungsdauer: 16 Initiativphasen\newline Kosten: 8 AsP\newline Fertigkeiten: Dämonisch, Eigenschaften, Einfluss\newline Erlernen: Alch, Mag 16; Ach, Bor 18; 40 EP}
}


\newglossaryentry{krabbelnderSchrecken_Talent}
{
    name={Krabbelnder Schrecken},
    description={Dein Opfer wird von einer Myriade Insekten und Kleintieren bedeckt. Misslingt eine Konterprobe (Willenskraft, 16), ist es handlungsunfähig. Gelingt die Konterprobe, sind alle Proben um –4 erschwert.\newline Probenschwierigkeit: 12\newline Modifikationen: Objekt verfluchen (–4, Einzelobjekt, 8 AsP; um das Objekt zu berühren, muss eine Konterprobe (MU, 16) bestanden werden.)\newline Vorbereitungszeit: 8 Aktionen\newline Ziel: Einzelperson\newline Reichweite: 16 Schritt\newline Wirkungsdauer: 16 Initiativphasen\newline Kosten: 16 AsP\newline Fertigkeiten: Dämonisch, Verständigung\newline Erlernen: Hex 18; Bor, Mag 20; 40 EP}
}


\newglossaryentry{nuntiovoloBotenvogel_Talent}
{
    name={Nuntiovolo Botenvogel},
    description={Du formst eine Rauchgestalt, die einen kleinen Gegenstand (wie einen Brief) mit etwa 50 Meilen pro Stunde an den gewünschten Zielort bringt. Der Zauber kann nur in der Nacht gewirkt werden und ist unzuverlässig: Wenn dem Spielleiter eine verdeckte Probe (6, I) misslingt, kommt die Rauchgestalt nicht an.\newline Mächtige Magie: Die Schwierigkeit der Probe ist 5/4/3/2.\newline Probenschwierigkeit: 12\newline Vorbereitungszeit: 8 Aktionen\newline Ziel: Zone\newline Reichweite: 100 Meilen\newline Wirkungsdauer: augenblicklich\newline Kosten: 8 AsP\newline Fertigkeiten: Dämonisch\newline Erlernen: Mag 18; Ach, Bor, Hex 20; 40 EP}
}


\newglossaryentry{pandämonium_Talent}
{
    name={Pandämonium},
    description={In einem Radius von 2 Schritt um das Ziel brechen Klauen, Mäuler und Tentakel hervor. Jedes Wesen ohne den Vorteil Unheilig erleidet 2W6 TP pro Initiativephase und kann sich nur mit einer Konterprobe (GE, 16) fortbewegen.\newline Mächtige Magie: Der Schaden steigt um 1W6 TP und der Radius erhöht sich um 4 Schritt.\newline Probenschwierigkeit: 12\newline Vorbereitungszeit: 8 Aktionen\newline Ziel: Zone\newline Reichweite: 16 Schritt\newline Wirkungsdauer: 1 Stunde\newline Kosten: 16 AsP\newline Fertigkeiten: Dämonisch\newline Erlernen: Bor 16; Hex 18; Mag 20; 40 EP}
}


\newglossaryentry{paniküberkommeeuch!_Talent}
{
    name={Panik überkomme euch!},
    description={Du erscheinst jedem, der dich sehen kann und der keine Konterprobe (Magieresistenz, 12) besteht, als ein Wesen mit Schreckgestalt II (S. 98).\newline Probenschwierigkeit: 12\newline Modifikationen: Objekt (–4, Einzelobjekt)\newline Vorbereitungszeit: 2 Aktionen\newline Ziel: selbst\newline Reichweite: Berührung\newline Wirkungsdauer: 16 Initiativphasen\newline Kosten: 8 AsP\newline Fertigkeiten: Dämonisch, Einfluss\newline Erlernen: Bor 18; 40 EP}
}


\newglossaryentry{reptileaNatternnest_Talent}
{
    name={Reptilea Natternnest},
    description={Alle Geschuppten in einem Radius von 64 Schritt streben auf dein Ziel zu und fallen dort über alles her, was in ihr Fressschema passt.\newline Mächtige Magie: Verdoppelt den Radius.\newline Probenschwierigkeit: 12\newline Modifikationen: Selemer Verhältnisse (–8, Wirkungsdauer 1 Woche, 32 AsP)\newline Krötenkunde (–4; du kannst den Zauber auf eine bestimmte Art von Echsenwesen beschränken oder eine Art ausnehmen)\newline Vorbereitungszeit: 16 Aktionen\newline Ziel: Zone\newline Reichweite: 16 Schritt\newline Wirkungsdauer: 8 Stunden\newline Kosten: 16 AsP\newline Fertigkeiten: Dämonisch, Verständigung\newline Erlernen: Ach 16; Bor 18; Mag 20; 40 EP}
}


\newglossaryentry{steinwandle!_Talent}
{
    name={Stein wandle!},
    description={Du erschaffst einen Golem aus Holz, Stein oder anderen Materialien (mehr zu Beschwörungen siehe S. 81). Der Golem hat sich nach 1 Stunde an seine Existenz gewöhnt und ist einsatzfähig.\newline Probenschwierigkeit: nach Golem\newline Vorbereitungszeit: frei wählbar\newline Ziel: Material für einen einzelnen Golem\newline Wirkungsdauer: bis die Bindung gelöst wird\newline Kosten: nach Golem, ein Viertel der Basiskosten als gAsP\newline Fertigkeiten: Dämonisch\newline Erlernen: Bor, Mag 18; 60 EP}
}


\newglossaryentry{schwarzundRot_Talent}
{
    name={Schwarz und Rot},
    description={In der Herzgegend deines Opfers entsteht ein schmerzhaftes rotes Mal, das sich langsam ausbreitet. Nach 2 und 4 Stunden erleidet das Opfer 1 Wunde. Solange das Opfer Schaden nimmt, sind Wundmalusse verdoppelt.\newline Mächtige Magie: Das Ziel erleidet nach 2 weiteren Stunden eine weitere Wunde.\newline Probenschwierigkeit: Magieresistenz\newline Vorbereitungszeit: 4 Aktionen\newline Ziel: Einzelperson\newline Reichweite: Berührung\newline Wirkungsdauer: augenblicklich\newline Kosten: 8 AsP\newline Fertigkeiten: Dämonisch, Eigenschaften\newline Erlernen: Bor 16; Mag 18; Hex 20; 40 EP}
}


\newglossaryentry{skelettariusTotenherr_Talent}
{
    name={Skelettarius Totenherr},
    description={Du erhebst eine Leiche als Untoten (mehr zu Beschwörungen siehe S. 81), der in zwei INI-Phasen einsatzfähig ist.\newline Probenschwierigkeit: nach Untotem\newline Modifikationen: Schnelle Erhebung (–4; der Untote ist sofort bereit.)\newline Vorbereitungszeit: frei wählbar\newline Ziel: Material für einen einzelnen Untoten\newline Wirkungsdauer: augenblicklich\newline Kosten: nach Untotem\newline Fertigkeiten: Dämonisch\newline Erlernen: Bor, Mag 16; Alch 20; 40 EP}
}


\newglossaryentry{tlalucsOdemPestgestank_Talent}
{
    name={Tlalucs Odem Pestgestank},
    description={Eine giftige Wolke breitet sich aus deinem Mund 8 Schritt weit kegelförmig aus (45°). Jedes Wesen in der Wolke nimmt 2W6 SP und muss eine Konterprobe (Zähigkeit, 16) bestehen, um nicht einen Malus von –4 auf alle Proben zu erleiden.\newline Mächtige Magie: Die SP steigen um 1W6.\newline Probenschwierigkeit: 12\newline Modifikationen: Miasmasphaero (–4; die Wolke breitet sich rund um dich aus.)\newline Miasmafaxius (–4, Einzelperson, 8 AsP)\newline Vorbereitungszeit: 2 Aktionen\newline Ziel: Zone\newline Reichweite: Berührung\newline Wirkungsdauer: 8 Initiativphasen\newline Kosten: 16 AsP\newline Fertigkeiten: Dämonisch\newline Erlernen: Mag 18; Bor, Dru, Hex 20; 40 EP}
}


\newglossaryentry{toteshandle!_Talent}
{
    name={Totes handle!},
    description={Du erschaffst aus einer Leiche einen untoten Diener (mehr zu Beschwörungen siehe S. 81). Der Untote hat sich nach 1 Stunde an seine Existenz gewöhnt und ist einsatzfähig.\newline Probenschwierigkeit: nach Untotem\newline Vorbereitungszeit: frei wählbar\newline Ziel: Material für einen einzelnen Untoten\newline Wirkungsdauer: bis die Bindung gelöst wird\newline Kosten: nach Untotem, ein Viertel der Basiskosten als gAsP\newline Fertigkeiten: Dämonisch\newline Erlernen: Bor, Mag 18; 60 EP}
}


\newglossaryentry{umbraportaSchattensprung_Talent}
{
    name={Umbraporta Schattensprung},
    description={Du trittst in einen Schatten hinein und wirst sofort durch den Limbus zu einem anderen Schatten transportiert.\newline Probenschwierigkeit: 12\newline Modifikationen: Schattenportal (–8, 16 AsP; Du kannst eine Person mitnehmen. Mächtige Magie erlaubt zwei/drei/vier/fünf Personen.)\newline Schattensprung (–8, 4 Aktionen, Wirkungsdauer 16 Initiativphasen, 16 AsP; du verschmilzt mit den Schatten und kannst dich in ihnen mit GS 12 bewegen. Du kannst nicht körperlich mit deiner Umwelt interagieren, aber mit einer Erschwernis von –4 zaubern. Ermöglicht Aufrechterhalten.)\newline Vorbereitungszeit: 1 Aktion\newline Ziel: Zone\newline Reichweite: 8 Schritt\newline Wirkungsdauer: augenblicklich\newline Kosten: 8 AsP\newline Fertigkeiten: Dämonisch, Kraft\newline Erlernen: Bor 18; Mag 20; 60 EP}
}


\newglossaryentry{weichesErstarre!_Talent}
{
    name={Weiches Erstarre!},
    description={Du lässt bis zu 8 Raumschritt Luft, Wasser oder ein anderes „weiches“ Material zu einer festen, harten Masse mit einer Härte von 8 erstarren. Lebende Materie, oder Material, das ein Wesen berührt, kann nicht beeinflusst werden.\newline Mächtige Magie: Die Härte steigt um +4.\newline Probenschwierigkeit: 12\newline Vorbereitungszeit: 2 Aktionen\newline Ziel: Zone\newline Reichweite: Berührung\newline Wirkungsdauer: 1 Stunde\newline Kosten: 8 AsP\newline Fertigkeiten: Dämonisch (nicht in Geo oder Mag), Erz, Verwandlung\newline Erlernen: Bor 14; Dru, Geo 16; Ach, Alch, Mag 18; 60 EP}
}


\newglossaryentry{adleraugeLuchsenohr_Talent}
{
    name={Adlerauge Luchsenohr},
    description={Du schärfst deine Sinne auf magische Weise. Alle Proben auf Sinnenschärfe und Wachsamkeit sind um +4 erleichtert.\newline Mächtige Magie: Der Bonus steigt um +2.\newline Probenschwierigkeit: 12\newline Modifikationen: Einzelsinn (–4, Wirkungsdauer 1 Stunde; nur ein Sinn ist betroffen.)\newline Vorbereitungszeit: 2 Aktionen\newline Ziel: selbst\newline Reichweite: Berührung\newline Wirkungsdauer: 4 Minuten\newline Kosten: 8 AsP\newline Fertigkeiten: Eigenschaften, Hellsicht\newline Erlernen: Elf 8; Ach, Alch, Dru, Hex, Geo, Mag, Sch 18; 20 EP}
}


\newglossaryentry{armatrutz_Talent}
{
    name={Armatrutz},
    description={Der RS deines Zieles steigt um 1.\newline Mächtige Magie: Je zwei Stufen verleihen +1 RS.\newline Probenschwierigkeit: 12\newline Modifikationen: Körperschild (–4, 1 AsP; der Armatrutz schützt nur eine Trefferzone.)\newline Vorbereitungszeit: 1 Aktion\newline Ziel: Einzelperson\newline Reichweite: Berührung\newline Wirkungsdauer: 4 Minuten\newline Kosten: 4 AsP\newline Fertigkeiten: Eigenschaften, Erz\newline Erlernen: Elf, Mag 12; Alch 16; Dru, Hex 18; 40 EP}
}


\newglossaryentry{atemnot_Talent}
{
    name={Atemnot},
    description={Du entziehst dem Opfer einen Teil seiner Kraft. Es erleidet 2 Punkte Erschöpfung. Du regenerierst die Hälfte der angerichteten Erschöpfung.\newline Mächtige Magie: Verursacht 1 weiteren Punkt Erschöpfung.\newline Probenschwierigkeit: Magieresistenz\newline Vorbereitungszeit: 2 Aktionen\newline Ziel: Einzelperson\newline Reichweite: 4 Schritt\newline Wirkungsdauer: augenblicklich\newline Kosten: 8 AsP\newline Fertigkeiten: Eigenschaften\newline Erlernen: Dru 14; Ach 18; 40 EP}
}


\newglossaryentry{attributo_Talent}
{
    name={Attributo},
    description={Wähle ein Attribut aus. Proben auf dieses Attribut sind um +2, Fertigkeitsproben mit diesem Attribut um +1 erleichtert.\newline Mächtige Magie: Der Bonus steigt um +2/+1.\newline Probenschwierigkeit: 12\newline Vorbereitungszeit: 8 Aktionen\newline Ziel: Einzelperson\newline Reichweite: Berührung\newline Wirkungsdauer: 1 Stunde\newline Kosten: 8 AsP\newline Fertigkeiten: Eigenschaften\newline Erlernen: Alch 8; Ach, Dru, Elf, Geo, Hex, Mag, Sch 14; 40 EP}
}


\newglossaryentry{axxeleratusBlitzgeschwind_Talent}
{
    name={Axxeleratus Blitzgeschwind},
    description={Die GS deines Zieles erhöht sich um +4 und alle AT und VT sind um +2 erleichtert, Ausweichen um weitere +2.\newline Mächtige Magie: Die GS steigt um weitere +2.\newline Probenschwierigkeit: 12\newline Vorbereitungszeit: 0 Aktionen\newline Ziel: Einzelperson\newline Reichweite: 4 Schritt\newline Wirkungsdauer: 16 Initiativphasen\newline Kosten: 8 AsP\newline Fertigkeiten: Eigenschaften\newline Erlernen: Elf 12; Ach, Sch 14; Mag 18; 40 EP}
}


\newglossaryentry{corpofessoGliederschmerz_Talent}
{
    name={Corpofesso Gliederschmerz},
    description={Dein Ziel erleidet eine plötzliche Muskelschwäche. Wenn es anstrengende Bewegungen unternimmt (Laufen, Kämpfen), erleidet es alle DH* Initiativephasen 1 Punkt Erschöpfung.\newline Probenschwierigkeit: Magieresistenz\newline Vorbereitungszeit: 2 Aktionen\newline Ziel: Einzelwesen\newline Reichweite: 16 Schritt\newline Wirkungsdauer: 16 Initiativphasen\newline Kosten: 8 AsP\newline Fertigkeiten: Eigenschaften\newline Erlernen: Alch, Mag 16; Elf, Hex 20; 40 EP}
}


\newglossaryentry{corpofrigoKälteschock_Talent}
{
    name={Corpofrigo Kälteschock},
    description={Du entziehst deinem Opfer die Körperwärme, bis sich sein Körper mit Raureif überzieht. Seine GS sinkt auf die Hälfte und alle Athletik-Proben zur Bewegung sind um –4 erschwert.\newline Mächtige Magie: Die GS sinkt auf ein Viertel/Achtel/Sechzehntel/Zweiunddreißigstel und der Malus steigt um –2.\newline Probenschwierigkeit: Magieresistenz\newline Vorbereitungszeit: 1 Aktion\newline Ziel: Einzelwesen\newline Reichweite: 32 Schritt\newline Wirkungsdauer: 8 Initiativphasen\newline Kosten: 4 AsP\newline Fertigkeiten: Eigenschaften, Eis\newline Erlernen: Ach, Dru 12; Mag 16; Alch 18; 40 EP}
}


\newglossaryentry{einsmitderNatur_Talent}
{
    name={Eins mit der Natur},
    description={Das Ziel kann in der Wildnis kann es natürliche Gefahren wie mit der Gabe Gefahreninstinkt erahnen. Beherrscht es die Gabe bereits, sind damit verbundene Proben um +4 erleichtert. Wirkt nicht unter der Erde, auf dem Meer oder in dämonisch pervertiertem Gebiet.\newline Mächtige Magie: Überleben-Proben sind um +2 erleichtert.\newline Probenschwierigkeit: 12\newline Vorbereitungszeit: 4 Minuten\newline Ziel: Einzelperson\newline Reichweite: Berührung\newline Wirkungsdauer: 1 Tag\newline Kosten: 8 AsP\newline Fertigkeiten: Eigenschaften, Humus\newline Erlernen: Geo 8; Dru 14; Elf, Hex, Mag 20; 40 EP}
}


\newglossaryentry{eiseskälteKämpferherz_Talent}
{
    name={Eiseskälte Kämpferherz},
    description={Das Ziel empfindet keinerlei Schmerzen mehr. Es erhält den Vorteil Kalte Wut (S. 43). Proben zum Ignorieren von Wundschmerz sind um +4 erleichtert.\newline Mächtige Magie: Der Bonus steigt um +2.\newline Probenschwierigkeit: 12\newline Vorbereitungszeit: 0 Aktionen\newline Ziel: Einzelperson\newline Reichweite: Berührung\newline Wirkungsdauer: 16 Initiativphasen\newline Kosten: 8 AsP\newline Fertigkeiten: Eigenschaften\newline Erlernen: Ach, Elf 18; 40 EP}
}


\newglossaryentry{falkenaugeMeisterschuss_Talent}
{
    name={Falkenauge Meisterschuss},
    description={Dein nächster Fernkampfangriff ist um +4 erleichtert.\newline Mächtige Magie: Der Bonus steigt um +2.\newline Probenschwierigkeit: 12\newline Modifikationen: Dauerndes Band (–8, 8 AsP; der Zauber wirkt auf alle Fernkampfangriffe während der Wirkungsdauer.)\newline Vorbereitungszeit: 1 Aktion\newline Ziel: Einzelperson\newline Reichweite: Berührung\newline Wirkungsdauer: 8 Initiativphasen\newline Kosten: 4 AsP\newline Fertigkeiten: Eigenschaften\newline Erlernen: Elf 12; 20 EP}
}


\newglossaryentry{firnlauf_Talent}
{
    name={Firnlauf},
    description={Jede noch so dünne Eis- oder Schneeschicht trägt dich wie trockener Boden. Unter diesen Bedingungen erleidest du keinerlei Erschwernisse oder Geschwindigkeitsabzüge.\newline Probenschwierigkeit: 12\newline Modifikationen: Verankerung (–4, 2 AsP; du verankerst dich so auf einer Eisfläche, dass du selbst dann nicht stürzt, wenn sie in bedrohliche Schieflage oder schnelle Bewegung gerät.)\newline Vorbereitungszeit: 2 Aktionen\newline Ziel: selbst\newline Reichweite: Berührung\newline Wirkungsdauer: 1 Stunde\newline Kosten: 8 AsP\newline Fertigkeiten: Eigenschaften, Eis\newline Erlernen: Dru, Elf 16; Alch, Mag 20; 20 EP}
}


\newglossaryentry{großeVerwirrung_Talent}
{
    name={Große Verwirrung},
    description={Dein Opfer kann sich nicht mehr konzentrieren. Alle Proben auf KL und IN sind um –4 erschwert, Proben auf Fertigkeiten mit KL oder IN um –2.\newline Mächtige Magie: Der Malus steigt um –2/–1.\newline Probenschwierigkeit: Magieresistenz\newline Vorbereitungszeit: 0 Aktionen\newline Ziel: Einzelperson\newline Reichweite: Berührung\newline Wirkungsdauer: 1 Stunde\newline Kosten: 4 AsP\newline Fertigkeiten: Eigenschaften\newline Erlernen: Dru, Geo 8; Sch 14; 20 EP}
}


\newglossaryentry{hexengalle_Talent}
{
    name={Hexengalle},
    description={Du spuckst auf deinen Gegner. Dein Speichel verwandelt sich in ätzende Säure und das getroffene Opfer erleidet 2W6 SP und wird bei misslungener Konterprobe (Zähigkeit, 16) bis zu seiner übernächsten Initiativephase handlungsunfähig. Ballistischer Zauber.\newline Mächtige Magie: Die SP steigen um 1W6.\newline Probenschwierigkeit: 12\newline Modifikationen: Drachenspeichel (–4; der Zauber wirkt auch gegen unbelebte Ziele.)\newline Vorbereitungszeit: 0 Aktionen\newline Ziel: selbst\newline Reichweite: 4 Schritt\newline Wirkungsdauer: augenblicklich\newline Kosten: 8 AsP\newline Fertigkeiten: Eigenschaften\newline Erlernen: Ach, Hex 16; 40 EP}
}


\newglossaryentry{hexenkrallen_Talent}
{
    name={Hexenkrallen},
    description={Deine Fingernägel werden lang, scharf und hart wie Raubtierklauen. Deine Hände richten 2W6 Waffenschaden an und verlieren die Eigenschaft Zerbrechlich.\newline Mächtige Magie: Erhöht den Schaden um +2.\newline Probenschwierigkeit: 12\newline Vorbereitungszeit: 0 Aktionen\newline Ziel: selbst\newline Reichweite: Berührung\newline Wirkungsdauer: 16 Initiativphasen\newline Kosten: 8 AsP\newline Fertigkeiten: Eigenschaften, Verwandlung\newline Erlernen: Hex 18; 20 EP}
}


\newglossaryentry{katzenaugen_Talent}
{
    name={Katzenaugen},
    description={Du ignorierst eine Stufe Dunkelheit, aber grelles Licht erschwert alle Proben um –2. Bei absoluter Dunkelheit ist der Zauber wirkungslos.\newline Mächtige Magie: Du ignorierst 2/3 Stufen Dunkelheit.\newline Probenschwierigkeit: 12\newline Vorbereitungszeit: 4 Aktionen\newline Ziel: Einzelperson\newline Reichweite: Berührung\newline Wirkungsdauer: 1 Stunde\newline Kosten: 4 AsP\newline Fertigkeiten: Eigenschaften\newline Erlernen: Hex 16; Elf 18; 20 EP}
}


\newglossaryentry{krötensprung_Talent}
{
    name={Krötensprung},
    description={Der nächste Sprung deines Zieles ist gewaltig. Er kann bis zu 8 Schritt Weite und 4 Schritt Höhe überwinden und die effektive Sturzhöhe (S. 34) sinkt um 4 Schritt.\newline Mächtige Magie: Die Weite steigt um +4 Schritt,  die (Sturz-)Höhe um +2 Schritt.\newline Probenschwierigkeit: 12\newline Modifikationen: Krötengang (–4, Wirkungsdauer 16 Initiativphasen, 8 AsP; dein Ziel kann beliebig oft springen.)\newline Vorbereitungszeit: 2 Aktionen\newline Ziel: Einzelperson\newline Reichweite: Berührung\newline Wirkungsdauer: 16 Initiativphasen\newline Kosten: 4 AsP\newline Fertigkeiten: Eigenschaften\newline Erlernen: Hex 16; Ach, Elf 18; 20 EP}
}


\newglossaryentry{memoransGedächtniskraft_Talent}
{
    name={Memorans Gedächtniskraft},
    description={Du kannst dir sämtliche Bilder, Schriftstücke, Inschriften und ähnliches für immer einprägen. Eine Seite eines Buches kostet dich etwa eine Minute.\newline Probenschwierigkeit: 12\newline Modifikationen: Drachengedächtnis (–8, Wirkungsdauer 16 Initiativphasen; alles, was du während der Wirkungsdauer siehst, ist dir für immer ins Gedächtnis gebrannt.)\newline Vorbereitungszeit: 4 Minuten\newline Ziel: selbst\newline Reichweite: 4 Schritt\newline Wirkungsdauer: 4 Minuten\newline Kosten: 8 AsP\newline Fertigkeiten: Eigenschaften, Hellsicht\newline Erlernen: Alch, Mag 16; Dru 20; 40 EP}
}


\newglossaryentry{movimentoDauerlauf_Talent}
{
    name={Movimento Dauerlauf},
    description={Verdoppelt das DH* (S. 34) des Ziels und das Intervall, in dem körperliche Anstrengung Erschöpfung verursacht.\newline Mächtige Magie: Je zwei Stufen verdreifachen/vervierfachen das DH* und das Intervall.\newline Probenschwierigkeit: 12\newline Vorbereitungszeit: 8 Aktionen\newline Ziel: Einzelwesen\newline Reichweite: Berührung\newline Wirkungsdauer: 8 Stunden\newline Kosten: 8 AsP\newline Fertigkeiten: Eigenschaften\newline Erlernen: Elf 8; Alch, Dru, Hex, Mag, Sch 18; 20 EP}
}


\newglossaryentry{plumbumbarumschwererArm_Talent}
{
    name={Plumbumbarum schwerer Arm},
    description={Alle AT des Ziels sind um –4 erschwert.\newline Mächtige Magie: Erhöht den Malus um –2.\newline Probenschwierigkeit: Magieresistenz\newline Modifikationen: Demotivation (–8; die Erschwernis wirkt auch auf Zauberproben.)\newline Vorbereitungszeit: 1 Aktion\newline Ziel: Einzelwesen\newline Reichweite: 8 Schritt\newline Wirkungsdauer: 8 Initiativphasen\newline Kosten: 4 AsP\newline Fertigkeiten: Eigenschaften\newline Erlernen: Mag, Dru, Geo, Hex 12; Ach, Alch, Elf, Sch 16; 20 EP}
}


\newglossaryentry{satuariasHerrlichkeit_Talent}
{
    name={Satuarias Herrlichkeit},
    description={Dein Aussehen weckt Begehren in allen an deinem Geschlecht und deiner Spezies interessierten Individuen. Ihnen gegenüber sind Betören-Proben um +4 erleichtert.\newline Mächtige Magie: Der Bonus steigt um +2.\newline Probenschwierigkeit: 12\newline Vorbereitungszeit: 4 Aktionen\newline Ziel: selbst\newline Reichweite: Berührung\newline Wirkungsdauer: 1 Stunde\newline Kosten: 8 AsP\newline Fertigkeiten: Eigenschaften, Illusion\newline Erlernen: Hex 12; 20 EP}
}


\newglossaryentry{seelenwanderung_Talent}
{
    name={Seelenwanderung},
    description={Wenn du ein Körperteil (z.B. Haare oder Blut) deines Opfers besitzt, kannst du mit ihm den Körper tauschen. Geistige Attribute, Fertigkeiten und Vorteile bleiben erhalten, körperliche werden getauscht. Wird einer der Körper während des Tausches vernichtet, endet der Tausch und der ursprüngliche Besitzer des vernichteten Körpers stirbt.\newline Probenschwierigkeit: Magieresistenz\newline Modifikationen: Tiersinne (–4, Einzelwesen) Fremdtausch (–8; zwei Ziele tauschen Körper.)\newline Vorbereitungszeit: 16 Aktionen\newline Ziel: Einzelperson\newline Reichweite: 8 Meilen\newline Wirkungsdauer: 4 Minuten\newline Kosten: 16 AsP\newline Fertigkeiten: Eigenschaften, Verständigung\newline Erlernen: Ach, Dru, Geo, Hex 20; 60 EP\newline Anmerkungen: Gerüchten zufolge gibt es eine Möglichkeit, mit der der Zaubernde den Körper seines Zieles permanent übernehmen kann.}
}


\newglossaryentry{sensattacoMeisterstreich_Talent}
{
    name={Sensattaco Meisterstreich},
    description={Dein Ziel erkennt intuitiv die Lücken in der Verteidigung des Gegners. Seine AT sind um +2 erleichtert und seine Chance auf einen Triumph bei einer AT steigt um 1 auf dem W20 (zum Beispiel von 20 auf 19–20).\newline Mächtige Magie: Der Bonus steigt um +1.\newline Probenschwierigkeit: 12\newline Vorbereitungszeit: 1 Aktion\newline Ziel: Einzelperson\newline Reichweite: Berührung\newline Wirkungsdauer: 16 Initiativphasen\newline Kosten: 8 AsP\newline Fertigkeiten: Eigenschaften, Hellsicht\newline Erlernen: Mag 18; 40 EP}
}


\newglossaryentry{spinnenlauf_Talent}
{
    name={Spinnenlauf},
    description={Deine Hände und Füße haften an Oberflächen, sodass du mit GS 1 an glatten Wänden und sogar Decken ohne Griffen klettern kannst. Erlaubt Aufrechterhalten.\newline Mächtige Magie: Erhöht die GS um +1.\newline Probenschwierigkeit: 12\newline Vorbereitungszeit: 4 Aktionen\newline Ziel: Einzelperson\newline Reichweite: Berührung\newline Wirkungsdauer: 4 Minuten\newline Kosten: 8 AsP\newline Fertigkeiten: Eigenschaften\newline Erlernen: Hex 16; Ach, Elf 18; 40 EP}
}


\newglossaryentry{standfestKatzengleich_Talent}
{
    name={Standfest Katzengleich},
    description={Deine Geschicklichkeit erhöht sich. Alle Proben, um auf den Beinen zu bleiben, sind um +4 erleichtert und deine Patzerchance im Kampf sinkt um 1 auf dem W20.\newline Mächtige Magie: Der Bonus steigt um +2.\newline Probenschwierigkeit: 12\newline Vorbereitungszeit: 2 Aktionen\newline Ziel: selbst\newline Reichweite: Berührung\newline Wirkungsdauer: 16 Initiativphasen\newline Kosten: 4 AsP\newline Fertigkeiten: Eigenschaften\newline Erlernen: Elf 14; Ach 20; 40 EP}
}


\newglossaryentry{warmesBlut_Talent}
{
    name={Warmes Blut},
    description={Durch diesen Zauber siehst du die Wärmestrahlung deiner Umgebung. Kaltes erscheint schwarz bis grünblau, Warmblüter sind gelb und Feuer ist orange bis tiefrot.\newline Probenschwierigkeit: 12\newline Vorbereitungszeit: 4 Aktionen\newline Ziel: selbst\newline Reichweite: Berührung\newline Wirkungsdauer: 1 Stunde\newline Kosten: 4 AsP\newline Fertigkeiten: Eigenschaften, Feuer, Hellsicht\newline Erlernen: Ach 8; Alch 20; 20 EP}
}


\newglossaryentry{wellenlauf_Talent}
{
    name={Wellenlauf},
    description={Wasser ist für dich ein fester Untergrund. Du erleidest keine Abzüge durch ungünstige Position und wirst nicht durch Wellen und Strömungen beeinflusst. Erlaubt Aufrechterhalten.\newline Probenschwierigkeit: 12\newline Modifikationen: Sinken (–4; du kannst die Zauberwirkung nach Belieben unterdrücken und wieder aktiv werden lassen. Bei Aktivierung unter Wasser wirst du an die Wasseroberfläche gehoben.)\newline Wasserwand (–8; du kannst selbst an Wasserfällen hochklettern, wofür einfache Klettern-Proben anfallen.)\newline Vorbereitungszeit: 2 Aktionen\newline Ziel: selbst\newline Reichweite: Berührung\newline Wirkungsdauer: 4 Minuten\newline Kosten: 4 AsP\newline Fertigkeiten: Eigenschaften, Wasser\newline Erlernen: Dru, Geo 12; Elf 14; Ach, Mag 20; 20 EP}
}


\newglossaryentry{wipfellauf_Talent}
{
    name={Wipfellauf},
    description={Du kannst dich durch Baumkronen und Unterholz bewegen, als wären sie eine normale Straße. Unter diesen Bedingungen erleidest du keinerlei Erschwernisse oder Geschwindigkeitsabzüge.\newline Probenschwierigkeit: 12\newline Modifikationen: Kopfüber (–8; du kannst selbst kopfüber an Bäumen laufen.)\newline Vorbereitungszeit: 2 Aktionen\newline Ziel: selbst\newline Reichweite: Berührung\newline Wirkungsdauer: 1 Stunde\newline Kosten: 16 AsP\newline Fertigkeiten: Eigenschaften, Humus\newline Erlernen: Elf 16; Ach, Dru, Geo, Mag 20; 40 EP}
}


\newglossaryentry{zaubernahrungHungerbann_Talent}
{
    name={Zaubernahrung Hungerbann},
    description={Für einen Tag spürst du keinerlei Hunger und die Kraft des Zaubers ernährt dich. Du musst spätestens vor vier Tagen echtes Essen zu dir genommen haben.\newline Mächtige Magie: Die letzte echte Mahlzeit kann bis zu zwei Tage länger her sein.\newline Probenschwierigkeit: 12\newline Modifikationen: Durstbann (–4, 8 AsP; auch jeglicher Durst wird vom Zauber gestillt.)\newline Vorbereitungszeit: 4 Minuten\newline Ziel: selbst\newline Reichweite: Berührung\newline Wirkungsdauer: 1 Tag\newline Kosten: 4 AsP\newline Fertigkeiten: Eigenschaften, Einfluss\newline Erlernen: Elf 18; 20 EP}
}


\newglossaryentry{zungelähmen_Talent}
{
    name={Zunge lähmen},
    description={Dein Opfer kann seine Zunge nicht mehr bewegen und kann keine verständlichen Äußerungen von sich geben. Wenn ein Zauber oder eine Liturgie eine gesprochene Formel oder Gebet benötigt, muss das Ziel die Zaubertechnik ignorieren.\newline Probenschwierigkeit: Magieresistenz\newline Vorbereitungszeit: 2 Aktionen\newline Ziel: Einzelperson\newline Reichweite: 8 Schritt\newline Wirkungsdauer: 1 Stunde\newline Kosten: 8 AsP\newline Fertigkeiten: Eigenschaften\newline Erlernen: Dru 14; Geo 16; Alch, Hex, Sch 18; 20 EP}
}


\newglossaryentry{alpgestalt_Talent}
{
    name={Alpgestalt},
    description={Du erscheinst deinem Opfer als grauenvolle Gestalt aus den Niederhöllen. Es kann nicht wegsehen und ist vor Angst handlungsunfähig. Wird der Sichtkontakt unterbrochen, endet der Zauber.\newline Probenschwierigkeit: Magieresistenz\newline Modifikationen: Fremdgestalt (–4; du kannst eine andere Person wählen, die als Alpgestalt erscheint.)\newline Vorbereitungszeit: 4 Aktionen\newline Ziel: Einzelwesen\newline Reichweite: 8 Schritt\newline Wirkungsdauer: 4 Minuten\newline Kosten: 16 AsP\newline Fertigkeiten: Einfluss\newline Erlernen: Dru 18; 40 EP}
}


\newglossaryentry{ängstelindern_Talent}
{
    name={Ängste lindern},
    description={Auf dem Ziel lastende Furcht-Effekte sinken um eine Stufe.\newline Probenschwierigkeit: 12\newline Vorbereitungszeit: 4 Aktionen\newline Ziel: Einzelperson\newline Reichweite: Berührung\newline Wirkungsdauer: augenblicklich\newline Kosten: 4 AsP\newline Fertigkeiten: Einfluss\newline Erlernen: Hex 8; Geo 14; Dru, Mag 16; Alch 18; 40 EP}
}


\newglossaryentry{bandundFessel_Talent}
{
    name={Band und Fessel},
    description={Dein Opfer kann einen kreisförmigen Bereich von 4 Schritt Radius nicht verlassen. Alle zwei Stunden darf es versuchen, sich mit einer Konterprobe (MU, 16) vom Zauber zu befreien. Dabei handelt es sich um eine mentale Barriere. Wird das Ziel mit Gewalt über die Barriere getragen – wogegen es sich wehren wird – endet der Effekt.\newline Probenschwierigkeit: Magieresistenz\newline Vorbereitungszeit: 4 Aktionen\newline Ziel: Einzelwesen\newline Reichweite: 8 Schritt\newline Wirkungsdauer: 8 Stunden\newline Kosten: 8 AsP\newline Fertigkeiten: Einfluss\newline Erlernen: Elf 14; Hex 16; Dru, Mag 18; Ach, Bor, Geo 20; 40 EP}
}


\newglossaryentry{bannbaladin_Talent}
{
    name={Bannbaladin},
    description={In Rededuellen mit dem Ziel bist du immun gegen die Auswirkungen von ungewohnter Umgebung (S. 55).\newline Probenschwierigkeit: Magieresistenz\newline Mächtige Magie: Für je zwei Stufen gilt dein Vorhaben in einem Rededuell als um eine Stufe ungefährlicher (S. 57).\newline Modifikationen: Gemeinsame Erinnerung (–4; du teilst eine von dir bestimmte positive Erinnerung mit dem Ziel.)\newline Vorbereitungszeit: 2 Aktionen\newline Ziel: Einzelperson\newline Reichweite: 4 Schritt\newline Wirkungsdauer: 1 Stunde\newline Kosten: 8 AsP\newline Fertigkeiten: Einfluss\newline Erlernen: Elf 8; Mag 12; Alch, Dru, Hex, Srl 18; 40 EP}
}


\newglossaryentry{blitzdichfind_Talent}
{
    name={Blitz dich find},
    description={Dein Ziel ist geblendet und erleidet eine Erschwernis von –2 auf Proben.\newline Mächtige Magie: Erhöht den Malus um –1.\newline Probenschwierigkeit: Magieresistenz\newline Vorbereitungszeit: 0 Aktionen\newline Ziel: Einzelwesen\newline Reichweite: 8 Schritt\newline Wirkungsdauer: 4 Initiativphasen\newline Kosten: 4 AsP\newline Fertigkeiten: Einfluss\newline Erlernen: Elf 8; Mag 12; Dru, Hex 14; Ach, Geo, Sch, Srl 16; 40 EP}
}


\newglossaryentry{böserBlick_Talent}
{
    name={Böser Blick},
    description={Dein Opfer hat schreckliche Angst vor dir und erleidet einen Furcht-Effekt Stufe 2.\newline Mächtige Magie: Der Furcht-Effekt steigt um eine Stufe.\newline Probenschwierigkeit: Magieresistenz\newline Vorbereitungszeit: 0 Aktionen\newline Ziel: Einzelwesen\newline Reichweite: 8 Schritt\newline Wirkungsdauer: 1 Stunde\newline Kosten: 8 AsP\newline Fertigkeiten: Einfluss\newline Erlernen: Dru 8; Geo 12; Ach 14; Hex 20; 40 EP}
}


\newglossaryentry{dichterundDenker_Talent}
{
    name={Dichter und Denker},
    description={Dein Opfer kann sich nur noch in Reimen ausdrücken. Zauber können nur noch gewirkt werden, wenn ihm ein passender Reim auf den Zaubernamen einfällt.\newline Probenschwierigkeit: Magieresistenz\newline Vorbereitungszeit: 2 Aktionen\newline Ziel: Einzelperson\newline Reichweite: 4 Schritt\newline Wirkungsdauer: 1 Stunde\newline Kosten: 8 AsP\newline Fertigkeiten: Einfluss\newline Erlernen: Sch 20; 20 EP}
}


\newglossaryentry{großeGier_Talent}
{
    name={Große Gier},
    description={Du erweckst im Ziel ein dringendes Bedürfnis nach einem Gegenstand, einer Handlung oder einem Ort. Das Opfer wird aber keine großen Risiken eingehen, um das Bedürfnis zu erfüllen. Der Zauber ist um 2 bis 4 Punkte erleichtert, wenn er eine Charakterschwäche des Ziels besonders anspricht.\newline Mächtige Magie: Das Ziel nimmt Schmerzen oder eine peinliche Situation/kleinere Verletzungen oder einen momentanen Gesichtsverlust/schwere Verletzungen oder dauerhaften Verlust seines Rufes/jedes Risiko auf sich.\newline Probenschwierigkeit: Magieresistenz\newline Vorbereitungszeit: 1 Aktion\newline Ziel: Einzelperson\newline Reichweite: Berührung\newline Wirkungsdauer: 1 Stunde\newline Kosten: 8 AsP\newline Fertigkeiten: Einfluss\newline Erlernen: Hex 8; Geo 16; 40 EP}
}


\newglossaryentry{halluzination_Talent}
{
    name={Halluzination},
    description={Deinem Opfer erscheint eine Halluzination deiner Wahl („ein Oger“, „köstlicher Tharf“), die es mit allen fünf Sinnen wahrzunehmen glaubt. Wird das Opfer von der Halluzination „getötet“, fällt es für 16 Initiativphasen in Ohnmacht. Nach dem Erwachen sind alle durch die Halluzination verursachten Wunden verschwunden.\newline Probenschwierigkeit: Magieresistenz\newline Modifikationen: Traumbilder (–8; die Halluzination erscheint im Traum eines Schlafenden.)\newline Vorbereitungszeit: 2 Aktionen\newline Ziel: Einzelperson\newline Reichweite: Berührung\newline Wirkungsdauer: 4 Minuten\newline Kosten: 8 AsP\newline Fertigkeiten: Einfluss\newline Erlernen: Dru 14; Geo 12; Hex, Mag, Srl 20; 40 EP}
}


\newglossaryentry{harmloseGestalt_Talent}
{
    name={Harmlose Gestalt},
    description={Mit dieser Illusion (Sicht, Gehör, Geruch) erscheinst du den Umstehenden als kleines Kind, alter Krüppel, Orkfrau oder als eine andere harmlose Gestalt. Deine Kleidung ist von der Illusion betroffen, zusätzliche Gegenstände wie ein Wanderstab oder ein Rucksack nicht. Erlaubt Aufrechterhalten.\newline Probenschwierigkeit: 12\newline Vorbereitungszeit: 4 Aktionen\newline Ziel: selbst\newline Reichweite: Berührung\newline Wirkungsdauer: 4 Minuten\newline Kosten: 4 AsP\newline Fertigkeiten: Einfluss, Illusion\newline Erlernen: Hex 8; Srl 14; Ach, Bor, Dru, Mag, Sch 16; Elf 18; 40 EP}
}


\newglossaryentry{herrüberdasTierreich_Talent}
{
    name={Herr über das Tierreich},
    description={Du zwingst ein Tier unter deinen Bann. Es verhält sich, als wäre es gut dressiert und dir bedingungslos loyal und erfüllt deine Befehle. Nur Befehle, die seinen Instinkten klar widersprechen, kann es mit einer Konterprobe (MU, 16) widerstehen.\newline Probenschwierigkeit: Magieresistenz\newline Modifikationen: Herr der Fliegen (–4; du beherrschst einen Schwarm Kleintiere.)\newline Vorbereitungszeit: 2 Aktionen\newline Ziel: Tier\newline Reichweite: 4 Schritt\newline Wirkungsdauer: 1 Stunde\newline Kosten: 8 AsP\newline Fertigkeiten: Einfluss\newline Erlernen: Dru 8; Geo 12; Elf, Hex 18; Ach, Mag 20; 40 EP}
}


\newglossaryentry{hexenknoten_Talent}
{
    name={Hexenknoten},
    description={Wenn den Umstehenden keine Konterprobe (MR, 16) gelingt, erscheint ihnen der Zauber als eine bis zu 4 Schritt lange, furchteinflößende Barriere, die sie keinesfalls durchschreiten wollen. Das Aussehen der Barriere bestimmt die Hexe.\newline Mächtige Magie: Erhöht die Länge der Barriere um 4 Schritt.\newline Probenschwierigkeit: 12\newline Vorbereitungszeit: 4 Aktionen\newline Ziel: Zone\newline Reichweite: 4 Schritt\newline Wirkungsdauer: 4 Minuten\newline Kosten: 8 AsP\newline Fertigkeiten: Einfluss, Illusion\newline Erlernen: Hex 8; Dru, Mag, Srl 20; 40 EP}
}


\newglossaryentry{hilfreicheTatze,rettendeSchwinge_Talent}
{
    name={Hilfreiche Tatze, rettende Schwinge},
    description={Du wählst eine Tierart. Befindet sich ein Tier dieser Art in einem Radius von 1 Meile, eilt es herbei. Du kannst das Tier um einen Gefallen bitten, den dieses wenn möglich erfüllen wird. Dann trollt sich das Tier.\newline Mächtige Magie: Verdoppelt den Radius.\newline Probenschwierigkeit: Magieresistenz\newline Vorbereitungszeit: 8 Aktionen\newline Ziel: Zone\newline Reichweite: Berührung\newline Wirkungsdauer: 1 Stunde\newline Kosten: 8 AsP\newline Fertigkeiten: Einfluss, Verständigung\newline Erlernen: Geo 14; Elf 16; Ach, Dru 18; 20 EP}
}


\newglossaryentry{horriphobusSchreckgestalt_Talent}
{
    name={Horriphobus Schreckgestalt},
    description={Dein Opfer hat schreckliche Angst vor dir und erleidet einen Furcht-Effekt Stufe 2.\newline Mächtige Magie: Der Furcht-Effekt steigt um eine Stufe.\newline Probenschwierigkeit: Magieresistenz\newline Vorbereitungszeit: 0 Aktionen\newline Ziel: Einzelwesen\newline Reichweite: 8 Schritt\newline Wirkungsdauer: 1 Stunde\newline Kosten: 8 AsP\newline Fertigkeiten: Einfluss\newline Erlernen: Mag 12; Bor 16; Alch 18; 40 EP}
}


\newglossaryentry{ignorantiaUngesehen_Talent}
{
    name={Ignorantia Ungesehen},
    description={Zufällige Beobachter bemerken dich nicht und Umstehende verlieren das Interesse. Wenn jemand allein mit dir ist oder sich konzentriert, muss ihm eine Konterprobe (Wachsamkeit, 16) gelingen, um dich bewusst wahrzunehmen. Erlaubt Aufrechterhalten.\newline Probenschwierigkeit: 12\newline Vorbereitungszeit: 4 Aktionen\newline Ziel: selbst\newline Reichweite: Berührung\newline Wirkungsdauer: 16 Minuten\newline Kosten: 8 AsP\newline Fertigkeiten: Einfluss, Illusion\newline Erlernen: Srl 16; Mag 18; 40 EP}
}


\newglossaryentry{imperaviHandlungszwang_Talent}
{
    name={Imperavi Handlungszwang},
    description={Dein Ziel muss einen einzigen Befehl von dir ausführen und darf währenddessen nicht gegen dich vorgehen. Widerspricht der Befehl den tiefsten Überzeugungen oder dem Selbsterhaltungstrieb des Zieles, kann es mit einer Konterprobe (Willenskraft, 16) widerstehen. Eine Verschachtelung von Befehlen („Folge allen weiteren Befehlen!“) ist nicht möglich.\newline Probenschwierigkeit: Magieresistenz\newline Modifikationen: Sofort (–4, 0 Aktionen, 8 AsP; das Ziel muss einen kurzen Befehl sofort ausführen.)\newline Später (–4; der Zauber wirkt erst, sobald ein einfacher Auslöser eintritt.)\newline Dauerhafter Diener (–4, 4 Minuten, Berührung, Wirkungsdauer 16 Tage, 32 AsP; du kannst bis zu 16 Befehle äußern.)\newline Vorbereitungszeit: 4 Aktionen\newline Ziel: Einzelperson\newline Reichweite: 2 Schritt\newline Wirkungsdauer: 8 Stunden oder bis der Befehl erfüllt ist\newline Kosten: 16 AsP\newline Fertigkeiten: Einfluss\newline Erlernen: Mag, Dru 16; Bor 18; 60 EP}
}


\newglossaryentry{juckreiz,dämlicher!_Talent}
{
    name={Juckreiz, dämlicher!},
    description={Die Person mit der niedrigsten KL im Radius von 8 Schritt wird von einem heftigen Juckreiz befallen, insbesondere dort, wo es sich nicht kratzen kann. Erschwernisse sind Spielleiterentscheid.\newline Probenschwierigkeit: Magieresistenz\newline Vorbereitungszeit: 2 Aktionen\newline Ziel: Zone\newline Reichweite: Berührung\newline Wirkungsdauer: 16 Minuten\newline Kosten: 4 AsP\newline Fertigkeiten: Einfluss\newline Erlernen: Sch 18; 20 EP}
}


\newglossaryentry{koboldgeschenk_Talent}
{
    name={Koboldgeschenk},
    description={Du überreichst deinem Opfer einen maximal faustgroßen Gegenstand, der für ihn als ähnlich großer Gegenstand deiner Wahl erscheint. Das Opfer sucht selbst Erklärungen für das Verhalten des Gegenstandes – so wird zum Beispiel ein wegspringender Frosch zu einem wegrollenden Edelstein.\newline Probenschwierigkeit: Magieresistenz\newline Vorbereitungszeit: 2 Aktionen\newline Ziel: Einzelperson\newline Reichweite: Berührung\newline Wirkungsdauer: 16 Minuten\newline Kosten: 8 AsP\newline Fertigkeiten: Einfluss\newline Erlernen: Sch 8; 20 EP}
}


\newglossaryentry{krähenruf_Talent}
{
    name={Krähenruf },
    description={Du rufst einen Krähenschwarm aus bis zu 100 Meilen Entfernung zur Hilfe, der sofort erscheint und an deiner Seite kämpft (WS 3, Koloss I, INI 6, GS 8, VT 3, RW 2, AT 10, TP 2W6–2, Zusätzliche AT I).\newline Mächtige Magie: WS, AT und TP des Schwarms steigen um je +1.\newline Probenschwierigkeit: 12\newline Vorbereitungszeit: 2 Aktionen\newline Ziel: Zone\newline Reichweite: Berührung\newline Wirkungsdauer: 16 Initiativphasen\newline Kosten: 8 AsP\newline Fertigkeiten: Einfluss, Kraft, Verständigung\newline Erlernen: Hex 12; 40 EP}
}


\newglossaryentry{kusch!_Talent}
{
    name={Kusch!},
    description={Das verzauberte Tier flieht vor dir.\newline Probenschwierigkeit: Magieresistenz\newline Modifikationen: Schrecken des Schwarms (–4; du verscheuchst einen Schwarm Kleintiere.)\newline Vorbereitungszeit: 0 Aktionen\newline Ziel: einzelnes Tier\newline Reichweite: 8 Schritt\newline Wirkungsdauer: 1 Stunde\newline Kosten: 8 AsP\newline Fertigkeiten: Einfluss\newline Erlernen: Geo, Sch 16; 20 EP}
}


\newglossaryentry{lachdichgesund_Talent}
{
    name={Lach dich gesund},
    description={Du erzählst deinem Ziel einen hervorragenden Witz, der es in einen kichernden Rauschzustand versetzt. Währenddessen erhält es 2W6+4 Heilpunkte. Für jede Überschreitung der WS wird eine Wunde geheilt. Das Ziel muss dir vertrauen und du kannst dich nicht selbst mit dem Zauber belegen.\newline Mächtige Magie: Erhöht die Heilpunkte um 4.\newline Probenschwierigkeit: 12+Wund-Mod des Ziels\newline Modifikationen: Tanz dich wach (–4; du tanzt gemeinsam mit deinem Ziel, das statt Wunden Erschöpfung regeneriert.)\newline Vorbereitungszeit: 8 Aktionen\newline Ziel: Einzelperson\newline Reichweite: Berührung\newline Wirkungsdauer: 4 Minuten\newline Kosten: 8 AsP\newline Fertigkeiten: Einfluss, Humus\newline Erlernen: Sch 14; 40 EP}
}


\newglossaryentry{lachkrampf_Talent}
{
    name={Lachkrampf},
    description={Dein Opfer bekommt einen schweren Lachkrampf, der es von jeglicher Beteiligung in Rededuellen abhält. Nach 1 Minute darf es mit einer Konterprobe (Willenskraft, 16) versuchen, den Zauber zu beenden.\newline Probenschwierigkeit: Magieresistenz\newline Modifikationen: Schluckauf (2 AsP; dein Opfer erhält einen kräftigen Schluckauf. Auswirkungen sind Spielleiterentscheid.)\newline Blähungen (–4; alle Proben zur gesellschaftlichen Interaktion sind um –4 erschwert.)\newline Vorbereitungszeit: 0 Aktionen\newline Ziel: Einzelperson\newline Reichweite: 4 Schritt\newline Wirkungsdauer: 16 Minuten\newline Kosten: 8 AsP\newline Fertigkeiten: Einfluss\newline Erlernen: Sch 8; 20 EP}
}


\newglossaryentry{levthansFeuer_Talent}
{
    name={Levthans Feuer},
    description={Das Ziel entbrennt in heißer Leidenschaft für dich. Auf einer Skala von abstoßend/uninteressant/neutral/begehrenswert/unwiderstehlich steigt seine Einstellung dir gegenüber um eine Stufe. In einem folgenden Liebesspiel kannst du deine gesamte Wund- und AsP-Regeneration auf das Ziel übertragen oder die gesamte Regeneration des Ziels stehlen.\newline Mächtige Magie: Steigert die Einstellung um eine weitere Stufe.\newline Probenschwierigkeit: Magieresistenz\newline Vorbereitungszeit: 2 Aktionen\newline Ziel: Einzelperson\newline Reichweite: Berührung\newline Wirkungsdauer: 4 Stunden\newline Kosten: 8 AsP\newline Fertigkeiten: Einfluss, Verständigung\newline Erlernen: Hex 16; 20 EP}
}


\newglossaryentry{memorabiaFalsifir_Talent}
{
    name={Memorabia Falsifir},
    description={Dein Opfer verdrängt sämtliche Erinnerungen an einen von dir bestimmten Zeitraum von maximal 1 Stunde. Seitdem darf maximal 1 Stunde vergangen sein.\newline Mächtige Magie: Der Zeitraum und die maximal vergangene Zeit erhöhen sich auf 1 Tag/1 Woche/1 Monat/1 Jahr.\newline Probenschwierigkeit: Magieresistenz\newline Modifikationen: Falsche Erinnerung (–4; du kannst deinem Opfer eine neue Erinnerung grob vorgeben. Die Details füllt sein Unterbewusstsein aus.)\newline Tilgung von Wissen (–4; du kannst wählen, welche Ereignisse vergessen werden und welche nicht.)\newline Permanenz (–4, Wirkungsdauer bis die Bindung gelöst wird, 16 AsP, davon 2 gAsP)\newline Vorbereitungszeit: 4 Minuten\newline Ziel: Einzelperson\newline Reichweite: Berührung\newline Wirkungsdauer: 1 Woche\newline Kosten: 16 AsP\newline Fertigkeiten: Einfluss\newline Erlernen: Elf, Mag 18; Ach, Dru, Hex 20; 60 EP}
}


\newglossaryentry{papperlapapp_Talent}
{
    name={Papperlapapp},
    description={Jeder, der sich dir auf 4 Schritt nähert, muss eine Konterprobe (Magieresistenz, 12) ablegen. Misslingt sie, spricht er für den Rest der Wirkungsdauer nur noch in einer Phantasiesprache. Er würfelt 1W20 und kann sich nur noch mit jenen unterhalten, die ebenfalls dem Zauber unterliegen und das gleiche Ergebnis gewürfelt haben.\newline Probenschwierigkeit: 12\newline Vorbereitungszeit: 4 Aktionen\newline Ziel: selbst\newline Reichweite: Berührung\newline Wirkungsdauer: 1 Stunde\newline Kosten: 16 AsP\newline Fertigkeiten: Einfluss\newline Erlernen: Sch 20; 40 EP}
}


\newglossaryentry{respondamiWahrheitszwang_Talent}
{
    name={Respondami Wahrheitszwang},
    description={Das Ziel muss eine Ja/Nein-Frage wahrheitsgemäß beantworten. Ist die Frage nicht mit Ja oder Nein zu beantworten, erleidet es 1W6 SP.\newline Mächtige Magie: Erlaubt eine weitere Frage.\newline Probenschwierigkeit: Magieresistenz\newline Vorbereitungszeit: 1 Aktion\newline Ziel: Einzelperson\newline Reichweite: 2 Schritt\newline Wirkungsdauer: 4 Minuten\newline Kosten: 4 AsP\newline Fertigkeiten: Einfluss\newline Erlernen: Dru, Geo, Mag 14; Elf, Hex 16; Alch 18; 40 EP}
}


\newglossaryentry{sanftmut_Talent}
{
    name={Sanftmut},
    description={Das verzauberte Tier verliert seine Angriffslust, solange es nicht angegriffen oder gereizt wird.\newline Mächtige Magie: Die Lethargie hält sogar dann, wenn das Tier gereizt/leicht verletzt/schwer verletzt wird.\newline Probenschwierigkeit: Magieresistenz\newline Vorbereitungszeit: 1 Aktion\newline Ziel: Tier\newline Reichweite: 8 Schritt\newline Wirkungsdauer: 4 Minuten\newline Kosten: 8 AsP\newline Fertigkeiten: Einfluss\newline Erlernen: Geo, Hex 12; Dru, Elf 16; Ach 18; 20 EP}
}


\newglossaryentry{schabernack_Talent}
{
    name={Schabernack},
    description={Deinem Opfer geschieht ein kleines Missgeschick – es stolpert, bekommt im Gespräch plötzlich Blähungen, oder ihm rutscht eine unanständige Bemerkung heraus.\newline Probenschwierigkeit: Magieresistenz\newline Vorbereitungszeit: 2 Aktionen\newline Ziel: Einzelperson\newline Reichweite: 8 Schritt\newline Wirkungsdauer: augenblicklich\newline Kosten: 4 AsP\newline Fertigkeiten: Einfluss\newline Erlernen: Sch 8; 20 EP}
}


\newglossaryentry{schelmenlaune_Talent}
{
    name={Schelmenlaune},
    description={Deine Stimmung überträgt sich auf jede Person in einem Umreis von 8 Schritt, der keine Konterprobe (Magieresistenz, 12) gelingt. Da Schelme grundsätzlich positiv gestimmt sind, sorgt dies meist für Massenhysterien.\newline Probenschwierigkeit: 12\newline Vorbereitungszeit: 4 Aktionen\newline Ziel: selbst\newline Reichweite: Berührung\newline Wirkungsdauer: 16 Minuten\newline Kosten: 16 AsP\newline Fertigkeiten: Einfluss\newline Erlernen: Sch 20; 20 EP}
}


\newglossaryentry{schelmenrausch_Talent}
{
    name={Schelmenrausch},
    description={Du versetzt dein Ziel in einen friedvoll-glücklichen Rauschzustand. Alle seine Proben sind um –4 erschwert. Das Ziel erwacht aus der Trance, wenn es in direkter Lebensgefahr schwebt.\newline Probenschwierigkeit: Magieresistenz\newline Vorbereitungszeit: 2 Aktionen\newline Ziel: Einzelperson\newline Reichweite: 4 Schritt\newline Wirkungsdauer: 4 Minuten\newline Kosten: 8 AsP\newline Fertigkeiten: Einfluss\newline Erlernen: Sch 14; 40 EP}
}


\newglossaryentry{schwarzerSchrecken_Talent}
{
    name={Schwarzer Schrecken},
    description={Das Opfer leidet an einer Angst vor einer Farbe oder Form deiner Wahl, meist wird die Farbe Schwarz gewählt. Wird die Angst ausgelöst, steht es unter einem Furcht-Effekt Stufe 1.\newline Mächtige Magie: Der Furcht-Effekt steigt um eine Stufe.\newline Probenschwierigkeit: Magieresistenz\newline Vorbereitungszeit: 2 Aktionen\newline Ziel: Einzelperson\newline Reichweite: 8 Schritt\newline Wirkungsdauer: 1 Woche\newline Kosten: 8 AsP\newline Fertigkeiten: Einfluss\newline Erlernen: Bor 12; Mag 18; 20 EP}
}


\newglossaryentry{seidenzungeElfenwort_Talent}
{
    name={Seidenzunge Elfenwort},
    description={Das Ziel denkt nicht zu genau über deine Worte nach und findet dich überzeugend. Alle Überreden-Proben gegen dein Ziel sind um +4 erleichtert.\newline Mächtige Magie: Der Bonus steigt um +2.\newline Probenschwierigkeit: Magieresistenz\newline Vorbereitungszeit: 1 Aktion\newline Ziel: Einzelperson\newline Reichweite: 4 Schritt\newline Wirkungsdauer: 16 Initiativphasen\newline Kosten: 8 AsP\newline Fertigkeiten: Einfluss\newline Erlernen: Elf 18; Sch 20; 40 EP}
}


\newglossaryentry{somnigravis_Talent}
{
    name={Somnigravis},
    description={Ein Ziel in einer ruhenden Position sinkt langsam in einen Tiefschlaf, aus dem es durch lauten Lärm oder Schmerzen geweckt werden kann.\newline Probenschwierigkeit: Magieresistenz\newline Modifikationen: Beliebiges Wesen (–4; der Zauber wirkt auf alle Wesen, die Schlaf kennen.)\newline Ohnmacht (–8, 1 Aktion, Wirkungsdauer 16 Initiativphasen; das Opfer wird schlagartig ohnmächtig und ist nicht zu wecken.)\newline Vorbereitungszeit: 2 Aktionen\newline Ziel: Einzelperson\newline Reichweite: 4 Schritt\newline Wirkungsdauer: 4 Stunden\newline Kosten: 8 AsP\newline Fertigkeiten: Einfluss\newline Erlernen: Elf 12; Mag 14; Alch 16; Ach, Dru, Geo, Hex, Sch, Srl 18; 40 EP}
}


\newglossaryentry{vipernblick_Talent}
{
    name={Vipernblick},
    description={Du erscheinst deinem Opfer als grauenvolle Gestalt aus den Niederhöllen. Es kann nicht wegsehen und ist vor Angst handlungsunfähig. Wird der Sichtkontakt unterbrochen, endet der Zauber.\newline Probenschwierigkeit: Magieresistenz\newline Modifikationen: Fremdgestalt (–4; du kannst eine andere Person wählen, die als Alpgestalt erscheint.)\newline Vorbereitungszeit: 4 Aktionen\newline Ziel: Einzelwesen\newline Reichweite: 8 Schritt\newline Wirkungsdauer: 4 Minuten\newline Kosten: 16 AsP\newline Fertigkeiten: Einfluss\newline Erlernen: Hex, Ach 18; 40 EP}
}


\newglossaryentry{widerwilleUngemach_Talent}
{
    name={Widerwille Ungemach},
    description={Das verzauberte Objekt wird von Umstehenden gemieden. Der Spielleiter erwähnt es nur, wenn die Spieler danach suchen und ihnen eine Konterprobe (Magieresistenz oder Wachsamkeit, 16) gelingt.\newline Probenschwierigkeit: 12\newline Modifikationen: Räumlicher Widerwille (–8, Zone; der Zauber betrifft einen ganzen Raum)\newline Permanenz (–4, Wirkungsdauer bis die Bindung gelöst wird, 16 AsP, davon 2 gAsP)\newline Vorbereitungszeit: 4 Minuten\newline Ziel: Einzelobjekt\newline Reichweite: Berührung\newline Wirkungsdauer: 1 Monat\newline Kosten: 8 AsP\newline Fertigkeiten: Einfluss, Illusion\newline Erlernen: Alch, Mag, Srl 18; 40 EP}
}


\newglossaryentry{zauberzwang_Talent}
{
    name={Zauberzwang},
    description={Du erlegst dem Opfer eine Aufgabe auf, die nicht tödlich sein darf. Stehen die moralischen Vorstellungen des Opfers der Aufgabe entgegen, kann es mit einer Konterprobe (Willenskraft, 16) widerstehen. Ignoriert das Opfer die Aufgabe oder lässt sie absichtlich scheitern, erleidet es pro Woche 1 Wunde, die während der Wirkungsdauer nicht geheilt oder regeneriert werden kann.\newline Probenschwierigkeit: Magieresistenz\newline Vorbereitungszeit: 1 Stunde\newline Ziel: Einzelperson\newline Reichweite: Berührung\newline Wirkungsdauer: Bis die Aufgabe gelöst wurde, maximal 1 Monat\newline Kosten: 32 AsP\newline Fertigkeiten: Einfluss\newline Erlernen: Hex 14; Dru 18; Mag 20; 60 EP}
}


\newglossaryentry{zwingtanz_Talent}
{
    name={Zwingtanz},
    description={Dein Opfer verliert die Kontrolle über seinen Körper. Es beginnt wie wild zu tanzen und erleidet nach dem Ende der Wirkungsdauer 2 Punkte Erschöpfung. Angriffe auf das Opfer haben eine Schwierigkeit von 16.\newline Mächtige Magie: Das Opfer erleidet einen weiteren Punkt Erschöpfung.\newline Probenschwierigkeit: Magieresistenz\newline Vorbereitungszeit: 2 Aktionen\newline Ziel: Einzelperson\newline Reichweite: 8 Schritt\newline Wirkungsdauer: 4 Minuten\newline Kosten: 16 AsP\newline Fertigkeiten: Einfluss\newline Erlernen: Dru 8; Geo 12; 40 EP}
}


\newglossaryentry{caldofrigoheißundkalt_Talent}
{
    name={Caldofrigo heiß und kalt},
    description={Du veränderst die Temperaturstufe (S. 35) eines Objektes um zwei Stufen.\newline Mächtige Magie: Du veränderst die Temperatur um eine weitere Stufe.\newline Probenschwierigkeit: 12\newline Modifikationen: Zone (–4, selbst, 32 AsP; der Zauber betrifft einen Radius von 8 Schritt um dich herum.)\newline Ferne Zone (–8, Zone, 8 Schritt, 32 AsP)\newline Vorbereitungszeit: 16 Aktionen\newline Ziel: Einzelobjekt\newline Reichweite: Berührung\newline Wirkungsdauer: 16 Minuten\newline Kosten: 8 AsP\newline Fertigkeiten: Eis, Feuer, Umwelt (Zone), Verwandlung\newline Erlernen: Ach, Dru, Mag 16; Alch, Geo, Srl 18; 60 EP}
}


\newglossaryentry{herbeirufungdesEises_Talent}
{
    name={Herbeirufung des Eises},
    description={Ruft ein Elementarwesen des jeweiligen Elements herbei (mehr zu Herbeirufungen S. 81), das in deiner unmittelbaren Nähe erscheint.\newline Probenschwierigkeit: 16/24/32 (Diener/Dschinn/Meister)\newline Vorbereitungszeit: frei wählbar\newline Ziel: einzelnes Elementar\newline Wirkungsdauer: augenblicklich\newline Kosten: 16/32/64 AsP (Diener/Dschinn/Meister)\newline Fertigkeiten: Eis\newline Erlernen: Ach, Dru, Mag 16; Alch 18; 60 EP\newline Anmerkung: Nicht überall sind die Wahren Namen und damit die Beschwörung von elementaren Dienern und vor allem Meistern bekannt. }
}


\newglossaryentry{frigifaxius_Talent}
{
    name={Frigifaxius},
    description={Eine Strahl aus elementarem Eis fügt dem Ziel 4W6 TP zu und verursacht Erfrieren (S. 98). Ballistischer Zauber.\newline Mächtige Magie: Die TP steigen um 2W6.\newline Probenschwierigkeit: 12\newline Modifikationen: Gezielter Strahl (–4; du kannst die Trefferzone bestimmen.)\newline Vorbereitungszeit: 1 Aktion\newline Ziel: Einzelwesen, Einzelobjekt\newline Reichweite: 16 Schritt\newline Wirkungsdauer: augenblicklich\newline Kosten: 16 AsP\newline Fertigkeiten: Eis\newline Erlernen: Ach, Dru, Mag 18; 40 EP}
}


\newglossaryentry{frigisphaero_Talent}
{
    name={Frigisphaero},
    description={Du erschaffst eine elementare Kugel, die du mit Konzentration und Blickkontakt 16 Schritt pro Initiativephase bewegen kannst. Die Kugel explodiert, wenn du die Konzentration oder den Blickkontakt verlierst, du sie absichtlich zündest oder die Wirkungsdauer endet. Die Explosion richtet 4W6 TP an und verursacht Erfrieren (S. 98). Pro Schritt Entfernung fällt der niedrigste Würfel weg.\newline Mächtige Magie: Die TP steigen um 1W6.\newline Probenschwierigkeit: 12\newline Modifikationen: Vorgegebene Bewegung (–4; du gibst der Kugel die Bewegung bis zur Explosion vor, Konzentration und Blickkontakt zur Kugel sind nicht nötig.)\newline Vorbereitungszeit: 2 Aktionen\newline Ziel: Zone\newline Reichweite: 2 Schritt\newline Wirkungsdauer: 2 Initiativphasen\newline Kosten: 16 AsP\newline Fertigkeiten: Eis\newline Erlernen: Ach, Mag 18; Dru 20; 60 EP}
}


\newglossaryentry{glacioflumenFlussausEis_Talent}
{
    name={Glacioflumen Fluss aus Eis},
    description={Du erzeugst eine dünne Schicht aus Eis auf dem Boden. Die Schicht gilt als eisiger Untergrund, was alle Proben im Kampf und zum Stehenbleiben um –4 erschwert. Ihre Fläche beträgt 16 Rechtschritt, ihre Form kannst du bestimmen.\newline Mächtige Magie: Verdoppelt die Fläche.\newline Probenschwierigkeit: 12\newline Vorbereitungszeit: 2 Aktionen\newline Ziel: Zone\newline Reichweite: 16 Schritt\newline Wirkungsdauer: 4 Minuten\newline Kosten: 8 AsP\newline Fertigkeiten: Eis\newline Erlernen: Elf 20; 20 EP}
}


\newglossaryentry{gletscherwand_Talent}
{
    name={Gletscherwand},
    description={Eine 3 Schritt hohe und bis zu 4 Schritt lange Wand aus blankem Eis wächst entlang einer von dir bestimmten Linie aus dem Boden. Sie verfügt über eine Härte von 16. Wer sich der Wand nähert, erleidet bei misslungener Konterprobe (KO, 12) Erfrieren (S. 98).\newline Mächtige Magie: Die Härte der Wand steigt um 8, die maximale Länge um 2 Schritt und die Höhe um 1 Schritt.\newline Probenschwierigkeit: 12\newline Vorbereitungszeit: 16 Aktionen\newline Ziel: Zone\newline Reichweite: 8 Schritt\newline Wirkungsdauer: 16 Minuten\newline Kosten: 8 AsP\newline Fertigkeiten: Eis\newline Erlernen: Dru 16; Mag 18; 40 EP}
}


\newglossaryentry{leibdesEises_Talent}
{
    name={Leib des Eises},
    description={Du harmonierst mit dem Element Eis. Du bist immun gegen Kälte und Eisschaden.\newline Probenschwierigkeit: 12\newline Modifikationen: Reise ins Eis (–4; du kannst dich mit 1 Schritt pro Initiativephase in Eis und Schnee bewegen, als würdest du darin tauchen. Im Eis brauchst du nicht zu atmen.)\newline Begleiter (–4; der Zauber betrifft auch eine weitere Person, mit der du permanent Hautkontakt halten musst.)\newline Leib aus Eis (–8; Deine Kreaturenklasse wird zu "Elementar" mit allen entsprechenden Eigenschaften (S. 99). Du kannst während der Wirkungsdauer keine Zauber wirken.)\newline Vorbereitungszeit: 8 Aktionen\newline Ziel: selbst\newline Reichweite: Berührung\newline Wirkungsdauer: 1 Stunde\newline Kosten: 16 AsP\newline Fertigkeiten: Eis, Verwandlung\newline Erlernen: Ach, Elf 18; Dru 20; 60 EP}
}


\newglossaryentry{metamorphoGletscherform_Talent}
{
    name={Metamorpho Gletscherform},
    description={Du formst Eis mit bloßen Händen in die wundersamsten Formen. Die Kosten, Zauberdauer und Modifikationen Mächtige Magie sind Meisterentscheid. Beispiele:\newline 4 AsP, 1x Mächtige Magie, 4 Aktionen: Du formst einen Eiszapfen um zu einer einfachen Waffe aus Eis (identische Werte, aber zerbrechlich).\newline 8 AsP, keine Mächtige Magie, eine halbe Stunde: Ein Eisblock wird zu einem Iglu.\newline 16 AsP, 2x Mächtige Magie, 1 Stunde: Du ziehst eine einfache Brücke über eine Eisspalte.\newline 128+ AsP, 4x Mächtige Magie, 1 Woche: Du errichtest einen Eispalast, wie ihn die Firnelfen bewohnen.\newline Probenschwierigkeit: 12\newline Modifikationen: Schnee (–4; du kannst auch mit Schnee arbeiten, der dabei zu Eis wird.)\newline Wasser (–8; du kannst auch mit Wasser arbeiten, das dabei gefriert.)\newline Permanenz (–4, ein Viertel der Basiskosten als gAsP; das so geformte Eis schmilzt auch bei hohen Temperaturen nicht mehr.)\newline Vorbereitungszeit: nach Vorhaben\newline Ziel: Einzelobjekt\newline Reichweite: Berührung\newline Wirkungsdauer: bis die Bindung gelöst wird\newline Kosten: nach Vorhaben\newline Fertigkeiten: Eis, Verwandlung\newline Erlernen: Elf 16; Dru, Mag 20; 60 EP}
}


\newglossaryentry{manifestoElement_Talent}
{
    name={Manifesto Element},
    description={Du beschwörst eine etwa faustgroße Menge oder kleine Manifestation des Elements herbei, mit dem du den Zauber wirkst. Sie verschwindet nach dem Ende der Wirkungsdauer.\newline Fertigkeit Eis: Du kannst zum Beispiel einen Schneeball oder einen Eiszapfen erscheinen lassen, ein Getränk kühlen, oder deine Körpertemperatur etwas senken.\newline Fertigkeit Erz: Du kannst zum Beispiel eine Hand voll Sand oder einen Stein erscheinen lassen oder das Gewicht eines Gegenstandes kurzzeitig etwas erhöhen.\newline Fertigkeit Feuer: Du kannst zum Beispiel eine kleine Flamme über deiner Hand tanzen lassen, deine Körpertemperatur etwas erhöhen, eine Pfeife anzünden  oder einen Funkenschwarm erzeugen.\newline Fertigkeit Humus: Du kannst zum Beispiel eine Knospe erblühen lassen, eine Hand voll Kuhmist herbeizaubern, ein Gänseblümchen erscheinen lassen, oder einen kleinen Kratzer schließen.\newline Fertigkeit Luft: Du kannst zum Beispiel einen frischen Windhauch rufen, einen Wohlgeruch erzeugen, oder einen kleinen Luftwirbel erschaffen.\newline Fertigkeit Wasser: Du kannst zum Beispiel einen Becher mit Wasser füllen, deine Hände abspülen, eine Pfütze bilden oder tagsüber einen kleinen Regenbogen glitzern lassen.\newline Probenschwierigkeit: 12\newline Modifikationen: Demanifesto (–4; du kannst die Manifestation nach Belieben verschwinden und wieder auftauchen lassen.)\newline Vorbereitungszeit: 4 Aktionen\newline Ziel: Zone\newline Reichweite: Berührung\newline Wirkungsdauer: 1 Stunde\newline Kosten: 2 AsP\newline Fertigkeiten: Eis, Erz, Feuer, Humus, Luft, Wasser\newline Erlernen: Ach, Alch 8; Dru, Geo, Mag 14; Hex, Srl 18; Elf 20; 20 EP}
}


\newglossaryentry{pfeildesEises_Talent}
{
    name={Pfeil des Eises},
    description={Du verzauberst einen Pfeil (oder Bolzen oder Wurfwaffe), sodass er im Flug die Macht des Elements freisetzt. Der Pfeil verursacht Eisschaden und Erfrieren (S. 98). Unbelebte Gegenstände werden bei einem Treffer mit Eis überzogen.\newline Probenschwierigkeit: 12\newline Modifikationen: Geschütz (–4; du verzauberst ein größeres Geschoss wie das einer Balliste.)\newline Permanenz (–4, 4 AsP, davon 1 gAsP, Wirkungsdauer bis die Bindung gelöst wird oder der Pfeil verschossen wurde)\newline Vorbereitungszeit: 1 Aktion\newline Ziel: Einzelobjekt\newline Reichweite: Berührung\newline Wirkungsdauer: 8 Initiativphasen\newline Kosten: 4 AsP\newline Fertigkeiten: Eis\newline Erlernen: Elf 18; 40 EP}
}


\newglossaryentry{stillstand_Talent}
{
    name={Stillstand},
    description={Du erschaffst eine Zone mit 8 Schritt Radius, die alle Bewegungen außer deiner verlangsamt. Andere Ziele verlieren ihre nächste Aktion, wenn sie eine andere Aktion als Konzentration nutzen. Trifft eine Waffe ein Wesen in der Zone, richtet sie nur halben Schaden an.\newline Probenschwierigkeit: 12\newline Modifikationen: Begleiter (–4; die Zone bewegt sich mit dir.)\newline Vorbereitungszeit: 4 Aktionen\newline Ziel: Zone\newline Reichweite: Berührung\newline Wirkungsdauer: 16 Initiativphasen\newline Kosten: 16 AsP\newline Fertigkeiten: Eis, Umwelt\newline Erlernen: Ach 16; 60 EP}
}


\newglossaryentry{warmesgefriere!_Talent}
{
    name={Warmes gefriere!},
    description={Das verzauberte Material von maximal 2 Kubikmetern Volumen kühlt schlagartig auf die Temperaturstufe Grimmfrost ab (S. 35) und bleibt für die Wirkungsdauer gefroren. Während der Wirkungsdauer ist die Härte des Materials halbiert, nach dem Ende der Wirkungsdauer immer noch um 2 verringert.\newline Mächtige Magie: Du kannst 1 weiteren Kubikmeter einfrieren.\newline Probenschwierigkeit: 12\newline Vorbereitungszeit: 1 Aktion\newline Ziel: Einzelobjekt\newline Reichweite: Berührung\newline Wirkungsdauer: 1 Stunde\newline Kosten: 8 AsP\newline Fertigkeiten: Eis, Verwandlung\newline Erlernen: Bor, Mag 18; 40 EP}
}


\newglossaryentry{zornderElemente_Talent}
{
    name={Zorn der Elemente},
    description={Du schleuderst eine Handvoll des Elements, mit dem du den Zauber wirkst, auf deinen Gegner. Die Menge vervielfacht sich im Flug und richtet 2W6 TP an. Ballistischer Zauber.\newline Fertigkeit Eis: Erfrieren (S. 98)\newline Fertigkeit Erz: Niederschmettern (S. 98)\newline Fertigkeit Feuer: Nachbrennen (S. 98)\newline Fertigkeit Humus: Fesseln (S. 98)\newline Fertigkeit Luft: Zurückstoßen (S. 98)\newline Fertigkeit Wasser: Ertränken (S. 98)\newline Mächtige Magie: Erhöht die TP um +4.\newline Probenschwierigkeit: 12\newline Vorbereitungszeit: 0 Aktionen\newline Ziel: Einzelwesen\newline Reichweite: 16 Schritt\newline Wirkungsdauer: augenblicklich\newline Kosten: 8 AsP\newline Fertigkeiten: Eis, Erz, Feuer, Humus, Luft, Wasser\newline Erlernen: Geo, Dru, Ach 12; 40 EP}
}


\newglossaryentry{adamantiumErzstruktur_Talent}
{
    name={Adamantium Erzstruktur},
    description={Du stärkst die Struktur eines Gegenstandes von maximal 16 Stein Gewicht. Seine Härte wird verdoppelt. Bei Waffen steigen der Waffenschaden um 2 Punkte, bei Rüstungen der RS um 1.\newline Mächtige Magie: Waffenschaden einer Waffe +1, für je 2 Stufen RS einer Rüstung +1.\newline Probenschwierigkeit: 12\newline Modifikationen: Adamantenquader (–4; das maximale Gewicht wird verdoppelt. Mehrfach wählbar.)\newline Kristallglanz (–4; während der Wirkungsdauer erscheint die Oberfläche des Materials wie die eines beliebigen anderen erzenen Materials.)\newline Zauberklinge (–4; während der Wirkungsdauer gilt die Waffe als magisch.)\newline Vorbereitungszeit: 16 Aktionen\newline Ziel: Einzelobjekt\newline Reichweite: Berührung\newline Wirkungsdauer: 1 Stunde\newline Kosten: 8 AsP\newline Fertigkeiten: Erz, Verwandlung\newline Erlernen: Ach 12; Geo, Mag 14; Dru 16; Alch 20; 40 EP}
}


\newglossaryentry{archofaxius_Talent}
{
    name={Archofaxius},
    description={Eine Strahl aus elementarem Erz fügt dem Ziel 4W6 TP zu und verursacht Niederschmettern (S. 98). Ballistischer Zauber.\newline Mächtige Magie: Die TP steigen um 2W6.\newline Probenschwierigkeit: 12\newline Modifikationen: Gezielter Strahl (–4; du kannst die Trefferzone bestimmen.)\newline Vorbereitungszeit: 1 Aktion\newline Ziel: Einzelwesen, Einzelobjekt\newline Reichweite: 16 Schritt\newline Wirkungsdauer: augenblicklich\newline Kosten: 16 AsP\newline Fertigkeiten: Erz\newline Erlernen: Ach, Dru, Geo, Mag 18; 40 EP}
}


\newglossaryentry{archosphaero_Talent}
{
    name={Archosphaero},
    description={Du erschaffst eine elementare Kugel, die du mit Konzentration und Blickkontakt 16 Schritt pro Initiativephase bewegen kannst. Die Kugel explodiert, wenn du die Konzentration oder den Blickkontakt verlierst, du sie absichtlich zündest oder die Wirkungsdauer endet. Die Explosion richtet 4W6 TP an und verursacht Niederschmettern (S. 98). Pro Schritt Entfernung fällt der niedrigste Würfel weg.\newline Mächtige Magie: Die TP steigen um 1W6.\newline Probenschwierigkeit: 12\newline Modifikationen: Vorgegebene Bewegung (–4; du gibst der Kugel die Bewegung bis zur Explosion vor, Konzentration und Blickkontakt zur Kugel sind nicht nötig.)\newline Vorbereitungszeit: 2 Aktionen\newline Ziel: Zone\newline Reichweite: 2 Schritt\newline Wirkungsdauer: 2 Initiativphasen\newline Kosten: 16 AsP\newline Fertigkeiten: Erz\newline Erlernen: Ach, Mag 18; Geo, Dru 20; 60 EP}
}


\newglossaryentry{herbeirufungdesErzes_Talent}
{
    name={Herbeirufung des Erzes},
    description={Ruft ein Elementarwesen des jeweiligen Elements herbei (mehr zu Herbeirufungen S. 81), das in deiner unmittelbaren Nähe erscheint.\newline Probenschwierigkeit: 16/24/32 (Diener/Dschinn/Meister)\newline Vorbereitungszeit: frei wählbar\newline Ziel: einzelnes Elementar\newline Wirkungsdauer: augenblicklich\newline Kosten: 16/32/64 AsP (Diener/Dschinn/Meister)\newline Fertigkeiten: Erz\newline Erlernen: Geo 14; Ach, Dru, Mag 16; Alch 18; 60 EP\newline Anmerkung: Nicht überall sind die Wahren Namen und damit die Beschwörung von elementaren Dienern und vor allem Meistern bekannt. }
}


\newglossaryentry{fortifexarkaneWand_Talent}
{
    name={Fortifex arkane Wand},
    description={Es entsteht eine maximal 2x2 Schritt große, 5 Finger dicke, unsichtbare Wand. Die Wand ist undurchdringlich für Gegenstände bis zum Gewicht einer Rotzenkugel, schwerere Gegenstände werden nur leicht verlangsamt.\newline Mächtige Magie: Die Wand ist bis zum Gewicht eines schweren Menschen/Pferdes mit Reiter/einer Kutsche/eines Schiffes undurchdringlich.\newline Probenschwierigkeit: 12\newline Modifikationen: Begleiter (–4; die Wand bewegt sich mit dir.)\newline Schimmernder Schild (–8; erschafft einen unsichtbaren und unzerstörbaren Schild. Die übrigen Werte gleichen einem Holzschild.)\newline Vorbereitungszeit: 4 Aktionen\newline Ziel: Zone\newline Reichweite: 4 Schritt\newline Wirkungsdauer: 4 Minuten\newline Kosten: 8 AsP\newline Fertigkeiten: Erz, Umwelt\newline Erlernen: Mag 16; Ach 18; Alch 20; 40 EP}
}


\newglossaryentry{kraftdesErzes_Talent}
{
    name={Kraft des Erzes},
    description={Der schwerste Metallgegenstand in Reichweite wird stark magnetisch. Im Radius von 8 Schritt um den Magneten müssen alle Träger von Metallgegenständen eine Konterprobe (KK, 16) ablegen. Misslingt sie, werden ihnen Metallgegenstände aus den Händen gerissen und schlittern in Richtung des Magneten – bei einer Rüstung mit ihrem Träger. Selbst wenn die Konterprobe gelingt, sind alle Proben im Umgang mit Metallgegenständen um –4 erschwert.\newline Mächtige Magie: Der Malus steigt um –2.\newline Probenschwierigkeit: 12\newline Vorbereitungszeit: 16 Aktionen\newline Ziel: Einzelobjekt\newline Reichweite: 8 Schritt\newline Wirkungsdauer: 1 Stunde\newline Kosten: 8 AsP\newline Fertigkeiten: Erz, Verwandlung\newline Erlernen: Geo 12; Dru 16; Ach 18; Alch, Mag 20; 40 EP}
}


\newglossaryentry{leibdesErzes_Talent}
{
    name={Leib des Erzes},
    description={Du harmonierst mit dem Element Erz. Du bist immun gegen Erzschaden und dein RS gegen Metallwaffen steigt um 1.\newline Probenschwierigkeit: 12\newline Modifikationen: Reise ins Element (–4; du kannst dich mit 1 Schritt pro Initiativephase durch Stein und Erz bewegen, als würdest du darin tauchen. Im Erz brauchst du nicht zu atmen.)\newline Begleiter (–4; der Zauber betrifft auch eine weitere Person, mit der du permanent Hautkontakt halten musst.)\newline Leib aus Erz (–8; Deine Kreaturenklasse wird zu "Elementar" mit allen entsprechenden Eigenschaften (S. 99). Du kannst während der Wirkungsdauer keine Zauber wirken.)\newline Vorbereitungszeit: 8 Aktionen\newline Ziel: selbst\newline Reichweite: Berührung\newline Wirkungsdauer: 1 Stunde\newline Kosten: 16 AsP\newline Fertigkeiten: Erz, Verwandlung\newline Erlernen: Elf 18; Geo, Dru 20; 60 EP}
}


\newglossaryentry{paralysisstarrwieStein_Talent}
{
    name={Paralysis starr wie Stein},
    description={Dein Ziel erstarrt zu einer unzerstörbaren Statue, der auch Gifte oder Krankheiten nichts anhaben können. Die Statue fühlt und hört nichts, kann aber sehen und riechen.\newline Probenschwierigkeit: Magieresistenz\newline Vorbereitungszeit: 2 Aktionen\newline Ziel: Einzelwesen\newline Reichweite: 8 Schritt\newline Wirkungsdauer: 1 Stunde\newline Kosten: 8 AsP\newline Fertigkeiten: Erz, Verwandlung\newline Erlernen: Mag 12; Geo 16; Alch, Dru 18; 60 EP}
}


\newglossaryentry{pfeildesErzes_Talent}
{
    name={Pfeil des Erzes},
    description={Du verzauberst einen Pfeil (oder Bolzen oder Wurfwaffe), sodass er im Flug die Macht des Elements freisetzt. Der Pfeil verursacht Erzschaden und Niederschmettern (S. 98). Unbelebte Gegenstände erleiden bei einem Treffer doppelten Schaden.\newline Probenschwierigkeit: 12\newline Modifikationen: Geschütz (–4; du verzauberst ein größeres Geschoss wie das einer Balliste.)\newline Permanenz (–4, 4 AsP, davon 1 gAsP, Wirkungsdauer bis die Bindung gelöst wird oder der Pfeil verschossen wurde)\newline Vorbereitungszeit: 1 Aktion\newline Ziel: Einzelobjekt\newline Reichweite: Berührung\newline Wirkungsdauer: 8 Initiativphasen\newline Kosten: 4 AsP\newline Fertigkeiten: Erz\newline Erlernen: Mag 18; 20 EP}
}


\newglossaryentry{staubwandle!_Talent}
{
    name={Staub wandle!},
    description={Du erschaffst einen Golem aus Sand (mehr zu Beschwörungen siehe S. 81). Der Golem hat sich nach 1 Stunde an seine Existenz gewöhnt und ist einsatzfähig.\newline Probenschwierigkeit: nach Golem\newline Vorbereitungszeit: frei wählbar\newline Ziel: einzelner Rohling\newline Wirkungsdauer: bis die Bindung gelöst wird\newline Kosten: nach Golem, ein Viertel der Basiskosten als gAsP\newline Fertigkeiten: Erz\newline Erlernen: Mag 18; 60 EP}
}


\newglossaryentry{wandausErz_Talent}
{
    name={Wand aus Erz},
    description={Eine 3 Schritt hohe und bis zu 4 Schritt lange Wand aus massivem Erz wächst entlang einer von dir bestimmten Linie aus dem Boden. Sie verfügt über eine Härte von 32. Sie kann mit einer Konterprobe (Klettern, 20) in 8 Aktionen überwunden werden.\newline Mächtige Magie: Die Höhe steigt um 1 Schritt. Die Länge der Wand steigt um bis zu 2 Schritt. Die Härte der Wand steigt um 16.\newline Probenschwierigkeit: 12\newline Vorbereitungszeit: 16 Aktionen\newline Ziel: Zone\newline Reichweite: 8 Schritt\newline Wirkungsdauer: 16 Minuten\newline Kosten: 8 AsP\newline Fertigkeiten: Erz\newline Erlernen: Geo14; Mag 18; Dru 20; 40 EP}
}


\newglossaryentry{zagibuUbigaz_Talent}
{
    name={Zagibu Ubigaz},
    description={Du lässt einen Schatz von maximal 1 Stein Gewicht unwiederbringlich zu Staub zerfallen. Magische oder karmale Gegenstände sind hiervon nicht betroffen.\newline Probenschwierigkeit: 12\newline Vorbereitungszeit: 1 Aktion\newline Ziel: Zone\newline Reichweite: Berührung\newline Wirkungsdauer: augenblicklich\newline Kosten: 4 AsP\newline Fertigkeiten: Erz, Verwandlung\newline Erlernen: Sch 8; Mag 20; 20 EP}
}


\newglossaryentry{custodosigilDiebesbann_Talent}
{
    name={Custodosigil Diebesbann},
    description={Du sicherst ein Gefäß von maximal 4 Stein mit einem magischen Siegel. Öffnet jemand außer dir das Gefäß, muss ihm eine Konterprobe (Schlösser knacken, 16) gelingen, sonst geht der Inhalt in Flammen auf.\newline Mächtige Magie: Verdoppelt das Gewicht.\newline Probenschwierigkeit: 12\newline Modifikationen: Personalisierung (–2; eine weitere Person kann das Gefäß öffnen. Mehrfach wählbar.)\newline Permanenz (–4, 4 AsP, davon 1 gAsP, Wirkungsdauer bis die Bindung gelöst wird)\newline Vorbereitungszeit: 1 Stunde\newline Ziel: Einzelobjekt\newline Reichweite: Berührung\newline Wirkungsdauer: 1 Jahr\newline Kosten: 4 AsP\newline Fertigkeiten: Feuer\newline Erlernen: Ach, Mag 12; Alch, Srl 16; 20 EP}
}


\newglossaryentry{herbeirufungdesFeuers_Talent}
{
    name={Herbeirufung des Feuers},
    description={Ruft ein Elementarwesen des jeweiligen Elements herbei (mehr zu Herbeirufungen S. 81), das in deiner unmittelbaren Nähe erscheint.\newline Probenschwierigkeit: 16/24/32 (Diener/Dschinn/Meister)\newline Vorbereitungszeit: frei wählbar\newline Ziel: einzelnes Elementar\newline Wirkungsdauer: augenblicklich\newline Kosten: 16/32/64 AsP (Diener/Dschinn/Meister)\newline Fertigkeiten: Feuer\newline Erlernen: Geo 14; Ach, Dru, Mag 16; Alch 18; 60 EP\newline Anmerkung: Nicht überall sind die Wahren Namen und damit die Beschwörung von elementaren Dienern und vor allem Meistern bekannt. }
}


\newglossaryentry{ignifaxiusFlammenstrahl_Talent}
{
    name={Ignifaxius Flammenstrahl},
    description={Eine Flammenlanze fügt dem Ziel 4W6 TP zu und verursacht Nachbrennen (S. 98). Ballistischer Zauber.\newline Mächtige Magie: Die TP steigen um 2W6.\newline Probenschwierigkeit: 12\newline Modifikationen: Gezielter Strahl (–4; du kannst die Trefferzone bestimmen.)\newline Vorbereitungszeit: 1 Aktion\newline Ziel: Einzelwesen, Einzelobjekt\newline Reichweite: 16 Schritt\newline Wirkungsdauer: augenblicklich\newline Kosten: 16 AsP\newline Fertigkeiten: Feuer\newline Erlernen: Mag 14; Ach 18; Dru, Srl 20; 40 EP}
}


\newglossaryentry{ignifugoFeuerbann_Talent}
{
    name={Ignifugo Feuerbann},
    description={Du löschst ein Feuer bis zur Größe eines Kaminfeuers.\newline Mächtige Magie: Du kannst ein Feuer bis zur Größe eines großen Lagerfeuers/eines Scheiterhaufens/eines brennenden Schiffes/einer brennenden Häusergruppe löschen.\newline Probenschwierigkeit: 12\newline Modifikationen: Feuerdiebstahl (–4, Wirkungsdauer 1 Stunde, 1 AsP; du kannst ein maximal fackelgroßes Feuer von seiner alten Quelle nehmen und in deiner Hand mit dir tragen.)\newline Selbstlöschung (–4, 0 Aktionen, 4 AsP; du löschst dich selbst.)\newline Vorbereitungszeit: 2 Aktionen\newline Ziel: Zone\newline Reichweite: 32 Schritt\newline Wirkungsdauer: augenblicklich\newline Kosten: 8 AsP\newline Fertigkeiten: Feuer, Umwelt\newline Erlernen: Dru 14; Alch, Geo, Sch, Srl 16; Mag 18; 40 EP}
}


\newglossaryentry{ignimorphoFeuerform_Talent}
{
    name={Ignimorpho Feuerform},
    description={Du kannst ein bestehendes Feuer bis zur Größe einer Fackel formen oder verlöschen lassen. Die Verformung geschieht mit einer GS von 8 Schritt pro Initiativephase und benötigt Konzentration.\newline Mächtige Magie: Du kannst ein Feuer bis zur Größe eines Kaminfeuers/eines großen Lagerfeuers/eines Scheiterhaufens/eines brennenden Schiffes manipulieren.\newline Probenschwierigkeit: 12\newline Vorbereitungszeit: 4 Aktionen\newline Ziel: Einzelobjekt\newline Reichweite: 16 Schritt\newline Wirkungsdauer: 8 Initiativphasen\newline Kosten: 8 AsP\newline Fertigkeiten: Feuer\newline Erlernen: Elf, Srl 18; 20 EP}
}


\newglossaryentry{igniplanoFlächenbrand_Talent}
{
    name={Igniplano Flächenbrand},
    description={Du beschwörst flammende Urgewalten aus der Erde hervor. In einem Kreis mit 4 Schritt Radius fügen Flammensäulen jedem Wesen 4W6 TP zu, die Nachbrennen (S. 98) verursachen. Alles Brennbare in der Zone geht augenblicklich in Flammen auf.\newline Mächtige Magie: Die TP steigen um +2W6.\newline Probenschwierigkeit: 12\newline Vorbereitungszeit: 4 Aktionen\newline Ziel: Zone\newline Reichweite: 32 Schritt\newline Wirkungsdauer: augenblicklich\newline Kosten: 32 AsP\newline Fertigkeiten: Feuer\newline Erlernen: -; 60 EP}
}


\newglossaryentry{ignisphaero_Talent}
{
    name={Ignisphaero},
    description={Du erschaffst eine elementare Kugel, die du mit Konzentration und Blickkontakt 16 Schritt pro Initiativephase bewegen kannst. Die Kugel explodiert, wenn du die Konzentration oder den Blickkontakt verlierst, du sie absichtlich zündest oder die Wirkungsdauer endet. Die Explosion richtet 4W6 TP an und verursacht Nachbrennen (S. 98). Pro Schritt Entfernung fällt der niedrigste Würfel weg.\newline Mächtige Magie: Die TP steigen um 1W6.\newline Probenschwierigkeit: 12\newline Modifikationen: Vorgegebene Bewegung (–4; du gibst der Kugel die Bewegung bis zur Explosion vor, Konzentration und Blickkontakt zur Kugel sind nicht nötig.)\newline Vorbereitungszeit: 2 Aktionen\newline Ziel: Zone\newline Reichweite: 2 Schritt\newline Wirkungsdauer: 2 Initiativphasen\newline Kosten: 16 AsP\newline Fertigkeiten: Feuer\newline Erlernen: Mag 18; Ach, Geo 20; 60 EP}
}


\newglossaryentry{leibdesFeuers_Talent}
{
    name={Leib des Feuers},
    description={Du harmonierst mit dem Element Feuer. Du bist immun gegen Hitze und Feuerschaden.\newline Probenschwierigkeit: 12\newline Modifikationen: Reise ins Feuer (–4; du kannst dich mit 1 Schritt pro Initiativephase durch Feuer und Lava bewegen, als würdest du darin tauchen. Im Feuer brauchst du nicht zu atmen.)\newline Begleiter (–4; der Zauber betrifft auch eine weitere Person, mit der du permanent Hautkontakt halten musst.)\newline Leib aus Feuer (–8; Deine Kreaturenklasse wird zu "Elementar" mit allen entsprechenden Eigenschaften (S. 99). Du kannst während der Wirkungsdauer keine Zauber wirken.)\newline Vorbereitungszeit: 8 Aktionen\newline Ziel: selbst\newline Reichweite: Berührung\newline Wirkungsdauer: 1 Stunde\newline Kosten: 16 AsP\newline Fertigkeiten: Feuer, Verwandlung\newline Erlernen: Geo, Hex 16; Alch, Elf, Dru, Mag 20; 60 EP}
}


\newglossaryentry{pfeildesFeuers_Talent}
{
    name={Pfeil des Feuers},
    description={Du verzauberst einen Pfeil (oder Bolzen oder Wurfwaffe), sodass er im Flug die Macht des Elements freisetzt. Der Pfeil verursacht Feuerschaden und Nachbrennen (S. 98). Unbelebte Gegenstände gehen bei einem Treffer in Flammen auf.\newline Probenschwierigkeit: 12\newline Modifikationen: Geschütz (–4; du verzauberst ein größeres Geschoss wie das einer Balliste.)\newline Permanenz (–4, 4 AsP, davon 1 gAsP, Wirkungsdauer bis die Bindung gelöst wird oder der Pfeil verschossen wurde)\newline Vorbereitungszeit: 1 Aktion\newline Ziel: Einzelobjekt\newline Reichweite: Berührung\newline Wirkungsdauer: 8 Initiativphasen\newline Kosten: 4 AsP\newline Fertigkeiten: Feuer\newline Erlernen: Mag 20; 40 EP}
}


\newglossaryentry{wandausFlammen_Talent}
{
    name={Wand aus Flammen},
    description={Eine 3 Schritt hohe und bis zu 4 Schritt lange Wand aus lodernden Flammen wächst entlang einer von dir bestimmten Linie aus dem Boden. Wer sich der Wand nähert, erleidet bei misslungener Konterprobe (KO, 12) Nachbrennen (S. 98). Die Durchquerung verursacht 4W6 SP.\newline Mächtige Magie: Der Schaden beim Durchqueren der Wand steigt um 2W6, die maximale Länge um 2 Schritt und die Höhe um 1 Schritt.\newline Probenschwierigkeit: 12\newline Vorbereitungszeit: 16 Aktionen\newline Ziel: Zone\newline Reichweite: 8 Schritt\newline Wirkungsdauer: 16 Minuten\newline Kosten: 8 AsP\newline Fertigkeiten: Feuer\newline Erlernen: Dru, Geo 16; Mag 18; 40 EP}
}


\newglossaryentry{analysArcanstruktur_Talent}
{
    name={Analys Arcanstruktur},
    description={Du analysierst die magische Struktur eines arkanen Artefakts oder eines magischen Wesens. Das entspricht einem Analysegrad von 1 für die Strukturanalyse (mehr dazu siehe S. 80).\newline Mächtige Magie: Der Analysegrad steigt um 1.\newline Probenschwierigkeit: 16\newline Vorbereitungszeit: 1 Stunde\newline Ziel: Einzelobjekt, Einzelwesen\newline Reichweite: Berührung\newline Wirkungsdauer: augenblicklich\newline Kosten: 8 AsP\newline Fertigkeiten: Hellsicht, Kraft\newline Erlernen: Mag 12; Ach 14; Alch, Dru, Elf, Geo, Hex 16; 40 EP}
}


\newglossaryentry{blickaufsWesen_Talent}
{
    name={Blick aufs Wesen},
    description={Der Zauber offenbart dir die Fertigkeiten deines Ziels. Du erhältst einen groben Eindruck von seinen körperlichen, geistigen, handwerklichen, kämpferischen und zauberischen Fähigkeiten.\newline Mächtige Magie: Du erfährst auch besonders hohe Fertigkeiten/markante Vorteile/Eigenheiten/Vorlieben und Vorgehen.\newline Probenschwierigkeit: Magieresistenz\newline Modifikationen: Leuchtende Persönlichkeit (–4; du erkennst in einer Gruppe von bis zu 8 Personen diejenige, die in einem von dir gewählten Bereich am meisten heraussticht. Durch eine Konterprobe (MR, 16) können sich die Personen vor der Entdeckung schützen.)\newline Vorbereitungszeit: 16 Aktionen\newline Ziel: Einzelwesen\newline Reichweite: 8 Schritt\newline Wirkungsdauer: augenblicklich\newline Kosten: 8 AsP\newline Fertigkeiten: Hellsicht\newline Erlernen: Elf 12; Geo, Mag 14; Ach, Alch, Dru, Hex, Sch 16; 40 EP}
}


\newglossaryentry{blickdurchfremdeAugen_Talent}
{
    name={Blick durch fremde Augen},
    description={Du blickst während der Wirkungsdauer durch die Augen deines Opfers, das davon nichts ahnt. Du kannst nur Opfer verzaubern, die du sehr gut kennst oder von denen du ein Körperteil besitzt.\newline Mächtige Magie: Es reicht, wenn dir das Ziel persönlich bekannt ist/du es eine Weile beobachtet hast/du es vom Hörensagen kennst.\newline Probenschwierigkeit: Magieresistenz\newline Modifikationen: Mehrere Sinne (–4; du teilst einen weiteren Sinn mit dem Opfer. Mehrfach wählbar.)\newline Fremder Zauber (–8; du kannst während der Wirkungsdauer aus den Augen deines Opfers zaubern. Du musst dabei selbstverständlich auf Gesten und Formeln verzichten.)\newline Vorbereitungszeit: 4 Minuten\newline Ziel: Einzelwesen\newline Reichweite: 16 Meilen\newline Wirkungsdauer: 1 Stunde\newline Kosten: 8 AsP\newline Fertigkeiten: Hellsicht, Verständigung\newline Erlernen: Dru, Sch 18; Mag 20; 40 EP}
}


\newglossaryentry{blickindieGedanken_Talent}
{
    name={Blick in die Gedanken},
    description={Du siehst die Gedanken deines Ziels als verschwommene Bilder vor dir. Weiß das Ziel, dass seine Gedanken gelesen werden, kann es dich mit einer Konterprobe (Willenskraft, 20) in die Irre führen.\newline Mächtige Liturgie: Du erhältst einen deutlichen/klaren Einblick in die Gedanken deines Ziels.\newline Probenschwierigkeit: Magieresistenz\newline Modifikationen: Kampfsinn (nur Elf, –4; du kannst die AT des Gegners vorhersehen, VT gegen diesen Gegner sind um +4 erleichtert.)\newline Traumreise (–4, Wirkungsdauer 1 Stunde; du kannst an den Träumen des Ziels teilhaben.)\newline Verhandlungssinn (nicht Elf, –4, Wirkungsdauer 1 Stunde; du kannst die Argumente und Absichten deines Ziels vorhersehen, gesellschaftliche Proben sind um +2 erleichtert.)\newline Vorbereitungszeit: 4 Aktionen\newline Ziel: Einzelperson\newline Reichweite: 4 Schritt\newline Wirkungsdauer: 16 Initiativphasen\newline Kosten: 8 AsP\newline Fertigkeiten: Hellsicht\newline Erlernen: Elf, Mag 14; Dru, Hex 16; Geo 18; 40 EP}
}


\newglossaryentry{blickindieVergangenheit_Talent}
{
    name={Blick in die Vergangenheit},
    description={Die Geschichte des Ortes rauscht vor deinen Augen vorbei. Während der Wirkungsdauer erkennst du die wichtigsten Ereignisse des letzten Monats.\newline Mächtige Magie: Du blickst 1 Jahr/10 Jahre/100 Jahre/1000 Jahre zurück.\newline Probenschwierigkeit: 12\newline Modifikationen: Objekt (–4, Einzelobjekt)\newline Vorbereitungszeit: 16 Aktionen\newline Ziel: Zone\newline Reichweite: Berührung\newline Wirkungsdauer: 16 Initiativphasen\newline Kosten: 16 AsP\newline Fertigkeiten: Hellsicht, Temporal\newline Erlernen: Dru, Geo 18; Mag 20; 20 EP}
}


\newglossaryentry{exposamiLebenskraft_Talent}
{
    name={Exposami Lebenskraft},
    description={Du nimmst Lebewesen als grün leuchtende Flecken wahr. Der Zauber kann alle Elemente bis auf Erz und Eis durchdringen.\newline Mächtige Magie: Du kannst Angehörige verschiedener Spezies/verschiedene Individuen unterscheiden und wiedererkennen.\newline Probenschwierigkeit: 12\newline Modifikationen: Reinheit der Aura (–8; der Zauber zeigt den Gesundheitszustand des Ziels, dämonische Verseuchungen usw.)\newline Vorbereitungszeit: 1 Aktion\newline Ziel: Einzelperson\newline Reichweite: 16 Schritt\newline Wirkungsdauer: 8 Initiativphasen\newline Kosten: 4 AsP\newline Fertigkeiten: Hellsicht\newline Erlernen: Elf 12; Ach, Alch, Dru, Geo 16; Hex, Mag 18; 20 EP}
}


\newglossaryentry{gefunden!_Talent}
{
    name={Gefunden!},
    description={Du erspürst die Richtung, in der sich ein mindestens rucksackgroßer Gegenstand aus deinem Besitz befindet.\newline Mächtige Magie: Der Gegenstand kann kopf-/faust-/münzgroß sein.\newline Probenschwierigkeit: 12\newline Modifikationen: Fremdbesitz (–4; du musst den Gegenstand nur gesehen haben.)\newline Hörensagen (–8, der Gegenstand muss dir nur ausführlich beschrieben worden sein.)\newline Vorbereitungszeit: 16 Aktionen\newline Ziel: selbst\newline Reichweite: 4 Meilen\newline Wirkungsdauer: 1 Stunde\newline Kosten: 8 AsP\newline Fertigkeiten: Hellsicht\newline Erlernen: Sch 14; Ach, Alch, Hex, Mag, Srl 18; 20 EP}
}


\newglossaryentry{koboldvision_Talent}
{
    name={Koboldvision},
    description={Du bist in der Lage, während der Wirkungsdauer in die Feenwelt zu blicken, sofern sich ein Übergang in der Nähe befindet. Achaz blicken mit diesem Zauber in entrückte Globulen.\newline Probenschwierigkeit: 12\newline Vorbereitungszeit: 4 Aktionen\newline Ziel: selbst\newline Reichweite: Berührung\newline Wirkungsdauer: 1 Stunde\newline Kosten: 4 AsP\newline Fertigkeiten: Hellsicht, Kraft\newline Erlernen: Sch 14; Ach 20; 20 EP}
}


\newglossaryentry{madasSpiegel_Talent}
{
    name={Madas Spiegel},
    description={Die Spiegelung des Mondes auf der Wasseroberfläche verwandelt sich in das Abbild des Ziels. Zusätzlich kannst du kleine Teile des Hintergrunds erkennen und spürst flüchtig die Gefühlseindrücke deines Zieles. Dieses erinnert sich währenddessen an dich.\newline Probenschwierigkeit: 12\newline Vorbereitungszeit: 4 Minuten\newline Ziel: Einzelperson\newline Reichweite: dereweit\newline Wirkungsdauer: 16 Initiativphasen\newline Kosten: 8 AsP\newline Fertigkeiten: Hellsicht, Verständigung\newline Erlernen: Hex 18; Dru, Srl 20; 20 EP}
}


\newglossaryentry{oculusAstralis_Talent}
{
    name={Oculus Astralis},
    description={Du wirfst einen Blick in die Welt der Magie. Während nichtmagische Gegenstände nur schwer zu erkennen sind, erscheint Magie als Ansammlung pulsierender Kraftfäden (Analysegrad 1 der Intensitätsanalyse, S. 80). Während der Wirkungsdauer kannst du deine Sinnenschärfe zur Strukturanalyse (S. 80) verwenden, wodurch du 1 Punkt Erschöpfung erleidest. Außerdem kannst du dich im Limbus orientieren.\newline Mächtige Magie: Der Analysegrad der Intensitätsanalyse steigt um +1.\newline Probenschwierigkeit: 12\newline Vorbereitungszeit: 8 Aktionen\newline Ziel: selbst\newline Reichweite: Berührung\newline Wirkungsdauer: 16 Minuten\newline Kosten: 8 AsP\newline Fertigkeiten: Hellsicht, Kraft\newline Erlernen: Mag 18; 60 EP}
}


\newglossaryentry{odemArcanum_Talent}
{
    name={Odem Arcanum},
    description={Du nimmst magische Kraft um das Ziel kurz als roten Schimmer wahr. Das entspricht einem Analysegrad von 1 für die Intensitätsanalyse (mehr dazu S. 80).\newline Mächtige Magie: Der Analysegrad steigt um 1.\newline Probenschwierigkeit: 12\newline Modifikationen: Sichtbereich (–4, Zone; wirkt auf alle Objekte im Sichtfeld.)\newline Vorbereitungszeit: 1 Aktion\newline Ziel: Einzelwesen, Einzelobjekt\newline Reichweite: 8 Schritt\newline Wirkungsdauer: 2 Initiativphasen\newline Kosten: 4 AsP\newline Fertigkeiten: Hellsicht, Kraft\newline Erlernen: Ach, Alch, Elf, Mag 8; Dru, Geo, Hex 12; Bor, Sch, Srl 14; 40 EP}
}


\newglossaryentry{penetrizzelTiefenblick_Talent}
{
    name={Penetrizzel Tiefenblick},
    description={Du kannst durch eine Wand von bis zu 1/2 Schritt Dicke sehen, solange die Wand nicht aus einem magischen oder magieabweisenden Material besteht.\newline Mächtige Magie: Die mögliche Dicke steigt um 1/2 Schritt.\newline Probenschwierigkeit: 12\newline Vorbereitungszeit: 2 Aktionen\newline Ziel: selbst\newline Reichweite: Berührung\newline Wirkungsdauer: 8 Initiativphasen\newline Kosten: 4 AsP\newline Fertigkeiten: Hellsicht\newline Erlernen: Mag 14; Elf, Sch, Srl 16; 20 EP}
}


\newglossaryentry{pestilenzerspüren_Talent}
{
    name={Pestilenz erspüren},
    description={Du erspürst die Krankheit in deinem Ziel. Du erfährst die Art der Krankheit, ihren Verlauf und die Ansteckungsgefahr.\newline Mächtige Magie: Anschließende Proben zur Heilung der Krankheit sind um +2 erleichtert.\newline Probenschwierigkeit: 12\newline Modifikationen: Gift entdecken (–4, 2 Aktionen; statt Krankheiten erkennst du Gifte.)\newline Vorbereitungszeit: 1 Stunde\newline Ziel: Einzelwesen\newline Reichweite: Berührung\newline Wirkungsdauer: augenblicklich\newline Kosten: 4 AsP\newline Fertigkeiten: Hellsicht\newline Erlernen: Dru 12; Hex 14; Elf 18; 20 EP}
}


\newglossaryentry{seelentiererkennen_Talent}
{
    name={Seelentier erkennen},
    description={Du erkennst das Seelentier deines Zieles. Wenn dir das Tier und seine assoziierten Eigenschaften bekannt sind, erleichtert dir das alle künftigen gesellschaftlichen Proben dem Ziel gegenüber um +2.\newline Probenschwierigkeit: Magieresistenz\newline Vorbereitungszeit: 4 Aktionen\newline Ziel: Einzelperson\newline Reichweite: 4 Schritt\newline Wirkungsdauer: augenblicklich\newline Kosten: 4 AsP\newline Fertigkeiten: Hellsicht\newline Erlernen: Hex, Sch 18; 40 EP}
}


\newglossaryentry{sensibarEmpathicus_Talent}
{
    name={Sensibar Empathicus},
    description={Du kannst die Gefühle deines Gegenübers erahnen, wodurch deine Menschenkenntnis gegen das Ziel um +4 steigt.\newline Mächtige Magie: Der Bonus steigt um +2.\newline Probenschwierigkeit: Magieresistenz\newline Vorbereitungszeit: 4 Aktionen\newline Ziel: Einzelperson\newline Reichweite: 4 Schritt\newline Wirkungsdauer: 1 Stunde\newline Kosten: 8 AsP\newline Fertigkeiten: Hellsicht\newline Erlernen: Elf 8; Hex, Mag 12; Ach, Dru, Geo, Sch, Srl 14; Alch 16; 40 EP}
}


\newglossaryentry{tiergedanken_Talent}
{
    name={Tiergedanken},
    description={Du siehst die Gedanken des Tiers als verschwommene Bilder.\newline Mächtige Magie: Zusätzlich kannst du das Tier um einen kleineren/größeren/gefährlichen Gefallen bitten. Ob das Tier den Gefallen ausführen kann und will, ist Spielleiterentscheid.\newline Probenschwierigkeit: Magieresistenz\newline Vorbereitungszeit: 4 Aktionen\newline Ziel: Tier\newline Reichweite: 8 Schritt\newline Wirkungsdauer: 4 Minuten\newline Kosten: 8 AsP\newline Fertigkeiten: Hellsicht, Verständigung\newline Erlernen: Elf, Geo 14; Ach, Dru, Hex 16; Sch 18; 20 EP}
}


\newglossaryentry{xenographusSchriftenkunde_Talent}
{
    name={Xenographus Schriftenkunde},
    description={Du kannst den Sinn eines geschriebenen Satzes verstehen, auch wenn du Schrift und Sprache nicht kennst.\newline Mächtige Magie: Verdoppelt die Anzahl der Sätze.\newline Probenschwierigkeit: 12\newline Vorbereitungszeit: 8 Aktionen\newline Ziel: Einzelobjekt\newline Reichweite: 4 Schritt\newline Wirkungsdauer: augenblicklich\newline Kosten: 4 AsP\newline Fertigkeiten: Hellsicht\newline Erlernen: Alch 18; Mag 20; 40 EP}
}


\newglossaryentry{balsamSalabunde_Talent}
{
    name={Balsam Salabunde},
    description={Dein Ziel erhält 2W6+4 Heilpunkte, für jede Überschreitung der WS heilst du eine Wunde.\newline Mächtige Magie: Erhöht die Heilpunkte um 4.\newline Probenschwierigkeit: 12+Wund-Mod. des Ziels\newline Modifikationen: Blutung stoppen (Probenschwierigkeit 16, 8 Aktionen; stoppt eine Blutung)\newline Sofortige Regeneration (–16, 4 Aktionen, 32 AsP; das Ziel erhält für 32 Initiativephasen den Vorteil Regeneration I (S. 97).)\newline Vorbereitungszeit: 16 Minuten\newline Ziel: Einzelwesen\newline Reichweite: Berührung\newline Wirkungsdauer: augenblicklich\newline Kosten: 8 AsP\newline Fertigkeiten: Humus, Verwandlung\newline Erlernen: Elf, Mag 8; Ach, Geo 12; Alch, Dru 14; Srl 16; 60 EP}
}


\newglossaryentry{herbeirufungdesHumus_Talent}
{
    name={Herbeirufung des Humus},
    description={Ruft ein Elementarwesen des jeweiligen Elements herbei (mehr zu Herbeirufungen S. 81), das in deiner unmittelbaren Nähe erscheint.\newline Probenschwierigkeit: 16/24/32 (Diener/Dschinn/Meister)\newline Vorbereitungszeit: frei wählbar\newline Ziel: einzelnes Elementar\newline Wirkungsdauer: augenblicklich\newline Kosten: 16/32/64 AsP (Diener/Dschinn/Meister)\newline Fertigkeiten: Humus\newline Erlernen: Geo 14; Ach, Dru, Mag 16; Alch 18; 60 EP\newline Anmerkung: Nicht überall sind die Wahren Namen und damit die Beschwörung von elementaren Dienern und vor allem Meistern bekannt. }
}


\newglossaryentry{fesselranken_Talent}
{
    name={Fesselranken},
    description={Aus dem Boden unter deinem Opfer wachsen Ranken hervor, die es mit einem Umklammern-Manöver (–4) festhalten, aus der es sich mit einer Aktion Konflikt und einer Konterprobe (GE oder KK, 16) befreien kann. Beachte, dass diese Probe ebenfalls durch das Umklammern erschwert ist.\newline Probenschwierigkeit: 12\newline Modifikationen: Dornenfessel (–8; misslungene Konterproben fügen dem Opfer 2W6 TP zu.)\newline Vorbereitungszeit: 1 Aktion\newline Ziel: Einzelperson\newline Reichweite: 8 Schritt\newline Wirkungsdauer: 4 Initiativphasen\newline Kosten: 4 AsP\newline Fertigkeiten: Humus\newline Erlernen: Dru, Geo 12; Hex 16; 20 EP}
}


\newglossaryentry{haselbuschundGinsterkraut_Talent}
{
    name={Haselbusch und Ginsterkraut},
    description={Du leitest das Wachstum einer Pflanze, kannst es beschleunigen und formen. Die Kosten, Zauberdauer und Modifikationen Mächtige Magie sind Meisterentscheid. Beispiele:\newline 4 AsP, keine Mächtige Magie, 1 Minute: Aus einem Samen wächst eine kleine Heilpflanze.\newline 8 AsP, 1x Mächtige Magie, eine halbe Stunde: Ein Baum trägt drei Monate zu früh Früchte.\newline 4 AsP, 2x Mächtige Magie, 4 Minuten: Die Rinde eines Baumes zeigt eine Botschaft;\newline 128 AsP, 4x Mächtige Magie, 4 Tage: Eine Baumkrone wird zu einem Baumhaus mit Möbeln, Dach und Fenstern\newline Probenschwierigkeit: 12\newline Vorbereitungszeit: nach Projekt\newline Ziel: Pflanze\newline Reichweite: Berührung\newline Wirkungsdauer: augenblicklich\newline Kosten: nach Projekt\newline Fertigkeiten: Humus, Verwandlung\newline Erlernen: Elf 14; Geo 18; Alch, Dru, Hex, Mag 20; 60 EP}
}


\newglossaryentry{hexenspeichel_Talent}
{
    name={Hexenspeichel},
    description={Dein Ziel erhält 2W6+4 Heilpunkte, für jede Überschreitung der WS heilst du eine Wunde.\newline Mächtige Magie: Erhöht die Heilpunkte um 4.\newline Probenschwierigkeit: 12+Wund-Mod. des Ziels\newline Modifikationen: Blutung stoppen (Probenschwierigkeit 16, 8 Aktionen; stoppt eine Blutung), Geheime Zutat (–4, Einzelobjekt, Wirkungsdauer 1 Stunde; der Zauber heilt den Esser des verzauberten Gerichts.)\newline Vorbereitungszeit: 16 Minuten\newline Ziel: Einzelwesen\newline Reichweite: Berührung\newline Wirkungsdauer: augenblicklich\newline Kosten: 8 AsP\newline Fertigkeiten: Humus, Verwandlung\newline Erlernen: Hex 8; 60 EP}
}


\newglossaryentry{humofaxius_Talent}
{
    name={Humofaxius},
    description={Eine Strahl aus elementarem Humus fügt dem Ziel 4W6 TP zu und verursacht Fesseln (S. 98). Ballistischer Zauber.\newline Mächtige Magie: Die TP steigen um 2W6.\newline Probenschwierigkeit: 12\newline Modifikationen: Gezielter Strahl (–4; du kannst die Trefferzone bestimmen.)\newline Vorbereitungszeit: 1 Aktion\newline Ziel: Einzelwesen, Einzelobjekt\newline Reichweite: 16 Schritt\newline Wirkungsdauer: augenblicklich\newline Kosten: 16 AsP\newline Fertigkeiten: Humus\newline Erlernen: Ach, Dru, Geo, Mag 18; 40 EP}
}


\newglossaryentry{humosphaero_Talent}
{
    name={Humosphaero},
    description={Du erschaffst eine elementare Kugel, die du mit Konzentration und Blickkontakt 16 Schritt pro Initiativephase bewegen kannst. Die Kugel explodiert, wenn du die Konzentration oder den Blickkontakt verlierst, du sie absichtlich zündest oder die Wirkungsdauer endet. Die Explosion richtet 4W6 TP an und verursacht Fesseln (S. 98). Pro Schritt Entfernung fällt der niedrigste Würfel weg.\newline Mächtige Magie: Die TP steigen um 1W6.\newline Probenschwierigkeit: 12\newline Modifikationen: Vorgegebene Bewegung (–4; du gibst der Kugel die Bewegung bis zur Explosion vor, Konzentration und Blickkontakt zur Kugel sind nicht nötig.)\newline Vorbereitungszeit: 2 Aktionen\newline Ziel: Zone\newline Reichweite: 2 Schritt\newline Wirkungsdauer: 2 Initiativphasen\newline Kosten: 16 AsP\newline Fertigkeiten: Humus\newline Erlernen: Ach, Mag 18; Geo, Dru 20; 60 EP}
}


\newglossaryentry{klarumPurum_Talent}
{
    name={Klarum Purum},
    description={Du stoppst die Wirkung eines Giftes bis Stufe 16 sofort.\newline Mächtige Magie: Die maximal aufgehobene Giftstufe steigt um 4.\newline Probenschwierigkeit: 12\newline Modifikationen: Schutz (Wirkungsdauer 8 Stunden; die verzauberte Person ist resistent gegen Gifte bis zur entsprechenden Giftstufe (S. 35). War sie bereits resistent, ist sie immun.)\newline Vorbereitungszeit: 2 Aktionen\newline Ziel: Einzelwesen\newline Reichweite: Berührung\newline Wirkungsdauer: augenblicklich\newline Kosten: 8 AsP\newline Fertigkeiten: Humus\newline Erlernen: Alch, Mag 12; Ach, Dru, Elf, Geo, Hex 18; 40 EP}
}


\newglossaryentry{kraftdesHumus_Talent}
{
    name={Kraft des Humus},
    description={Du erfüllst ein großes humusaffines Objekt wie einen Baum mit elementarer Lebenskraft. Wer im Radius von 8 Schritt eine Ruhepause verbringt, regeneriert eine zusätzliche Wunde.\newline Probenschwierigkeit: 12\newline Modifikationen: Heiliger Hain (–16, Wirkungsdauer bis die Bindung gelöst wird, 64 AsP, davon 16 gAsP)\newline Vorbereitungszeit: 1 Stunde\newline Ziel: Einzelwesen\newline Reichweite: Berührung\newline Wirkungsdauer: 1 Tag\newline Kosten: 16 AsP\newline Fertigkeiten: Humus\newline Erlernen: Dru, Geo 16; Hex 18; Ach, Elf 20; 40 EP}
}


\newglossaryentry{leibderErde_Talent}
{
    name={Leib der Erde},
    description={Du harmonierst mit dem Element Humus. Du bist immun gegen Gifte und Humusschaden.\newline Probenschwierigkeit: 12\newline Modifikationen: Reise in die Erde (–4; du kannst dich mit 2 Schritt pro Initiativephase durch Holz und Erde bewegen, als würdest du darin tauchen. Im Humus brauchst du nicht zu atmen.)\newline Begleiter (–4; der Zauber betrifft auch eine weitere Person, mit der du permanent Hautkontakt halten musst.)\newline Leib aus Humus (–8; Deine Kreaturenklasse wird zu "Elementar" mit allen entsprechenden Eigenschaften (S. 99). Du kannst während der Wirkungsdauer keine Zauber wirken.)\newline Vorbereitungszeit: 8 Aktionen\newline Ziel: selbst\newline Reichweite: Berührung\newline Wirkungsdauer: 1 Stunde\newline Kosten: 16 AsP\newline Fertigkeiten: Humus, Verwandlung\newline Erlernen: Geo, Hex 16; Elf, Dru, Mag 20; 60 EP}
}


\newglossaryentry{leidensbund_Talent}
{
    name={Leidensbund},
    description={Du übernimmst 2 Wunden von deinem Ziel.\newline Mächtige Magie: Du übernimmst 1 weitere Wunde.\newline Probenschwierigkeit: 12\newline Modifikationen: Krankheitsbund (Probenschwierigkeit Krankheitsstufe; du übernimmst die Krankheit deines Ziels.)\newline Giftbund (Probenschwierigkeit Giftstufe; du übernimmst das Gift von deinem Ziel.)\newline Heilender Dritter (–4; zwei Einzelpersonen; du überträgst die Wunden von einem Ziel auf ein freiwilliges anderes.)\newline Vorbereitungszeit: 4 Minuten\newline Ziel: Einzelperson\newline Reichweite: Berührung\newline Wirkungsdauer: augenblicklich\newline Kosten: 4 AsP\newline Fertigkeiten: Humus, Verständigung\newline Erlernen: Ach 18; Hex 20; 40 EP}
}


\newglossaryentry{pfeildesHumus_Talent}
{
    name={Pfeil des Humus},
    description={Du verzauberst einen Pfeil (oder Bolzen oder Wurfwaffe), sodass er im Flug die Macht des Elements freisetzt. Der Pfeil verursacht Humusschaden und Fesseln (S. 98). Unbelebte Gegenstände werden bei einem Treffer mit Ranken überzogen, die Klettern-Proben um +4 erleichtern.\newline Probenschwierigkeit: 12\newline Modifikationen: Geschütz (–4; du verzauberst ein größeres Geschoss wie das einer Balliste.)\newline Permanenz (–4, 4 AsP, davon 1 gAsP, Wirkungsdauer bis die Bindung gelöst wird oder der Pfeil verschossen wurde)\newline Vorbereitungszeit: 1 Aktion\newline Ziel: Einzelobjekt\newline Reichweite: Berührung\newline Wirkungsdauer: 8 Initiativphasen\newline Kosten: 4 AsP\newline Fertigkeiten: Humus\newline Erlernen: Elf 16; Ach 18; 40 EP}
}


\newglossaryentry{ruheKörper,RuheGeist_Talent}
{
    name={Ruhe Körper, Ruhe Geist },
    description={Dein Ziel sinkt in einen tiefen Schlaf, aus dem es nur mit Gewalt geweckt werden kann. Wird es nicht vorzeitig geweckt, regeneriert es zwei zusätzliche Wunden.\newline Mächtige Magie: Regeneriert eine weitere Wunde.\newline Probenschwierigkeit: 12\newline Vorbereitungszeit: 16 Aktionen\newline Ziel: Einzelperson\newline Reichweite: Berührung\newline Wirkungsdauer: 8 Stunden\newline Kosten: 8 AsP\newline Fertigkeiten: Humus\newline Erlernen: Elf 14; Ach, Alch, Dru, Hex, Mag 18; 20 EP}
}


\newglossaryentry{sumpfstrudel_Talent}
{
    name={Sumpfstrudel},
    description={Im Humusboden bildet sich ein morastiger Strudel. Wesen in der Umgebung müssen jede Initiativephase eine Konterprobe (KK, 16) ablegen, die um die Distanz zum Zentrum in Schritt erleichtert ist. Beim Misslingen wird es 2 Schritt in Richtung des Zentrums gezerrt. Im Zentrum erleidet jedes Ziel 8W6 SP und wird für die verbleibende Wirkungsdauer handlungsunfähig.\newline Probenschwierigkeit: 12\newline Modifikationen: Dornenranken (–8; jede misslungene Konterprobe verursacht 2W6 TP.)\newline Vorbereitungszeit: 8 Aktionen\newline Ziel: Zone\newline Reichweite: 64 Schritt\newline Wirkungsdauer: 8 Initiativphasen\newline Kosten: 32 AsP\newline Fertigkeiten: Humus, Umwelt\newline Erlernen: Geo 18; Dru 20; 60 EP}
}


\newglossaryentry{sumusElixiere_Talent}
{
    name={Sumus Elixiere},
    description={Du stärkst ein frisch gebrautes, heilendes oder kräftigendes Elixier. Es erhält eine zusätzliche Stufe Hohe Qualität.\newline Probenschwierigkeit: 12\newline Modifikationen: Potenzierung (–8; das Elixier muss nicht gerade frisch gebraut worden sein.)\newline Vorbereitungszeit: 4 Minuten\newline Ziel: Einzelobjekt\newline Reichweite: Berührung\newline Wirkungsdauer: 1 Tag\newline Kosten: 4 AsP\newline Fertigkeiten: Humus\newline Erlernen: Geo 14; Alch, Dru 16; Hex 18; Mag 20; 20 EP}
}


\newglossaryentry{tierebesprechen_Talent}
{
    name={Tiere besprechen},
    description={Dein Ziel erhält 4W6+8 Heilpunkte, für jede Überschreitung der WS wird eine Wunde geheilt.\newline Mächtige Magie: Erhöht die Heilpunkte um 8.\newline Probenschwierigkeit: 12+Wund-Mod des Ziels\newline Modifikationen: Blutung stoppen (Probenschwierigkeit 16, 8 Aktionen; du stoppst eine Blutung.)\newline Bann des Siechtums (Du beendest ein Gift oder eine Krankheit bis maximal Stufe 16. Mächtige Magie erhöht die maximal aufgehobene Gift-/Krankheitsstufe um 4.)\newline Vorbereitungszeit: 4 Minuten\newline Ziel: Tier\newline Reichweite: Berührung\newline Wirkungsdauer: augenblicklich\newline Kosten: 8 AsP\newline Fertigkeiten: Humus, Verwandlung\newline Erlernen: Hex 12; Dru, Geo 20; 20 EP}
}


\newglossaryentry{wandausDornen_Talent}
{
    name={Wand aus Dornen},
    description={Eine 3 Schritt hohe und bis zu 4 Schritt lange Wand aus spitzen Dornen wächst entlang einer von dir bestimmten Linie aus dem Boden. Sie hat eine Härte von 16 und verfügt über Regeneration I (S. 97). Um sie zu durchqueren, musst du sie mit einer Konterprobe (GE, 16) betreten. Nach je 4 Initiativephasen kannst du eine Konterprobe (GE, 16) ablegen, um die Wand zu verlassen. Durch den Versuch erleidest du 2W6 TP.\newline Mächtige Magie: Die Höhe der Wand steigt um 1 Schritt, die Länge um bis zu 2 Schritt, und die Härte um 8.\newline Probenschwierigkeit: 12\newline Vorbereitungszeit: 16 Aktionen\newline Ziel: Zone\newline Reichweite: 8 Schritt\newline Wirkungsdauer: 16 Minuten\newline Kosten: 8 AsP\newline Fertigkeiten: Humus\newline Erlernen: Geo, Mag 18; Dru 20; 40 EP}
}


\newglossaryentry{weisheitderBäume_Talent}
{
    name={Weisheit der Bäume},
    description={Du verwandelst dich in einen prächtigen Baum. Während deiner Zeit als Baum verlierst du das Bewusstsein, dafür fügen dir Gifte und Krankheiten keinen Schaden zu. Erleidest du mehr als 4 Wunden, verwandelst du dich zurück.\newline Probenschwierigkeit: 12\newline Modifikationen: Bewusst (–8; du bleibst bei Bewusstsein und kannst dich jederzeit zurückverwandeln.)\newline Vorbereitungszeit: 1 Stunde\newline Ziel: selbst\newline Reichweite: Berührung\newline Wirkungsdauer: 1 Jahr\newline Kosten: 8 AsP\newline Fertigkeiten: Humus, Verwandlung\newline Erlernen: Geo, Dru 16; Elf 20; 20 EP}
}


\newglossaryentry{aureolusGüldenglanz_Talent}
{
    name={Aureolus Güldenglanz},
    description={Eine feste Oberfläche von bis zu einem Rechtschritt sieht so aus, als wäre sie aus purem Gold. Dabei handelt es sich um eine Illusion (Sicht).\newline Mächtige Magie: Verdoppelt die Fläche.\newline Probenschwierigkeit: 12\newline Modifikationen: Farbenspiel (–4; du kannst auch eine andere Farbe oder ein Muster wählen.)\newline Wasserfläche (–4; du kannst auch andere Oberflächen wie deine Haut verzaubern.)\newline Vorbereitungszeit: 2 Aktionen\newline Ziel: Einzelobjekt\newline Reichweite: Berührung\newline Wirkungsdauer: 1 Tag\newline Kosten: 2 AsP\newline Fertigkeiten: Illusion\newline Erlernen: Srl 12; Mag 16; 20 EP}
}


\newglossaryentry{aurisNasusOculus_Talent}
{
    name={Auris Nasus Oculus},
    description={Eine statische Illusion (Sicht, Gehör oder Geruch) deiner Wahl erscheint. Ihre maximale Größe beträgt PW Illusionx4 Raumschritt.\newline Mächtige Magie: Die Illusion betrifft einen weiteren, oben genannten Sinn.\newline Probenschwierigkeit: 12\newline Modifikationen: Bewegte Illusion (-4; die Illusion führt eine beim Zaubern bestimmte Bewegung aus.)\newline Kontrollierte Illusion (–4; du kannst die Illusion aktiv steuern, was Konzentration erfordert.)\newline Vorbereitungszeit: 2 Aktionen\newline Ziel: Zone\newline Reichweite: 32 Schritt\newline Wirkungsdauer: 16 Minuten\newline Kosten: 8 AsP\newline Fertigkeiten: Illusion\newline Erlernen: Srl 12; Mag 14; 60 EP}
}


\newglossaryentry{blendwerk_Talent}
{
    name={Blendwerk},
    description={Eine Illusion (Sicht) deiner Wahl erscheint, die Bewegungen nach deinen Vorstellungen ausführt. Ihre maximale Größe beträgt PW Illusion x4 Raumschritt. Erfordert Konzentration.\newline Probenschwierigkeit: 12\newline Vorbereitungszeit: 2 Aktionen\newline Ziel: Zone\newline Reichweite: 32 Schritt\newline Wirkungsdauer: 16 Minuten\newline Kosten: 8 AsP\newline Fertigkeiten: Illusion\newline Erlernen: Sch 8; Mag, Srl 18; 40 EP}
}


\newglossaryentry{chamaelioniMimikry_Talent}
{
    name={Chamaelioni Mimikry},
    description={Du verschmilzt vollkommen mit der Umgebung und erhältst den Vorteil Tarnung (S. 98), solange du dich nicht bewegst. Proben auf Heimlichkeit sind um +4 erleichtert. Erlaubt Aufrechterhalten.\newline Mächtige Magie: Der Bonus steigt um +2.\newline Probenschwierigkeit: 12\newline Modifikationen: Tarnung (–4; du kannst dich mit 1 Schritt pro Aktion bewegen.)\newline Dinge Tarnen (–4, Einzelobjekt; du kannst einen maximal menschengroßen Gegenstand tarnen.)\newline Andere Tarnen (–4; der Zauber betrifft auch einen Begleiter, mit dem du permanent Hautkontakt haben musst. Mehrfach wählbar.)\newline Vorbereitungszeit: 0 Aktionen\newline Ziel: selbst\newline Reichweite: Berührung\newline Wirkungsdauer: 4 Minuten\newline Kosten: 8 AsP\newline Fertigkeiten: Illusion, Verwandlung\newline Erlernen: Elf 14; Hex, Mag 18; 40 EP}
}


\newglossaryentry{deliciosoGaumenschmaus_Talent}
{
    name={Delicioso Gaumenschmaus},
    description={Du kannst Geruch und Geschmack eines Gegenstandes – meist einer Speise – bestimmen. Dabei handelt es sich um eine Illusion (Geschmack und Geruch). Der Gegenstand kann maximal 1 Stein wiegen.\newline Mächtige Magie: Verdoppelt das maximale Gewicht.\newline Probenschwierigkeit: 12\newline Modifikationen: Aromenvielfalt (–4; du kannst die Geschmäcker einzelner Teile der Speise getrennt voneinander festlegen.)\newline Völlerei (–4; die Speise scheint während der Wirkungsdauer nicht zu sättigen.)\newline Vorbereitungszeit: 4 Aktionen\newline Ziel: Einzelobjekt\newline Reichweite: Berührung\newline Wirkungsdauer: 1 Stunde\newline Kosten: 2 AsP\newline Fertigkeiten: Illusion\newline Erlernen: Srl 14; Mag 16; Sch 18; 20 EP}
}


\newglossaryentry{duplicatus_Talent}
{
    name={Duplicatus},
    description={Ein mit dir verschwimmender Doppelgänger erscheint und verwirrt deine Gegner. Bei Nah- und Fernkampfangriffen wird ausgewürfelt, ob der Angriff dich oder einen Doppelgänger trifft. Der Doppelgänger ist eine Illusion (Sicht).\newline Mächtige Magie: Erschafft einen weiteren Doppelgänger.\newline Probenschwierigkeit: 12\newline Vorbereitungszeit: 1 Aktion\newline Ziel: Einzelperson\newline Reichweite: Berührung\newline Wirkungsdauer: 16 Initiativphasen\newline Kosten: 8 AsP\newline Fertigkeiten: Illusion\newline Erlernen: Mag, Srl 14; 40 EP}
}


\newglossaryentry{favilludoFunkentanz_Talent}
{
    name={Favilludo Funkentanz},
    description={Du erzeugst einen kunterbunten Funkenschwarm in einfachen geometrischen Formen deiner Wahl, der eine Person umgibt. Er ist offensichtlich als Illusion (Sicht) erkennbar.\newline Probenschwierigkeit: 12\newline Modifikationen: Subtiles Leuchten (–4, 4 AsP; alle Proben des Ziels auf Beeinflussung sind um +2 erleichtert.)\newline Leuchtender Panzer (–4, 4 AsP; AT auf das Ziel sind um –2 erschwert, gezielte Schläge unmöglich.)\newline Vorbereitungszeit: 0 Aktionen\newline Ziel: Einzelperson\newline Reichweite: 2 Schritt\newline Wirkungsdauer: 4 Minuten\newline Kosten: 1 AsP\newline Fertigkeiten: Illusion\newline Erlernen: Srl 12; Mag 14; Sch 18; 20 EP}
}


\newglossaryentry{flirrenderFunkelglanz_Talent}
{
    name={Flirrender Funkelglanz},
    description={Du erzeugst eine illusionäre Explosion aus Farben und Formen vor den Augen deines Ziels. Gelingt ihm keine Konterprobe (IN, 16), so ist es in seiner nächsten Initiativphase zu keiner Aktion fähig.\newline Probenschwierigkeit: 12\newline Vorbereitungszeit: 0 Aktionen\newline Ziel: Einzelperson\newline Reichweite: 4 Schritt\newline Wirkungsdauer: augenblicklich\newline Kosten: 4 AsP\newline Fertigkeiten: Illusion\newline Erlernen: Srl 16; Sch 20; 40 EP}
}


\newglossaryentry{impersonaMaskenbild_Talent}
{
    name={Impersona Maskenbild},
    description={Durch eine Illusion (Sicht) erscheint dein Gesicht und dein Haar wie das einer anderen Person. Du kannst nur Ziele kopieren, die du sehr gut kennst.\newline Mächtige Magie: Es reicht, wenn dir das Ziel persönlich bekannt ist/du es eine Weile beobachtet hast/du es vom Hörensagen kennst.\newline Probenschwierigkeit: 12\newline Modifikationen: Mal tarnen (–4, 4 AsP; die Illusion betrifft einen anderen Körperteil, wo sie Narben, Tätowierungen und eventuell sogar Dämonenmale verstecken oder erscheinen lassen kann.)\newline Karikatur (nur Srl; das Gesicht ist grotesk überzeichnet und eindeutig als Karikatur erkennbar.)\newline Vorbereitungszeit: 4 Minuten\newline Ziel: selbst\newline Reichweite: Berührung\newline Wirkungsdauer: 4 Stunden\newline Kosten: 8 AsP\newline Fertigkeiten: Illusion\newline Erlernen: Mag, Srl 16; 40 EP}
}


\newglossaryentry{lockrufundFeenfüße_Talent}
{
    name={Lockruf und Feenfüße},
    description={Einem kleinen, maximal 1 Stein schweren Objekt wachsen Feenfüße und es läuft zu dir. Die Feenfüße sind eine Illusion (Sicht).\newline Mächtige Magie: Verdoppelt das Gewicht.\newline Probenschwierigkeit: 12\newline Modifikationen: Da lang! (–4; du kannst ein anderes Ziel als dich selbst angeben.)\newline Erbsenparade (–4; der Zauber betrifft eine Gruppe von Gegenständen, die zusammen nicht mehr wiegen als das Maximalgewicht.)\newline Feenflügel (–8; das Objekt fliegt zu dir.)\newline Vorbereitungszeit: 0 Aktionen\newline Ziel: Einzelobjekt\newline Reichweite: 8 Schritt\newline Wirkungsdauer: augenblicklich\newline Kosten: 4 AsP\newline Fertigkeiten: Illusion, Umwelt\newline Erlernen: Sch 12; 20 EP}
}


\newglossaryentry{menetekelFlammenschrift_Talent}
{
    name={Menetekel Flammenschrift},
    description={Du lässt bis zu 64 Schriftzeichen mit einer Länge von bis zu 8 Schritt Gesamtgröße erscheinen. Die Zeichen sehen wie Flammenlinien, verschmiertes Blut oder astrales Glühen oder Ähnliches aus und sind eine Illusion (Sicht).\newline Mächtige Magie: Die Zahl der Schriftzeichen und deren maximale Größe verdoppelt sich.\newline Probenschwierigkeit: 12\newline Vorbereitungszeit: 8 Aktionen\newline Ziel: Zone\newline Reichweite: 32 Schritt\newline Wirkungsdauer: 1 Stunde\newline Kosten: 4 AsP\newline Fertigkeiten: Illusion\newline Erlernen: Mag, Srl 14; 20 EP}
}


\newglossaryentry{projektimagoEbenbild_Talent}
{
    name={Projektimago Ebenbild},
    description={Du lässt eine Illusion (Sicht und Gehör) deiner selbst an einem Ort erscheinen, den du schon einmal gesehen haben musst. Die Illusion bewegt sich während der Wirkungsdauer genau wie du.\newline Probenschwierigkeit: 12\newline Vorbereitungszeit: 4 Aktionen\newline Ziel: Zone\newline Reichweite: 4 Meilen\newline Wirkungsdauer: 16 Initiativphasen\newline Kosten: 8 AsP\newline Fertigkeiten: Illusion\newline Erlernen: Srl 16; Ach, Mag 20; 20 EP}
}


\newglossaryentry{reflectimagoSpiegelschein_Talent}
{
    name={Reflectimago Spiegelschein},
    description={Durch eine Illusion (Sicht) kannst du während der Wirkungsdauer bis zu 4 Objekten in deinem Sichtbereich eine verspiegelte Oberfläche geben, die eine Stunde hält. Das Objekt kann maximal so groß wie eine Tür sein.\newline Mächtige Magie: Die Zahl der Objekte steigt um 2.\newline Probenschwierigkeit: 12\newline Modifikationen: Handspiegel (–4, 1 AsP; ein Teil deines Körpers wird verspiegelt.)\newline Spiegelsaal (–4; du kannst die Spiegel in der Luft entstehen lassen.)\newline Optik-Trick (–4; du kannst die Formen und Brennweiten der Spiegel verändern.)\newline Vorbereitungszeit: 4 Aktionen\newline Ziel: selbst\newline Reichweite: Berührung\newline Wirkungsdauer: 16 Initiativephasen\newline Kosten: 8 AsP\newline Fertigkeiten: Illusion, Verwandlung\newline Erlernen: Mag 14; Srl 16; 20 EP}
}


\newglossaryentry{schelmenmaske_Talent}
{
    name={Schelmenmaske},
    description={Durch eine Illusion (Sicht) gleicht dein Körper dem eines Anderen. Du kannst nur Personen kopieren, die du sehr gut kennst.\newline Mächtige Magie: Es reicht, wenn dir das Ziel persönlich bekannt ist/du es eine Weile beobachtet hast/du es vom Hörensagen kennst.\newline Probenschwierigkeit: 12\newline Modifikationen: Tier (–4; auch etwa menschengroße Tiere können gewählt werden.)\newline Vorbereitungszeit: 2 Aktionen\newline Ziel: selbst\newline Reichweite: Berührung\newline Wirkungsdauer: 4 Minuten\newline Kosten: 8 AsP\newline Fertigkeiten: Illusion\newline Erlernen: Sch 12; Mag, Srl 20; 40 EP}
}


\newglossaryentry{seidenweichSchuppengleich_Talent}
{
    name={Seidenweich Schuppengleich},
    description={Du gibst durch eine Illusion (Tastsinn) dem verzauberten Gegenstand eine Textur deiner Wahl.\newline Probenschwierigkeit: 12\newline Modifikationen: Zähneknirschen (–4; die Illusion betrifft nur Gaumen und Zunge.)\newline Vorbereitungszeit: 4 Aktionen\newline Ziel: Einzelobjekt\newline Reichweite: Berührung\newline Wirkungsdauer: 8 Stunden\newline Kosten: 4 AsP\newline Fertigkeiten: Illusion\newline Erlernen: Sch 12; Srl 20; 20 EP}
}


\newglossaryentry{vocolimbohohlerKlang_Talent}
{
    name={Vocolimbo hohler Klang},
    description={Von einem Ort deiner Wahl aus erklingt eine Illusion (Gehör) als hohle Stimme, die eine von dir gewählte, bis zu 16 Worte lange Botschaft spricht.\newline Mächtige Magie: Verdoppelt die Länge der Botschaft.\newline Probenschwierigkeit: 12\newline Modifikationen: Zeitversetzt (–2; die Botschaft ertönt erst nach einem Zeitraum von bis zu 4 Initiativphasen. Mehrfach wählbar.)\newline Vox Memoriae (–4, Wirkungsdauer bis zur Sommersonnenwende, 8 AsP; die Botschaft ertönt erst, wenn ein beim Zaubern bestimmtes Ereignis eintritt.)\newline Vorbereitungszeit: 2 Aktionen\newline Ziel: Zone\newline Reichweite: 32 Schritt\newline Wirkungsdauer: 16 Initiativphasen\newline Kosten: 2 AsP\newline Fertigkeiten: Illusion\newline Erlernen: Srl 12; Mag 14; Sch 20; 20 EP}
}


\newglossaryentry{vogelzwitschernGlockenspiel_Talent}
{
    name={Vogelzwitschern Glockenspiel},
    description={Ein bestimmtes Geräusch, welches das Ziel erzeugt, klingt wie ein anderes Geräusch ähnlicher Lautstärke. Illusion (Gehör).\newline Probenschwierigkeit: 12\newline Modifikationen: Orchester (–4; du kannst ein weiteres Geräusch des gleichen Zieles verändern. Mehrfach wählbar.)\newline Spieluhr (–4, Einzelobjekt)\newline Permanenz (–4, Wirkungsdauer bis die Bindung gelöst wird, 8 AsP, davon 2 gAsP)\newline Vorbereitungszeit: 2 Aktionen\newline Ziel: Einzelperson\newline Reichweite: 2 Schritt\newline Wirkungsdauer: 8 Stunden\newline Kosten: 4 AsP\newline Fertigkeiten: Illusion\newline Erlernen: Srl 12; Mag 14; Sch 18; Alch 20; 20 EP}
}


\newglossaryentry{weihrauchwolkeWohlgeruch_Talent}
{
    name={Weihrauchwolke Wohlgeruch},
    description={Du verleihst deinem Ziel einen angenehmen Geruch deiner Wahl. Illusion (Geruch).\newline Probenschwierigkeit: 12\newline Modifikationen: Geruchsspender (–4, Ziel Einzelobjekt)\newline Gestank (–4, kombinierbar mit Geruchsspender; auch üble Gerüche sind möglich.)\newline Ausgedehnte Wirkung (–4, Wirkungsdauer 1 Woche, 8 AsP)\newline Künstlicher Geruch (–8; du kannst beliebige Gerüche erzeugen – auch solche, die eventuell noch nicht existieren.)\newline Vorbereitungszeit: 2 Aktionen\newline Ziel: Einzelperson\newline Reichweite: Berührung\newline Wirkungsdauer: 8 Stunden\newline Kosten: 4 AsP\newline Fertigkeiten: Illusion\newline Erlernen: Mag, Srl 14; Sch 20; 20 EP}
}


\newglossaryentry{applicatusZauberspeicher_Talent}
{
    name={Applicatus Zauberspeicher},
    description={Der nächste Zauber, den du während der Wirkungsdauer sprichst, wird für einen Tag im Ziel gespeichert. Der Zauber wird ausgelöst, sobald das Objekt bewegt wird oder ein einfacher, von dir bestimmter Auslöser eintritt. Er trifft denjenigen, der den Zauber auslöst oder eine Zone darum.\newline Probenschwierigkeit: 12\newline Modifikationen: Tragbar (–4; das Objekt kann bewegt werden.)\newline Hauswächter (–4, 16 AsP, nicht mit tragbar kombinierbar; der Zauber bleibt ein Jahr lang gespeichert.)\newline Vorbereitungszeit: 16 Aktionen\newline Ziel: Einzelobjekt\newline Reichweite: Berührung\newline Wirkungsdauer: 4 Minuten\newline Kosten: 8 AsP\newline Fertigkeiten: Kraft\newline Erlernen: Alch 14; Ach, Mag, Srl 18; 40 EP}
}


\newglossaryentry{arcanoviArtefakt_Talent}
{
    name={Arcanovi Artefakt},
    description={Der Arcanovi ist der bindende Spruch eines Artefaktes, in das du weitere Zauber sprechen kannst (mehr zu Artefakten siehe S. 78).\newline Probenschwierigkeit: nach Artefakt\newline Vorbereitungszeit: 8 Stunden\newline Ziel: Einzelobjekt\newline Reichweite: Berührung\newline Wirkungsdauer: nach Artefakt\newline Kosten: 8 AsP\newline Fertigkeiten: Kraft\newline Erlernen: Alch 12; Ach, Mag, Srl 16; Geo, Hex 18; Dru, Elf 20; 60 EP}
}


\newglossaryentry{augedesLimbus_Talent}
{
    name={Auge des Limbus },
    description={Du reißt eine Öffnung in die Barriere zwischen der Welt und dem Limbus. Ein Sog entsteht, in dessen Umgebung alle Wesen jede Initiativephase eine Konterprobe (KK, 16) ablegen, die um die Distanz zur Öffnung in Schritt erleichtert ist. Beim Misslingen wird es 1W6 Schritt in Richtung der Öffnung gezerrt. Wird ein Ziel durch die Öffnung gerissen, erleidet es 4W6 SP und findet sich im Limbus wieder.\newline Probenschwierigkeit: 12\newline Modifikationen: Tor in die Niederhöllen (–16; der Strudel führt direkt in die siebte Sphäre.)\newline Vorbereitungszeit: 8 Aktionen\newline Ziel: Zone\newline Reichweite: 16 Schritt\newline Wirkungsdauer: 8 Initiativphasen\newline Kosten: 32 AsP\newline Fertigkeiten: Kraft\newline Erlernen: Ach, Bor, Mag 20; 60 EP}
}


\newglossaryentry{fulminictusDonnerkeil_Talent}
{
    name={Fulminictus Donnerkeil},
    description={Eine unsichtbare Welle magischer Kraft fügt deinem Ziel 2W6 SP zu. Ein Fulminictus verursacht keine Wundschmerzeffekte.\newline Mächtige Magie: Erhöht die SP um 4.\newline Probenschwierigkeit: 12\newline Modifikationen: Welle des Schmerzes (–4, 16 AsP; der Zauber trifft alle Ziele in bis zu 4 Schritt Entfernung.)\newline Welle der Reinigung (1 AsP; der Zauber tötet alle Schädlinge und andere winzige Wesen in bis zu 8 Schritt Entfernung.)\newline Vorbereitungszeit: 0 Aktionen\newline Ziel: Einzelwesen\newline Reichweite: 8 Schritt\newline Wirkungsdauer: augenblicklich\newline Kosten: 8 AsP\newline Fertigkeiten: Kraft\newline Erlernen: Elf 8; Mag 14; Bor, Hex 18; 40 EP}
}


\newglossaryentry{invercanoSpiegeltrick_Talent}
{
    name={Invercano Spiegeltrick},
    description={Du verwandelst deine Hände in silbern glänzende Spiegel. Du kannst den nächsten, während der Wirkungsdauer auf dich gewirkten Zauber auf den Zauberer zurückwerfen, wenn dir eine Konterprobe (IN, 12) gelingt.\newline Mächtige Magie: Ein weiterer Zauber wird zurückgeworfen.\newline Probenschwierigkeit: 12\newline Modifikationen: Silberschild (–8, Einzelobjekt, Wirkungsdauer 1 Stunde; du verzauberst einen Schild, dessen Träger die Zauberwirkung nutzen kann.)\newline Vorbereitungszeit: 0 Aktionen\newline Ziel: selbst\newline Reichweite: Berührung\newline Wirkungsdauer: 16 Initiativphasen\newline Kosten: 16 AsP\newline Fertigkeiten: Kraft\newline Erlernen: Elf, Mag 18; 60 EP}
}


\newglossaryentry{körperloseReise_Talent}
{
    name={Körperlose Reise},
    description={Du trennst deinen Geist vom Körper. Dein Körper bleibt totengleich zurück, während sich dein Geist mit der Geschwindigkeit eines Pferdes bewegen kann. Er kann jederzeit den Limbus betreten oder verlassen. Im Limbus bewegt er sich mit einer Geschwindigkeit von 100 Meilen pro Stunde. Um massive Hindernisse zu durchqueren, sind nach Spielleiterentscheid Willenskraft-Proben notwendig, antimagische Materialien oder Orte bleiben dir verwehrt. Der Zauber endet erst, wenn Geist und Körper wieder vereint sind – nach dem Ende der Wirkungsdauer erleidest du allerdings alle 4 Minuten 1 Punkt Erschöpfung, bis Körper und Geist vereint sind.\newline Probenschwierigkeit: 12\newline Modifikationen: Manifestation (–4; du kannst Umstehenden während der Wirkungsdauer als geisterhafte Gestalt erscheinen.)\newline Fernzauber (–8; du kannst weiterhin Zauber wirken. Dabei musst du auf Geste und gesprochene Formel verzichten.)\newline Vorbereitungszeit: 1 Stunde\newline Ziel: selbst\newline Reichweite: Berührung\newline Wirkungsdauer: 1 Stunde\newline Kosten: 16 AsP\newline Fertigkeiten: Kraft\newline Erlernen: Dru 18; Ach, Bor, Geo, Hex, Mag 20; 60 EP}
}


\newglossaryentry{limbusversiegeln_Talent}
{
    name={Limbus versiegeln},
    description={Du verhinderst in einem kreisförmigen Bereich von 16 Schritt Radius sämtliche Wechsel zwischen dem Limbus und der dritten Sphäre.\newline Mächtige Magie: Der Radius erhöht sich um 8 Schritt.\newline Probenschwierigkeit: 12\newline Vorbereitungszeit: 4 Aktionen\newline Ziel: Zone\newline Reichweite: 16 Schritt\newline Wirkungsdauer: 1 Stunde\newline Kosten: 8 AsP\newline Fertigkeiten: Kraft\newline Erlernen: Mag 20; 20 EP}
}


\newglossaryentry{magischerRaub_Talent}
{
    name={Magischer Raub},
    description={Du kannst bei deinem nächsten Zauber teilweise oder ganz auf die AsP des Ziels zugreifen. Der nächste Zauber kann nicht auf das Ziel des magischen Raubs gewirkt werden.\newline Probenschwierigkeit: Magieresistenz\newline Vorbereitungszeit: 16 Aktionen\newline Ziel: Einzelperson\newline Reichweite: Berührung\newline Wirkungsdauer: 4 Stunden\newline Kosten: 4 AsP\newline Fertigkeiten: Kraft, Verständigung\newline Erlernen: Dru 12; Geo 14; Mag 18; Ach, Bor, Hex 20; 40 EP}
}


\newglossaryentry{tauschrausch_Talent}
{
    name={Tauschrausch},
    description={Zwei Gegenstände deiner Wahl, die jeweils maximal 2 Stein schwer sind und maximal 4 Schritt voneinander entfernt stehen, tauschen auf der Stelle ihren Platz. Die Gegenstände dürfen nicht magisch, geweiht, befestigt oder mit einem Lebewesen in Kontakt sein.\newline Mächtige Magie: Das maximale Gewicht und die maximale Distanz verdoppeln sich.\newline Probenschwierigkeit: 12\newline Modifikationen: Einmischen (–8; auch magische und geweihte Gegenstände können gewählt werden.)\newline Vorbereitungszeit: 1 Aktion\newline Ziel: zwei Objekte\newline Reichweite: 8 Schritt\newline Wirkungsdauer: augenblicklich\newline Kosten: 8 AsP\newline Fertigkeiten: Kraft, Umwelt\newline Erlernen: Sch, Srl 20; 20 EP}
}


\newglossaryentry{transversalisTeleport_Talent}
{
    name={Transversalis Teleport},
    description={Du teleportierst dich an einen beliebigen Ort. Der Ort darf maximal 4 Meilen entfernt sein und du musst schon einmal dort gewesen sein.\newline Mächtige Magie: Vervierfacht die Distanz.\newline Probenschwierigkeit: 12\newline Modifikationen: Anhalter (–4; du kannst eine Person mitnehmen. Mehrfach wählbar.)\newline Lastenteleport (–8, Einzelobjekt; du kannst eine Last von maximal 32 Stein teleportieren.)\newline Vorbereitungszeit: 1 Aktion\newline Ziel: selbst\newline Reichweite: Berührung\newline Wirkungsdauer: augenblicklich\newline Kosten: 16 AsP\newline Fertigkeiten: Kraft\newline Erlernen: Mag 18; 60 EP}
}


\newglossaryentry{verschwindibus_Talent}
{
    name={Verschwindibus},
    description={Dein bis zu 2 Stein schweres Ziel verschwindet für die Wirkungsdauer im näheren Limbus. Der Gegenstand darf nicht magisch, geweiht, befestigt oder mit einem Lebewesen in Kontakt sein.\newline Mächtige Magie: Verdoppelt das maximale Gewicht.\newline Probenschwierigkeit: 12\newline Modifikationen: Einmischen (–8; auch magische und geweihte Gegenstände können gewählt werden.)\newline Vorbereitungszeit: 0 Aktionen\newline Ziel: Einzelobjekt\newline Reichweite: 16 Schritt\newline Wirkungsdauer: 4 Minuten\newline Kosten: 8 AsP\newline Fertigkeiten: Kraft, Umwelt\newline Erlernen: Sch 8; Srl 16; 20 EP}
}


\newglossaryentry{zauberklingeGeisterspeer_Talent}
{
    name={Zauberklinge Geisterspeer},
    description={Die verzauberte Waffe gilt während der Wirkungsdauer als magisch.\newline Probenschwierigkeit: 12\newline Modifikationen: Schnellverzauberung (–4, 2 Aktionen, Wirkungsdauer 1 Stunde)\newline Personalisierung (–4; der Zauber wirkt nur, solange der beim Zaubern gewählte Träger die Waffe führt.)\newline Namenssigille (–8; erfordert den wahren Namen eines Dämons. Wann immer der Dämon Wunden durch diese Waffe erleidet, erleidet er eine zusätzliche Wunde.)\newline Permanenz (–4, Wirkungsdauer bis die Bindung gelöst wird, 16 AsP, davon 2 gAsP)\newline Vorbereitungszeit: 16 Aktionen\newline Ziel: Einzelobjekt\newline Reichweite: Berührung\newline Wirkungsdauer: 1 Woche\newline Kosten: 8 AsP\newline Fertigkeiten: Kraft, Verwandlung\newline Erlernen: Ach, Mag 18; Elf, Hex, Geo, Dru 20; 40 EP}
}


\newglossaryentry{aeolitusWindgebraus_Talent}
{
    name={Aeolitus Windgebraus},
    description={Ein kräftiger Windstoß breitet sich von deinem Mund kegelförmig (45°) 16 Schritt weit aus und verursacht Niederschmettern, wenn eine Konterprobe (KK, 16) misslingt. Ballistischer Zauber.\newline Probenschwierigkeit: 12\newline Modifikationen: Langer Atem (–8, 8 AsP; der Windstoß hält KO Initiativphasen lang an. Erfordert Konzentration.)\newline Sturm (–4; bei misslungener Konterprobe verursacht der Zauber auch Zurückstoßen (S. 98).)\newline Winde der anderen Art (–4; der Wind bringt lieblichen Duft oder widerlichen Gestank.)\newline Vorbereitungszeit: 0 Aktionen\newline Ziel: Zone\newline Reichweite: Berührung\newline Wirkungsdauer: augenblicklich\newline Kosten: 4 AsP\newline Fertigkeiten: Luft\newline Erlernen: Elf 8; Geo 12; Ach, Alch, Dru, Mag 14; Sch, Srl 18; 40 EP}
}


\newglossaryentry{aerofugoVakuum_Talent}
{
    name={Aerofugo Vakuum },
    description={Du entfernst sämtliche Luft aus einem Bereich mit bis zu 4 Schritt Radius. Flammen erlöschen, Lebewesen erleiden alle DH* Initiativephasen 1 Punkt Erschöpfung und Luftelementare erleiden 1 Wunde pro Initiativephase.\newline Mächtige Magie: Der Radius steigt um 2 Schritt.\newline Probenschwierigkeit: 12\newline Vorbereitungszeit: 4 Aktionen\newline Ziel: Zone\newline Reichweite: 8 Schritt\newline Wirkungsdauer: 4 Initiativphasen\newline Kosten: 16 AsP\newline Fertigkeiten: Luft, Umwelt\newline Erlernen: Ach, Alch, Mag 20; 40 EP}
}


\newglossaryentry{aerogeloAtemqual_Talent}
{
    name={Aerogelo Atemqual},
    description={Die Luft in einem Quader von 4 Schritt Kantenlänge wird so dicht wie Wasser. In der Zone kann man sich nur schwimmend fortbewegen und Kämpfer gelten als unter Wasser (–8).\newline Mächtige Magie: Die Kantenlänge steigt um 2 Schritt.\newline Probenschwierigkeit: 12\newline Vorbereitungszeit: 4 Aktionen\newline Ziel: Zone\newline Reichweite: 16 Schritt\newline Wirkungsdauer: 1 Stunde\newline Kosten: 8 AsP\newline Fertigkeiten: Luft, Umwelt\newline Erlernen: Mag 20; 40 EP}
}


\newglossaryentry{aeropulvissanfterFall_Talent}
{
    name={Aeropulvis sanfter Fall},
    description={Halbiert die effektive Höhe eines Sturzes oder Sprunges (kumulativ zur Akrobatik-Probe).\newline Probenschwierigkeit: 12\newline Vorbereitungszeit: 4 Aktionen\newline Ziel: selbst\newline Reichweite: Berührung\newline Wirkungsdauer: 16 Initiativphasen\newline Kosten: 8 AsP\newline Fertigkeiten: Luft\newline Erlernen: Geo 16; Elf 18; Mag 20; 40 EP}
}


\newglossaryentry{aufgeblasenabgehoben_Talent}
{
    name={Aufgeblasen abgehoben},
    description={Dein Opfer wirkt aufgeblasen und schwebt langsam immer höher. Dabei ist es wie ein Ballon dem Wind ausgeliefert. Wenn der Zauber endet, sinkt es wieder langsam zu Boden.\newline Probenschwierigkeit: Magieresistenz\newline Modifikationen: Höhe begrenzen (–4; die maximale Höhe beträgt 8 Schritt.)\newline Fesselballon (–4; das Opfer bewegt sich nur vertikal, bleibt aber sonst an Ort und Stelle.)\newline Tierballons (–4; der Zauber wirkt auf Tiere.)\newline Kunstflug (–8; das Opfer fliegt Figuren, die du mit deinen Armen vorgibst. Erfordert Konzentration.)\newline Vorbereitungszeit: 4 Aktionen\newline Ziel: Einzelperson\newline Reichweite: 8 Schritt\newline Wirkungsdauer: 1 Stunde\newline Kosten: 8 AsP\newline Fertigkeiten: Luft, Umwelt, Verwandlung\newline Erlernen: Sch 14; 40 EP}
}


\newglossaryentry{herbeirufungderLuft_Talent}
{
    name={Herbeirufung der Luft},
    description={Ruft ein Elementarwesen des jeweiligen Elements herbei (mehr zu Herbeirufungen S. 81), das in deiner unmittelbaren Nähe erscheint.\newline Probenschwierigkeit: 16/24/32 (Diener/Dschinn/Meister)\newline Vorbereitungszeit: frei wählbar\newline Ziel: einzelnes Elementar\newline Wirkungsdauer: augenblicklich\newline Kosten: 16/32/64 AsP (Diener/Dschinn/Meister)\newline Fertigkeiten: Luft\newline Erlernen: Geo 14; Ach, Dru, Mag 16; Alch 18; 60 EP\newline Anmerkung: Nicht überall sind die Wahren Namen und damit die Beschwörung von elementaren Dienern und vor allem Meistern bekannt. }
}


\newglossaryentry{leibdesWindes_Talent}
{
    name={Leib des Windes},
    description={Du harmonierst mit dem Element Luft. Du bist immun gegen Luftschaden, wiegst nur noch die Hälfte und Winde beeinflussen dich nicht.\newline Probenschwierigkeit: 12\newline Modifikationen: Reise im Wind (–4; du kannst dich mit GS Schritt pro Initiativephase durch die Luft bewegen, als würdest du darin tauchen.)\newline Begleiter (–4; der Zauber betrifft auch eine weitere Person, mit der du permanent Hautkontakt halten musst.)\newline Leib aus Luft (–8; Deine Kreaturenklasse wird zu "Elementar" mit allen entsprechenden Eigenschaften (S. 99). Du kannst während der Wirkungsdauer keine Zauber wirken.)\newline Vorbereitungszeit: 8 Aktionen\newline Ziel: selbst\newline Reichweite: Berührung\newline Wirkungsdauer: 1 Stunde\newline Kosten: 16 AsP\newline Fertigkeiten: Luft, Verwandlung\newline Erlernen: Elf 16; Hex 18; Ach, Dru, Geo, Mag 20; 60 EP}
}


\newglossaryentry{nebelleib_Talent}
{
    name={Nebelleib},
    description={Du verwandelst dich in Nebel, während deine Ausrüstung zurückbleibt. In deiner Nebelgestalt bist du immun gegen profanen Schaden, kannst durch schmalste Öffnungen dringen und mit deiner gewöhnlichen GS bis zu 8 Schritt über dem Boden fliegen. Dafür kannst du nicht mit anderen Wesen kommunizieren, keine Zauber wirken und musst bei starkem Wind Willenskraft-Proben ablegen, um nicht weggeweht zu werden.\newline Mächtige Magie: Deine GS steigt um +4.\newline Probenschwierigkeit: 12\newline Vorbereitungszeit: 8 Aktionen\newline Ziel: selbst\newline Reichweite: Berührung\newline Wirkungsdauer: 1 Stunde\newline Kosten: 16 AsP\newline Fertigkeiten: Luft, Verwandlung, Wasser\newline Erlernen: Dru, Geo 18; 40 EP}
}


\newglossaryentry{nebelwandundMorgendunst_Talent}
{
    name={Nebelwand und Morgendunst},
    description={Du erzeugst eine nahezu undurchsichtige Nebelwolke mit etwa 8 Schritt Radius und beliebiger, auch komplizierter, Form. Wind beeinflusst die Wolke nicht.\newline Mächtige Magie: Verdoppelt den Radius.\newline Probenschwierigkeit: 12\newline Modifikationen: Dunst (–4; nur dünner Dunst erscheint, aber der Radius vervierfacht sich.)\newline Nebelbilder (–8; du kannst die Form der Nebelwolke stets verändern.)\newline Geisternebel (–4, im Nebel bilden sich schreckliche Fratzen, die bei misslungener Konterprobe (MU, 16) einen Furcht-Effekt Stufe 1 verursachen.)\newline Begleiter (–4; die Wolke bewegt sich mit dir.)\newline Vorbereitungszeit: 2 Aktionen\newline Ziel: Zone\newline Reichweite: 8 Schritt\newline Wirkungsdauer: 2 Stunden\newline Kosten: 8 AsP\newline Fertigkeiten: Luft, Umwelt, Wasser\newline Erlernen: Elf 12; Ach, Dru, Geo, Srl16; Mag, Hex 20; 40 EP}
}


\newglossaryentry{orcanofaxius_Talent}
{
    name={Orcanofaxius},
    description={Eine Strahl aus elementarer Luft fügt dem Ziel 4W6 TP zu und verursacht Zurückstoßen (S. 98). Ballistischer Zauber.\newline Mächtige Magie: Die TP steigen um 2W6.\newline Probenschwierigkeit: 12\newline Modifikationen: Gezielter Strahl (–4; du kannst die Trefferzone bestimmen.)\newline Vorbereitungszeit: 1 Aktion\newline Ziel: Einzelwesen, Einzelobjekt\newline Reichweite: 16 Schritt\newline Wirkungsdauer: augenblicklich\newline Kosten: 16 AsP\newline Fertigkeiten: Luft\newline Erlernen: Ach, Dru, Geo, Mag 18; 40 EP}
}


\newglossaryentry{orcanosphaero_Talent}
{
    name={Orcanosphaero},
    description={Du erschaffst eine elementare Kugel, die du mit Konzentration und Blickkontakt 16 Schritt pro Initiativephase bewegen kannst. Die Kugel explodiert, wenn du die Konzentration oder den Blickkontakt verlierst, du sie absichtlich zündest oder die Wirkungsdauer endet. Die Explosion richtet 4W6 TP an und verursacht Zurückstoßen (S. 98). Pro Schritt Entfernung fällt der niedrigste Würfel weg.\newline Mächtige Magie: Die TP steigen um 1W6.\newline Probenschwierigkeit: 12\newline Modifikationen: Vorgegebene Bewegung (–4; du gibst der Kugel die Bewegung bis zur Explosion vor, Konzentration und Blickkontakt zur Kugel sind nicht nötig.)\newline Vorbereitungszeit: 2 Aktionen\newline Ziel: Zone\newline Reichweite: 2 Schritt\newline Wirkungsdauer: 2 Initiativphasen\newline Kosten: 16 AsP\newline Fertigkeiten: Luft\newline Erlernen: Ach, Mag 18; Geo, Dru 20; 60 EP}
}


\newglossaryentry{orkanwand_Talent}
{
    name={Orkanwand},
    description={Eine 8 Schritt hohe und bis zu 4 Schritt lange Wand aus rasenden Windhosen wächst entlang einer von dir bestimmten Linie aus dem Boden. Um sich ihr zu nähern, muss dir eine Konterprobe (KK, 16)  gelingen, sonst wirst du zurückgestoßen (S. 98). Das Durchqueren verursacht 2 Punkte Erschöpfung.\newline Mächtige Magie: Die Höhe steigt um 4 Schritt, die Länge steigt um bis zu 2 Schritt und das Durchqueren verursacht 1 weiteren Punkt Erschöpfung.\newline Probenschwierigkeit: 12\newline Vorbereitungszeit: 8 Aktionen\newline Ziel: Zone\newline Reichweite: 8 Schritt\newline Wirkungsdauer: 16 Minuten\newline Kosten: 8 AsP\newline Fertigkeiten: Luft\newline Erlernen: Geo, Mag 18; Dru 20; 40 EP}
}


\newglossaryentry{pfeilderLuft_Talent}
{
    name={Pfeil der Luft},
    description={Du verzauberst einen Pfeil (oder Bolzen oder Wurfwaffe), sodass er im Flug die Macht des Elements freisetzt. Der Pfeil verursacht Luftschaden und Zurückstoßen (S. 98). Die Reichweite für diesen Schuss ist verdoppelt.\newline Probenschwierigkeit: 12\newline Modifikationen: Geschütz (–4; du verzauberst ein größeres Geschoss wie das einer Balliste.)\newline Permanenz (–4, 4 AsP, davon 1 gAsP, Wirkungsdauer bis die Bindung gelöst wird oder der Pfeil verschossen wurde)\newline Vorbereitungszeit: 1 Aktion\newline Ziel: Einzelobjekt\newline Reichweite: Berührung\newline Wirkungsdauer: 8 Initiativphasen\newline Kosten: 4 AsP\newline Fertigkeiten: Luft\newline Erlernen: Elf 14; Mag 18; 40 EP}
}


\newglossaryentry{radau_Talent}
{
    name={Radau},
    description={Du verzauberst deinen Besen, sodass er ein zufälliges Ziel in einem Radius von 8 Schritt angreift (INI 8, RW 1, AT 8, TP 2W6). Fällt bei einer AT eine 1–4 (wird für den Besen passiv gewürfelt: bei einer VT eine 17–20), sucht sich der Besen ein neues Ziel. Während der Wirkungsdauer ist der Besen unzerbrechlich und magisch.\newline Mächtige Magie: AT und TP steigen um +2.\newline Probenschwierigkeit: 12\newline Modifikationen: Gezielte Wut (–4, Einzelwesen; der Besen greift nur ein von dir vorbestimmtes Ziel an.)\newline Vorbereitungszeit: 1 Aktion\newline Ziel: Einzelobjekt (Hexenbesen)\newline Reichweite: 16 Schritt\newline Wirkungsdauer: 16 Initiativphasen\newline Kosten: 8 AsP\newline Fertigkeiten: Luft, Umwelt\newline Erlernen: Hex 12; 40 EP}
}


\newglossaryentry{sapefactaZauberschwamm_Talent}
{
    name={Sapefacta Zauberschwamm},
    description={Deine Kleidung ist so sauber, als wäre sie frisch gewaschen.\newline Probenschwierigkeit: 12\newline Modifikationen: Läusekamm (–4; du wirst auch von Kleintieren und Ungeziefer befreit.)\newline Vorbereitungszeit: 4 Aktionen\newline Ziel: selbst\newline Reichweite: Berührung\newline Wirkungsdauer: augenblicklich\newline Kosten: 1 AsP\newline Fertigkeiten: Luft, Umwelt, Wasser\newline Erlernen: Alch 12; Mag 14; Dru, Srl 16; Hex 20; 20 EP}
}


\newglossaryentry{solidiridWegausLicht_Talent}
{
    name={Solidirid Weg aus Licht},
    description={Vor dir entsteht eine Brücke aus Licht, die in allen Regenbogenfarben schillert. Die Brücke ist maximal 8 Schritt lang, 1 Schritt breit und kann einen Höhenunterschied von bis zu 1 Schritt pro 4 Schritt Länge überwinden. Von oben ist die Brücke undurchdringlich, aber von unten kann sie problemlos durchdrungen werden.\newline Mächtige Magie: Die Länge der Brücke steigt um bis zu 4 Schritt, die Breite um bis zu 1 Schritt.\newline Probenschwierigkeit: 12\newline Modifikationen: Bogentreppe (–4; der maximale Höhenunterschied ist verdoppelt. Mehrfach wählbar.)\newline Vorbereitungszeit: 8 Aktionen\newline Ziel: Zone\newline Reichweite: Berührung\newline Wirkungsdauer: 4 Minuten\newline Kosten: 8 AsP\newline Fertigkeiten: Luft, Umwelt\newline Erlernen: Elf 16; Mag 20, 40 EP}
}


\newglossaryentry{wettermeisterschaft_Talent}
{
    name={Wettermeisterschaft},
    description={Das Wetter in einer Zone von bis zu 1 Meile Radius verändert sich nach deinem Willen. Du kannst folgende Skalen um insgesamt zwei Stufen verändern:\newline Niederschlag: trocken/Nieselregen/Regen/starker Regen/Wolkenbruch\newline Wind: windstill/leichte Brise/steife Brise/Sturm/Orkan\newline Temperatur: heiß/warm/mittel/kühl/kalt.\newline Mächtige Magie: Du kannst die Skalen um eine weitere Stufe verändern.\newline Probenschwierigkeit: 12\newline Vorbereitungszeit: 4 Minuten\newline Ziel: Zone\newline Reichweite: 1 Meile\newline Wirkungsdauer: 1 Stunde\newline Kosten: 32 AsP\newline Fertigkeiten: Luft, Umwelt\newline Erlernen: Dru 14; Geo 16; Ach 18; Hex, Mag 20; 40 EP}
}


\newglossaryentry{windgeflüster_Talent}
{
    name={Windgeflüster},
    description={Eine Botschaft von maximal 16 Worten wird von Luftelementaren binnen weniger Minuten zu deinem Empfänger weitergeflüstert. Ist dieser unaufmerksam oder befindet er sich an einem von anderen Elementen umschlossenen Ort, verhallt die Botschaft ungehört.\newline Mächtige Magie: Du kannst weitere 8 Worte übertragen.\newline Probenschwierigkeit: 12\newline Modifikationen: Sturmgebrüll (–4; die Botschaft wird laut wie Sturmböen und Donnerhall wiedergegeben. Sie ist kaum zu überhören, aber auch alles andere als privat.)\newline Vorbereitungszeit: 2 Aktionen\newline Ziel: Einzelperson\newline Reichweite: 8 Meilen\newline Wirkungsdauer: augenblicklich\newline Kosten: 4 AsP\newline Fertigkeiten: Luft, Verständigung\newline Erlernen: Dru 12; Geo 14; Elf 16; 20 EP}
}


\newglossaryentry{windhose_Talent}
{
    name={Windhose},
    description={Du erzeugst einen Wirbelsturm, den du mit einer Geschwindigkeit von 4 Schritt pro Initiativephase bewegen kannst. Der Wirbelsturm hat einen Radius von 2 Schritt. Wer immer ihn berührt, erleidet pro Initiativephase 1W6 TP durch umherfliegende Kleinteile und muss eine Konterprobe (KK, 16) ablegen, um nicht in eine zufällige Richtung zurückgestoßen (S. 98) zu werden. Erfordert Konzentration.\newline Mächtige Magie: Gegenstände und Tiere bis zur Größe eines Kopfes/Rucksacks/Menschen/ Pferdes wirbeln durch die Luft und verursachen zusätzliche 1W6 TP pro Initiativphase. Der Radius des Wirbelsturms steigt um je 1 Schritt.\newline Probenschwierigkeit: 12\newline Modifikationen: Beweglich (–4; erhöht die Geschwindigkeit des Wirbelsturms um 4 Schritt pro Initiativephase. Mehrfach wählbar.)\newline Vorbereitungszeit: 8 Aktionen\newline Ziel: Zone\newline Reichweite: 64 Schritt\newline Wirkungsdauer: 16 Initiativphasen\newline Kosten: 16 AsP\newline Fertigkeiten: Luft, Umwelt\newline Erlernen: Dru 14; Geo 16; Mag 20; 40 EP}
}


\newglossaryentry{windstille_Talent}
{
    name={Windstille},
    description={In einem Radius von 32 Schritt senkst du die Windstärke auf einer Skala von windstill/leichte Brise/steife Brise/Sturm/Orkan um zwei Stufen.\newline Mächtige Magie: Du senkst die Windstärke um eine weitere Stufe.\newline Probenschwierigkeit: 12\newline Modifikationen: Anderes Zentrum (–4, Zone, 16 Schritt)\newline Vorbereitungszeit: 16 Aktionen\newline Ziel: selbst\newline Reichweite: Berührung\newline Wirkungsdauer: 4 Minuten\newline Kosten: 8 AsP\newline Fertigkeiten: Luft, Umwelt\newline Erlernen: Elf 12; Mag 20; 20 EP}
}


\newglossaryentry{chronoklassisUrfossil_Talent}
{
    name={Chronoklassis Urfossil},
    description={Du rufst ein Objekt oder Lebewesen aus der Vergangenheit in die Gegenwart. Dabei kannst du maximal einen Zeitabstand von 1 Jahr überwinden und musst den Ort und die Zeit des Zieles kennen. Nach dem Ende des Zaubers fällt das Ziel zurück in die Vergangenheit und hat keine Erinnerung an den Vorfall.\newline Mächtige Magie: Verzehnfacht den Zeitabstand.\newline Probenschwierigkeit: 12\newline Modifikationen: Unbekannter Ort (–8; der Ort muss nicht bekannt sein.)\newline Unbekannte Zeit (–8; die Zeit muss nicht bekannt sein.)\newline Ausgangszustand (–4; am Ziel bleiben keine Spuren seiner Zeit in der Gegenwart zurück)\newline Vorbereitungszeit: 8 Stunden\newline Ziel: Einzelobjekt oder Einzelwesen\newline Reichweite: dereweit\newline Wirkungsdauer: 1 Stunde\newline Kosten: 16 AsP\newline Fertigkeiten: Temporal\newline Erlernen: Ach 20; 40 EP}
}


\newglossaryentry{chrononautosZeitenfahrt_Talent}
{
    name={Chrononautos Zeitenfahrt},
    description={Du erschaffst ein Tor durch die Zeiten. Wer es durchschreitet, gelangt an einen von dir bestimmten Zeitpunkt, der maximal 1 Jahr zurück liegen darf. Das Tor existiert nur in eine Richtung – Zeitreisende werden am Ende der Wirkungsdauer automatisch in ihre Zeit zurück gerissen.\newline Mächtige Magie: Verzehnfacht den Zeitabstand.\newline Probenschwierigkeit: 12\newline Vorbereitungszeit: 8 Stunden\newline Ziel: Zone\newline Reichweite: 2 Schritt\newline Wirkungsdauer: 8 Stunden\newline Kosten: 64 AsP\newline Fertigkeiten: Temporal\newline Erlernen: Ach 20; 80 EP}
}


\newglossaryentry{eisenrostundPatina_Talent}
{
    name={Eisenrost und Patina},
    description={Du lässt einen sehr kleinen (S. 46), metallischen Gegenstand altern. Innerhalb der nächsten 16 Initiativphasen verschiebt sich sein Zustand auf der Skala neu/benutzt/verschlissen/alt/unbrauchbar/zerstört um zwei Stufen. Magische und geweihte Gegenstände sind gegen den Zauber immun.\newline Mächtige Magie: Du kannst den Zustand des Gegenstandes um eine weitere Stufe verschieben.\newline Probenschwierigkeit: 12\newline Modifikationen: Größeres Ziel (–4; das Ziel kann eine Kategorie größer sein. Mehrfach wählbar.)\newline Rostträger (–4; du kannst das Objekt auch mit einem beliebigen Gegenstand wie deiner Waffe berühren.)\newline Vorbereitungszeit: 0 Aktionen\newline Ziel: Einzelobjekt\newline Reichweite: Berührung\newline Wirkungsdauer: augenblicklich\newline Kosten: 4 AsP\newline Fertigkeiten: Temporal, Verwandlung\newline Erlernen: Mag 12; Ach, Alch, Dru 14; Sch 16; 40 EP}
}


\newglossaryentry{lastdesAlters_Talent}
{
    name={Last des Alters},
    description={Dein Opfer altert schlagartig um 8 Jahre, Auswirkungen davon sind Spielleiterentscheid.\newline Mächtige Magie: Das Opfer altert um weitere 4 Jahre.\newline Probenschwierigkeit: Magieresistenz\newline Modifikationen: Verschrumpelte Glieder (–4, 8 AsP, davon 1 gAsP; der Zauber betrifft nur ein einziges Körperteil des Opfers.)\newline Vorbereitungszeit: 1 Stunde\newline Ziel: Einzelperson\newline Reichweite: Berührung\newline Wirkungsdauer: bis die Bindung gelöst wird\newline Kosten: 16 AsP, davon 2 gAsP\newline Fertigkeiten: Temporal, Verwandlung\newline Erlernen: Bor 20; 20 EP}
}


\newglossaryentry{objectofixo_Talent}
{
    name={Objectofixo},
    description={Du fixierst einen Gegenstand von maximal 2 Stein auf einer waagrechten Fläche. Er kann nicht bewegt werden.\newline Mächtige Magie: Das maximale Gewicht verdoppelt sich.\newline Probenschwierigkeit: 12\newline Modifikationen: Bilderhaken (–4; du fixierst einen Gegenstand an einer senkrechten Fläche)\newline Spinnenfest (–8; du fixierst einen Gegenstand unter einer waagrechten Fläche.)\newline Lufthaken (–12; du fixierst einen Gegenstand in der Luft.)\newline Vorbereitungszeit: 4 Aktionen\newline Ziel: Einzelobjekt\newline Reichweite: Berührung\newline Wirkungsdauer: 4 Stunden\newline Kosten: 4 AsP\newline Fertigkeiten: Temporal, Verwandlung\newline Erlernen: Ach 14; Mag 16; Alch 18; 40 EP}
}


\newglossaryentry{tempusStasis_Talent}
{
    name={Tempus Stasis},
    description={Du hältst in einem Radius von 8 Schritt die Zeit an – für alles außer dich. Alle betroffenen Wesen und Gegenstände sind unbeweglich und völlig unzerstörbar. Blutungen, Gifte und Krankheiten werden angehalten und Geschosse und ballistische Zauber bleiben im Flug stehen und bewegen sich erst nach dem Ende der Wirkungsdauer weiter.\newline Mächtige Magie: Du kannst eine weitere Person von der Wirkung ausnehmen.\newline Probenschwierigkeit: 12\newline Vorbereitungszeit: 1 Aktion\newline Ziel: Zone\newline Reichweite: Berührung\newline Wirkungsdauer: 4 Initiativphasen\newline Kosten: 16 AsP\newline Fertigkeiten: Temporal\newline Erlernen: Ach, Mag 20; 60 EP}
}


\newglossaryentry{unberührtvonSatinav_Talent}
{
    name={Unberührt von Satinav},
    description={Du stoppst den natürlichen Verfall eines Gegenstandes von bis zu 16 Stein Gewicht. Der Gegenstand wird weder verrotten, noch von Parasiten befallen – er bleibt genau wie zum Zeitpunkt der Verzauberung.\newline Mächtige Magie: Das maximale Gewicht steigt um 8 Stein.\newline Probenschwierigkeit: 12\newline Vorbereitungszeit: 4 Minuten\newline Ziel: Einzelobjekt\newline Reichweite: Berührung\newline Wirkungsdauer: 1 Woche\newline Kosten: 4 AsP\newline Fertigkeiten: Temporal, Verwandlung\newline Erlernen: Ach 12; Alch 14; Mag 16; 20 EP}
}


\newglossaryentry{animatiostummerDiener_Talent}
{
    name={Animatio stummer Diener},
    description={Du verzauberst ein Objekt von maximal 4 Stein, sodass es eine einfache Bewegung wiederholt. Du musst die Bewegung während der Vorbereitungszeit sieben Mal eigenhändig ausführen und ein Signal bestimmen (z.B. ein Fingerschnippen oder ein Wort). Gibst du das Signal, führt der Gegenstand die Bewegung aus. Seine KK entspricht deiner KK während der Vorbereitungszeit.\newline Mächtige Magie: Das maximale Gewicht verdoppelt sich.\newline Probenschwierigkeit: 12\newline Modifikationen:  Mehrere Objekte (–4; du kannst bis zu FF Objekte gleichzeitig verzaubern, solange ihre Bewegungen zusammengehören.)\newline Mehrere Auslösende (–2; eine zusätzliche Person kann das Signal geben. Mehrfach wählbar.)\newline Geworfen (–4; du musst das Objekt während der Vorbereitungszeit nicht permanent in der Hand halten.)\newline Tagelöhner (4 AsP, Wirkungsdauer 8 Stunden)\newline Ad Infinitum Ad Nauseam (–8; das Objekt wiederholt die Bewegung immer wieder, bis du das Signal wiederholst.)\newline Vorbereitungszeit: 4 Minuten\newline Ziel: Einzelobjekt\newline Reichweite: Berührung\newline Wirkungsdauer: 1 Jahr\newline Kosten: 16 AsP\newline Fertigkeiten: Umwelt\newline Erlernen: Mag 18; 60 EP}
}


\newglossaryentry{dunkelheit_Talent}
{
    name={Dunkelheit},
    description={Um deinen Körper herum entsteht eine Zone der Dunkelheit von 4 Schritt Radius. Die Helligkeit sinkt um eine Stufe (etwa von normal auf Dämmerung, von Sternenlicht auf absolute Dunkelheit, S. 38). Du selbst bist von dem Zauber nicht betroffen.\newline Mächtige Magie: Senkt die Helligkeit um eine weitere Stufe.\newline Probenschwierigkeit: 12\newline Modifikationen: Begleiter (–4; die Dunkelheit bewegt sich mit dir.)\newline Vorbereitungszeit: 8 Aktionen\newline Ziel: Zone\newline Reichweite: Berührung\newline Wirkungsdauer: 4 Minuten\newline Kosten: 8 AsP\newline Fertigkeiten: Umwelt\newline Erlernen: Dru 8; Alch, Geo, Mag 16; Ach, Hex, Srl 18; Elf 20; 40 EP}
}


\newglossaryentry{flimFlamFunkel_Talent}
{
    name={Flim Flam Funkel},
    description={Ein bläuliches Licht erscheint und steigert die Helligkeit in einem Radius von 4 Schritt um eine Stufe (etwa von Mondlicht auf Dämmerung S. 38). Erlaubt Aufrechterhalten.\newline Mächtige Magie: Steigert die Helligkeit um eine weitere Stufe. Nach je 4 Schritt Radius nimmt die Helligkeit um eine Stufe weniger zu.\newline Probenschwierigkeit: 12\newline Modifikationen: Variable Helligkeit (–4; du kannst die Helligkeit beliebig reduzieren und wieder auf das ursprüngliche Maß erhöhen.)\newline Begleiter (–4; die Lichtkugel bewegt sich mit dir.)\newline Leuchtturm (–4; Die Lichtkugel erscheint in 10 Schritt Höhe.)\newline Lichtblitz (–8, 8 AsP; du kannst die Kugel in einer Aktion Bereit machen explodieren lassen. Alle Beobachter sind 2 Initiativphasen lang geblendet, ihre Proben sind um die doppelte Helligkeitsstufe erschwert.)\newline Vorbereitungszeit: 0 Aktionen\newline Ziel: Zone\newline Reichweite: 8 Schritt\newline Wirkungsdauer: 4 Minuten\newline Kosten: 1 AsP\newline Fertigkeiten: Umwelt\newline Erlernen: Ach, Alch, Elf, Mag 8; Dru, Hex 12;  Geo 14; Sch, Srl 16; 20 EP}
}


\newglossaryentry{foramenForaminor_Talent}
{
    name={Foramen Foraminor},
    description={Du öffnest ein Schloss beliebiger Bauart.\newline Probenschwierigkeit: Herstellungsschwierigkeit des Schlosses\newline Modifikationen: Riegel (–4, 8 AsP; auch ein schwerer Riegel wie an einem Stadttor kann hiermit geöffnet werden.)\newline Vorbereitungszeit: 2 Aktionen\newline Ziel: Einzelobjekt\newline Reichweite: Berührung\newline Wirkungsdauer: augenblicklich\newline Kosten: 4 AsP\newline Fertigkeiten: Umwelt\newline Erlernen: Alch, Mag 14; Srl 16; Sch 18; 40 EP}
}


\newglossaryentry{hexenholz_Talent}
{
    name={Hexenholz},
    description={Du kannst einen Gegenstand mit einer KK von 2 anheben und mit maximal 4 Schritt pro Initiativephase durch die Luft fliegen lassen. Besonders komplizierte und feine Bewegungen sind um bis zu 8 Punkte erschwert. Erfordert Konzentration.\newline Mächtige Magie: Erhöht KK und die Schritte pro Initiativephase um +2.\newline Probenschwierigkeit: 12\newline Modifikationen: Unsichtbarer Hieb (Wirkungsdauer augenblicklich, 4 AsP; der Zauber fügt einem Objekt 2W6 SP zu. Mächtige Magie erhöht den Schaden um +4.)\newline Magische Abwehr (–4, wird in einer Reaktion gewirkt, Wirkungsdauer augenblicklich, 4 AsP; die Zauberprobe gilt als VT gegen einen Nahkampfangriff in Reichweite.)\newline Fesselfeld (–8, 32 AsP; in einer Zone von 4 Schritt Radius wird die Bewegung jedes unbelebten Objekts mit einer KK von 2 behindert, was Kampfhandlungen mit Waffen fast unmöglich macht.)\newline Vorbereitungszeit: 1 Aktion\newline Ziel: Einzelobjekt\newline Reichweite: 8 Schritt\newline Wirkungsdauer: 4 Minuten\newline Kosten: 4 AsP\newline Erlernen: Hex 8; 60 EP}
}


\newglossaryentry{holterdipolter_Talent}
{
    name={Holterdipolter},
    description={In einem Radius von 8 Schritt um dich herum geht alles schief: Menschen stolpern, Knoten öffnen sich und Vasen fallen um. Dabei kommt niemand ernsthaft zu Schaden.\newline Probenschwierigkeit: 12\newline Vorbereitungszeit: 1 Aktion\newline Ziel: Zone\newline Reichweite: Berührung\newline Wirkungsdauer: 4 Minuten\newline Kosten: 16 AsP\newline Fertigkeiten: Umwelt\newline Erlernen: Sch 20; 40 EP}
}


\newglossaryentry{klickeradomms_Talent}
{
    name={Klickeradomms},
    description={Du zerbrichst einen Gegenstand, den du auch mit einem Fausthieb zertrümmern könntest. Der Gegenstand darf maximal 1 Stein wiegen.\newline Mächtige Magie: Das maximale Gewicht steigt um einen halben Stein.\newline Probenschwierigkeit: 12\newline Vorbereitungszeit: 0 Aktionen\newline Ziel: Einzelobjekt\newline Reichweite: 8 Schritt\newline Wirkungsdauer: augenblicklich\newline Kosten: 2 AsP\newline Fertigkeiten: Umwelt\newline Erlernen: Sch 8; Hex, Mag, Srl 18; 20 EP}
}


\newglossaryentry{kulminatioKugelblitz_Talent}
{
    name={Kulminatio Kugelblitz},
    description={Du schießt einen erratischen Kugelblitz auf dein Ziel. Der Kugelblitz nähert sich dem Ziel 4 Initiativephasen lang mit 3W6 Schritt pro Initiativephase (einmal auswürfeln). Wenn er das Ziel erreicht, richtet er 1W20 SP an.\newline Mächtige Magie: Die SP steigen um 4.\newline Probenschwierigkeit:12\newline Modifikationen: Schneller Blitz (–4; der Blitz bewegt sich um 1W6 Schritt schneller. Mehrfach wählbar.)\newline Wächter (–4, Wirkungsdauer 1 Stunde, nur Ach; der Kugelblitz verharrt still und verfolgt das erste Wesen (außer dir), das sich ihm auf weniger als 8 Schritt nähert.)\newline Wolke von A‘Tall (–8, 16 Aktionen, Zone, Wirkungsdauer 1 Stunde, 32 AsP; dir folgt eine Gewitterwolke. Auf dein Kommando lösen sich einmalig 2W6 Kugelblitze aus ihr.)\newline Vorbereitungszeit: 2 Aktionen\newline Ziel: Einzelwesen\newline Reichweite: 32 Schritt\newline Wirkungsdauer: augenblicklich\newline Kosten: 8 AsP\newline Fertigkeiten: Umwelt\newline Erlernen: Ach, Mag 20; 40 EP}
}


\newglossaryentry{mahlstrom_Talent}
{
    name={Mahlstrom},
    description={Du erzeugst einen Strudel im Wasser. Wesen in der Umgebung müssen jede Initiativephase eine Konterprobe (Schwimmen, 16) ablegen, die um die Distanz zum Zentrum in Schritt erleichtert ist. Beim Misslingen wird es 4 Schritt näher ins Zentrum des Mahlstroms gezogen. Im Zentrum erleidet jedes Ziel 4W6 SP und wird in die Tiefe gezogen.\newline Probenschwierigkeit: 12\newline Modifikationen: Schiffverschlinger (–4, 4 Minuten, Wirkungsdauer 1 Stunde, 64 AsP; der Strudel ist so groß, dass er sogar Schiffe verschlingen kann. Die Konterprobe ist nur um 1 pro 8 Schritt Distanz erleichtert.)\newline Vorbereitungszeit: 8 Aktionen\newline Ziel: Zone\newline Reichweite: 64 Schritt\newline Wirkungsdauer: 16 Initiativphasen\newline Kosten: 16 AsP\newline Fertigkeiten: Umwelt, Wasser\newline Erlernen: Dru 18; Geo, Mag 20; 40 EP}
}


\newglossaryentry{motoricusGeisterhand_Talent}
{
    name={Motoricus Geisterhand},
    description={Du kannst einen Gegenstand mit einer KK von 2 anheben und mit maximal 4 Schritt pro Initiativephase durch die Luft fliegen lassen. Besonders komplizierte und feine Bewegungen sind um bis zu 8 Punkte erschwert. Erfordert Konzentration.\newline Mächtige Magie: Erhöht KK und die Schritte pro Initiativephase um +2.\newline Probenschwierigkeit: 12\newline Modifikationen: Unsichtbarer Hieb (Wirkungsdauer augenblicklich; der Zauber fügt einem Objekt 2W6 SP zu. Mächtige Magie erhöht den Schaden um +4.)\newline Magische Abwehr (–4, wird in einer Reaktion gewirkt, Wirkungsdauer augenblicklich; die Zauberprobe gilt als VT gegen einen Nahkampfangriff in Reichweite.)\newline Fesselfeld (–8, 32 AsP; in einer Zone von 4 Schritt Radius wird die Bewegung jedes unbelebten Objekts mit einer KK von 2 behindert, was Kampfhandlungen mit Waffen fast unmöglich macht.)\newline Vorbereitungszeit: 1 Aktion\newline Ziel: Einzelobjekt\newline Reichweite: 8 Schritt\newline Wirkungsdauer: 4 Minuten\newline Kosten: 4 AsP\newline Fertigkeiten: Umwelt\newline Erlernen: Mag 12; Alch, Sch 14; Ach, Elf, Hex, Srl; 16; 60 EP}
}


\newglossaryentry{nackedei_Talent}
{
    name={Nackedei},
    description={Sämtliche Stoffkleidung des Opfers fällt plötzlich zu Boden. Metallische Rüstung ist nicht betroffen.\newline Probenschwierigkeit: 12\newline Modifikationen: Blechdose (–8; der Zauber betrifft auch Rüstungsteile aus Metall.)\newline Absatteln (–8; der Zauber betrifft auch Pferdegeschirr und Sattelgurte, wenn dein Ziel ein Reiter ist.)\newline Vorbereitungszeit: 0 Aktionen\newline Ziel: Einzelperson\newline Reichweite: 8 Schritt\newline Wirkungsdauer: augenblicklich\newline Kosten: 8 AsP\newline Fertigkeiten: Umwelt\newline Erlernen: Sch 12; 40 EP}
}


\newglossaryentry{nihilogravoSchwerelos_Talent}
{
    name={Nihilogravo Schwerelos},
    description={Du hebst in einer Zone von 4 Schritt Radius die Schwerkraft auf. Nach unten wird die Zone vom Boden (oder der Wasseroberfläche) begrenzt, nach oben hin ist sie unbegrenzt. Wer sie verlässt, unterliegt sofort wieder der Schwerkraft.\newline Mächtige Magie: Der Radius steigt um 2 Schritt.\newline Probenschwierigkeit: 12\newline Modifikationen: Begleiter (–4; die Zone bewegt sich mit dir.)\newline Levitation (–8, selbst, 8 AsP; statt einer Zone bist nur du selbst betroffen.)\newline Vorbereitungszeit: 8 Aktionen\newline Ziel: Zone\newline Reichweite: Berührung\newline Wirkungsdauer: 4 Minuten\newline Kosten: 16 AsP\newline Fertigkeiten: Umwelt\newline Erlernen: Mag, Sch 18; Alch 20; 60 EP}
}


\newglossaryentry{schelmenkleister_Talent}
{
    name={Schelmenkleister},
    description={Der Untergrund in einem Kreis von 4 Schritt Radius wird zäh und klebrig. Die GS aller Wesen darauf sinkt um 2 Punkte. Fällt sie auf 0, bleibt das betroffene Wesen am Boden kleben.\newline Mächtige Magie: Die GS sinkt um 1 weiteren Punkt und der Radius steigt um 2 Schritt.\newline Probenschwierigkeit: 12\newline Modifikationen: Schelmenschleim (–4; der Untergrund wird stattdessen glitschig und schleimig. Alle Proben, um auf den Beinen zu bleiben, sind um +4 erschwert.)\newline Vorbereitungszeit: 2 Aktionen\newline Ziel: Zone\newline Reichweite: 4 Schritt\newline Wirkungsdauer: 16 Initiativphasen\newline Kosten: 8 AsP\newline Fertigkeiten: Umwelt\newline Erlernen: Sch 14; 40 EP}
}


\newglossaryentry{silentiumSchweigekreis_Talent}
{
    name={Silentium Schweigekreis},
    description={Der Zauber dämpft alle Geräusche in einer Kugel mit 2 Schritt Radius. Entsprechende Wahrnehmungs-Proben sind um –4 erschwert.\newline Mächtige Magie: Der Malus steigt um –2.\newline Probenschwierigkeit: 12\newline Modifikationen: Begleiter (–4; die Kugel bewegt sich mit dir.)\newline Fremdbegleiter (–8; die Kugel bewegt sich mit einem Einzelwesen.)\newline Vorbereitungszeit: 2 Aktionen\newline Ziel: Zone\newline Reichweite: 2 Schritt\newline Wirkungsdauer: 4 Minuten\newline Kosten: 4 AsP\newline Fertigkeiten: Umwelt\newline Erlernen: Elf 8; Sch 12; Geo, Srl 14; Dru, Hex, Mag 18; 40 EP}
}


\newglossaryentry{spurlosTrittlos_Talent}
{
    name={Spurlos Trittlos},
    description={Du tarnst deine Fährte mit Magie. Alle Proben zur Verfolgung deiner Fährte sind um –4 erschwert.\newline Mächtige Magie: Der Malus steigt um –2.\newline Probenschwierigkeit: 12\newline Modifikationen: Andere Person (–4, Ziel Einzelperson)\newline Zone (–4, Zone, 16 AsP; der Zauber betrifft alle in einem Radius von 4 Schritt.)\newline Vorbereitungszeit: 4 Aktionen\newline Ziel: selbst\newline Reichweite: Berührung\newline Wirkungsdauer: 1 Stunde\newline Kosten: 4 AsP\newline Fertigkeiten: Umwelt\newline Erlernen: Elf 12; Dru, Geo 18; Hex 20; 20 EP}
}


\newglossaryentry{zappenduster_Talent}
{
    name={Zappenduster},
    description={Um deinen Körper herum entsteht eine Zone der Dunkelheit von 4 Schritt Radius. Die Helligkeit sinkt um eine Stufe (etwa von normal auf Dämmerung, von Sternenlicht auf absolute Dunkelheit, S. 38). Der Zauber gilt als Konterprobe (16) gegen alle Lichtzauber im Bereich, bei deren Gelingen die Lichtzauber aufgehoben werden. Alle profanen Lichtquellen im Bereich werden erstickt. Du selbst bist von dem Zauber nicht betroffen.\newline Mächtige Magie: Senkt die Helligkeit um eine weitere Stufe.\newline Probenschwierigkeit: 12\newline Vorbereitungszeit: 8 Aktionen\newline Ziel: Zone\newline Reichweite: Berührung\newline Wirkungsdauer: 4 Minuten\newline Kosten: 8 AsP\newline Fertigkeiten: Umwelt\newline Erlernen: Sch 16; 40 EP}
}


\newglossaryentry{arachneaKrabbeltier_Talent}
{
    name={Arachnea Krabbeltier},
    description={Alle Insekten, Spinnen, Maden und anderen wirbellosen Tiere in einem Radius von 64 Schritt streben auf dein Ziel zu und fallen dort über alles her, was in ihr Fressschema passt.\newline Mächtige Magie: Verdoppelt den Radius.\newline Probenschwierigkeit: 12\newline Modifikationen: Maraskaner Verhältnisse (–8, Wirkungsdauer 1 Woche, 32 AsP)\newline Artenkunde (–4; du kannst den Zauber auf eine bestimmte Art von Wesen beschränken oder eine Art ausnehmen)\newline Vorbereitungszeit: 16 Aktionen\newline Ziel: Zone\newline Reichweite: 16 Schritt\newline Wirkungsdauer: 8 Stunden\newline Kosten: 16 AsP\newline Fertigkeiten: Verständigung\newline Erlernen: Mag 20; 40 EP}
}


\newglossaryentry{cryptographoZauberschrift_Talent}
{
    name={Cryptographo Zauberschrift},
    description={Du verschlüsselst eine Nachricht, sodass sie nur noch mit einem Lösungswort gelesen werden kann. Die Verschlüsselung kann nur mit einer Freien Fertigkeit (z.B. Kryptographie) auf Stufe II oder mit einer Konterprobe (KL oder Strukturanalyse, 16) und 4 Stunden Zeitaufwand geknackt werden.\newline Probenschwierigkeit: 12\newline Modifikationen: Unlesbares Buch (–4, 16 AsP; ein ganzes Buch ist betroffen. Der Zeitaufwand zur kompletten Entzifferung beträgt mehrere Monate.)\newline Vorbereitungszeit: 4 Minuten\newline Ziel: Einzelobjekt\newline Reichweite: Berührung\newline Wirkungsdauer: 1 Jahr\newline Kosten: 4 AsP\newline Fertigkeiten: Verständigung\newline Erlernen: Mag 12; Alch 16; Ach, Hex 20; 20 EP}
}


\newglossaryentry{gedankenbilderElfenruf_Talent}
{
    name={Gedankenbilder Elfenruf},
    description={Du sendest eine Gedankenbotschaft an jeden im Radius von 1 Meile, der sie empfangen möchte und den Zauber ebenfalls beherrscht. Die Botschaft ist nicht auf Sprache angewiesen.\newline Mächtige Magie: Verdoppelt den Radius.\newline Probenschwierigkeit: 12\newline Modifikationen: Bestimmter Empfänger (–4; nur ein von dir gewählter Empfänger erhält die Botschaft.)\newline Kreis der Eingeweihten (–4; nur deine engsten Vertrauten erhalten die Botschaft, die Reichweite ist verhundertfacht.)\newline Sinneseindrücke (–4 pro Sinn; du überträgst deine aktuellen Sinneseindrücke.)\newline Illusionen (–4 pro Sinn; wie Sinneseindrücke, aber du kannst beliebige, scheinbar aktuelle und reale Sinneseindrücke übertragen.)\newline Kontakt (–4, Berührung; dein Ziel muss den Zauber nicht beherrschen.)\newline Erzwungene Botschaft (–4; der Empfang der Botschaft kann nur durch eine Konterprobe (MR, 12) verhindert werden.)\newline Vorbereitungszeit: 2 Aktionen\newline Ziel: Zone\newline Reichweite: Berührung\newline Wirkungsdauer: 16 Initiativphasen\newline Kosten: 4 AsP\newline Fertigkeiten: Verständigung\newline Erlernen: Elf 8; Sch 16; Alch, Mag 18; 20 EP}
}


\newglossaryentry{geisterruf_Talent}
{
    name={Geisterruf},
    description={Du rufst einen Geist herbei. Befinden sich Geister in einem Radius von 1 Meile, erscheint einer von ihnen in deiner Nähe. Du kannst den Geist um einen Gefallen bitten, doch er entscheidet, ob er den Gefallen erfüllt und welche Gegenleistung er verlangt. Ist kein Geist in der Nähe, passiert nichts.\newline Mächtige Magie: Verdoppelt den Radius.\newline Probenschwierigkeit: 12\newline Modifikationen: Namensruf (–4; du rufst einen dir bereits bekannten Geist herbei.)\newline Vorbereitungszeit: 4 Minuten\newline Ziel: Zone\newline Reichweite: Berührung\newline Wirkungsdauer: augenblicklich\newline Kosten: 8 AsP\newline Fertigkeiten: Verständigung\newline Erlernen: Dru 12; Geo, Hex, Mag 18; 40 EP}
}


\newglossaryentry{hexenblick_Talent}
{
    name={Hexenblick},
    description={Du erfährst, ob dein Ziel die gleiche Tradition beherrscht wie du.\newline Probenschwierigkeit: 12\newline Vorbereitungszeit: 1 Aktion\newline Ziel: Einzelperson\newline Reichweite: 2 Schritt\newline Wirkungsdauer: augenblicklich\newline Kosten: 1 AsP\newline Fertigkeiten: Verständigung\newline Erlernen: Hex 8; Bor 18; 20 EP\newline Anmerkung: Ein Ziel kann verhindern, dass der Zauber anschlägt, wenn ihm eine Konterprobe (MR, 16) gelingt.}
}


\newglossaryentry{kommKoboldkomm_Talent}
{
    name={Komm Kobold komm},
    description={Du rufst einen Kobold zur Hilfe. Befinden sich Kobolde in einem Radius von 1 Meile, eilt einer von ihnen herbei. Du kannst den Kobold um einen Gefallen bitten, aber er entscheidet selbst, ob er den Gefallen erfüllt.\newline Mächtige Magie: Verdoppelt den Radius.\newline Probenschwierigkeit: 12\newline Modifikationen: Namensruf (–4; du rufst einen dir bereits bekannten Kobold in einem Radius von 8 Meilen herbei.)\newline Vorbereitungszeit: 2 Aktionen\newline Ziel: Zone\newline Reichweite: Berührung\newline Wirkungsdauer: augenblicklich\newline Kosten: 8 AsP\newline Fertigkeiten: Verständigung\newline Erlernen: Sch 8; 20 EP}
}


\newglossaryentry{meistermindererGeister_Talent}
{
    name={Meister minderer Geister},
    description={Du rufst Mindergeister herbei. Befinden sich Mindergeister in einem Radius von 1 Meile, eilt bis zu einem Dutzend von ihnen herbei. Sie verhalten sich ihrem Naturell gemäß.\newline Mächtige Magie: Verdoppelt den Radius.\newline Probenschwierigkeit: 12\newline Vorbereitungszeit: 2 Aktionen\newline Ziel: Einzelwesen\newline Reichweite: 1 Meile\newline Wirkungsdauer: augenblicklich\newline Kosten: 8 AsP\newline Fertigkeiten: Verständigung\newline Erlernen: Sch 8; Geo 14; Dru 20; 20 EP}
}


\newglossaryentry{nekropathiaSeelenreise_Talent}
{
    name={Nekropathia Seelenreise},
    description={Du nimmst Kontakt mit einer Seele in Borons Hallen auf. Du kannst mit ihr sprechen, doch es bleibt ihr überlassen, ob und wie sie antwortet. Der Tod der Person darf maximal 1 Jahr zurückliegen und du benötigst einen persönlichen Gegenstand.\newline Mächtige Magie: Verzehnfacht den Abstand zum Todeszeitpunkt.\newline Probenschwierigkeit: 12\newline Vorbereitungszeit: 4 Minuten\newline Ziel: Einzelobjekt\newline Reichweite: Berührung\newline Wirkungsdauer: 1 Stunde\newline Kosten: 8 AsP\newline Fertigkeiten: Verständigung\newline Erlernen: Mag 18; Alch, Dru, Hex 20; 40 EP}
}


\newglossaryentry{objectovoco_Talent}
{
    name={Objectovoco},
    description={Du verzauberst einen Gegenstand, sodass er dir 4 Ja/Nein-Fragen beantwortet. Beachte, dass die meisten Gegenstände nicht gerade intelligent sind und nur Dinge wahrnehmen, mit denen sie direkt in Kontakt stehen.\newline Mächtige Magie: Dir werden 2 weitere Fragen beantwortet.\newline Probenschwierigkeit: 12\newline Vorbereitungszeit: 8 Aktionen\newline Ziel: Einzelobjekt\newline Reichweite: Berührung\newline Wirkungsdauer: 4 Minuten\newline Kosten: 4 AsP\newline Fertigkeiten: Verständigung\newline Erlernen: Mag 14; Dru, Elf 16; Geo 18; Hex, Sch 20; 20 EP}
}


\newglossaryentry{traumgestalt_Talent}
{
    name={Traumgestalt},
    description={Du kannst in die Träume eines schlafenden Zieles eindringen und ihm dort Botschaften überbringen. Woran sich das Ziel erinnern kann, ist Spielleiterentscheid. Du benötigst ein Körperteil des Opfers, um den Zauber zu wirken.\newline Probenschwierigkeit: Magieresistenz\newline Modifikationen: Marionettenspiel (–4; du kannst im Traum als jemand anders erscheinen.)\newline Traumgestalten (–8, Wirkungsdauer 1 Stunde, 16 AsP; du kannst einige Gefährten in den Traum mitnehmen.)\newline Vorbereitungszeit: 4 Minuten\newline Ziel: Einzelperson\newline Reichweite: 100 Meilen\newline Wirkungsdauer: 4 Minuten\newline Kosten: 8 AsP\newline Fertigkeiten: Verständigung\newline Erlernen: Hex 14; Dru, Elf, Sch 18; Mag 20; 40 EP}
}


\newglossaryentry{zauberwesenderNatur_Talent}
{
    name={Zauberwesen der Natur},
    description={Du rufst ein Feenwesen zur Hilfe. Befinden sich Feenwesen in einem Radius von 1 Meile, eilt eines von ihnen herbei. Du kannst das Feenwesen um einen Gefallen bitten, aber es entscheidet selbst, ob es den Gefallen erfüllt.\newline Mächtige Magie: Verdoppelt den Radius.\newline Probenschwierigkeit: 12\newline Modifikationen: Namensruf (–4; du rufst ein dir bereits bekanntes Feenwesen in einem Radius von 8 Meilen herbei.)\newline Vorbereitungszeit: 4 Minuten\newline Ziel: Zone\newline Reichweite: Berührung\newline Wirkungsdauer: augenblicklich\newline Kosten: 8 AsP\newline Fertigkeiten: Verständigung\newline Erlernen: Dru, Hex, Sch 18; Elf 20; 20 EP}
}


\newglossaryentry{abvenenumreineSpeise_Talent}
{
    name={Abvenenum reine Speise},
    description={Du reinigst eine Mahlzeit für 10 Personen von Verfallserscheinungen, Giften und Krankheitserregern, bis zu einer Gift-/Krankheitsstufe von 20. Der Geschmack ändert sich nicht.\newline Mächtige Magie: Die maximal aufgehobene Gift-/Krankheitsstufe steigt um 4.\newline Probenschwierigkeit: 12\newline Modifikationen: Schutz vor Übelkeit (–4; auch ungefährliche, aber unangenehme Inhalte wie Salz im Meerwasser werden entfernt.)\newline Schutz vor Vergiftung (–4, Wirkungsdauer 8 Stunden; du reinigst auch alles, was dem Essen hinzugefügt wird.)\newline Vorbereitungszeit: 8 Aktionen\newline Ziel: Einzelobjekte\newline Reichweite: 2 Schritt\newline Wirkungsdauer: augenblicklich\newline Kosten: 4 AsP\newline Fertigkeiten: Verwandlung\newline Erlernen: Dru, Elf, Hex, Mag 12; Alch, Geo 14; Ach, Sch 16; 40 EP}
}


\newglossaryentry{accuratumZaubernadel_Talent}
{
    name={Accuratum Zaubernadel},
    description={Du veränderst die Farbe und den Schnitt eines Kleidungsstückes nach deinen Wünschen.\newline Probenschwierigkeit: 12\newline Modifikationen: Haltbarkeit (–4, Wirkungsdauer 1 Jahr, 8 AsP)\newline Vorbereitungszeit: 16 Aktionen\newline Ziel: Einzelobjekt\newline Reichweite: Berührung\newline Wirkungsdauer: 1 Woche\newline Kosten: 4 AsP\newline Fertigkeiten: Verwandlung\newline Erlernen: Alch, Mag, Srl 16; Hex 20; 20 EP}
}


\newglossaryentry{adlerschwingeWolfsgestalt_Talent}
{
    name={Adlerschwinge Wolfsgestalt},
    description={Du verwandelst dich in das beim Erlernen gewählte Tier, wobei du deine geistigen Fähigkeiten behältst. Die körperlichen Fähigkeiten entsprechen denen des Tiers. Du kannst in Tiergestalt nicht zaubern. Erlaubt Aufrechterhalten.\newline Mächtige Magie: Das Tier ist ein überdurchschnittlicher/außergewöhnlicher/herausragender/einzigartiger Vertreter seiner Art, was die Werte des Tieres nach Spielleiterentscheid erhöht.\newline Probenschwierigkeit: 12\newline Modifikationen: Seelentier (Wirkungsdauer 1 Tag, nur Elf; du kannst dich in dein Seelentier verwandeln. Dabei können die Instinkte des Tieres überhand nehmen.)\newline Vorbereitungszeit: 8 Aktionen\newline Ziel: selbst\newline Reichweite: Berührung\newline Wirkungsdauer: 1 Stunde\newline Kosten: 16 AsP\newline Fertigkeiten: Verwandlung\newline Erlernen: Elf 8; Ach, Alch, Mag 18; Dru, Hex, Srl 20; normalerweise 40 EP\newline Anmerkung: Der Zauber muss für jedes Tier extra gelernt werden. Nur herausragende Verwandler können sich in Tiere verwandeln, die nicht zu den Größenklassen klein oder mittel gehören. Solche Verwandlungen sind prinzipiell um –8 erschwert. Die Lernkosten des Zaubers orientieren sich an der Tierart, wobei fliegende, giftige, sehr starke usw. Tiere teurer sind.\newline Sephrasto: Trage die gewählten Tiere in das Kommentarfeld ein. Wenn du mehrere Tiere wählst, dann erhöhe die EP-Kosten entsprechend.}
}


\newglossaryentry{bärenruheWinterschlaf_Talent}
{
    name={Bärenruhe Winterschlaf},
    description={Du versetzt dein Ziel in einen tiefen Winterschlaf. Während des Schlafes benötigt es keine Nahrung, und kein Wasser. Gifte, Krankheiten und Kälte fügen ihm keinen Schaden zu. Dafür regeneriert es auch nicht.\newline Probenschwierigkeit: 12\newline Modifikationen: Der lange Schlaf (–4, Wirkungsdauer 1 Monat, 16 AsP)\newline Vorbereitungszeit: 4 Minuten\newline Ziel: Einzelwesen\newline Reichweite: Berührung\newline Wirkungsdauer: 1 Tag\newline Kosten: 8 AsP\newline Fertigkeiten: Verwandlung\newline Erlernen: Elf 16; Ach, Mag 18; 20 EP}
}


\newglossaryentry{claudibusClavistibor_Talent}
{
    name={Claudibus Clavistibor},
    description={Du verriegelst und stärkst eine Tür. Sie kann nicht normal geöffnet werden und Versuche, das Schloss zu öffnen, sind um –4 erschwert. Die Härte der Tür steigt um 8 Punkte.\newline Mächtige Magie: Der Malus steigt um –2, die Härte um +4.\newline Probenschwierigkeit: 12\newline Modifikationen: Permanenz (–4, Wirkungsdauer bis die Bindung gelöst wird, 8 AsP, davon 2 gAsP)\newline Aktive Versiegelung (–4; der Zauber endet nicht, wenn die Tür auf- und zugeschlossen wird.)\newline Schlüsselmeister (–4; du kannst bis zu 8 Personen oder Schlüssel nennen, für die der Zauber nicht wirkt.)\newline Die Priesterkaiser kommen! (–8, 8 AsP; der Zauber betrifft alle Türen, die du in den nächsten 16 Initiativphasen berührst.)\newline Vorbereitungszeit: 1 Aktion\newline Ziel: Einzelobjekt\newline Reichweite: Berührung\newline Wirkungsdauer: 1 Stunde\newline Kosten: 4 AsP\newline Fertigkeiten: Verwandlung\newline Erlernen: Mag 12; Hex, Srl 14; Alch 16; Dru, Elf, Sch 20; 20 EP}
}


\newglossaryentry{desintegratusPulverstaub_Talent}
{
    name={Desintegratus Pulverstaub},
    description={Von deiner Hand geht 8 Schritt weit eine kegelförmige (15°) Welle astraler Kraft aus, die Gegenstände beschädigt. Der Kegel ist am Ende 2 Schritt breit. Der Schaden reicht aus, um alle Gegenstände bis zur Stabilität einer stabilen Holztür zu zerstören. Lebewesen, magische und geweihte Gegenstände sind gegen den Zauber immun und schützen andere Gegenstände hinter ihnen.\newline Mächtige Magie: Gegenstände bis zu einer Metallwaffe/Rüstung/dicke Stahlstangen/Hauswand werden zerstört.\newline Probenschwierigkeit: 12\newline Modifikationen: Hand der Vernichtung (–4, Einzelobjekt, 4 AsP; nur ein maximal apfelgroßer Gegenstand ist betroffen.)\newline Vorbereitungszeit: 2 Aktionen\newline Ziel: Zone\newline Reichweite: 8 Schritt\newline Wirkungsdauer: augenblicklich\newline Kosten: 16 AsP\newline Fertigkeiten: Verwandlung\newline Erlernen: Mag 18; Alch 20; 40 EP}
}


\newglossaryentry{langerLulatsch_Talent}
{
    name={Langer Lulatsch},
    description={Dein Opfer dehnt sich entweder auf seine dreifache Größe aus oder schrumpft auf ein Drittel. Das Gewicht des Opfers bleibt gleich. Alle seine körperlichen Tätigkeiten sind um –4 erschwert.\newline Probenschwierigkeit: Magieresistenz\newline Modifikationen: Langfinger/Wurstfinger (–4 pro Gliedmaß; der Zauber betrifft nur einzelne Gliedmaßen.)\newline Vorbereitungszeit: 1 Aktion\newline Ziel: Einzelperson\newline Reichweite: Berührung\newline Wirkungsdauer: 4 Minuten\newline Kosten: 8 AsP\newline Fertigkeiten: Verwandlung\newline Erlernen: Sch 12; 20 EP}
}


\newglossaryentry{leibderWogen_Talent}
{
    name={Leib der Wogen},
    description={Du harmonierst mit dem Element Wasser. Du bist immun gegen Wasserschaden. Strömungen und der Druck unter Wasser beeinflussen dich nicht.\newline Probenschwierigkeit: 12\newline Modifikationen: Reise ins Wasser (–4; du kannst dich mit GS Schritt pro Initiativephase durchs Wasser bewegen. Du musst im Wasser nicht atmen.)\newline Begleiter (–4; der Zauber betrifft auch eine weitere Person, mit der du permanent Hautkontakt halten musst.)\newline Leib aus Wasser (–8; Deine Kreaturenklasse wird zu "Elementar" mit allen entsprechenden Eigenschaften (S. 99). Du kannst während der Wirkungsdauer keine Zauber wirken.)\newline Vorbereitungszeit: 8 Aktionen\newline Ziel: selbst\newline Reichweite: Berührung\newline Wirkungsdauer: 1 Stunde\newline Kosten: 16 AsP\newline Fertigkeiten: Wasser, Verwandlung\newline Erlernen: Ach 18; Elf, Geo 20; 60 EP}
}


\newglossaryentry{objectoObscuro_Talent}
{
    name={Objecto Obscuro},
    description={Du machst einen Gegenstand vollkommen unsichtbar. Wird er mit mehr als 1 Schritt pro Initiativephase bewegt, endet der Zauber.\newline Probenschwierigkeit: 12\newline Vorbereitungszeit: 4 Aktionen\newline Ziel: Einzelobjekt\newline Reichweite: Berührung\newline Wirkungsdauer: 1 Stunde\newline Kosten: 8 AsP\newline Fertigkeiten: Verwandlung\newline Erlernen: Alch, Mag, Srl 18; Ach 20; 40 EP}
}


\newglossaryentry{pectetondoZauberhaar_Talent}
{
    name={Pectetondo Zauberhaar},
    description={Du kannst den Schnitt und die Farbe deiner Haare und deines Bartes verändern. Dein Haar wächst während der Wirkungsdauer nicht.\newline Probenschwierigkeit: 12\newline Modifikationen: Farbtopf (–4; auch unnatürliche Farben sind wählbar.)\newline Ohne Kamm (0 Aktionen, 1 AsP; eine bereits mit diesem Zauber geformte Frisur kann mit dieser Modifikation wieder in den vorgesehenen Zustand gebracht werden.)\newline Winterpelz und Katzenfell (–4; sämtliches Körperhaar kann verändert werden.)\newline Vorbereitungszeit: 2 Aktionen\newline Ziel: selbst\newline Reichweite: Berührung\newline Wirkungsdauer: 1 Woche\newline Kosten: 4 AsP\newline Fertigkeiten: Verwandlung\newline Erlernen: Alch, Mag, Srl 14; Hex 18; Sch 20; 20 EP}
}


\newglossaryentry{salanderMutander_Talent}
{
    name={Salander Mutander},
    description={Du verwandelst dein Ziel in ein beliebiges anderes, kleineres und leichteres Tier oder eine solche Pflanze. Du musst das Zielwesen schon einmal gesehen haben, es kann nicht übernatürlich sein. Dein Ziel hat in seiner neuen Form nur noch nebulöse Erinnerungen an sein vorheriges Selbst, behält aber seine alte WS und MR.\newline Probenschwierigkeit: Magieresistenz\newline Modifikationen: Lange Verwandlung (–4, Wirkungsdauer 1 Woche)\newline Permanenz (–4, Wirkungsdauer bis die Bindung gelöst wird, 16 AsP, davon 2 gAsP), Bärenfell und Froschschenkel (–8; du kannst einzelne Gliedmaßen verwandeln.)\newline Vorbereitungszeit: 4 Aktionen\newline Ziel: Einzelperson\newline Reichweite: Berührung\newline Wirkungsdauer: 8 Stunden\newline Kosten: 16 AsP\newline Fertigkeiten: Verwandlung\newline Erlernen: Ach, Mag 14; Alch, Hex 18; Dru 20; 60 EP}
}


\newglossaryentry{serpentialisSchlangenleib_Talent}
{
    name={Serpentialis Schlangenleib},
    description={Deine Arme verwandeln sich in grüne, 2 Schritt lange Giftschlangen, deren hinteres Ende noch mit deinem Rumpf verbunden ist. Die Schlangen können beide in einer Aktion Konflikt angreifen (WS 3, VT 10, RW 2, AT 10, TP 2W6) und ein Waffengift übertragen (Stufe 20, keine Verzögerung, Intervall 2 INI-Phasen, Wirkungsdauer 2 INI-Phasen, 2W6 SP).\newline Mächtige Magie: Die WS der Schlangen steigt um +1, AT, VT und TP um +2.\newline Probenschwierigkeit: 12\newline Modifikationen: Schlangengriff (Wirkungsdauer 4 Minuten, 4 AsP; die Arme verwandeln sich in kampfunfähige, aber geschickte Nattern, deren FF gleich deiner ist.)\newline Vorbereitungszeit: 2 Aktionen\newline Ziel: selbst\newline Reichweite: Berührung\newline Wirkungsdauer: 16 Initiativphasen\newline Kosten: 16 AsP\newline Fertigkeiten: Verwandlung\newline Erlernen: Ach 16; Hex, Mag 20; 40 EP}
}


\newglossaryentry{transformatioFormgestalt_Talent}
{
    name={Transformatio Formgestalt},
    description={Du verwandelst dein Ziel. Die Kosten, Zauberdauer und Modifikationen Mächtige Magie sind Spielleiterentscheid. Beispiele:\newline 4 AsP, keine Mächtige Magie, 4 Aktionen: Du verwandelst ein Glas Wasser in Wein.\newline 8 AsP, 1x Mächtige Magie, 16 Aktionen: Ein Stein wird zu einem Kurzschwert.\newline 8 AsP, 3x Mächtige Magie, 4 Minuten: Aus einem Knäuel Kamelhaar wird ein feiner mhanadischer Teppich.\newline 32 AsP, 4x Mächtige Magie, 1 Stunde: Aus einem Baumstamm wird eine lebensechte Marmorstatue.\newline Probenschwierigkeit: 12\newline Modifikationen: Dauernde Form (–4, Wirkungsdauer 1 Woche)\newline Permanenz (–4, Wirkungsdauer bis die Bindung gelöst wird, zusätzlich ein Achtel der Basiskosten als gAsP)\newline Vorbereitungszeit: nach Vorhaben\newline Ziel: Einzelobjekt\newline Reichweite: Berührung\newline Wirkungsdauer: 1 Stunde\newline Kosten: nach Vorhaben\newline Fertigkeiten: Verwandlung\newline Erlernen: Ach 18; 60 EP}
}


\newglossaryentry{transmutareKörperform_Talent}
{
    name={Transmutare Körperform},
    description={Du kannst das Aussehen deines Zieles nach Belieben ändern, musst dabei aber dabei Größe und Gewicht deines Zieles beibehalten.\newline Probenschwierigkeit: Magieresistenz\newline Modifikationen: Neues Leben (–8, Wirkungsdauer bis die Bindung gelöst wird, 32 AsP, davon 8 gAsP)\newline Vorbereitungszeit: 1 Tag\newline Ziel: Einzelwesen\newline Reichweite: Berührung\newline Wirkungsdauer: 1 Monat\newline Kosten: 16 AsP\newline Fertigkeiten: Verwandlung\newline Erlernen: Alch, Mag 20; 40 EP}
}


\newglossaryentry{unsichtbarerJäger_Talent}
{
    name={Unsichtbarer Jäger},
    description={Du wirst samt deiner am Körper getragenen Ausrüstung unsichtbar. Der Zauber endet, sobald du eine Aktion Konflikt oder Volle Offensive ausführst.\newline Probenschwierigkeit: 12\newline Modifikationen: Unhörbar Geruchlos (–8; der Zauber betrifft auch Gehör und Geruch.)\newline Vorbereitungszeit: 4 Aktionen\newline Ziel: selbst\newline Reichweite: Berührung\newline Wirkungsdauer: 1 Stunde\newline Kosten: 16 AsP\newline Fertigkeiten: Verwandlung\newline Erlernen: Elf 20; 60 EP}
}


\newglossaryentry{visibiliVanitar_Talent}
{
    name={Visibili Vanitar},
    description={Dein Ziel wird unsichtbar. Kleidung und andere Gegenstände sind nicht betroffen. Erlaubt Aufrechterhalten.\newline Probenschwierigkeit: 12\newline Vorbereitungszeit: 2 Aktionen\newline Ziel: Einzelperson\newline Reichweite: Berührung\newline Wirkungsdauer: 4 Minuten\newline Kosten: 8 AsP\newline Fertigkeiten: Verwandlung\newline Erlernen: Elf 12; Alch, Mag, Sch, Srl 16; 40 EP}
}


\newglossaryentry{wasseratem_Talent}
{
    name={Wasseratem},
    description={Du kannst unter Wasser atmen, jedoch nicht mehr an Land. Erlaubt Aufrechterhalten.\newline Probenschwierigkeit: 12\newline Vorbereitungszeit: 16 Aktionen\newline Ziel: Einzelperson\newline Reichweite: Berührung\newline Wirkungsdauer: 1 Stunde\newline Kosten: 8 AsP\newline Fertigkeiten: Verwandlung\newline Erlernen: Elf 12; Alch, Dru, Geo, Mag 20; 20 EP}
}


\newglossaryentry{aquafaxius_Talent}
{
    name={Aquafaxius},
    description={Eine Strahl aus elementarem Wasser fügt dem Ziel 4W6 TP zu und verursacht Ertränken (S. 98). Ballistischer Zauber.\newline Mächtige Magie: Die TP steigen um 2W6.\newline Probenschwierigkeit: 12\newline Modifikationen: Gezielter Strahl (–4; du kannst die Trefferzone bestimmen.)\newline Vorbereitungszeit: 1 Aktion\newline Ziel: Einzelwesen, Einzelobjekt\newline Reichweite: 16 Schritt\newline Wirkungsdauer: augenblicklich\newline Kosten: 16 AsP\newline Fertigkeiten: Wasser\newline Erlernen: Ach, Dru, Geo, Mag 18; 40 EP}
}


\newglossaryentry{aquasphaero_Talent}
{
    name={Aquasphaero},
    description={Du erschaffst eine elementare Kugel, die du mit Konzentration und Blickkontakt 16 Schritt pro Initiativephase bewegen kannst. Die Kugel explodiert, wenn du die Konzentration oder den Blickkontakt verlierst, du sie absichtlich zündest oder die Wirkungsdauer endet. Die Explosion richtet 4W6 TP an und verursacht Ertränken (S. 98). Pro Schritt Entfernung fällt der niedrigste Würfel weg.\newline Mächtige Magie: Die TP steigen um 1W6.\newline Probenschwierigkeit: 12\newline Modifikationen: Vorgegebene Bewegung (–4; du gibst der Kugel die Bewegung bis zur Explosion vor, Konzentration und Blickkontakt zur Kugel sind nicht nötig.)\newline Vorbereitungszeit: 2 Aktionen\newline Ziel: Zone\newline Reichweite: 2 Schritt\newline Wirkungsdauer: 2 Initiativphasen\newline Kosten: 16 AsP\newline Fertigkeiten: Wasser\newline Erlernen: Ach, Mag 18; Geo, Dru 20; 60 EP}
}


\newglossaryentry{herbeirufungdesWassers_Talent}
{
    name={Herbeirufung des Wassers},
    description={Ruft ein Elementarwesen des jeweiligen Elements herbei (mehr zu Herbeirufungen S. 81), das in deiner unmittelbaren Nähe erscheint.\newline Probenschwierigkeit: 16/24/32 (Diener/Dschinn/Meister)\newline Vorbereitungszeit: frei wählbar\newline Ziel: einzelnes Elementar\newline Wirkungsdauer: augenblicklich\newline Kosten: 16/32/64 AsP (Diener/Dschinn/Meister)\newline Fertigkeiten: Wasser\newline Erlernen: Geo 14; Ach, Dru, Mag 16; Alch 18; 60 EP\newline Anmerkung: Nicht überall sind die Wahren Namen und damit die Beschwörung von elementaren Dienern und vor allem Meistern bekannt. }
}


\newglossaryentry{pfeildesWassers_Talent}
{
    name={Pfeil des Wassers},
    description={Du verzauberst einen Pfeil (oder Bolzen oder Wurfwaffe), sodass er im Flug die Macht des Elements freisetzt. Der Pfeil verursacht Wasserschaden und Ertränken (S. 98). Der Pfeil wird durch Regen nicht beeinflusst und kann auch unter Wasser ohne Einschränkungen verwendet werden.\newline Probenschwierigkeit: 12\newline Modifikationen: Geschütz (–4; du verzauberst ein größeres Geschoss wie das einer Balliste.)\newline Permanenz (–4, 4 AsP, davon 1 gAsP, Wirkungsdauer bis die Bindung gelöst wird oder der Pfeil verschossen wurde)\newline Vorbereitungszeit: 1 Aktion\newline Ziel: Einzelobjekt\newline Reichweite: Berührung\newline Wirkungsdauer: 8 Initiativphasen\newline Kosten: 4 AsP\newline Fertigkeiten: Wasser\newline Erlernen: Ach 18; 20 EP}
}


\newglossaryentry{wasserwand_Talent}
{
    name={Wasserwand},
    description={Eine 3 Schritt hohe und bis zu 4 Schritt lange Wand aus rauschenden Wogen quillt entlang einer von dir bestimmten Linie aus dem Boden. Um die Wand zu durchqueren, musst du sie mit einer Konterprobe (KO, 16) betreten. Dann kannst du alle 2 Initiativphasen eine Konterprobe (KO, 16) zum Verlassen der Wand ablegen und erleidest dabei 1 Punkt Erschöpfung (egal ob die Probe gelingt).\newline Mächtige Magie: Die Höhe steigt um 1 Schritt und die Länge um bis zu 2 Schritt.\newline Probenschwierigkeit: 12\newline Vorbereitungszeit: 16 Aktionen\newline Ziel: Zone\newline Reichweite: 8 Schritt\newline Wirkungsdauer: 16 Minuten\newline Kosten: 8 AsP\newline Fertigkeiten: Wasser\newline Erlernen: Geo 16; Mag 18; Dru 20; 40 EP}
}


\newglossaryentry{bindungdesDolches_Talent}
{
    name={Bindung des Dolches},
    description={Du stellst eine enge magische Bindung zu deinem Ritualgegenstand (ein Dolch für Druiden, eine Sichel für Geoden) her, welche die Voraussetzung für alle weiteren Talente dieser Fertigkeit ist. Außerdem wird der Ritualgegenstand unzerbrechlich und gilt als magische Waffe.\newline Probenschwierigkeit: 12\newline Vorbereitungszeit: 8 Stunden\newline Ziel: Ritualgegenstand\newline Reichweite: Berührung\newline Wirkungsdauer: permanent\newline Kosten: 16 AsP\newline Fertigkeiten: Dolchzauber, Kraft\newline Erlernen: Dru 4; Geo 12; 20 EP}
}


\newglossaryentry{apportdesDolches_Talent}
{
    name={Apport des Dolches},
    description={Der Dolch kehrt fliegend mit einer Geschwindigkeit von 10 Meilen pro Stunde zu dir zurück. Er weicht Hindernissen aus oder durchbricht sie notfalls, wenn die KK von PW Dolchzauber/2 oder Umwelt/2 dafür ausreicht.\newline Probenschwierigkeit: 12\newline Vorbereitungszeit: 0 Aktionen\newline Ziel: Zone\newline Reichweite: 1 Meile\newline Wirkungsdauer: augenblicklich\newline Kosten: 1 AsP\newline Fertigkeiten: Dolchzauber, Umwelt\newline Erlernen: Dru 16; 20 EP}
}


\newglossaryentry{blutdesDolches(passiv)_Talent}
{
    name={Blut des Dolches (passiv)},
    description={Du kannst wie mit dem Vorteil Verbotene Pforten Lebenskraft für deine Zauber nutzen (S. 73). Verfügst du bereits über Verbotene Pforten, stellt jede selbst zugefügt Wunde WS+8 AsP zur Verfügung. Überschüssige AsP verfallen.\newline Erlernen: Dru 18; 40 EP}
}


\newglossaryentry{dämonenbanndesDolches_Talent}
{
    name={Dämonenbann des Dolches},
    description={Du ziehst mit deinem Ritualgegenstand einen Kreis von maximal 8 Schritt Radius. Dämonen können diese Linie nicht übertreten, wenn ihre Beschwörungsschwierigkeit maximal 20 beträgt.\newline Mächtige Magie: Erhöht die Beschwörungsschwierigkeit um 4.\newline Probenschwierigkeit: 12\newline Vorbereitungszeit: 4 Aktionen\newline Ziel: Zone\newline Reichweite: Berührung\newline Wirkungsdauer: 1 Stunde\newline Kosten: 8 AsP\newline Fertigkeiten: Antimagie, Dolchzauber\newline Erlernen: Dru 18; 40 EP}
}


\newglossaryentry{erntedesDolches_Talent}
{
    name={Ernte des Dolches},
    description={Die nächste mit dem Ritualgegenstand geerntete Pflanze ist viermal so lange haltbar.\newline   Probenschwierigkeit: 12\newline Vorbereitungszeit: 0 Aktionen\newline Ziel: Ritualgegenstand\newline Reichweite: Berührung\newline Wirkungsdauer: 16 Initiativphasen\newline Kosten: 1 AsP\newline Fertigkeiten: Dolchzauber, Temporal, Verwandlung\newline Erlernen: Dru, Geo 12; 20 EP}
}


\newglossaryentry{geisterbanndesDolches_Talent}
{
    name={Geisterbann des Dolches},
    description={Du ziehst mit deinem Ritualgegenstand einen Kreis von maximal 8 Schritt Radius. Geister können diese Linie nicht übertreten.\newline Probenschwierigkeit: 12\newline Vorbereitungszeit: 4 Aktionen\newline Ziel: Zone\newline Reichweite: Berührung\newline Wirkungsdauer: 1 Stunde\newline Kosten: 4 AsP\newline Fertigkeiten: Antimagie, Dolchzauber, Verständigung\newline Erlernen: Dru 16; 20 EP}
}


\newglossaryentry{gespürdesDolches_Talent}
{
    name={Gespür des Dolches},
    description={Dein Ritualgegenstand kühlt ab, wenn er dämonisch verseuchten Boden berührt. So erfährst du, ob die Umgebung normal/leicht verseucht/deutlich verseucht/stark verseucht/ein kleines Unheiligtum/ein mittleres Unheiligtum/ein mächtiges Unheiligtum ist.\newline Probenschwierigkeit: 12\newline Vorbereitungszeit: 0 Aktionen\newline Ziel: Ritualgegenstand\newline Reichweite: Berührung\newline Wirkungsdauer: augenblicklich\newline Kosten: 1 AsP\newline Fertigkeiten: Dolchzauber, Hellsicht\newline Erlernen: Dru, Geo 18; 20 EP}
}


\newglossaryentry{kontrollederMiniatur_Talent}
{
    name={Kontrolle der Miniatur},
    description={Einmal pro Tag kannst du alle Sinne des Opfers der Miniatur der Herrschaft (s.u.) übernehmen und alle seine Handlungen bestimmen. Widersprechen die Handlungen den moralischen Prinzipien oder dem Selbsterhaltungstrieb des Opfers, kann es mit einer Konterprobe (Willenskraft, 16) widerstehen.\newline Probenschwierigkeit: Magieresistenz\newline Vorbereitungszeit: 8 Aktionen\newline Ziel: Miniatur der Herrschaft\newline Reichweite: Berührung\newline Wirkungsdauer: 1 Stunde\newline Kosten: 8 AsP\newline Fertigkeiten: Dolchzauber, Einfluss, Hellsicht, Verständigung\newline Erlernen: Dru 18; 40 EP}
}


\newglossaryentry{lebenskraftdesDolches_Talent}
{
    name={Lebenskraft des Dolches},
    description={Du entziehst dem Boden Lebenskraft und regenerierst in der nächsten Ruhephase eine zusätzliche Wunde (nicht kumulativ).\newline Probenschwierigkeit: 12\newline Vorbereitungszeit: 0 Aktionen\newline Ziel: selbst\newline Reichweite: Berührung\newline Wirkungsdauer: augenblicklich\newline Kosten: 4 AsP\newline Fertigkeiten: Dolchzauber, Humus\newline Erlernen: Geo 12; Dru 14; 40 EP}
}


\newglossaryentry{leibdesDolches(passiv)_Talent}
{
    name={Leib des Dolches (passiv)},
    description={Wenn du bei einer Verwandlung deines Körpers (egal ob freiwillig oder unfreiwillig) deinen Ritualgegenstand bei dir trägst, hast du ihn auch nach der Rückverwandlung wieder bei dir.\newline Erlernen: Dru 16; 20 EP}
}


\newglossaryentry{lichtdesDolches_Talent}
{
    name={Licht des Dolches},
    description={Du ignorierst eine Stufe Dunkelheit, aber grelles Licht erschwert alle Proben um –2. Bei absoluter Dunkelheit ist der Zauber wirkungslos.\newline Mächtige Magie: Du ignorierst 2/3 Stufen Dunkelheit.\newline Probenschwierigkeit: 12\newline Vorbereitungszeit: 4 Aktionen\newline Ziel: Einzelperson\newline Reichweite: Berührung\newline Wirkungsdauer: 1 Stunde\newline Kosten: 4 AsP\newline Fertigkeiten: Dolchzauber, Eigenschaften\newline Erlernen: Dru, Geo 14; 40 EP}
}


\newglossaryentry{miniaturderHerrschaft_Talent}
{
    name={Miniatur der Herrschaft},
    description={Aus Lehm und einem Körperteil des Opfers (z.B. ein Haar oder einen Blutstropfen) fertigst du eine Miniatur der Herrschaft. Während der Wirkungsdauer kannst du einmal pro Tag einen Einflusszauber oder einen Fluch der Pestilenz auf die Miniatur sprechen. Dieser Zauber wird direkt auf das Opfer übertragen, solange es nicht weiter als 1 Meile entfernt ist. Das Opfer kann wie gewöhnlich eine MR-Probe ablegen, um dem Zauber zu widerstehen.\newline Mächtige Magie: Verdoppelt die Entfernung.\newline Probenschwierigkeit: 12\newline Vorbereitungszeit: 1 Stunde\newline Ziel: Einzelobjekt\newline Reichweite: Berührung\newline Wirkungsdauer: 1 Woche\newline Kosten: 16 AsP\newline Fertigkeiten: Dolchzauber, Einfluss, Verständigung\newline Erlernen: Dru 12; 40 EP}
}


\newglossaryentry{opferdolch(passiv)_Talent}
{
    name={Opferdolch (passiv)},
    description={Du kannst Tierblut blutmagisch nutzen. Verfügst du zusätzlich über den Vorteil Blutmagie, kannst du aus Tieren insgesamt 2xWS AsP gewinnen.\newline Erlernen: Dru 18; 20 EP\newline Anmerkung: Finstere Druiden hüten Varianten dieses Rituals, deren Wirkung nicht auf Tiere beschränkt ist.}
}


\newglossaryentry{schmerzenderMiniatur_Talent}
{
    name={Schmerzen der Miniatur},
    description={Mit Dolchstichen in eine Miniatur der Herrschaft (s.o.) kannst du dem Opfer Schmerzen zufügen. Dadurch sind geistige oder körperliche Tätigkeiten um –2 erschwert.\newline Probenschwierigkeit: Magieresistenz\newline Vorbereitungszeit: 0 Aktionen\newline Ziel: Miniatur der Herrschaft\newline Reichweite: Berührung\newline Wirkungsdauer: 1 Stunde\newline Kosten: 1 AsP\newline Fertigkeiten: Dolchzauber, Einfluss, Verständigung\newline Erlernen: Dru 14; 20 EP}
}


\newglossaryentry{schneidedesDolches_Talent}
{
    name={Schneide des Dolches},
    description={Du kannst mit dem Dolch natürliches Gestein schneiden wie Wachs.\newline Probenschwierigkeit: 12\newline Vorbereitungszeit: 0 Aktionen\newline Ziel: Einzelobjekt\newline Reichweite: Berührung\newline Wirkungsdauer: 1 Stunde\newline Kosten: 8 AsP\newline Fertigkeiten: Dolchzauber, Erz\newline Erlernen: Geo 16; Dru 18; 20 EP}
}


\newglossaryentry{schutzdesDolches_Talent}
{
    name={Schutz des Dolches },
    description={Dir steht die Natur schützend zur Seite. Du schwimmst leichter auf Wasser, Schnee bildet wärmende Höhlen und du findest überall Verstecke. Verbergen, Schwimmen,\newline Überleben oder andere mit der Natur zusammenhängende Proben sind um +4 erleichtert.\newline Mächtige Magie: Der Bonus steigt um +2.\newline Probenschwierigkeit: 12\newline Vorbereitungszeit: 4 Aktionen\newline Ziel: selbst\newline Reichweite: Berührung\newline Wirkungsdauer: 1 Stunde\newline Kosten: 8 AsP\newline Fertigkeiten: Dolchzauber, Umwelt\newline Erlernen: Geo 14; Dru 18; 40 EP}
}


\newglossaryentry{verletzungderMiniatur_Talent}
{
    name={Verletzung der Miniatur},
    description={Du fügst dir mit dem Dolch selbst eine oder mehrere Wunden zu. Das Opfer deiner Minitatur der Herrschaft (s.o.) erleidet genau so viele Wunden. Sollte das Opfer durch diese Wunden sterben, kann es eine Konterprobe (KO, 16) ablegen, um die Wunden abzuwehren.\newline Probenschwierigkeit: 12\newline Vorbereitungszeit: 8 Aktionen\newline Ziel: Miniatur der Herrschaft\newline Reichweite: Berührung\newline Wirkungsdauer: augenblicklich\newline Kosten: 8 AsP\newline Fertigkeiten: Dolchzauber, Verständigung\newline Erlernen: Dru 18; 40 EP}
}


\newglossaryentry{wegdesDolches_Talent}
{
    name={Weg des Dolches},
    description={Dein Dolch zeigt in die Richtung des Ortes, an dem er geweiht wurde.  Eine Überleben-Proben zur Orientierung ist um +4 Punkte erleichtert.\newline Probenschwierigkeit: 12\newline Vorbereitungszeit: 0 Aktionen\newline Ziel: Zone\newline Reichweite: dereweit\newline Wirkungsdauer: 16 Initiativphasen\newline Kosten: 4 AsP\newline Fertigkeiten: Dolchzauber, Hellsicht, Umwelt\newline Erlernen: Dru 12; Geo 14; 20 EP}
}


\newglossaryentry{weisungdesDolches_Talent}
{
    name={Weisung des Dolches},
    description={Dein Dolch leuchtet rot auf, wenn er in Richtung einer Kraftlinie gehalten wird. Die Leuchtstärke hängt von Entfernung und Stärke der Kraftlinie ab.\newline Probenschwierigkeit: 12\newline Vorbereitungszeit: 0 Aktionen\newline Ziel: Zone\newline Reichweite: 8 Meilen\newline Wirkungsdauer: 1 Stunde\newline Kosten: 1 AsP\newline Fertigkeiten: Dolchzauber, Hellsicht, Kraft\newline Erlernen: Dru 12; Geo 14; 20 EP}
}


\newglossaryentry{bindungdesIama_Talent}
{
    name={Bindung des Iama},
    description={Du stellst eine enge magische Bindung zu deinem Ritualgegenstand her, welche die Voraussetzung für alle weiteren Talente dieser Fertigkeit ist. Außerdem wird der Ritualgegenstand unzerbrechlich.\newline Probenschwierigkeit: 12\newline Vorbereitungszeit: 8 Stunden\newline Ziel: Ritualgegenstand\newline Reichweite: Berührung\newline Wirkungsdauer: permanent\newline Kosten: 16 AsP\newline Fertigkeiten: Elfenlieder, Kraft\newline Erlernen: Elf 4; 20 EP}
}


\newglossaryentry{apportdesIama_Talent}
{
    name={Apport des Iama},
    description={Der Ritualgegenstand kehrt fliegend mit einer Geschwindigkeit von 10 Meilen pro Stunde zu dir zurück. Er weicht Hindernissen aus oder durchbricht sie notfalls, wenn die KK von PW Elfenlieder/2 oder Umwelt/2 dafür ausreicht.\newline Probenschwierigkeit: 12\newline Vorbereitungszeit: 0 Aktionen\newline Ziel: Zone\newline Reichweite: 1 Meile\newline Wirkungsdauer: augenblicklich\newline Kosten: 1 AsP\newline Fertigkeiten: Elfenlieder, Umwelt\newline Erlernen: Elf 16; 20 EP}
}


\newglossaryentry{erinnerungsmelodie_Talent}
{
    name={Erinnerungsmelodie},
    description={Du stimmst dich auf eine vergangene Situation ein, um dich an ein Detail zu erinnern. Die dafür notwendige KL-Probe ist um +4 erleichtert.\newline Mächtige Magie: Der Bonus steigt um +2.\newline Probenschwierigkeit: 12\newline Vorbereitungszeit: 1 Stunde\newline Ziel: selbst\newline Reichweite: Berührung\newline Wirkungsdauer: augenblicklich\newline Kosten: 1 AsP\newline Fertigkeiten: Eigenschaften, Elfenlieder\newline Erlernen: Elf 12; 20 EP}
}


\newglossaryentry{friedenslied_Talent}
{
    name={Friedenslied},
    description={Alle Humanoide und Tiere in Hörreichweite verfallen in eine friedfertige Stimmung, auch wenn sie deine Musik nicht verstehen können. Aggressive Handlungen erfordern eine Konterprobe (Willenskraft, 16). Erfordert Konzen­tration, erlaubt Aufrechterhalten.\newline Probenschwierigkeit: 12\newline Vorbereitungszeit: 4 Aktionen\newline Ziel: Zone\newline Reichweite: Berührung\newline Wirkungsdauer: 1 Stunde\newline Kosten: 8 AsP\newline Fertigkeiten: Einfluss, Elfenlieder\newline Erlernen: Elf 16; 40 EP}
}


\newglossaryentry{liedderLieder_Talent}
{
    name={Lied der Lieder },
    description={Alle Humanoide in Hörreichweite sind von deiner immer berauschenderen Musik völlig gefesselt. Nur mit einer Konterprobe (Willenskraft, 16) kann sich ein Zuhörer für 16 Initiativphasen dem Zauber des Liedes entziehen. Ansonsten erwacht er, wenn du aufhörst zu spielen oder er eine Wunde erleidet. Erfordert Konzentration, erlaubt Aufrechterhalten.\newline Probenschwierigkeit: 12\newline Vorbereitungszeit: 16 Aktionen\newline Ziel: Zone\newline Reichweite: Berührung\newline Wirkungsdauer: 1 Stunde\newline Kosten: 8 AsP\newline Fertigkeiten: Einfluss, Elfenlieder\newline Erlernen: Elf 16; 40 EP}
}


\newglossaryentry{liedderReinheit_Talent}
{
    name={Lied der Reinheit},
    description={Dein Lied schwächt dämonische Einflüsse in deiner Umgebung, natürliches Leben kehrt zurück. Auf der Skala normal/leicht verseucht/deutlich verseucht/stark verseucht/kleines Unheiligtum/mittleres Unheiligtum/mächtiges Unheiligtum sinkt das Gebiet um eine Stufe.\newline Mächtige Magie: Das Gebiet sinkt um eine weitere Stufe.\newline Probenschwierigkeit: 12\newline Vorbereitungszeit: 1 Stunde\newline Ziel: Zone\newline Reichweite: Berührung\newline Wirkungsdauer: 1 Jahr\newline Kosten: 16 AsP\newline Fertigkeiten: Antimagie, Elfenlieder, Umwelt\newline Erlernen: Elf 18; 20 EP\newline Anmerkung: Verschiebst du die Skala auf normal, kehren die dämonischen Einflüsse nicht von selbst zurück – außer die Ursache der Verseuchung existiert weiterhin. }
}


\newglossaryentry{lieddesTrostes_Talent}
{
    name={Lied des Trostes},
    description={Dein Lied hilft dem Ziel, erlebte Schicksalsschläge zu verkraften und neuen Lebensmut zu fassen.\newline Probenschwierigkeit: 12\newline Vorbereitungszeit: 1 Stunde\newline Ziel: Einzelperson\newline Reichweite: 4 Schritt\newline Wirkungsdauer: augenblicklich\newline Kosten: 1 AsP\newline Fertigkeiten: Eigenschaften, Einfluss, Elfenlieder\newline Erlernen: Elf 8; 20 EP\newline Anmerkung: Wiederholte Anwendung kann helfen, eine entsprechende Eigenheit zu verändern.}
}


\newglossaryentry{melodiederKunstfertigkeit_Talent}
{
    name={Melodie der Kunstfertigkeit},
    description={Die Melodie erleichtert alle Proben zur Herstellung eines bestimmten Handwerksstückes um +2.\newline Mächtige Magie: Der Bonus steigt um +1.\newline Probenschwierigkeit: 12\newline Vorbereitungszeit: 16 Aktionen\newline Ziel: selbst\newline Reichweite: Berührung\newline Wirkungsdauer: bis zur Fertigstellung.\newline Kosten: 16 AsP\newline Fertigkeiten: Eigenschaften, Elfenlieder\newline Erlernen: Elf 18; 60 EP}
}


\newglossaryentry{sorgenlied_Talent}
{
    name={Sorgenlied},
    description={Das Lied vermittelt dir einen vagen Eindruck vom Wohlergehen des Zieles, mit dem du befreundet sein musst. Ist es gesund, krank oder befindet sich in Gefahr? Ist es zufrieden mit seinem Leben?\newline Probenschwierigkeit: 12\newline Vorbereitungszeit: 1 Stunde\newline Ziel: Einzelperson\newline Reichweite: dereweit\newline Wirkungsdauer: augenblicklich\newline Kosten: 1 AsP\newline Fertigkeiten: Elfenlieder, Hellsicht, Verständigung\newline Erlernen: Elf 8; 20 EP}
}


\newglossaryentry{melodiedesWindes_Talent}
{
    name={Melodie des Windes},
    description={Du öffnest deine Sinne für Eindrücke aus der Windrichtung, die von der Harmonie des Ortes, seinen Pflanzen und Bewohnern künden. Die Stimmungen, Visionen oder Geräusche bleiben dabei stets vage. Erfordert Konzentration.\newline Probenschwierigkeit: 12\newline Vorbereitungszeit: 16 Aktionen\newline Ziel: Zone\newline Reichweite: Berührung\newline Wirkungsdauer: 1 Stunde\newline Kosten: 4 AsP\newline Fertigkeiten: Elfenlieder, Hellsicht, Luft\newline Erlernen: Elf 12; 20 EP}
}


\newglossaryentry{verwandlungdesIama(passiv)_Talent}
{
    name={Verwandlung des Iama (passiv)},
    description={Wenn du bei einer Verwandlung deines Körpers (egal ob freiwillig oder unfreiwillig) dein Iama bei dir trägst, hast du es auch nach der Rückverwandlung wieder bei dir.\newline Erlernen: Elf 12; 20 EP}
}


\newglossaryentry{zaubermelodie_Talent}
{
    name={Zaubermelodie},
    description={Die Melodie erleichtert alle allgemeinen Zauber um +2.\newline Mächtige Magie: Der Bonus steigt um +1.\newline Probenschwierigkeit: 12\newline Vorbereitungszeit: 16 Aktionen\newline Ziel: selbst\newline Reichweite: Berührung\newline Wirkungsdauer: 1 Stunde\newline Kosten: 8 AsP\newline Fertigkeiten: Eigenschaften, Elfenlieder, Kraft\newline Erlernen: Elf 16; 40 EP}
}


\newglossaryentry{blutgeistaufnehmen_Talent}
{
    name={Blutgeist aufnehmen},
    description={Du verzehrst das frische Herz eines Tieres, um den Blutgeist des Tieres aufzunehmen. Damit beginnt die Besessenheit, die drei Effekte hat: Erstens kannst du die Zauber Stärke des Blutgeists und Körper des Blutgeists nutzen. Diese Zauber beziehen sich immer auf das aufgenommene Tier (siehe S. 160). Zweitens kannst du alle dem Tier zugeordneten allgemeinen Zauber wirken, ohne dass du sie erlernen musst. Die Probe wird dabei auf die Fertigkeit Gaben des Blutgeists abgelegt. Drittens erhältst du die Eigenheit „Gefahr des Blutes“: Wann immer du in Kontakt mit fremdem Blut kommst, kann der Blutgeist die Oberhand gewinnen. Meist gerätst du in einen mörderischen Blutrausch, aber auch teilweise oder ganze Verwandlungen oder andere, zum Tier passende Effekte sind denkbar. Positive und negative Effekte verschwinden mit dem Ende der Besessenheit. Du kannst niemals zwei Blutgeister gleichzeitig aufnehmen.\newline Probenschwierigkeit: 12\newline Vorbereitungszeit: 1 Stunde\newline Ziel: selbst\newline Reichweite: Berührung\newline Wirkungsdauer: bis die Bindung gelöst wird\newline Kosten: 8 AsP, davon 8 gAsP\newline Fertigkeiten: Verwandlung, Gaben des Blutgeists\newline Erlernen: Ana 4; 60 EP\newline Sephrasto: Die Tiertabelle ist in Sephrasto über die kostenlosen Tiergeist-Vorteile abgebildet, wähle einen davon aus. Dies schaltet alle Zauber des Tiergeists kostenlos über die Fertigkeit Gaben des Blutgeists frei - setze bei allen einen Haken.}
}


\newglossaryentry{stärkedesBlutgeists_Talent}
{
    name={Stärke des Blutgeists},
    description={Der Blutgeist stärkt dich. Du erhältst einen Bonus von +2 auf alle bei dem Tier des Blutgeists angegebenen Werte (S. 160).\newline Mächtige Magie: Der Bonus steigt um +1.\newline Probenschwierigkeit: 12\newline Vorbereitungszeit: 4 Aktionen\newline Ziel: selbst\newline Reichweite: Berührung\newline Wirkungsdauer: 1 Stunde\newline Kosten: 4 AsP\newline Fertigkeiten: Eigenschaften, Gaben des Blutgeists\newline Erlernen: Ana 8; 40 EP}
}


\newglossaryentry{körperdesBlutgeists_Talent}
{
    name={Körper des Blutgeists},
    description={Du setzt eine der beim Tier deines Blutgeists angegebenen Verwandlungen (S. 160) ein. Du kannst auch beide Verwandlungen gleichzeitig nutzen, dafür musst du den Zauber zwei Mal wirken.\newline Mächtige Magie: Falls zutreffend, geschieht das Folgende: Probenerleichterungen steigen um +2, TP um +2, die Giftstufe um +4, die Flugweiten und -höhen um 4/2 Schritt. Je zwei Stufen erhöhen einen RS-Bonus um +1.\newline Probenschwierigkeit: 12\newline Vorbereitungszeit: 4 Aktionen\newline Ziel: Zone\newline Reichweite: selbst\newline Wirkungsdauer: 1 Stunde\newline Kosten: 8 AsP\newline Fertigkeiten: Verwandlung, Gaben des Blutgeists\newline Erlernen: Ana 12; 60 EP}
}


\newglossaryentry{attributo(Blutgeist)_Talent}
{
    name={Attributo (Blutgeist)},
    description={Wähle ein Attribut aus. Proben auf dieses Attribut sind um +2, Fertigkeitsproben mit diesem Attribut um +1 erleichtert.\newline Mächtige Magie: Der Bonus steigt um +2/+1.\newline Probenschwierigkeit: 12\newline Vorbereitungszeit: 8 Aktionen\newline Ziel: Einzelperson\newline Reichweite: Berührung\newline Wirkungsdauer: 1 Stunde\newline Kosten: 8 AsP}
}


\newglossaryentry{axxeleratusBlitzgeschwind(Blutgeist)_Talent}
{
    name={Axxeleratus Blitzgeschwind (Blutgeist)},
    description={Die GS deines Zieles erhöht sich um +4 und alle AT und VT sind um +2 erleichtert, Ausweichen um weitere +2.\newline Mächtige Magie: Die GS steigt um weitere +2.\newline Probenschwierigkeit: 12\newline Vorbereitungszeit: 0 Aktionen\newline Ziel: Einzelperson\newline Reichweite: 4 Schritt\newline Wirkungsdauer: 16 Initiativphasen\newline Kosten: 8 AsP}
}


\newglossaryentry{wipfellauf(Blutgeist)_Talent}
{
    name={Wipfellauf (Blutgeist)},
    description={Du kannst dich durch Baumkronen und Unterholz bewegen, als wären sie eine normale Straße. Unter diesen Bedingungen erleidest du keinerlei Erschwernisse oder Geschwindigkeitsabzüge.\newline Probenschwierigkeit: 12\newline Modifikationen: Kopfüber (–8; du kannst selbst kopfüber an Bäumen laufen.)\newline Vorbereitungszeit: 2 Aktionen\newline Ziel: selbst\newline Reichweite: Berührung\newline Wirkungsdauer: 1 Stunde\newline Kosten: 16 AsP}
}


\newglossaryentry{eiseskälteKämpferherz(Blutgeist)_Talent}
{
    name={Eiseskälte Kämpferherz (Blutgeist)},
    description={Das Ziel empfindet keinerlei Schmerzen mehr. Es erhält den Vorteil Kalte Wut (S. 43). Proben zum Ignorieren von Wundschmerz sind um +4 erleichtert.\newline Mächtige Magie: Der Bonus steigt um +2.\newline Probenschwierigkeit: 12\newline Vorbereitungszeit: 0 Aktionen\newline Ziel: Einzelperson\newline Reichweite: Berührung\newline Wirkungsdauer: 16 Initiativphasen\newline Kosten: 8 AsP}
}


\newglossaryentry{ruheKörper,RuheGeist(Blutgeist)_Talent}
{
    name={Ruhe Körper, Ruhe Geist (Blutgeist)},
    description={Dein Ziel sinkt in einen tiefen Schlaf, aus dem es nur mit Gewalt geweckt werden kann. Wird es nicht vorzeitig geweckt, regeneriert es zwei zusätzliche Wunden.\newline Mächtige Magie: Regeneriert eine weitere Wunde.\newline Probenschwierigkeit: 12\newline Vorbereitungszeit: 16 Aktionen\newline Ziel: Einzelperson\newline Reichweite: Berührung\newline Wirkungsdauer: 8 Stunden\newline Kosten: 8 AsP}
}


\newglossaryentry{sanftmut(Blutgeist)_Talent}
{
    name={Sanftmut (Blutgeist)},
    description={Das verzauberte Tier verliert seine Angriffslust, solange es nicht angegriffen oder gereizt wird.\newline Mächtige Magie: Die Lethargie hält sogar dann, wenn das Tier gereizt/leicht verletzt/schwer verletzt wird.\newline Probenschwierigkeit: Magieresistenz\newline Vorbereitungszeit: 1 Aktion\newline Ziel: Tier\newline Reichweite: 8 Schritt\newline Wirkungsdauer: 4 Minuten\newline Kosten: 8 AsP}
}


\newglossaryentry{standfestKatzengleich(Blutgeist)_Talent}
{
    name={Standfest Katzengleich (Blutgeist)},
    description={Deine Geschicklichkeit erhöht sich. Alle Proben, um auf den Beinen zu bleiben, sind um +4 erleichtert und deine Patzerchance im Kampf sinkt um 1 auf dem W20.\newline Mächtige Magie: Der Bonus steigt um +2.\newline Probenschwierigkeit: 12\newline Vorbereitungszeit: 2 Aktionen\newline Ziel: selbst\newline Reichweite: Berührung\newline Wirkungsdauer: 16 Initiativphasen\newline Kosten: 4 AsP}
}


\newglossaryentry{zaubernahrungHungerbann(Blutgeist)_Talent}
{
    name={Zaubernahrung Hungerbann (Blutgeist)},
    description={Für einen Tag spürst du keinerlei Hunger und die Kraft des Zaubers ernährt dich. Du musst spätestens vor vier Tagen echtes Essen zu dir genommen haben.\newline Mächtige Magie: Die letzte echte Mahlzeit kann bis zu zwei Tage länger her sein.\newline Probenschwierigkeit: 12\newline Modifikationen: Durstbann (–4, 8 AsP; auch jeglicher Durst wird vom Zauber gestillt.)\newline Vorbereitungszeit: 4 Minuten\newline Ziel: selbst\newline Reichweite: Berührung\newline Wirkungsdauer: 1 Tag\newline Kosten: 4 AsP}
}


\newglossaryentry{memoransGedächtniskraft(Blutgeist)_Talent}
{
    name={Memorans Gedächtniskraft (Blutgeist)},
    description={Du kannst dir sämtliche Bilder, Schriftstücke, Inschriften und ähnliches für immer einprägen. Eine Seite eines Buches kostet dich etwa eine Minute.\newline Probenschwierigkeit: 12\newline Modifikationen: Drachengedächtnis (–8, Wirkungsdauer 16 Initiativphasen; alles, was du während der Wirkungsdauer siehst, ist dir für immer ins Gedächtnis gebrannt.)\newline Vorbereitungszeit: 4 Minuten\newline Ziel: selbst\newline Reichweite: 4 Schritt\newline Wirkungsdauer: 4 Minuten\newline Kosten: 8 AsP}
}


\newglossaryentry{psychostabilis(Blutgeist)_Talent}
{
    name={Psychostabilis (Blutgeist)},
    description={Magieresistenz-Proben des Ziels sind um +4 erleichtert.\newline Mächtige Magie: Erhöht den Bonus um +2.\newline Probenschwierigkeit: 12\newline Modifikationen: Schnellsteigerung (–4; Wirkungsdauer 16 Initiativphasen; verdoppelt die Erleichterung.)\newline Stabilisierung (–8; das Ziel darf sofort eine MR-Probe gegen einen auf es wirkenden Zauber wiederholen. Gelingt sie, wird dieser Zauber für die Wirkungsdauer des Psychostabilis unterdrückt.)\newline Vorbereitungszeit: 2 Aktionen\newline Ziel: Einzelperson\newline Reichweite: Berührung\newline Wirkungsdauer: 1 Stunde\newline Kosten: 8 AsP}
}


\newglossaryentry{seelentiererkennen(Blutgeist)_Talent}
{
    name={Seelentier erkennen (Blutgeist)},
    description={Du erkennst das Seelentier deines Zieles. Wenn dir das Tier und seine assoziierten Eigenschaften bekannt sind, erleichtert dir das alle künftigen gesellschaftlichen Proben dem Ziel gegenüber um +2.\newline Probenschwierigkeit: Magieresistenz\newline Vorbereitungszeit: 4 Aktionen\newline Ziel: Einzelperson\newline Reichweite: 4 Schritt\newline Wirkungsdauer: augenblicklich\newline Kosten: 4 AsP}
}


\newglossaryentry{xenographusSchriftenkunde(Blutgeist)_Talent}
{
    name={Xenographus Schriftenkunde (Blutgeist)},
    description={Du kannst den Sinn eines geschriebenen Satzes verstehen, auch wenn du Schrift und Sprache nicht kennst.\newline Mächtige Magie: Verdoppelt die Anzahl der Sätze.\newline Probenschwierigkeit: 12\newline Vorbereitungszeit: 8 Aktionen\newline Ziel: Einzelobjekt\newline Reichweite: 4 Schritt\newline Wirkungsdauer: augenblicklich\newline Kosten: 4 AsP}
}


\newglossaryentry{blickaufsWesen(Blutgeist)_Talent}
{
    name={Blick aufs Wesen (Blutgeist)},
    description={Der Zauber offenbart dir die Fertigkeiten deines Ziels. Du erhältst einen groben Eindruck von seinen körperlichen, geistigen, handwerklichen, kämpferischen und zauberischen Fähigkeiten.\newline Mächtige Magie: Du erfährst auch besonders hohe Fertigkeiten/markante Vorteile/Eigenheiten/Vorlieben und Vorgehen.\newline Probenschwierigkeit: Magieresistenz\newline Modifikationen: Leuchtende Persönlichkeit (–4; du erkennst in einer Gruppe von bis zu 8 Personen diejenige, die in einem von dir gewählten Bereich am meisten heraussticht. Durch eine Konterprobe (MR, 16) können sich die Personen vor der Entdeckung schützen.)\newline Vorbereitungszeit: 16 Aktionen\newline Ziel: Einzelwesen\newline Reichweite: 8 Schritt\newline Wirkungsdauer: augenblicklich\newline Kosten: 8 AsP}
}


\newglossaryentry{exposamiLebenskraft(Blutgeist)_Talent}
{
    name={Exposami Lebenskraft (Blutgeist)},
    description={Du nimmst Lebewesen als grün leuchtende Flecken wahr. Der Zauber kann alle Elemente bis auf Erz und Eis durchdringen.\newline Mächtige Magie: Du kannst Angehörige verschiedener Spezies/verschiedene Individuen unterscheiden und wiedererkennen.\newline Probenschwierigkeit: 12\newline Modifikationen: Reinheit der Aura (–8; der Zauber zeigt den Gesundheitszustand des Ziels, dämonische Verseuchungen usw.)\newline Vorbereitungszeit: 1 Aktion\newline Ziel: Einzelperson\newline Reichweite: 16 Schritt\newline Wirkungsdauer: 8 Initiativphasen\newline Kosten: 4 AsP}
}


\newglossaryentry{hexenkrallen(Blutgeist)_Talent}
{
    name={Hexenkrallen (Blutgeist)},
    description={Deine Fingernägel werden lang, scharf und hart wie Raubtierklauen. Deine Hände richten 2W6 Waffenschaden an und verlieren die Eigenschaft Zerbrechlich.\newline Mächtige Magie: Erhöht den Schaden um +2.\newline Probenschwierigkeit: 12\newline Vorbereitungszeit: 0 Aktionen\newline Ziel: selbst\newline Reichweite: Berührung\newline Wirkungsdauer: 16 Initiativphasen\newline Kosten: 8 AsP}
}


\newglossaryentry{silentiumSchweigekreis(Blutgeist)_Talent}
{
    name={Silentium Schweigekreis (Blutgeist)},
    description={Der Zauber dämpft alle Geräusche in einer Kugel mit 2 Schritt Radius. Entsprechende Wahrnehmungs-Proben sind um –4 erschwert.\newline Mächtige Magie: Der Malus steigt um –2.\newline Probenschwierigkeit: 12\newline Modifikationen: Begleiter (–4; die Kugel bewegt sich mit dir.)\newline Fremdbegleiter (–8; die Kugel bewegt sich mit einem Einzelwesen.)\newline Vorbereitungszeit: 2 Aktionen\newline Ziel: Zone\newline Reichweite: 2 Schritt\newline Wirkungsdauer: 4 Minuten\newline Kosten: 4 AsP}
}


\newglossaryentry{adleraugeLuchsenohr(Blutgeist)_Talent}
{
    name={Adlerauge Luchsenohr (Blutgeist)},
    description={Du schärfst deine Sinne auf magische Weise. Alle Proben auf Sinnenschärfe und Wachsamkeit sind um +4 erleichtert.\newline Mächtige Magie: Der Bonus steigt um +2.\newline Probenschwierigkeit: 12\newline Modifikationen: Einzelsinn (–4, Wirkungsdauer 1 Stunde; nur ein Sinn ist betroffen.)\newline Vorbereitungszeit: 2 Aktionen\newline Ziel: selbst\newline Reichweite: Berührung\newline Wirkungsdauer: 4 Minuten\newline Kosten: 8 AsP}
}


\newglossaryentry{falkenaugeMeisterschuss(Blutgeist)_Talent}
{
    name={Falkenauge Meisterschuss (Blutgeist)},
    description={Dein nächster Fernkampfangriff ist um +4 erleichtert.\newline Mächtige Magie: Der Bonus steigt um +2.\newline Probenschwierigkeit: 12\newline Modifikationen: Dauerndes Band (–8, 8 AsP; der Zauber wirkt auf alle Fernkampfangriffe während der Wirkungsdauer.)\newline Vorbereitungszeit: 1 Aktion\newline Ziel: Einzelperson\newline Reichweite: Berührung\newline Wirkungsdauer: 8 Initiativphasen\newline Kosten: 4 AsP}
}


\newglossaryentry{pfeilderLuft(Blutgeist)_Talent}
{
    name={Pfeil der Luft (Blutgeist)},
    description={Du verzauberst einen Pfeil (oder Bolzen oder Wurfwaffe), sodass er im Flug die Macht des Elements freisetzt. Der Pfeil verursacht Luftschaden und Zurückstoßen (S. 98). Die Reichweite für diesen Schuss ist verdoppelt.\newline Probenschwierigkeit: 12\newline Modifikationen: Geschütz (–4; du verzauberst ein größeres Geschoss wie das einer Balliste.)\newline Permanenz (–4, 4 AsP, davon 1 gAsP, Wirkungsdauer bis die Bindung gelöst wird oder der Pfeil verschossen wurde)\newline Vorbereitungszeit: 1 Aktion\newline Ziel: Einzelobjekt\newline Reichweite: Berührung\newline Wirkungsdauer: 8 Initiativphasen\newline Kosten: 4 AsP}
}


\newglossaryentry{aeropulvissanfterFall(Blutgeist)_Talent}
{
    name={Aeropulvis sanfter Fall (Blutgeist)},
    description={Halbiert die effektive Höhe eines Sturzes oder Sprunges (kumulativ zur Akrobatik-Probe).\newline Probenschwierigkeit: 12\newline Vorbereitungszeit: 4 Aktionen\newline Ziel: selbst\newline Reichweite: Berührung\newline Wirkungsdauer: 16 Initiativphasen\newline Kosten: 8 AsP}
}


\newglossaryentry{einsmitderNatur(Blutgeist)_Talent}
{
    name={Eins mit der Natur (Blutgeist)},
    description={Das Ziel kann in der Wildnis kann es natürliche Gefahren wie mit der Gabe Gefahreninstinkt erahnen. Beherrscht es die Gabe bereits, sind damit verbundene Proben um +4 erleichtert. Wirkt nicht unter der Erde, auf dem Meer oder in dämonisch pervertiertem Gebiet.\newline Mächtige Magie: Überleben-Proben sind um +2 erleichtert.\newline Probenschwierigkeit: 12\newline Vorbereitungszeit: 4 Minuten\newline Ziel: Einzelperson\newline Reichweite: Berührung\newline Wirkungsdauer: 1 Tag\newline Kosten: 8 AsP}
}


\newglossaryentry{katzenaugen(Blutgeist)_Talent}
{
    name={Katzenaugen (Blutgeist)},
    description={Du ignorierst eine Stufe Dunkelheit, aber grelles Licht erschwert alle Proben um –2. Bei absoluter Dunkelheit ist der Zauber wirkungslos.\newline Mächtige Magie: Du ignorierst 2/3 Stufen Dunkelheit.\newline Probenschwierigkeit: 12\newline Vorbereitungszeit: 4 Aktionen\newline Ziel: Einzelperson\newline Reichweite: Berührung\newline Wirkungsdauer: 1 Stunde\newline Kosten: 4 AsP}
}


\newglossaryentry{foramenForaminor(Blutgeist)_Talent}
{
    name={Foramen Foraminor (Blutgeist)},
    description={Du öffnest ein Schloss beliebiger Bauart.\newline Probenschwierigkeit: Herstellungsschwierigkeit des Schlosses\newline Modifikationen: Riegel (–4, 8 AsP; auch ein schwerer Riegel wie an einem Stadttor kann hiermit geöffnet werden.)\newline Vorbereitungszeit: 2 Aktionen\newline Ziel: Einzelobjekt\newline Reichweite: Berührung\newline Wirkungsdauer: augenblicklich\newline Kosten: 4 AsP}
}


\newglossaryentry{wasseratem(Blutgeist)_Talent}
{
    name={Wasseratem (Blutgeist)},
    description={Du kannst unter Wasser atmen, jedoch nicht mehr an Land. Erlaubt Aufrechterhalten.\newline Probenschwierigkeit: 12\newline Vorbereitungszeit: 16 Aktionen\newline Ziel: Einzelperson\newline Reichweite: Berührung\newline Wirkungsdauer: 1 Stunde\newline Kosten: 8 AsP}
}


\newglossaryentry{wellenlauf(Blutgeist)_Talent}
{
    name={Wellenlauf (Blutgeist)},
    description={Wasser ist für dich ein fester Untergrund. Du erleidest keine Abzüge durch ungünstige Position und wirst nicht durch Wellen und Strömungen beeinflusst. Erlaubt Aufrechterhalten.\newline Probenschwierigkeit: 12\newline Modifikationen: Sinken (–4; du kannst die Zauberwirkung nach Belieben unterdrücken und wieder aktiv werden lassen. Bei Aktivierung unter Wasser wirst du an die Wasseroberfläche gehoben.)\newline Wasserwand (–8; du kannst selbst an Wasserfällen hochklettern, wofür einfache Klettern-Proben anfallen.)\newline Vorbereitungszeit: 2 Aktionen\newline Ziel: selbst\newline Reichweite: Berührung\newline Wirkungsdauer: 4 Minuten\newline Kosten: 4 AsP}
}


\newglossaryentry{hilfreicheTatze,rettendeSchwinge(Blutgeist)_Talent}
{
    name={Hilfreiche Tatze, rettende Schwinge (Blutgeist)},
    description={Du wählst eine Tierart. Befindet sich ein Tier dieser Art in einem Radius von 1 Meile, eilt es herbei. Du kannst das Tier um einen Gefallen bitten, den dieses wenn möglich erfüllen wird. Dann trollt sich das Tier.\newline Mächtige Magie: Verdoppelt den Radius.\newline Probenschwierigkeit: Magieresistenz\newline Vorbereitungszeit: 8 Aktionen\newline Ziel: Zone\newline Reichweite: Berührung\newline Wirkungsdauer: 1 Stunde\newline Kosten: 8 AsP}
}


\newglossaryentry{harmloseGestalt(Blutgeist)_Talent}
{
    name={Harmlose Gestalt (Blutgeist)},
    description={Mit dieser Illusion (Sicht, Gehör, Geruch) erscheinst du den Umstehenden als kleines Kind, alter Krüppel, Orkfrau oder als eine andere harmlose Gestalt. Deine Kleidung ist von der Illusion betroffen, zusätzliche Gegenstände wie ein Wanderstab oder ein Rucksack nicht. Erlaubt Aufrechterhalten.\newline Probenschwierigkeit: 12\newline Vorbereitungszeit: 4 Aktionen\newline Ziel: selbst\newline Reichweite: Berührung\newline Wirkungsdauer: 4 Minuten\newline Kosten: 4 AsP}
}


\newglossaryentry{sensibarEmpathicus(Blutgeist)_Talent}
{
    name={Sensibar Empathicus (Blutgeist)},
    description={Du kannst die Gefühle deines Gegenübers erahnen, wodurch deine Menschenkenntnis gegen das Ziel um +4 steigt.\newline Mächtige Magie: Der Bonus steigt um +2.\newline Probenschwierigkeit: Magieresistenz\newline Vorbereitungszeit: 4 Aktionen\newline Ziel: Einzelperson\newline Reichweite: 4 Schritt\newline Wirkungsdauer: 1 Stunde\newline Kosten: 8 AsP}
}


\newglossaryentry{seidenzungeElfenwort(Blutgeist)_Talent}
{
    name={Seidenzunge Elfenwort (Blutgeist)},
    description={Das Ziel denkt nicht zu genau über deine Worte nach und findet dich überzeugend. Alle Überreden-Proben gegen dein Ziel sind um +4 erleichtert.\newline Mächtige Magie: Der Bonus steigt um +2.\newline Probenschwierigkeit: Magieresistenz\newline Vorbereitungszeit: 1 Aktion\newline Ziel: Einzelperson\newline Reichweite: 4 Schritt\newline Wirkungsdauer: 16 Initiativphasen\newline Kosten: 8 AsP}
}


\newglossaryentry{firnlauf(Blutgeist)_Talent}
{
    name={Firnlauf (Blutgeist)},
    description={Jede noch so dünne Eis- oder Schneeschicht trägt dich wie trockener Boden. Unter diesen Bedingungen erleidest du keinerlei Erschwernisse oder Geschwindigkeitsabzüge.\newline Probenschwierigkeit: 12\newline Modifikationen: Verankerung (–4, 2 AsP; du verankerst dich so auf einer Eisfläche, dass du selbst dann nicht stürzt, wenn sie in bedrohliche Schieflage oder schnelle Bewegung gerät.)\newline Vorbereitungszeit: 2 Aktionen\newline Ziel: selbst\newline Reichweite: Berührung\newline Wirkungsdauer: 1 Stunde\newline Kosten: 8 AsP}
}


\newglossaryentry{spinnenlauf(Blutgeist)_Talent}
{
    name={Spinnenlauf (Blutgeist)},
    description={Deine Hände und Füße haften an Oberflächen, sodass du mit GS 1 an glatten Wänden und sogar Decken ohne Griffen klettern kannst. Erlaubt Aufrechterhalten.\newline Mächtige Magie: Erhöht die GS um +1.\newline Probenschwierigkeit: 12\newline Vorbereitungszeit: 4 Aktionen\newline Ziel: Einzelperson\newline Reichweite: Berührung\newline Wirkungsdauer: 4 Minuten\newline Kosten: 8 AsP}
}


\newglossaryentry{ängstelindern(Blutgeist)_Talent}
{
    name={Ängste lindern (Blutgeist)},
    description={Auf dem Ziel lastende Furcht-Effekte sinken um eine Stufe.\newline Probenschwierigkeit: 12\newline Vorbereitungszeit: 4 Aktionen\newline Ziel: Einzelperson\newline Reichweite: Berührung\newline Wirkungsdauer: augenblicklich\newline Kosten: 4 AsP}
}


\newglossaryentry{armatrutz(Blutgeist)_Talent}
{
    name={Armatrutz (Blutgeist)},
    description={Der RS deines Zieles steigt um 1.\newline Mächtige Magie: Je zwei Stufen verleihen +1 RS.\newline Probenschwierigkeit: 12\newline Modifikationen: Körperschild (–4, 1 AsP; der Armatrutz schützt nur eine Trefferzone.)\newline Vorbereitungszeit: 1 Aktion\newline Ziel: Einzelperson\newline Reichweite: Berührung\newline Wirkungsdauer: 4 Minuten\newline Kosten: 4 AsP}
}


\newglossaryentry{kusch!(Blutgeist)_Talent}
{
    name={Kusch! (Blutgeist)},
    description={Das verzauberte Tier flieht vor dir.\newline Probenschwierigkeit: Magieresistenz\newline Modifikationen: Schrecken des Schwarms (–4; du verscheuchst einen Schwarm Kleintiere.)\newline Vorbereitungszeit: 0 Aktionen\newline Ziel: einzelnes Tier\newline Reichweite: 8 Schritt\newline Wirkungsdauer: 1 Stunde\newline Kosten: 8 AsP}
}


\newglossaryentry{krähenruf(Blutgeist)_Talent}
{
    name={Krähenruf  (Blutgeist)},
    description={Du rufst einen Krähenschwarm aus bis zu 100 Meilen Entfernung zur Hilfe, der sofort erscheint und an deiner Seite kämpft (WS 3, Koloss I, INI 6, GS 8, VT 3, RW 2, AT 10, TP 2W6–2, Zusätzliche AT I).\newline Mächtige Magie: WS, AT und TP des Schwarms steigen um je +1.\newline Probenschwierigkeit: 12\newline Vorbereitungszeit: 2 Aktionen\newline Ziel: Zone\newline Reichweite: Berührung\newline Wirkungsdauer: 16 Initiativphasen\newline Kosten: 8 AsP}
}


\newglossaryentry{nekropathiaSeelenreise(Blutgeist)_Talent}
{
    name={Nekropathia Seelenreise (Blutgeist)},
    description={Du nimmst Kontakt mit einer Seele in Borons Hallen auf. Du kannst mit ihr sprechen, doch es bleibt ihr überlassen, ob und wie sie antwortet. Der Tod der Person darf maximal 1 Jahr zurückliegen und du benötigst einen persönlichen Gegenstand.\newline Mächtige Magie: Verzehnfacht den Abstand zum Todeszeitpunkt.\newline Probenschwierigkeit: 12\newline Vorbereitungszeit: 4 Minuten\newline Ziel: Einzelobjekt\newline Reichweite: Berührung\newline Wirkungsdauer: 1 Stunde\newline Kosten: 8 AsP}
}


\newglossaryentry{atemnot(Blutgeist)_Talent}
{
    name={Atemnot (Blutgeist)},
    description={Du entziehst dem Opfer einen Teil seiner Kraft. Es erleidet 2 Punkte Erschöpfung. Du regenerierst die Hälfte der angerichteten Erschöpfung.\newline Mächtige Magie: Verursacht 1 weiteren Punkt Erschöpfung.\newline Probenschwierigkeit: Magieresistenz\newline Vorbereitungszeit: 2 Aktionen\newline Ziel: Einzelperson\newline Reichweite: 4 Schritt\newline Wirkungsdauer: augenblicklich\newline Kosten: 8 AsP}
}


\newglossaryentry{serpentialisSchlangenleib(Blutgeist)_Talent}
{
    name={Serpentialis Schlangenleib (Blutgeist)},
    description={Deine Arme verwandeln sich in grüne, 2 Schritt lange Giftschlangen, deren hinteres Ende noch mit deinem Rumpf verbunden ist. Die Schlangen können beide in einer Aktion Konflikt angreifen (WS 3, VT 10, RW 2, AT 10, TP 2W6) und ein Waffengift übertragen (Stufe 20, keine Verzögerung, Intervall 2 INI-Phasen, Wirkungsdauer 2 INI-Phasen, 2W6 SP).\newline Mächtige Magie: Die WS der Schlangen steigt um +1, AT, VT und TP um +2.\newline Probenschwierigkeit: 12\newline Modifikationen: Schlangengriff (Wirkungsdauer 4 Minuten, 4 AsP; die Arme verwandeln sich in kampfunfähige, aber geschickte Nattern, deren FF gleich deiner ist.)\newline Vorbereitungszeit: 2 Aktionen\newline Ziel: selbst\newline Reichweite: Berührung\newline Wirkungsdauer: 16 Initiativphasen\newline Kosten: 16 AsP}
}


\newglossaryentry{vipernblick(Blutgeist)_Talent}
{
    name={Vipernblick (Blutgeist)},
    description={Du erscheinst deinem Opfer als grauenvolle Gestalt aus den Niederhöllen. Es kann nicht wegsehen und ist vor Angst handlungsunfähig. Wird der Sichtkontakt unterbrochen, endet der Zauber.\newline Probenschwierigkeit: Magieresistenz\newline Modifikationen: Fremdgestalt (–4; du kannst eine andere Person wählen, die als Alpgestalt erscheint.)\newline Vorbereitungszeit: 4 Aktionen\newline Ziel: Einzelwesen\newline Reichweite: 8 Schritt\newline Wirkungsdauer: 4 Minuten\newline Kosten: 16 AsP}
}


\newglossaryentry{warmesBlut(Blutgeist)_Talent}
{
    name={Warmes Blut (Blutgeist)},
    description={Durch diesen Zauber siehst du die Wärmestrahlung deiner Umgebung. Kaltes erscheint schwarz bis grünblau, Warmblüter sind gelb und Feuer ist orange bis tiefrot.\newline Probenschwierigkeit: 12\newline Vorbereitungszeit: 4 Aktionen\newline Ziel: selbst\newline Reichweite: Berührung\newline Wirkungsdauer: 1 Stunde\newline Kosten: 4 AsP}
}


\newglossaryentry{horriphobusSchreckgestalt(Blutgeist)_Talent}
{
    name={Horriphobus Schreckgestalt (Blutgeist)},
    description={Dein Opfer hat schreckliche Angst vor dir und erleidet einen Furcht-Effekt Stufe 2.\newline Mächtige Magie: Der Furcht-Effekt steigt um eine Stufe.\newline Probenschwierigkeit: Magieresistenz\newline Vorbereitungszeit: 0 Aktionen\newline Ziel: Einzelwesen\newline Reichweite: 8 Schritt\newline Wirkungsdauer: 1 Stunde\newline Kosten: 8 AsP}
}


\newglossaryentry{sensattacoMeisterstreich(Blutgeist)_Talent}
{
    name={Sensattaco Meisterstreich (Blutgeist)},
    description={Dein Ziel erkennt intuitiv die Lücken in der Verteidigung des Gegners. Seine AT sind um +2 erleichtert und seine Chance auf einen Triumph bei einer AT steigt um 1 auf dem W20 (zum Beispiel von 20 auf 19–20).\newline Mächtige Magie: Der Bonus steigt um +1.\newline Probenschwierigkeit: 12\newline Vorbereitungszeit: 1 Aktion\newline Ziel: Einzelperson\newline Reichweite: Berührung\newline Wirkungsdauer: 16 Initiativphasen\newline Kosten: 8 AsP}
}


\newglossaryentry{krötensprung(Blutgeist)_Talent}
{
    name={Krötensprung (Blutgeist)},
    description={Der nächste Sprung deines Zieles ist gewaltig. Er kann bis zu 8 Schritt Weite und 4 Schritt Höhe überwinden und die effektive Sturzhöhe (S. 35) sinkt um 4 Schritt.\newline Mächtige Magie: Die Weite steigt um +4 Schritt,  die (Sturz-)Höhe um +2 Schritt.\newline Probenschwierigkeit: 12\newline Modifikationen: Krötengang (–4, Wirkungsdauer 16 Initiativphasen, 8 AsP; dein Ziel kann beliebig oft springen.)\newline Vorbereitungszeit: 2 Aktionen\newline Ziel: Einzelperson\newline Reichweite: Berührung\newline Wirkungsdauer: 16 Initiativphasen\newline Kosten: 4 AsP}
}


\newglossaryentry{spurlosTrittlos(Blutgeist)_Talent}
{
    name={Spurlos Trittlos (Blutgeist)},
    description={Du tarnst deine Fährte mit Magie. Alle Proben zur Verfolgung deiner Fährte sind um –4 erschwert.\newline Mächtige Magie: Der Malus steigt um –2.\newline Probenschwierigkeit: 12\newline Modifikationen: Andere Person (–4, Ziel Einzelperson)\newline Zone (–4, Zone, 16 AsP; der Zauber betrifft alle in einem Radius von 4 Schritt.)\newline Vorbereitungszeit: 4 Aktionen\newline Ziel: selbst\newline Reichweite: Berührung\newline Wirkungsdauer: 1 Stunde\newline Kosten: 4 AsP}
}


\newglossaryentry{abvenenumreineSpeise(Blutgeist)_Talent}
{
    name={Abvenenum reine Speise (Blutgeist)},
    description={Du reinigst eine Mahlzeit für 10 Personen von Verfallserscheinungen, Giften und Krankheitserregern, bis zu einer Gift-/Krankheitsstufe von 20. Der Geschmack ändert sich nicht.\newline Mächtige Magie: Die maximal aufgehobene Gift-/Krankheitsstufe steigt um 4.\newline Probenschwierigkeit: 12\newline Modifikationen: Schutz vor Übelkeit (–4; auch ungefährliche, aber unangenehme Inhalte wie Salz im Meerwasser werden entfernt.)\newline Schutz vor Vergiftung (–4, Wirkungsdauer 8 Stunden; du reinigst auch alles, was dem Essen hinzugefügt wird.)\newline Vorbereitungszeit: 8 Aktionen\newline Ziel: Einzelobjekte\newline Reichweite: 2 Schritt\newline Wirkungsdauer: augenblicklich\newline Kosten: 4 AsP}
}


\newglossaryentry{movimentoDauerlauf(Blutgeist)_Talent}
{
    name={Movimento Dauerlauf (Blutgeist)},
    description={Verdoppelt das DH* (S. 34) des Ziels und das Intervall, in dem körperliche Anstrengung Erschöpfung verursacht.\newline Mächtige Magie: Je zwei Stufen verdreifachen/vervierfachen das DH* und das Intervall.\newline Probenschwierigkeit: 12\newline Vorbereitungszeit: 8 Aktionen\newline Ziel: Einzelwesen\newline Reichweite: Berührung\newline Wirkungsdauer: 8 Stunden\newline Kosten: 8 AsP}
}


\newglossaryentry{bärenruheWinterschlaf(Blutgeist)_Talent}
{
    name={Bärenruhe Winterschlaf (Blutgeist)},
    description={Du versetzt dein Ziel in einen tiefen Winterschlaf. Während des Schlafes benötigt es keine Nahrung, und kein Wasser. Gifte, Krankheiten und Kälte fügen ihm keinen Schaden zu. Dafür regeneriert es auch nicht.\newline Probenschwierigkeit: 12\newline Modifikationen: Der lange Schlaf (–4, Wirkungsdauer 1 Monat, 16 AsP)\newline Vorbereitungszeit: 4 Minuten\newline Ziel: Einzelwesen\newline Reichweite: Berührung\newline Wirkungsdauer: 1 Tag\newline Kosten: 8 AsP}
}


\newglossaryentry{motoricusGeisterhand(Blutgeist)_Talent}
{
    name={Motoricus Geisterhand (Blutgeist)},
    description={Du kannst einen Gegenstand mit einer KK von 2 anheben und mit maximal 4 Schritt pro Initiativephase durch die Luft fliegen lassen. Besonders komplizierte und feine Bewegungen sind um bis zu 8 Punkte erschwert. Erfordert Konzentration.\newline Mächtige Magie: Erhöht KK und die Schritte pro Initiativephase um +2.\newline Probenschwierigkeit: 12\newline Modifikationen: Unsichtbarer Hieb (Wirkungsdauer augenblicklich; der Zauber fügt einem Objekt 2W6 SP zu. Mächtige Magie erhöht den Schaden um +4.)\newline Magische Abwehr (–4, wird in einer Reaktion gewirkt, Wirkungsdauer augenblicklich; die Zauberprobe gilt als VT gegen einen Nahkampfangriff in Reichweite.)\newline Fesselfeld (–8, 32 AsP; in einer Zone von 4 Schritt Radius wird die Bewegung jedes unbelebten Objekts mit einer KK von 2 behindert, was Kampfhandlungen mit Waffen fast unmöglich macht.)\newline Vorbereitungszeit: 1 Aktion\newline Ziel: Einzelobjekt\newline Reichweite: 8 Schritt\newline Wirkungsdauer: 4 Minuten\newline Kosten: 4 AsP}
}


\newglossaryentry{blutdes(Tieres)_Talent}
{
    name={Blut des (Tieres)},
    description={Du wirst von der animalischen Essenz deines Tiergeistes erfüllt. Jedem, der keine Konterprobe (MU, 16) besteht, erscheinst du als ein Wesen mit Schreckgestalt II (S. 98).\newline Probenschwierigkeit: 12\newline Vorbereitungszeit: 0 Aktionen\newline Ziel: selbst\newline Reichweite: Berührung\newline Wirkungsdauer: 16 Initiativphasen\newline Kosten: 8 AsP\newline Fertigkeiten: Verwandlung, Gaben des Odun\newline Erlernen: Dur 8; 40 EP}
}


\newglossaryentry{hauchdes(Tieres)_Talent}
{
    name={Hauch des (Tieres)},
    description={Der Tiergeist stärkt und unterstützt dich. Du erhältst einen Bonus von +2 auf alle bei deinem Tier angegebenen Werte (S. 160).\newline Mächtige Magie: Der Bonus steigt um +1.\newline Probenschwierigkeit: 12\newline Vorbereitungszeit: 4 Aktionen\newline Ziel: selbst\newline Reichweite: Berührung\newline Wirkungsdauer: 1 Stunde\newline Kosten: 4 AsP\newline Fertigkeiten: Eigenschaften, Gaben des Odun\newline Erlernen: Dur 4; 40 EP}
}


\newglossaryentry{hautdes(Tieres)_Talent}
{
    name={Haut des (Tieres)},
    description={Du setzt eine der bei deinem Tier angegebenen Verwandlungen (S. 160) ein. Du kannst auch beide Verwandlungen gleichzeitig nutzen, dafür musst du den Zauber zwei Mal wirken.\newline Mächtige Magie: Falls zutreffend, geschieht das Folgende: Probenerleichterungen steigen um +2, TP um +2, die Giftstufe um +4, die Flugweiten  und -höhen um 4/2 Schritt. Je zwei Stufen erhöhen einen RS-Bonus um +1.\newline Probenschwierigkeit: 12\newline Vorbereitungszeit: 4 Aktionen\newline Ziel: Zone\newline Reichweite: selbst\newline Wirkungsdauer: 1 Stunde\newline Kosten: 8 AsP\newline Fertigkeiten: Verwandlung, Gaben des Odun\newline Erlernen: Dur 12; 60 EP}
}


\newglossaryentry{rufdes(Tieres)_Talent}
{
    name={Ruf des (Tieres)},
    description={Du rufst ein Exemplar deines Tiers herbei. Befindet sich ein Wesen dieser Art in einem Radius von 4 Meilen, eilt es herbei. Du kannst dieses Wesen um einen Gefallen bitten, das dieses wenn möglich erfüllen wird. Dann trollt sich das Wesen.\newline Mächtige Magie: Verdoppelt den Radius und die Zahl der herbeieilenden Wesen.\newline Probenschwierigkeit: 12\newline Vorbereitungszeit: 16 Aktionen\newline Ziel: Zone\newline Reichweite: Berührung\newline Wirkungsdauer: 1 Stunde\newline Kosten: 16 AsP\newline Fertigkeiten: Einfluss, Verständigung, Gaben des Odun\newline Erlernen: Dur 14; 20 EP}
}


\newglossaryentry{wesendes(Tieres)_Talent}
{
    name={Wesen des (Tieres)},
    description={Du siehst die Gedanken des Tiers als verschwommene Bilder.\newline Mächtige Magie: Zusätzlich kannst du das Tier um einen kleineren/größeren/gefährlichen Gefallen bitten. Ob das Tier den Gefallen ausführen kann und will, ist Spielleiterentscheid.\newline Probenschwierigkeit: Magieresistenz\newline Vorbereitungszeit: 4 Aktionen\newline Ziel: Tier\newline Reichweite: 8 Schritt\newline Wirkungsdauer: 1 Stunde\newline Kosten: 8 AsP\newline Fertigkeiten: Hellsicht, Verständigung, Gaben des Odun\newline Erlernen: Dur 14; 20 EP}
}


\newglossaryentry{seeledes(Tieres)_Talent}
{
    name={Seele des (Tieres)},
    description={Du verwandelst dich in das Tier, wobei du deine geistigen Fähigkeiten behältst. Die körperlichen Fähigkeiten entsprechen denen des Tiers. Du kannst in Tiergestalt nicht zaubern. Erlaubt Aufrechterhalten.\newline Mächtige Magie: Das Tier ist ein überdurchschnittlicher/außergewöhnlicher/herausragender/einzigartiger Vertreter seiner Art, was die Werte des Tieres nach Spielleiterentscheid erhöht.\newline Probenschwierigkeit: 12\newline Modifikationen: Tierseele (Wirkungsdauer 1 Tag; Die Instinkte des Tieres können überhand nehmen.)\newline Vorbereitungszeit: 8 Aktionen\newline Ziel: selbst\newline Reichweite: Berührung\newline Wirkungsdauer: 1 Stunde\newline Kosten: 16 AsP\newline Fertigkeiten: Verwandlung, Gaben des Odun\newline Erlernen: Dur 16; normalerweise 40 EP\newline Anmerkung: Die Lernkosten des Zaubers orientieren sich an der Tierart, wobei fliegende, giftige, sehr starke usw. Tiere teurer sind.}
}


\newglossaryentry{abvenenumreineSpeise(Tiergeist)_Talent}
{
    name={Abvenenum reine Speise (Tiergeist)},
    description={Du reinigst eine Mahlzeit für 10 Personen von Verfallserscheinungen, Giften und Krankheitserregern, bis zu einer Gift-/Krankheitsstufe von 20. Der Geschmack ändert sich nicht.\newline Mächtige Magie: Die maximal aufgehobene Gift-/Krankheitsstufe steigt um 4.\newline Probenschwierigkeit: 12\newline Modifikationen: Schutz vor Übelkeit (–4; auch ungefährliche, aber unangenehme Inhalte wie Salz im Meerwasser werden entfernt.)\newline Schutz vor Vergiftung (–4, Wirkungsdauer 8 Stunden; du reinigst auch alles, was dem Essen hinzugefügt wird.)\newline Vorbereitungszeit: 8 Aktionen\newline Ziel: Einzelobjekte\newline Reichweite: 2 Schritt\newline Wirkungsdauer: augenblicklich\newline Kosten: 4 AsP}
}


\newglossaryentry{adleraugeLuchsenohr(Tiergeist)_Talent}
{
    name={Adlerauge Luchsenohr (Tiergeist)},
    description={Du schärfst deine Sinne auf magische Weise. Alle Proben auf Sinnenschärfe und Wachsamkeit sind um +4 erleichtert.\newline Mächtige Magie: Der Bonus steigt um +2.\newline Probenschwierigkeit: 12\newline Modifikationen: Einzelsinn (–4, Wirkungsdauer 1 Stunde; nur ein Sinn ist betroffen.)\newline Vorbereitungszeit: 2 Aktionen\newline Ziel: selbst\newline Reichweite: Berührung\newline Wirkungsdauer: 4 Minuten\newline Kosten: 8 AsP}
}


\newglossaryentry{aeropulvissanfterFall(Tiergeist)_Talent}
{
    name={Aeropulvis sanfter Fall (Tiergeist)},
    description={Halbiert die effektive Höhe eines Sturzes oder Sprunges (kumulativ zur Akrobatik-Probe).\newline Probenschwierigkeit: 12\newline Vorbereitungszeit: 4 Aktionen\newline Ziel: selbst\newline Reichweite: Berührung\newline Wirkungsdauer: 16 Initiativphasen\newline Kosten: 8 AsP}
}


\newglossaryentry{armatrutz(Tiergeist)_Talent}
{
    name={Armatrutz (Tiergeist)},
    description={Der RS deines Zieles steigt um 1.\newline Mächtige Magie: Je zwei Stufen verleihen +1 RS.\newline Probenschwierigkeit: 12\newline Modifikationen: Körperschild (–4, 1 AsP; der Armatrutz schützt nur eine Trefferzone.)\newline Vorbereitungszeit: 1 Aktion\newline Ziel: Einzelperson\newline Reichweite: Berührung\newline Wirkungsdauer: 4 Minuten\newline Kosten: 4 AsP}
}


\newglossaryentry{atemnot(Tiergeist)_Talent}
{
    name={Atemnot (Tiergeist)},
    description={Du entziehst dem Opfer einen Teil seiner Kraft. Es erleidet 2 Punkte Erschöpfung. Du regenerierst die Hälfte der angerichteten Erschöpfung.\newline Mächtige Magie: Verursacht 1 weiteren Punkt Erschöpfung.\newline Probenschwierigkeit: Magieresistenz\newline Vorbereitungszeit: 2 Aktionen\newline Ziel: Einzelperson\newline Reichweite: 4 Schritt\newline Wirkungsdauer: augenblicklich\newline Kosten: 8 AsP}
}


\newglossaryentry{attributo(Tiergeist)_Talent}
{
    name={Attributo (Tiergeist)},
    description={Wähle ein Attribut aus. Proben auf dieses Attribut sind um +2, Fertigkeitsproben mit diesem Attribut um +1 erleichtert.\newline Mächtige Magie: Der Bonus steigt um +2/+1.\newline Probenschwierigkeit: 12\newline Vorbereitungszeit: 8 Aktionen\newline Ziel: Einzelperson\newline Reichweite: Berührung\newline Wirkungsdauer: 1 Stunde\newline Kosten: 8 AsP}
}


\newglossaryentry{axxeleratusBlitzgeschwind(Tiergeist)_Talent}
{
    name={Axxeleratus Blitzgeschwind (Tiergeist)},
    description={Die GS deines Zieles erhöht sich um +4 und alle AT und VT sind um +2 erleichtert, Ausweichen um weitere +2.\newline Mächtige Magie: Die GS steigt um weitere +2.\newline Probenschwierigkeit: 12\newline Vorbereitungszeit: 0 Aktionen\newline Ziel: Einzelperson\newline Reichweite: 4 Schritt\newline Wirkungsdauer: 16 Initiativphasen\newline Kosten: 8 AsP}
}


\newglossaryentry{blickaufsWesen(Tiergeist)_Talent}
{
    name={Blick aufs Wesen (Tiergeist)},
    description={Der Zauber offenbart dir die Fertigkeiten deines Ziels. Du erhältst einen groben Eindruck von seinen körperlichen, geistigen, handwerklichen, kämpferischen und zauberischen Fähigkeiten.\newline Mächtige Magie: Du erfährst auch besonders hohe Fertigkeiten/markante Vorteile/Eigenheiten/Vorlieben und Vorgehen.\newline Probenschwierigkeit: Magieresistenz\newline Modifikationen: Leuchtende Persönlichkeit (–4; du erkennst in einer Gruppe von bis zu 8 Personen diejenige, die in einem von dir gewählten Bereich am meisten heraussticht. Durch eine Konterprobe (MR, 16) können sich die Personen vor der Entdeckung schützen.)\newline Vorbereitungszeit: 16 Aktionen\newline Ziel: Einzelwesen\newline Reichweite: 8 Schritt\newline Wirkungsdauer: augenblicklich\newline Kosten: 8 AsP}
}


\newglossaryentry{bärenruheWinterschlaf(Tiergeist)_Talent}
{
    name={Bärenruhe Winterschlaf (Tiergeist)},
    description={Du versetzt dein Ziel in einen tiefen Winterschlaf. Während des Schlafes benötigt es keine Nahrung, und kein Wasser. Gifte, Krankheiten und Kälte fügen ihm keinen Schaden zu. Dafür regeneriert es auch nicht.\newline Probenschwierigkeit: 12\newline Modifikationen: Der lange Schlaf (–4, Wirkungsdauer 1 Monat, 16 AsP)\newline Vorbereitungszeit: 4 Minuten\newline Ziel: Einzelwesen\newline Reichweite: Berührung\newline Wirkungsdauer: 1 Tag\newline Kosten: 8 AsP}
}


\newglossaryentry{einsmitderNatur(Tiergeist)_Talent}
{
    name={Eins mit der Natur (Tiergeist)},
    description={Das Ziel kann in der Wildnis kann es natürliche Gefahren wie mit der Gabe Gefahreninstinkt erahnen. Beherrscht es die Gabe bereits, sind damit verbundene Proben um +4 erleichtert. Wirkt nicht unter der Erde, auf dem Meer oder in dämonisch pervertiertem Gebiet.\newline Mächtige Magie: Überleben-Proben sind um +2 erleichtert.\newline Probenschwierigkeit: 12\newline Vorbereitungszeit: 4 Minuten\newline Ziel: Einzelperson\newline Reichweite: Berührung\newline Wirkungsdauer: 1 Tag\newline Kosten: 8 AsP}
}


\newglossaryentry{eiseskälteKämpferherz(Tiergeist)_Talent}
{
    name={Eiseskälte Kämpferherz (Tiergeist)},
    description={Das Ziel empfindet keinerlei Schmerzen mehr. Es erhält den Vorteil Kalte Wut (S. 43). Proben zum Ignorieren von Wundschmerz sind um +4 erleichtert.\newline Mächtige Magie: Der Bonus steigt um +2.\newline Probenschwierigkeit: 12\newline Vorbereitungszeit: 0 Aktionen\newline Ziel: Einzelperson\newline Reichweite: Berührung\newline Wirkungsdauer: 16 Initiativphasen\newline Kosten: 8 AsP}
}


\newglossaryentry{exposamiLebenskraft(Tiergeist)_Talent}
{
    name={Exposami Lebenskraft (Tiergeist)},
    description={Du nimmst Lebewesen als grün leuchtende Flecken wahr. Der Zauber kann alle Elemente bis auf Erz und Eis durchdringen.\newline Mächtige Magie: Du kannst Angehörige verschiedener Spezies/verschiedene Individuen unterscheiden und wiedererkennen.\newline Probenschwierigkeit: 12\newline Modifikationen: Reinheit der Aura (–8; der Zauber zeigt den Gesundheitszustand des Ziels, dämonische Verseuchungen usw.)\newline Vorbereitungszeit: 1 Aktion\newline Ziel: Einzelperson\newline Reichweite: 16 Schritt\newline Wirkungsdauer: 8 Initiativphasen\newline Kosten: 4 AsP}
}


\newglossaryentry{falkenaugeMeisterschuss(Tiergeist)_Talent}
{
    name={Falkenauge Meisterschuss (Tiergeist)},
    description={Dein nächster Fernkampfangriff ist um +4 erleichtert.\newline Mächtige Magie: Der Bonus steigt um +2.\newline Probenschwierigkeit: 12\newline Modifikationen: Dauerndes Band (–8, 8 AsP; der Zauber wirkt auf alle Fernkampfangriffe während der Wirkungsdauer.)\newline Vorbereitungszeit: 1 Aktion\newline Ziel: Einzelperson\newline Reichweite: Berührung\newline Wirkungsdauer: 8 Initiativphasen\newline Kosten: 4 AsP}
}


\newglossaryentry{firnlauf(Tiergeist)_Talent}
{
    name={Firnlauf (Tiergeist)},
    description={Jede noch so dünne Eis- oder Schneeschicht trägt dich wie trockener Boden. Unter diesen Bedingungen erleidest du keinerlei Erschwernisse oder Geschwindigkeitsabzüge.\newline Probenschwierigkeit: 12\newline Modifikationen: Verankerung (–4, 2 AsP; du verankerst dich so auf einer Eisfläche, dass du selbst dann nicht stürzt, wenn sie in bedrohliche Schieflage oder schnelle Bewegung gerät.)\newline Vorbereitungszeit: 2 Aktionen\newline Ziel: selbst\newline Reichweite: Berührung\newline Wirkungsdauer: 1 Stunde\newline Kosten: 8 AsP}
}


\newglossaryentry{foramenForaminor(Tiergeist)_Talent}
{
    name={Foramen Foraminor (Tiergeist)},
    description={Du öffnest ein Schloss beliebiger Bauart.\newline Probenschwierigkeit: Herstellungsschwierigkeit des Schlosses\newline Modifikationen: Riegel (–4, 8 AsP; auch ein schwerer Riegel wie an einem Stadttor kann hiermit geöffnet werden.)\newline Vorbereitungszeit: 2 Aktionen\newline Ziel: Einzelobjekt\newline Reichweite: Berührung\newline Wirkungsdauer: augenblicklich\newline Kosten: 4 AsP}
}


\newglossaryentry{harmloseGestalt(Tiergeist)_Talent}
{
    name={Harmlose Gestalt (Tiergeist)},
    description={Mit dieser Illusion (Sicht, Gehör, Geruch) erscheinst du den Umstehenden als kleines Kind, alter Krüppel, Orkfrau oder als eine andere harmlose Gestalt. Deine Kleidung ist von der Illusion betroffen, zusätzliche Gegenstände wie ein Wanderstab oder ein Rucksack nicht. Erlaubt Aufrechterhalten.\newline Probenschwierigkeit: 12\newline Vorbereitungszeit: 4 Aktionen\newline Ziel: selbst\newline Reichweite: Berührung\newline Wirkungsdauer: 4 Minuten\newline Kosten: 4 AsP}
}


\newglossaryentry{hexenkrallen(Tiergeist)_Talent}
{
    name={Hexenkrallen (Tiergeist)},
    description={Deine Fingernägel werden lang, scharf und hart wie Raubtierklauen. Deine Hände richten 2W6 Waffenschaden an und verlieren die Eigenschaft Zerbrechlich.\newline Mächtige Magie: Erhöht den Schaden um +2.\newline Probenschwierigkeit: 12\newline Vorbereitungszeit: 0 Aktionen\newline Ziel: selbst\newline Reichweite: Berührung\newline Wirkungsdauer: 16 Initiativphasen\newline Kosten: 8 AsP}
}


\newglossaryentry{hilfreicheTatze,rettendeSchwinge(Tiergeist)_Talent}
{
    name={Hilfreiche Tatze, rettende Schwinge (Tiergeist)},
    description={Du wählst eine Tierart. Befindet sich ein Tier dieser Art in einem Radius von 1 Meile, eilt es herbei. Du kannst das Tier um einen Gefallen bitten, den dieses wenn möglich erfüllen wird. Dann trollt sich das Tier.\newline Mächtige Magie: Verdoppelt den Radius.\newline Probenschwierigkeit: Magieresistenz\newline Vorbereitungszeit: 8 Aktionen\newline Ziel: Zone\newline Reichweite: Berührung\newline Wirkungsdauer: 1 Stunde\newline Kosten: 8 AsP}
}


\newglossaryentry{horriphobusSchreckgestalt(Tiergeist)_Talent}
{
    name={Horriphobus Schreckgestalt (Tiergeist)},
    description={Dein Opfer hat schreckliche Angst vor dir und erleidet einen Furcht-Effekt Stufe 2.\newline Mächtige Magie: Der Furcht-Effekt steigt um eine Stufe.\newline Probenschwierigkeit: Magieresistenz\newline Vorbereitungszeit: 0 Aktionen\newline Ziel: Einzelwesen\newline Reichweite: 8 Schritt\newline Wirkungsdauer: 1 Stunde\newline Kosten: 8 AsP}
}


\newglossaryentry{katzenaugen(Tiergeist)_Talent}
{
    name={Katzenaugen (Tiergeist)},
    description={Du ignorierst eine Stufe Dunkelheit, aber grelles Licht erschwert alle Proben um –2. Bei absoluter Dunkelheit ist der Zauber wirkungslos.\newline Mächtige Magie: Du ignorierst 2/3 Stufen Dunkelheit.\newline Probenschwierigkeit: 12\newline Vorbereitungszeit: 4 Aktionen\newline Ziel: Einzelperson\newline Reichweite: Berührung\newline Wirkungsdauer: 1 Stunde\newline Kosten: 4 AsP}
}


\newglossaryentry{krähenruf(Tiergeist)_Talent}
{
    name={Krähenruf  (Tiergeist)},
    description={Du rufst einen Krähenschwarm aus bis zu 100 Meilen Entfernung zur Hilfe, der sofort erscheint und an deiner Seite kämpft (WS 3, Koloss I, INI 6, GS 8, VT 3, RW 2, AT 10, TP 2W6–2, Zusätzliche AT I).\newline Mächtige Magie: WS, AT und TP des Schwarms steigen um je +1.\newline Probenschwierigkeit: 12\newline Vorbereitungszeit: 2 Aktionen\newline Ziel: Zone\newline Reichweite: Berührung\newline Wirkungsdauer: 16 Initiativphasen\newline Kosten: 8 AsP}
}


\newglossaryentry{krötensprung(Tiergeist)_Talent}
{
    name={Krötensprung (Tiergeist)},
    description={Der nächste Sprung deines Zieles ist gewaltig. Er kann bis zu 8 Schritt Weite und 4 Schritt Höhe überwinden und die effektive Sturzhöhe (S. 35) sinkt um 4 Schritt.\newline Mächtige Magie: Die Weite steigt um +4 Schritt,  die (Sturz-)Höhe um +2 Schritt.\newline Probenschwierigkeit: 12\newline Modifikationen: Krötengang (–4, Wirkungsdauer 16 Initiativphasen, 8 AsP; dein Ziel kann beliebig oft springen.)\newline Vorbereitungszeit: 2 Aktionen\newline Ziel: Einzelperson\newline Reichweite: Berührung\newline Wirkungsdauer: 16 Initiativphasen\newline Kosten: 4 AsP}
}


\newglossaryentry{kusch!(Tiergeist)_Talent}
{
    name={Kusch! (Tiergeist)},
    description={Das verzauberte Tier flieht vor dir.\newline Probenschwierigkeit: Magieresistenz\newline Modifikationen: Schrecken des Schwarms (–4; du verscheuchst einen Schwarm Kleintiere.)\newline Vorbereitungszeit: 0 Aktionen\newline Ziel: einzelnes Tier\newline Reichweite: 8 Schritt\newline Wirkungsdauer: 1 Stunde\newline Kosten: 8 AsP}
}


\newglossaryentry{memoransGedächtniskraft(Tiergeist)_Talent}
{
    name={Memorans Gedächtniskraft (Tiergeist)},
    description={Du kannst dir sämtliche Bilder, Schriftstücke, Inschriften und ähnliches für immer einprägen. Eine Seite eines Buches kostet dich etwa eine Minute.\newline Probenschwierigkeit: 12\newline Modifikationen: Drachengedächtnis (–8, Wirkungsdauer 16 Initiativphasen; alles, was du während der Wirkungsdauer siehst, ist dir für immer ins Gedächtnis gebrannt.)\newline Vorbereitungszeit: 4 Minuten\newline Ziel: selbst\newline Reichweite: 4 Schritt\newline Wirkungsdauer: 4 Minuten\newline Kosten: 8 AsP}
}


\newglossaryentry{motoricusGeisterhand(Tiergeist)_Talent}
{
    name={Motoricus Geisterhand (Tiergeist)},
    description={Du kannst einen Gegenstand mit einer KK von 2 anheben und mit maximal 4 Schritt pro Initiativephase durch die Luft fliegen lassen. Besonders komplizierte und feine Bewegungen sind um bis zu 8 Punkte erschwert. Erfordert Konzentration.\newline Mächtige Magie: Erhöht KK und die Schritte pro Initiativephase um +2.\newline Probenschwierigkeit: 12\newline Modifikationen: Unsichtbarer Hieb (Wirkungsdauer augenblicklich; der Zauber fügt einem Objekt 2W6 SP zu. Mächtige Magie erhöht den Schaden um +4.)\newline Magische Abwehr (–4, wird in einer Reaktion gewirkt, Wirkungsdauer augenblicklich; die Zauberprobe gilt als VT gegen einen Nahkampfangriff in Reichweite.)\newline Fesselfeld (–8, 32 AsP; in einer Zone von 4 Schritt Radius wird die Bewegung jedes unbelebten Objekts mit einer KK von 2 behindert, was Kampfhandlungen mit Waffen fast unmöglich macht.)\newline Vorbereitungszeit: 1 Aktion\newline Ziel: Einzelobjekt\newline Reichweite: 8 Schritt\newline Wirkungsdauer: 4 Minuten\newline Kosten: 4 AsP}
}


\newglossaryentry{movimentoDauerlauf(Tiergeist)_Talent}
{
    name={Movimento Dauerlauf (Tiergeist)},
    description={Verdoppelt das DH* (S. 34) des Ziels und das Intervall, in dem körperliche Anstrengung Erschöpfung verursacht.\newline Mächtige Magie: Je zwei Stufen verdreifachen/vervierfachen das DH* und das Intervall.\newline Probenschwierigkeit: 12\newline Vorbereitungszeit: 8 Aktionen\newline Ziel: Einzelwesen\newline Reichweite: Berührung\newline Wirkungsdauer: 8 Stunden\newline Kosten: 8 AsP}
}


\newglossaryentry{nekropathiaSeelenreise(Tiergeist)_Talent}
{
    name={Nekropathia Seelenreise (Tiergeist)},
    description={Du nimmst Kontakt mit einer Seele in Borons Hallen auf. Du kannst mit ihr sprechen, doch es bleibt ihr überlassen, ob und wie sie antwortet. Der Tod der Person darf maximal 1 Jahr zurückliegen und du benötigst einen persönlichen Gegenstand.\newline Mächtige Magie: Verzehnfacht den Abstand zum Todeszeitpunkt.\newline Probenschwierigkeit: 12\newline Vorbereitungszeit: 4 Minuten\newline Ziel: Einzelobjekt\newline Reichweite: Berührung\newline Wirkungsdauer: 1 Stunde\newline Kosten: 8 AsP}
}


\newglossaryentry{pfeilderLuft(Tiergeist)_Talent}
{
    name={Pfeil der Luft (Tiergeist)},
    description={Du verzauberst einen Pfeil (oder Bolzen oder Wurfwaffe), sodass er im Flug die Macht des Elements freisetzt. Der Pfeil verursacht Luftschaden und Zurückstoßen (S. 98). Die Reichweite für diesen Schuss ist verdoppelt.\newline Probenschwierigkeit: 12\newline Modifikationen: Geschütz (–4; du verzauberst ein größeres Geschoss wie das einer Balliste.)\newline Permanenz (–4, 4 AsP, davon 1 gAsP, Wirkungsdauer bis die Bindung gelöst wird oder der Pfeil verschossen wurde)\newline Vorbereitungszeit: 1 Aktion\newline Ziel: Einzelobjekt\newline Reichweite: Berührung\newline Wirkungsdauer: 8 Initiativphasen\newline Kosten: 4 AsP}
}


\newglossaryentry{psychostabilis(Tiergeist)_Talent}
{
    name={Psychostabilis (Tiergeist)},
    description={Magieresistenz-Proben des Ziels sind um +4 erleichtert.\newline Mächtige Magie: Erhöht den Bonus um +2.\newline Probenschwierigkeit: 12\newline Modifikationen: Schnellsteigerung (–4; Wirkungsdauer 16 Initiativphasen; verdoppelt die Erleichterung.)\newline Stabilisierung (–8; das Ziel darf sofort eine MR-Probe gegen einen auf es wirkenden Zauber wiederholen. Gelingt sie, wird dieser Zauber für die Wirkungsdauer des Psychostabilis unterdrückt.)\newline Vorbereitungszeit: 2 Aktionen\newline Ziel: Einzelperson\newline Reichweite: Berührung\newline Wirkungsdauer: 1 Stunde\newline Kosten: 8 AsP}
}


\newglossaryentry{ruheKörper,RuheGeist(Tiergeist)_Talent}
{
    name={Ruhe Körper, Ruhe Geist (Tiergeist)},
    description={Dein Ziel sinkt in einen tiefen Schlaf, aus dem es nur mit Gewalt geweckt werden kann. Wird es nicht vorzeitig geweckt, regeneriert es zwei zusätzliche Wunden.\newline Mächtige Magie: Regeneriert eine weitere Wunde.\newline Probenschwierigkeit: 12\newline Vorbereitungszeit: 16 Aktionen\newline Ziel: Einzelperson\newline Reichweite: Berührung\newline Wirkungsdauer: 8 Stunden\newline Kosten: 8 AsP}
}


\newglossaryentry{sanftmut(Tiergeist)_Talent}
{
    name={Sanftmut (Tiergeist)},
    description={Das verzauberte Tier verliert seine Angriffslust, solange es nicht angegriffen oder gereizt wird.\newline Mächtige Magie: Die Lethargie hält sogar dann, wenn das Tier gereizt/leicht verletzt/schwer verletzt wird.\newline Probenschwierigkeit: Magieresistenz\newline Vorbereitungszeit: 1 Aktion\newline Ziel: Tier\newline Reichweite: 8 Schritt\newline Wirkungsdauer: 4 Minuten\newline Kosten: 8 AsP}
}


\newglossaryentry{seelentiererkennen(Tiergeist)_Talent}
{
    name={Seelentier erkennen (Tiergeist)},
    description={Du erkennst das Seelentier deines Zieles. Wenn dir das Tier und seine assoziierten Eigenschaften bekannt sind, erleichtert dir das alle künftigen gesellschaftlichen Proben dem Ziel gegenüber um +2.\newline Probenschwierigkeit: Magieresistenz\newline Vorbereitungszeit: 4 Aktionen\newline Ziel: Einzelperson\newline Reichweite: 4 Schritt\newline Wirkungsdauer: augenblicklich\newline Kosten: 4 AsP}
}


\newglossaryentry{seidenzungeElfenwort(Tiergeist)_Talent}
{
    name={Seidenzunge Elfenwort (Tiergeist)},
    description={Das Ziel denkt nicht zu genau über deine Worte nach und findet dich überzeugend. Alle Überreden-Proben gegen dein Ziel sind um +4 erleichtert.\newline Mächtige Magie: Der Bonus steigt um +2.\newline Probenschwierigkeit: Magieresistenz\newline Vorbereitungszeit: 1 Aktion\newline Ziel: Einzelperson\newline Reichweite: 4 Schritt\newline Wirkungsdauer: 16 Initiativphasen\newline Kosten: 8 AsP}
}


\newglossaryentry{sensattacoMeisterstreich(Tiergeist)_Talent}
{
    name={Sensattaco Meisterstreich (Tiergeist)},
    description={Dein Ziel erkennt intuitiv die Lücken in der Verteidigung des Gegners. Seine AT sind um +2 erleichtert und seine Chance auf einen Triumph bei einer AT steigt um 1 auf dem W20 (zum Beispiel von 20 auf 19–20).\newline Mächtige Magie: Der Bonus steigt um +1.\newline Probenschwierigkeit: 12\newline Vorbereitungszeit: 1 Aktion\newline Ziel: Einzelperson\newline Reichweite: Berührung\newline Wirkungsdauer: 16 Initiativphasen\newline Kosten: 8 AsP}
}


\newglossaryentry{sensibarEmpathicus(Tiergeist)_Talent}
{
    name={Sensibar Empathicus (Tiergeist)},
    description={Du kannst die Gefühle deines Gegenübers erahnen, wodurch deine Menschenkenntnis gegen das Ziel um +4 steigt.\newline Mächtige Magie: Der Bonus steigt um +2.\newline Probenschwierigkeit: Magieresistenz\newline Vorbereitungszeit: 4 Aktionen\newline Ziel: Einzelperson\newline Reichweite: 4 Schritt\newline Wirkungsdauer: 1 Stunde\newline Kosten: 8 AsP}
}


\newglossaryentry{serpentialisSchlangenleib(Tiergeist)_Talent}
{
    name={Serpentialis Schlangenleib (Tiergeist)},
    description={Deine Arme verwandeln sich in grüne, 2 Schritt lange Giftschlangen, deren hinteres Ende noch mit deinem Rumpf verbunden ist. Die Schlangen können beide in einer Aktion Konflikt angreifen (WS 3, VT 10, RW 2, AT 10, TP 2W6) und ein Waffengift übertragen (Stufe 20, keine Verzögerung, Intervall 2 INI-Phasen, Wirkungsdauer 2 INI-Phasen, 2W6 SP).\newline Mächtige Magie: Die WS der Schlangen steigt um +1, AT, VT und TP um +2.\newline Probenschwierigkeit: 12\newline Modifikationen: Schlangengriff (Wirkungsdauer 4 Minuten, 4 AsP; die Arme verwandeln sich in kampfunfähige, aber geschickte Nattern, deren FF gleich deiner ist.)\newline Vorbereitungszeit: 2 Aktionen\newline Ziel: selbst\newline Reichweite: Berührung\newline Wirkungsdauer: 16 Initiativphasen\newline Kosten: 16 AsP}
}


\newglossaryentry{silentiumSchweigekreis(Tiergeist)_Talent}
{
    name={Silentium Schweigekreis (Tiergeist)},
    description={Der Zauber dämpft alle Geräusche in einer Kugel mit 2 Schritt Radius. Entsprechende Wahrnehmungs-Proben sind um –4 erschwert.\newline Mächtige Magie: Der Malus steigt um –2.\newline Probenschwierigkeit: 12\newline Modifikationen: Begleiter (–4; die Kugel bewegt sich mit dir.)\newline Fremdbegleiter (–8; die Kugel bewegt sich mit einem Einzelwesen.)\newline Vorbereitungszeit: 2 Aktionen\newline Ziel: Zone\newline Reichweite: 2 Schritt\newline Wirkungsdauer: 4 Minuten\newline Kosten: 4 AsP}
}


\newglossaryentry{spinnenlauf(Tiergeist)_Talent}
{
    name={Spinnenlauf (Tiergeist)},
    description={Deine Hände und Füße haften an Oberflächen, sodass du mit GS 1 an glatten Wänden und sogar Decken ohne Griffen klettern kannst. Erlaubt Aufrechterhalten.\newline Mächtige Magie: Erhöht die GS um +1.\newline Probenschwierigkeit: 12\newline Vorbereitungszeit: 4 Aktionen\newline Ziel: Einzelperson\newline Reichweite: Berührung\newline Wirkungsdauer: 4 Minuten\newline Kosten: 8 AsP}
}


\newglossaryentry{spurlosTrittlos(Tiergeist)_Talent}
{
    name={Spurlos Trittlos (Tiergeist)},
    description={Du tarnst deine Fährte mit Magie. Alle Proben zur Verfolgung deiner Fährte sind um –4 erschwert.\newline Mächtige Magie: Der Malus steigt um –2.\newline Probenschwierigkeit: 12\newline Modifikationen: Andere Person (–4, Ziel Einzelperson)\newline Zone (–4, Zone, 16 AsP; der Zauber betrifft alle in einem Radius von 4 Schritt.)\newline Vorbereitungszeit: 4 Aktionen\newline Ziel: selbst\newline Reichweite: Berührung\newline Wirkungsdauer: 1 Stunde\newline Kosten: 4 AsP}
}


\newglossaryentry{standfestKatzengleich(Tiergeist)_Talent}
{
    name={Standfest Katzengleich (Tiergeist)},
    description={Deine Geschicklichkeit erhöht sich. Alle Proben, um auf den Beinen zu bleiben, sind um +4 erleichtert und deine Patzerchance im Kampf sinkt um 1 auf dem W20.\newline Mächtige Magie: Der Bonus steigt um +2.\newline Probenschwierigkeit: 12\newline Vorbereitungszeit: 2 Aktionen\newline Ziel: selbst\newline Reichweite: Berührung\newline Wirkungsdauer: 16 Initiativphasen\newline Kosten: 4 AsP}
}


\newglossaryentry{vipernblick(Tiergeist)_Talent}
{
    name={Vipernblick (Tiergeist)},
    description={Du erscheinst deinem Opfer als grauenvolle Gestalt aus den Niederhöllen. Es kann nicht wegsehen und ist vor Angst handlungsunfähig. Wird der Sichtkontakt unterbrochen, endet der Zauber.\newline Probenschwierigkeit: Magieresistenz\newline Modifikationen: Fremdgestalt (–4; du kannst eine andere Person wählen, die als Alpgestalt erscheint.)\newline Vorbereitungszeit: 4 Aktionen\newline Ziel: Einzelwesen\newline Reichweite: 8 Schritt\newline Wirkungsdauer: 4 Minuten\newline Kosten: 16 AsP}
}


\newglossaryentry{warmesBlut(Tiergeist)_Talent}
{
    name={Warmes Blut (Tiergeist)},
    description={Durch diesen Zauber siehst du die Wärmestrahlung deiner Umgebung. Kaltes erscheint schwarz bis grünblau, Warmblüter sind gelb und Feuer ist orange bis tiefrot.\newline Probenschwierigkeit: 12\newline Vorbereitungszeit: 4 Aktionen\newline Ziel: selbst\newline Reichweite: Berührung\newline Wirkungsdauer: 1 Stunde\newline Kosten: 4 AsP}
}


\newglossaryentry{wasseratem(Tiergeist)_Talent}
{
    name={Wasseratem (Tiergeist)},
    description={Du kannst unter Wasser atmen, jedoch nicht mehr an Land. Erlaubt Aufrechterhalten.\newline Probenschwierigkeit: 12\newline Vorbereitungszeit: 16 Aktionen\newline Ziel: Einzelperson\newline Reichweite: Berührung\newline Wirkungsdauer: 1 Stunde\newline Kosten: 8 AsP}
}


\newglossaryentry{wellenlauf(Tiergeist)_Talent}
{
    name={Wellenlauf (Tiergeist)},
    description={Wasser ist für dich ein fester Untergrund. Du erleidest keine Abzüge durch ungünstige Position und wirst nicht durch Wellen und Strömungen beeinflusst. Erlaubt Aufrechterhalten.\newline Probenschwierigkeit: 12\newline Modifikationen: Sinken (–4; du kannst die Zauberwirkung nach Belieben unterdrücken und wieder aktiv werden lassen. Bei Aktivierung unter Wasser wirst du an die Wasseroberfläche gehoben.)\newline Wasserwand (–8; du kannst selbst an Wasserfällen hochklettern, wofür einfache Klettern-Proben anfallen.)\newline Vorbereitungszeit: 2 Aktionen\newline Ziel: selbst\newline Reichweite: Berührung\newline Wirkungsdauer: 4 Minuten\newline Kosten: 4 AsP}
}


\newglossaryentry{wipfellauf(Tiergeist)_Talent}
{
    name={Wipfellauf (Tiergeist)},
    description={Du kannst dich durch Baumkronen und Unterholz bewegen, als wären sie eine normale Straße. Unter diesen Bedingungen erleidest du keinerlei Erschwernisse oder Geschwindigkeitsabzüge.\newline Probenschwierigkeit: 12\newline Modifikationen: Kopfüber (–8; du kannst selbst kopfüber an Bäumen laufen.)\newline Vorbereitungszeit: 2 Aktionen\newline Ziel: selbst\newline Reichweite: Berührung\newline Wirkungsdauer: 1 Stunde\newline Kosten: 16 AsP}
}


\newglossaryentry{xenographusSchriftenkunde(Tiergeist)_Talent}
{
    name={Xenographus Schriftenkunde (Tiergeist)},
    description={Du kannst den Sinn eines geschriebenen Satzes verstehen, auch wenn du Schrift und Sprache nicht kennst.\newline Mächtige Magie: Verdoppelt die Anzahl der Sätze.\newline Probenschwierigkeit: 12\newline Vorbereitungszeit: 8 Aktionen\newline Ziel: Einzelobjekt\newline Reichweite: 4 Schritt\newline Wirkungsdauer: augenblicklich\newline Kosten: 4 AsP}
}


\newglossaryentry{zaubernahrungHungerbann(Tiergeist)_Talent}
{
    name={Zaubernahrung Hungerbann (Tiergeist)},
    description={Für einen Tag spürst du keinerlei Hunger und die Kraft des Zaubers ernährt dich. Du musst spätestens vor vier Tagen echtes Essen zu dir genommen haben.\newline Mächtige Magie: Die letzte echte Mahlzeit kann bis zu zwei Tage länger her sein.\newline Probenschwierigkeit: 12\newline Modifikationen: Durstbann (–4, 8 AsP; auch jeglicher Durst wird vom Zauber gestillt.)\newline Vorbereitungszeit: 4 Minuten\newline Ziel: selbst\newline Reichweite: Berührung\newline Wirkungsdauer: 1 Tag\newline Kosten: 4 AsP}
}


\newglossaryentry{ängstelindern(Tiergeist)_Talent}
{
    name={Ängste lindern (Tiergeist)},
    description={Auf dem Ziel lastende Furcht-Effekte sinken um eine Stufe.\newline Probenschwierigkeit: 12\newline Vorbereitungszeit: 4 Aktionen\newline Ziel: Einzelperson\newline Reichweite: Berührung\newline Wirkungsdauer: augenblicklich\newline Kosten: 4 AsP}
}


\newglossaryentry{arngrimmsHöhle_Talent}
{
    name={Arngrimms Höhle},
    description={In einem abgeschlossenen Raum wie einer Jurte oder einer kleinen Höhle steigt die Temperaturstufe (S. 35) um +1, bis maximal auf normal.\newline Mächtige Magie: Für je zwei Stufen steigt die Temperaturstufe um weitere +1.\newline Probenschwierigkeit: 12\newline Vorbereitungszeit: 4 Minuten\newline Ziel: Zone\newline Reichweite: Berührung\newline Wirkungsdauer: 1 Tag\newline Kosten: 8 AsP\newline Fertigkeiten: Feuer, Geister der Stärkung, Umwelt\newline Erlernen: Smn 14; 20 EP}
}


\newglossaryentry{aufmerksamerWächter_Talent}
{
    name={Aufmerksamer Wächter},
    description={Du malst, schnitzt oder formst das Abbild eines Tieres. Dieses Tier wird mit seinen üblichen Sinnen Wache halten. Wenn es eine Gefahr entdeckt, stößt es einen Warnruf aus (eine Illusion (Gehör)). Was das Tier als Gefahr betrachtet, hängt von der Tierart ab: Eine Katze stellt für einen Affen keine Gefahr dar, für einen Vogel aber sehr wohl – bei einem Skorpion wäre es umgekehrt.\newline Probenschwierigkeit: 12\newline Vorbereitungszeit: 4 Minuten\newline Ziel: Einzelobjekt\newline Reichweite: Berührung\newline Wirkungsdauer: 1 Tag\newline Kosten: 4 AsP\newline Fertigkeiten: Hellsicht, Illusion, Geister der Stärkung\newline Erlernen: Smn 18; 20 EP}
}


\newglossaryentry{blutsbund_Talent}
{
    name={Blutsbund},
    description={Du segnest einen Blutsbund wie etwa eine Hochzeit oder Blutsbrüderschaft. Alle Versuche, einen Teilnehmer des Blutsbunds gegen den anderen handeln zu lassen (etwa durch Überreden oder einen Imperavi), sind um –4 erschwert. Stirbt einer der Partner, so nimmt der andere dies intuitiv wahr. Der Blutsbund selbst fügt jedem Teilnehmer eine Wunde zu.\newline Probenschwierigkeit: 12\newline Vorbereitungszeit: 1 Stunde\newline Ziel: zwei Personen\newline Reichweite: Berührung\newline Wirkungsdauer: permanent\newline Kosten: 16 AsP\newline Fertigkeiten: Eigenschaften, Geister der Stärkung, Verständigung\newline Erlernen: Smn 8; 0 EP}
}


\newglossaryentry{geisterklinge_Talent}
{
    name={Geisterklinge},
    description={Die verzauberte Waffe gilt während der Wirkungsdauer als magisch.\newline Probenschwierigkeit: 12\newline Modifikationen: Personalisierung (–4; der Zauber wirkt nur, solange der beim Zaubern gewählte Träger die Waffe führt.)\newline Namenssigille (–8; erfordert den wahren Namen eines Dämons. Wann immer der Dämon Wunden durch diese Waffe erleidet, erleidet er eine zusätzliche Wunde.)\newline Permanenz (–4, Wirkungsdauer bis die Bindung gelöst wird, 16 AsP, davon 2 gAsP)\newline Vorbereitungszeit: 16 Aktionen\newline Ziel: Einzelobjekt\newline Reichweite: Berührung\newline Wirkungsdauer: 1 Woche\newline Kosten: 8 AsP\newline Fertigkeiten: Kraft, Geister der Stärkung, Verwandlung\newline Erlernen: Smn 18; 40 EP}
}


\newglossaryentry{hauchdesElements_Talent}
{
    name={Hauch des Elements},
    description={Du bindest einen elementaren Effekt in ein maximal truhengroßes Objekt. Der Effekt bemisst sich nach dem gewählten Element.\newline Eis: Senkt die Temperatur des Objekts und in seinem Inneren um 1 Stufe.\newline Erz: Verdoppelt das Gewicht des Objekts und seines Inhaltes.\newline Feuer: Erhöht die Temperatur des Objekts und in seinem Inneren um 1 Stufe.\newline Humus: Vervierfacht die Wachstumsgeschwindigkeit von Pflanzen.\newline Luft: Verringert das Gewicht des Objekts und seines Inhalts um ein Viertel.\newline Wasser: Macht einen faustgroßen Teil biegsam wie Wachs.\newline Probenschwierigkeit: 12\newline Vorbereitungszeit: 1 Stunde\newline Ziel: Einzelobjekt\newline Reichweite: Berührung\newline Wirkungsdauer: bis die Bindung gelöst wird\newline Kosten: 1 AsP, davon 1 gAsP\newline Fertigkeiten: Eis, Erz, Feuer, Humus, Luft, Wasser, Geister der Stärkung\newline Erlernen: Smn 12; 40 EP\newline Anmerkung: Manche elementaren Wirkungen sind für manche Schamanen äußerst ungewöhnlich (zum Beispiel Eis für Achaz oder Erz für Gjalsker)}
}


\newglossaryentry{heimführungderHerde_Talent}
{
    name={Heimführung der Herde},
    description={Alle Tiere in einem Radius von 4 Meilen, die dir gehören, kommen zu dir.\newline Mächtige Magie: Verdoppelt den Radius.\newline Probenschwierigkeit: 12\newline Modifikationen: Namensruf (–4; der Ruf betrifft nur ein bestimmtes Tier in 8 Meilen Radius)\newline Vorbereitungszeit: 16 Aktionen\newline Ziel: Zone\newline Reichweite: Berührung\newline Wirkungsdauer: augenblicklich\newline Kosten: 8 AsP\newline Fertigkeiten: Geister der Stärkung, Geister rufen, Verständigung\newline Erlernen: Smn 14; 20 EP}
}


\newglossaryentry{mutderAhnen_Talent}
{
    name={Mut der Ahnen},
    description={Die Ahnengeister stärken den Mut von bis zu 8 Gefährten. Sie erhalten eine Erleichterung von +2 auf alle MU-Proben und auf ihnen lastende Furcht-Effekte gelten als eine Stufe niedriger.\newline Mächtige Magie: Verdoppelt die Zahl der Ziele.\newline Probenschwierigkeit: 12\newline Vorbereitungszeit: 16 Aktionen\newline Ziel: Zone\newline Reichweite: Berührung\newline Wirkungsdauer: 4 Stunden\newline Kosten: 16 AsP\newline Fertigkeiten: Eigenschaften, Geister der Stärkung\newline Erlernen: Smn 16; 40 EP}
}


\newglossaryentry{regentanz_Talent}
{
    name={Regentanz},
    description={Innerhalb des nächsten Tages regnet es auf die Felder im Radius von 1 Meile und das Saatgut kann wachsen.\newline Mächtige Magie: Verdoppelt den Radius.\newline Probenschwierigkeit: 12\newline Vorbereitungszeit: 1 Stunde\newline Ziel: Zone\newline Reichweite: Berührung\newline Wirkungsdauer: 1 Tag\newline Kosten: 16 AsP\newline Fertigkeiten: Luft, Umwelt, Geister der Stärkung, Wasser\newline Erlernen: Smn 14; 20 EP}
}


\newglossaryentry{reinesWasser_Talent}
{
    name={Reines Wasser},
    description={Du verwandelst 100 Liter Salzwasser in Trinkwasser.\newline Mächtige Magie: Du verwandelst 200/300/400/500 Liter.\newline Probenschwierigkeit: 12\newline Vorbereitungszeit: 4 Minuten\newline Ziel: Einzelobjekt\newline Reichweite: Berührung\newline Wirkungsdauer: augenblicklich\newline Kosten: 16 AsP\newline Fertigkeiten: Geister der Stärkung, Geister vertreiben, Verwandlung, Wasser\newline Erlernen: Smn 18; 20 EP}
}


\newglossaryentry{reitenderGeist_Talent}
{
    name={Reitender Geist},
    description={Das bezauberte Tier wird dir gegenüber brav und anhänglich. Proben im Umgang mit dem Tier sind um +4 erleichtert, Freie Fertigkeiten (wie Abrichter) gelten als eine Stufe höher.\newline Mächtige Magie: Erhöht den Bonus um +2, je zwei Stufen erhöhen Freie Fertigkeiten um eine weitere Stufe.\newline Probenschwierigkeit: 12\newline Modifikationen: Anderer Reiter (–4; die Wirkung bezieht sich auf eine andere Person als dich)\newline Vorbereitungszeit: 1 Stunde\newline Ziel: Tier\newline Reichweite: Berührung\newline Wirkungsdauer: bis die Bindung gelöst wird\newline Kosten: 8 AsP, davon 2 gAsP\newline Fertigkeiten: Einfluss, Geister der Stärkung\newline Erlernen: Smn 14; 40 EP\newline Anmerkung: Besonders große und starke Tierarten erhöhen die gAsP auf 4 oder sogar 8.}
}


\newglossaryentry{schutzderJurte_Talent}
{
    name={Schutz der Jurte},
    description={Du schützt einen abgeschlossenen Raum, wie eine Jurte oder einer Höhle, mit bis zu 16 Schritt Radius vor übernatürlichen Wesen. Dämonen, Daimonoiden und Untote können den Raum nicht betreten, wenn ihre Beschwörungsschwierigkeit bei maximal 20 liegt, bei Geistern und Feenwesen entscheidet der Spielleiter. Innerhalb des Raumes erleiden alle genannten Wesenheiten 1 Wunde pro Initiativephase. Wenn ein beweglicher Raum wie eine Jurte verzaubert wird, wirkt der Zauber unabhängig vom Ort, an dem die Jurte steht.\newline Mächtige Magie: Die maximale Beschwörungsschwierigkeit steigt um +4.\newline Probenschwierigkeit: 12\newline Modifikationen: Permanenz (–4, Wirkungsdauer bis die Bindung gelöst wird, 16 AsP, davon 2 gAsP)\newline Vorbereitungszeit: 1 Stunde\newline Ziel: Zone\newline Reichweite: Berührung\newline Wirkungsdauer: 1 Woche\newline Kosten: 16 AsP\newline Fertigkeiten: Antimagie, Geister der Stärkung\newline Geister vertreiben\newline Erlernen: Smn 14; 40 EP}
}


\newglossaryentry{schützendeRotte_Talent}
{
    name={Schützende Rotte},
    description={Du bemalst einen Schild mit Tieren. Proben zur Abwehr von ballistischen Zaubern (siehe S. 124) sind mit diesem Schild um +4 erleichtert. Der Schild erleidet dabei keinen Schaden.\newline Mächtige Magie: Der Bonus steigt um +2.\newline Probenschwierigkeit: 12\newline Vorbereitungszeit: 16 Aktionen\newline Ziel: Einzelobjekt\newline Reichweite: Berührung\newline Wirkungsdauer: bis die Bindung gelöst wird\newline Kosten: 4 AsP, davon 1 gAsP\newline Fertigkeiten: Antimagie, Geister der Stärkung, Geister vertreiben, Kraft\newline Erlernen: Smn 18; 20 EP}
}


\newglossaryentry{schutzgeist(Tier)_Talent}
{
    name={Schutzgeist (Tier)},
    description={Du bittest den Geist eines Tieres, über eine Gruppe von bis zu 8 Gefährten zu wachen. Sie erhalten einen Bonus von +1 auf alle beim entsprechenden Tier angegebenen Werte, solange sie sich nicht weiter als 1 Meile von dir entfernen. Es kann nur ein Schutzgeist gleichzeitig aktiv sein.\newline Mächtige Magie: Verdoppelt die Zahl der Ziele, je zwei Stufen erhöhen den Bonus um +1.\newline Probenschwierigkeit: 12\newline Vorbereitungszeit: 1 Stunde\newline Ziel: Zone\newline Reichweite: Berührung\newline Wirkungsdauer: bis die Bindung gelöst wird\newline Kosten: 16 AsP, davon 4 gAsP\newline Fertigkeiten: Eigenschaften, Geister der Stärkung\newline Erlernen: Smn 14; 40 EP\newline Anmerkung: Der Zauber muss für jedes Tier extra erlernt werden.\newline Sephrasto: Trage die gewählten Tiere in das Kommentarfeld ein. Wenn du mehrere Tiere wählst, dann erhöhe die EP-Kosten entsprechend. Die Tiertabelle ist in Sephrasto außerdem über die kostenlosen Tiergeist-Vorteile abgebildet; wenn du diese auswählst, erhältst du die Werte des Tiergeists im Regelanhang.}
}


\newglossaryentry{segenderErdgeister_Talent}
{
    name={Segen der Erdgeister},
    description={Du leitest das Wachstum einer Pflanze, kannst es beschleunigen und formen. Die Kosten, Zauberdauer und Modifikationen Mächtige Magie sind Meisterentscheid. Beispiele:\newline 4 AsP, keine Mächtige Magie, 1 Minute: Aus einem Samen wächst eine kleine Heilpflanze.\newline 8 AsP, 1x Mächtige Magie, eine halbe Stunde: Ein Baum trägt drei Monate zu früh Früchte.\newline 4 AsP, 2x Mächtige Magie, 4 Minuten: Die Rinde eines Baumes zeigt eine Botschaft;\newline 128 AsP, 4x Mächtige Magie, 4 Tage: Eine Baumkrone wird zu einem Baumhaus mit Möbeln, Dach und Fenstern\newline Probenschwierigkeit: 12\newline Vorbereitungszeit: nach Projekt\newline Ziel: Pflanze\newline Reichweite: Berührung\newline Wirkungsdauer: augenblicklich\newline Kosten: nach Projekt\newline Fertigkeiten: Humus, Geister der Stärkung, Verwandlung\newline Erlernen: Smn 16; 60 EP}
}


\newglossaryentry{tauschplatz_Talent}
{
    name={Tauschplatz},
    description={In einer Zone von 16 Schritt Radius sind Proben auf Einschüchtern oder Überreden um –4 erschwert. Davon sind all jene ausgenommen, die während der gesamten Vorbereitungszeit in der Zone waren.\newline Mächtige Magie: Verdoppelt den Radius und die Proben sind um weitere –2 erschwert.\newline Probenschwierigkeit: 12\newline Vorbereitungszeit: 1 Tag\newline Ziel: Zone\newline Reichweite: Berührung\newline Wirkungsdauer: bis die Bindung gelöst wird\newline Kosten: 16 AsP, davon 1 gAsP\newline Fertigkeiten: Einfluss, Geister der Stärkung\newline Erlernen: Smn 18; 20 EP}
}


\newglossaryentry{wegdesWindes_Talent}
{
    name={Weg des Windes},
    description={Verdoppelt das DH* (S. 34) deines Ziels und das Intervall, in dem körperliche Anstrengung Erschöpfung verursacht. Proben, um natürliche Hindernisse zu überwinden, sind um +4 erleichtert.\newline Mächtige Magie: Der Bonus steigt um +2 und je zwei Stufen verdreifachen/vervierfachen das DH* und das Intervall.\newline Probenschwierigkeit: 12\newline Vorbereitungszeit: 1 Stunde\newline Ziel: Einzelperson\newline Reichweite: Berührung\newline Wirkungsdauer: 1 Tag\newline Kosten: 8 AsP\newline Fertigkeiten: Eigenschaften, Geister der Stärkung\newline Erlernen: Smn 16; 20 EP}
}


\newglossaryentry{wegzeichen_Talent}
{
    name={Wegzeichen},
    description={Du vergräbst einen Talisman. Während der Wirkungsdauer kannst unter Aufwendung von 1 AsP erfahren, wo deine Wegzeichen liegen und diese intuitiv unterscheiden.\newline Probenschwierigkeit: 12\newline Modifikationen: Permanenz (–4, Wirkungsdauer bis die Bindung gelöst wird, Kosten 8 AsP, davon 1 gAsP)\newline Sprechendes Zeichen (–4; das Zeichen kann einen kurzen Satz als Botschaft enthalten, der mit einem AG von 2 (S. 80) gelesen werden kann)\newline Vorbereitungszeit: 16 Aktionen\newline Ziel: Einzelobjekt\newline Reichweite: Berührung\newline Wirkungsdauer: 1 Tag\newline Kosten: 4 AsP\newline Fertigkeiten: Hellsicht, Geister der Stärkung, Geister rufen, Verständigung\newline Erlernen: Smn 14; 20 EP}
}


\newglossaryentry{weidegründeFinden_Talent}
{
    name={Weidegründe Finden},
    description={Die Geister helfen dir auf dem Weg zu den nächsten Weidegründen. Entsprechende Überleben-Proben sind um +4 erleichtert.\newline Mächtige Magie: Der Bonus steigt um +2.\newline Probenschwierigkeit: 12\newline Vorbereitungszeit: 1 Stunde\newline Ziel: selbst\newline Reichweite: Berührung\newline Wirkungsdauer: 1 Tag\newline Kosten: 4 AsP\newline Fertigkeiten: Eigenschaften, Hellsicht, Geister der Stärkung, Geister rufen\newline Erlernen: Smn 12; 0 EP}
}


\newglossaryentry{wildFinden_Talent}
{
    name={Wild Finden},
    description={Du erfährst, ob und wo sich im Radius von 1 Meile jagdbares Wild befindet und um welche Tiere es sich handelt. Sind jagdbare Tiere in der Nähe, können Jäger gezielt in die richtige Richtung losziehen. Das senkt die Schwierigkeit der Jagd (S. 68) auf 12, so als ob sie sich in wildreichem Gebiet befänden.\newline Mächtige Magie: Verdoppelt den Radius.\newline Probenschwierigkeit: 12\newline Modifikationen: Gaben der Erde (die Wirkung bezieht sich auf nahrhafte Pflanzen und verringert die Schwierigkeit von Überleben-Proben)\newline Vorbereitungszeit: 1 Stunde\newline Ziel: Zone\newline Reichweite: Berührung\newline Wirkungsdauer: augenblicklich\newline Kosten: 8 AsP\newline Fertigkeiten: Eigenschaften, Hellsicht, Geister der Stärkung, Geister rufen\newline Erlernen: Smn 16; 20 EP}
}


\newglossaryentry{brazoraghsHieb_Talent}
{
    name={Brazoraghs Hieb},
    description={Die so verzauberte Nahkampfwaffe gilt als magisch und verursacht Niederwerfen.\newline Mächtige Magie: Waffenschaden +1.\newline Probenschwierigkeit: 12\newline Vorbereitungszeit: 1 Stunde\newline Ziel: Einzelobjekt\newline Reichweite: Berührung\newline Wirkungsdauer: bis die Bindung gelöst wird\newline Kosten: 16 AsP, davon 4 gAsP\newline Fertigkeiten: Eigenschaften, Geister des Zorns\newline Erlernen: Smn 20; 40 EP}
}


\newglossaryentry{fluchderVerwirrung_Talent}
{
    name={Fluch der Verwirrung},
    description={Der Fluch verwirrt alle Humanoide in einem Radius von 2 Schritt, denen keine MR-Gegenprobe (16) gelingt. Sie erleiden eine Erschwernis von –2 auf Proben.\newline Mächtige Magie: Erhöht den Malus um –1 und verdoppelt den Radius.\newline Probenschwierigkeit: 12\newline Vorbereitungszeit: 4 Aktionen\newline Ziel: Zone\newline Reichweite: 16 Schritt\newline Wirkungsdauer: 8 Initiativphasen\newline Kosten: 16 AsP\newline Fertigkeiten: Dämonisch, Einfluss, Geister des Zorns\newline Erlernen: Smn 18; 40 EP}
}


\newglossaryentry{fluchderWandlung_Talent}
{
    name={Fluch der Wandlung},
    description={Das Ziel verwandelt sich im Laufe einer Woche in ein Tier, typischerweise in eine Schlange (bei manchen Völkern sind auch andere Tiere verbreitet). Mit Abschluss der Verwandlung verliert es die Erinnerung an sein früheres Leben. In der letzten Woche der Wirkungsdauer verwandelt sich das Ziel in seine ursprüngliche Gestalt zurück und gewinnt dabei auch sein Gedächtnis wieder.\newline Probenschwierigkeit: Magieresistenz\newline Modifikationen: Permanenz (–4, Wirkungsdauer bis die Bindung gelöst wird, 16 AsP, davon 2 gAsP)\newline Vorbereitungszeit: 1 Stunde\newline Ziel: Einzelperson\newline Reichweite: 4 Meilen\newline Wirkungsdauer: 1 Monat\newline Kosten: 16 AsP\newline Fertigkeiten: Verwandlung, Geister des Zorns\newline Erlernen: Smn 16; 40 EP}
}


\newglossaryentry{fluchdes(Tieres)_Talent}
{
    name={Fluch des (Tieres)},
    description={Ein zorniger Tiergeist straft das Ziel. Es erleidet einen Malus von –2 auf alle Werte, die dem Tier zugeordnet sind (S. 160).\newline Mächtige Magie: Der Malus steigt um –1.\newline Probenschwierigkeit: Magieresistenz\newline Modifikationen: Permanenz (–4, Wirkungsdauer bis die Bindung gelöst wird, 16 AsP, davon 2 gAsP)\newline Vorbereitungszeit: 16 Aktionen\newline Ziel: Einzelperson\newline Reichweite: 32 Schritt\newline Wirkungsdauer: 1 Tag\newline Kosten: 8 AsP\newline Fertigkeiten: Einfluss, Geister des Zorns\newline Erlernen: Smn 16; 20 EP\newline Anmerkung: Der Zauber muss für jedes Tier extra erlernt werden.\newline Sephrasto: Trage die gewählten Tiere in das Kommentarfeld ein. Wenn du mehrere Tiere wählst, dann erhöhe die EP-Kosten entsprechend. Die Tiertabelle ist in Sephrasto außerdem über die kostenlosen Tiergeist-Vorteile abgebildet; wenn du diese auswählst, erhältst du die Werte des Tiergeists im Regelanhang.}
}


\newglossaryentry{fluchdesGewürms_Talent}
{
    name={Fluch des Gewürms},
    description={Dein Opfer wird von einer Myriade Insekten und Kleintieren bedeckt. Misslingt eine Konterprobe (Willenskraft, 16), ist es handlungsunfähig. Gelingt die Konterprobe, sind alle Proben um –4 erschwert.\newline Probenschwierigkeit: 12\newline Vorbereitungszeit: 8 Aktionen\newline Ziel: Einzelperson\newline Reichweite: 16 Schritt\newline Wirkungsdauer: 16 Initiativphasen\newline Kosten: 16 AsP\newline Fertigkeiten: Dämonisch, Verständigung, Geister des Zorns\newline Erlernen: Smn 18; 40 EP}
}


\newglossaryentry{fluchdesSiechtums_Talent}
{
    name={Fluch des Siechtums},
    description={Du infizierst dein Opfer mit einer dir bekannten Krankheit deiner Wahl, die dann ihren natürlichen Krankheitsverlauf nimmt. Die maximale Krankheitsstufe ist 16.\newline Mächtige Magie: Erhöht die maximale Krankheitsstufe um +4.\newline Probenschwierigkeit: Magieresistenz\newline Modifikationen: Einzelfall (–4; die Krankheit ist nicht ansteckend.)\newline Vorbereitungszeit: 4 Aktionen\newline Ziel: Einzelperson\newline Reichweite: 4 Schritt\newline Wirkungsdauer: augenblicklich\newline Kosten: 16 AsP\newline Fertigkeiten: Dämonisch, Geister des Zorns\newline Erlernen: Smn 18; 40 EP}
}


\newglossaryentry{fluchdesUnglücks_Talent}
{
    name={Fluch des Unglücks},
    description={Das Opfer wird vom Unglück verfolgt. Seine Chance auf einen Patzer steigt um 1 auf dem W20 (zum Beispiel von 1 auf 1–2).\newline Mächtige Magie: Je zwei Stufen steigern die Chance auf einen Patzer um einen weiteren Punkt.\newline Probenschwierigkeit: Magieresistenz\newline Modifikationen: Permanenz (–4, Wirkungsdauer bis die Bindung gelöst wird, 16 AsP, davon 2 gAsP)\newline Vorbereitungszeit: 16 Aktionen\newline Ziel: Einzelperson\newline Reichweite: 32 Schritt\newline Wirkungsdauer: 1 Tag\newline Kosten: 8 AsP\newline Fertigkeiten: Eigenschaften, Geister des Zorns\newline Erlernen: Smn 16; 20 EP}
}


\newglossaryentry{machtdesBlutes_Talent}
{
    name={Macht des Blutes},
    description={Dein Ziel muss eine körperliche Handlung deiner Wahl ausführen. Widerspricht der Befehl dem Selbsterhaltungstrieb des Zieles, kann es mit einer Konterprobe (Willenskraft, 16) widerstehen. Du fügst dir beim Wirken dieses Rituals eine Wunde zu.\newline Probenschwierigkeit: Magieresistenz\newline Vorbereitungszeit: 4 Aktionen\newline Ziel: Einzelperson\newline Reichweite: 8 Schritt\newline Wirkungsdauer: 8 Initiativphasen\newline Kosten: 4 AsP\newline Fertigkeiten: Einfluss, Geister des Zorns\newline Erlernen: Smn 20; 40 EP}
}


\newglossaryentry{machtderUngeformten_Talent}
{
    name={Macht der Ungeformten},
    description={Ruft einen Dämon herbei (mehr zur Dämonenbeschwörung siehe S. 81), der in deiner unmittelbaren Nähe erscheint.\newline Probenschwierigkeit: nach Dämon\newline Vorbereitungszeit: frei wählbar\newline Ziel: einzelner Dämon\newline Wirkungsdauer: augenblicklich\newline Kosten: nach Dämon\newline Fertigkeiten: Dämonisch, Geister rufen, Geister des Zorns\newline Erlernen: Smn 18; 120 EP}
}


\newglossaryentry{rikaisFluch_Talent}
{
    name={Rikais Fluch},
    description={Der Zauber schwächt Pflanzen und Gegenstände aus dem Element Humus in einem Radius von 2 Schritt. Pflanzen und verderbliches Material verrotten, die Härte hölzerner oder lederner Objekte sinkt um ein Viertel der Basishärte. Die Auswirkungen auf Lebewesen sind zu gering, um sie zu bemerken.\newline Mächtige Magie: Verdoppelt den Radius.\newline Probenschwierigkeit: 12\newline Vorbereitungszeit: 8 Aktionen\newline Ziel: Zone\newline Reichweite: 16 Schritt\newline Wirkungsdauer: augenblicklich\newline Kosten: 8 AsP\newline Fertigkeiten: Dämonisch, Geister des Zorns, Umwelt\newline Erlernen: Smn 18; 20 EP}
}


\newglossaryentry{tabu_Talent}
{
    name={Tabu},
    description={Um die verzauberte Zone von 16 Schritt Radius zu betreten, ist eine Konterprobe (Willenskraft, 12) nötig. Davon sind all jene ausgenommen, die während der gesamten Vorbereitungszeit in der Zone waren.\newline Mächtige Magie: Verdoppelt dens Radius.\newline Probenschwierigkeit: 12\newline Vorbereitungszeit: 4 Minuten\newline Ziel: Zone\newline Reichweite: Berührung\newline Wirkungsdauer: bis die Bindung gelöst wirdKosten: 16 AsP, davon 1 gAsP\newline Fertigkeiten: Einfluss, Geister vertreiben, Geister des Zorns\newline Erlernen: Smn 16; 20 EP}
}


\newglossaryentry{tairachsKrieger_Talent}
{
    name={Tairachs Krieger},
    description={Du erhebst eine Leiche als Untoten (mehr zu Beschwörungen siehe S. 81), der in zwei INI-Phasen einsatzfähig ist. Entspricht dem Skelettarius.\newline Probenschwierigkeit: nach Untotem\newline Modifikationen: Schnelle Erhebung (–4; der Untote ist sofort bereit.)\newline Vorbereitungszeit: frei wählbar\newline Ziel: Material für einen einzelnen Untoten\newline Wirkungsdauer: augenblicklich\newline Kosten: nach Untotem\newline Fertigkeiten: Dämonisch, Geister rufen, Geister des Zorns\newline Erlernen: Smn 18; 40 EP}
}


\newglossaryentry{tairachsSklaven_Talent}
{
    name={Tairachs Sklaven},
    description={Dein Geist fährt in den frischen Leichnam eines Lebewesens ein, der sich als von dir gesteuerter Untoter erhebt. Dabei sind keine Beschwörungs- oder Beherrschungsproben notwendig. Dein Körper bleibt währenddessen reglos zurück; sollte er vernichtet werden, endet der Zauber und du bist tot.\newline Probenschwierigkeit: 12\newline Vorbereitungszeit: 16 Aktionen\newline Ziel: Einzelobjekt\newline Reichweite: Berührung\newline Wirkungsdauer: 1 Stunde\newline Kosten: 8 AsP\newline Fertigkeiten: Verständigung, Geister des Zorns\newline Erlernen: Smn 20; 40 EP}
}


\newglossaryentry{blickinLiskasAugen_Talent}
{
    name={Blick in Liskas Augen},
    description={Du erhältst den Vorteil Prophezeien (S. 30) und erleidest keine Erschöpfung durch den Einsatz dieser Gabe. Besitzt du die Gabe bereits, sind entsprechende Proben um +4 erleichtert. Notwendige Proben kannst du auf Geister rufen statt auf Willenskraft ablegen.\newline Probenschwierigkeit: 12\newline Vorbereitungszeit: 1 Stunde\newline Ziel: selbst\newline Reichweite: Berührung\newline Wirkungsdauer: 1 Stunde\newline Kosten: 8 AsP\newline Fertigkeiten: Hellsicht, Geister rufen\newline Erlernen: Smn 12; 20 EP}
}


\newglossaryentry{blickinsGeisterreich_Talent}
{
    name={Blick ins Geisterreich},
    description={Du wirfst einen Blick in die Welt der Magie. Während nichtmagische Gegenstände nur schwer zu erkennen sind, erscheint Magie als Ansammlung pulsierender Kraftfäden (Analysegrad 1 der Intensitätsanalyse, S. 80). Während der Wirkungsdauer kannst du deine Sinnenschärfe zur Strukturanalyse (S. 80) verwenden, wodurch du 1 Punkt Erschöpfung erleidest. Außerdem kannst du dich im Limbus orientieren.\newline Mächtige Magie: Der Analysegrad der Intensitätsanalyse steigt um +1.\newline Probenschwierigkeit: 12\newline Vorbereitungszeit: 8 Aktionen\newline Ziel: selbst\newline Reichweite: Berührung\newline Wirkungsdauer: 16 Minuten\newline Kosten: 8 AsP\newline Fertigkeiten: Hellsicht, Geister rufen, Kraft\newline Erlernen: Smn 8; 60 EP}
}


\newglossaryentry{geisterbote_Talent}
{
    name={Geisterbote},
    description={Durch das Geisterreich schließt du eine Verbindung mit dem Ziel, von dem du einen Körperteil (z.B. ein Haar oder einen Blutstropfen) besitzt. Der Körperteil wird dabei verbraucht. Während der Wirkungsdauer kannst einmal am Tag einen beliebigen Zauber auf das Ziel sprechen, als könntest du es berühren. Die Zauber müssen wie üblich die Magieresistenz überwinden.\newline Probenschwierigkeit: 12\newline Vorbereitungszeit: 1 Stunde\newline Ziel: Einzelperson\newline Reichweite: aventurienweit\newline Wirkungsdauer: 1 Woche\newline Kosten: 16 AsP\newline Fertigkeiten: Geister rufen, Verständigung\newline Erlernen: Smn 14; 60 EP}
}


\newglossaryentry{machtdesErzes_Talent}
{
    name={Macht des Erzes},
    description={Ruft ein Elementarwesen des jeweiligen Elements herbei (mehr zu Herbeirufungen S. 81), das in deiner unmittelbaren Nähe erscheint.\newline Probenschwierigkeit: 16/24/32 (Diener/Dschinn/Meister)\newline Vorbereitungszeit: frei wählbar\newline Ziel: einzelnes Elementar\newline Wirkungsdauer: augenblicklich\newline Kosten: 16/32/64 AsP (Diener/Dschinn/Meister)\newline Fertigkeiten: Erz, Geister rufen\newline Erlernen: Smn 18; 60 EP}
}


\newglossaryentry{machtdesFeuers_Talent}
{
    name={Macht des Feuers},
    description={Ruft ein Elementarwesen des jeweiligen Elements herbei (mehr zu Herbeirufungen S. 81), das in deiner unmittelbaren Nähe erscheint.\newline Probenschwierigkeit: 16/24/32 (Diener/Dschinn/Meister)\newline Vorbereitungszeit: frei wählbar\newline Ziel: einzelnes Elementar\newline Wirkungsdauer: augenblicklich\newline Kosten: 16/32/64 AsP (Diener/Dschinn/Meister)\newline Fertigkeiten: Feuer, Geister rufen\newline Erlernen: Smn 18; 60 EP}
}


\newglossaryentry{machtdesHumus_Talent}
{
    name={Macht des Humus},
    description={Ruft ein Elementarwesen des jeweiligen Elements herbei (mehr zu Herbeirufungen S. 81), das in deiner unmittelbaren Nähe erscheint.\newline Probenschwierigkeit: 16/24/32 (Diener/Dschinn/Meister)\newline Vorbereitungszeit: frei wählbar\newline Ziel: einzelnes Elementar\newline Wirkungsdauer: augenblicklich\newline Kosten: 16/32/64 AsP (Diener/Dschinn/Meister)\newline Fertigkeiten: Humus, Geister rufen\newline Erlernen: Smn 16; 60 EP}
}


\newglossaryentry{machtderLuft_Talent}
{
    name={Macht der Luft},
    description={Ruft ein Elementarwesen des jeweiligen Elements herbei (mehr zu Herbeirufungen S. 81), das in deiner unmittelbaren Nähe erscheint.\newline Probenschwierigkeit: 16/24/32 (Diener/Dschinn/Meister)\newline Vorbereitungszeit: frei wählbar\newline Ziel: einzelnes Elementar\newline Wirkungsdauer: augenblicklich\newline Kosten: 16/32/64 AsP (Diener/Dschinn/Meister)\newline Fertigkeiten: Geister rufen, Luft\newline Erlernen: Smn 16; 60 EP}
}


\newglossaryentry{machtdesWassers_Talent}
{
    name={Macht des Wassers},
    description={Ruft ein Elementarwesen des jeweiligen Elements herbei (mehr zu Herbeirufungen S. 81), das in deiner unmittelbaren Nähe erscheint.\newline Probenschwierigkeit: 16/24/32 (Diener/Dschinn/Meister)\newline Vorbereitungszeit: frei wählbar\newline Ziel: einzelnes Elementar\newline Wirkungsdauer: augenblicklich\newline Kosten: 16/32/64 AsP (Diener/Dschinn/Meister)\newline Fertigkeiten: Geister rufen, Wasser\newline Erlernen: Smn 16; 60 EP}
}


\newglossaryentry{hilferuf_Talent}
{
    name={Hilferuf},
    description={Du rufst ein Feenwesen, einen Totengeist oder einige Mindergeister aus einem Radius von 1 Meile herbei. Das Wesen eilt zu dir und du kannst es um Hilfe bei einem unmittelbaren Problem bitten. Meistens hilft das Wesen, es kann aber auch einen Gegengefallen fordern oder (selten) die Hilfe ablehnen.\newline Mächtige Magie: Verdoppelt den Radius.\newline Probenschwierigkeit: 12\newline Modifikationen: Art (–4, du kannst dir die Art des helfenden Geists aussuchen)\newline Wesen (–4; du rufst ein dir bereits bekanntes Wesen aus bis zu 8 Meilen Radius herbei)\newline Vorbereitungszeit: 4 Aktionen\newline Ziel: Zone\newline Reichweite: Berührung\newline Wirkungsdauer: augenblicklich; der Geist erscheint nach 2W6 Initiativphasen\newline Kosten: 8 AsP\newline Fertigkeiten: Geister rufen, Verständigung\newline Erlernen: Smn 14; 60 EP}
}


\newglossaryentry{geistertausch_Talent}
{
    name={Geistertausch},
    description={Wenn du ein Körperteil (z.B. Haare oder Blut) deines Opfers besitzt, kannst du mit ihm den Körper tauschen. Geistige Attribute, Fertigkeiten und Vorteile bleiben erhalten, körperliche werden getauscht. Wird einer der Körper während des Tausches vernichtet, endet der Tausch und der ursprüngliche Besitzer des vernichteten Körpers stirbt.\newline Probenschwierigkeit: Magieresistenz\newline Modifikationen: Tiersinne (–4, Einzelwesen) Fremdtausch (–8; zwei Ziele tauschen Körper.)\newline Vorbereitungszeit: 16 Aktionen\newline Ziel: Einzelperson\newline Reichweite: 8 Meilen\newline Wirkungsdauer: 4 Minuten\newline Kosten: 16 AsP\newline Fertigkeiten: Eigenschaften, Geister rufen, Verständigung\newline Erlernen: Smn 18; 60 EP}
}


\newglossaryentry{gesangderWölfe_Talent}
{
    name={Gesang der Wölfe},
    description={Du stößt ein lautes Wolfsgeheul aus, das im Radius von 16 Meilen zu hören ist und von allen Kennern der Wolfssprache (z.B. Wölfe, Wolfskinder oder Schamanen) verstanden werden kann.\newline Mächtige Magie: Verdoppelt den Radius.\newline Probenschwierigkeit: 12\newline Modifikationen: Gespräch der Wölfe (–4, 8 AsP, Wirkungsdauer 1 Stunde; Du kannst während der Wirkungsdauer beliebig viele Nachrichten aussenden.)\newline Vorbereitungszeit: 4 Aktionen\newline Ziel: Zone\newline Reichweite: Berührung\newline Wirkungsdauer: 16 Initiativphasen\newline Kosten: 2 AsP\newline Fertigkeiten: Geister rufen, Verständigung\newline Erlernen: Smn 18; 20 EP\newline Anmerkung: Vereinzelt soll dieser Zauber auch in anderen Tiervarianten verbreitet sein.}
}


\newglossaryentry{lockruf(Wesen)_Talent}
{
    name={Lockruf (Wesen)},
    description={Du rufst das beim Erlernen gewählte Wesen, meist ein Tier, herbei. Befindet sich ein Wesen dieser Art in einem Radius von 4 Meilen, eilt es herbei. Du kannst dieses Wesen um einen Gefallen bitten, das dieses wenn möglich erfüllen wird. Dann trollt sich das Wesen.\newline Mächtige Magie: Verdoppelt den Radius und die Zahl der herbeieilenden Wesen.\newline Probenschwierigkeit: 12\newline Vorbereitungszeit: 16 Aktionen\newline Ziel: Zone\newline Reichweite: Berührung\newline Wirkungsdauer: 1 Stunde\newline Kosten: 16 AsP\newline Fertigkeiten: Einfluss, Geister rufen, Verständigung\newline Erlernen: Smn 14; 20 EP\newline Anmerkung: Der Zauber muss für jedes Wesen extra gelernt werden. Übliche Wesen sind Wölfe (Nivesen), Wildschweine (Goblins), Rinder (Ferkinas), Schlinger (Achaz), Seeschlangen (Achaz), Haie (Tocamujac), Oger (Ork), Mammuts (Gjalsker).\newline Sephrasto: Trage die gewählten Wesen in das Kommentarfeld ein. Wenn du mehrere Wesen wählst, dann erhöhe die EP-Kosten entsprechend.}
}


\newglossaryentry{ratderAhnen_Talent}
{
    name={Rat der Ahnen},
    description={Du rufst einen Totengeist in den Körper eines seiner Nachfahren herab. Dort verhält er sich seinem Charakter gemäß; er kann Rat geben oder die Stammeskrieger anführen. Es ist nicht möglich, die Bindung zu lösen. Der Totengeist entscheidet selbst, wann er den Körper verlässt – notfalls ist ein Großer Geisterbann (S. 166) nötig. Der Tote muss im letzten Jahr verstorben sein.\newline Mächtige Magie: Der Tote kann seit 10 Jahren/100 Jahren/1000 Jahren/seit Menschengedenken verstorben sein.\newline Probenschwierigkeit: 12\newline Vorbereitungszeit: 1 Stunde\newline Ziel: Einzelperson\newline Reichweite: Berührung\newline Wirkungsdauer: augenblicklich\newline Kosten: 4 AsP, davon 2 gAsP\newline Fertigkeiten: Geister rufen, Verständigung\newline Erlernen: Smn 12; 20 EP}
}


\newglossaryentry{raubderGeisterkraft_Talent}
{
    name={Raub der Geisterkraft},
    description={Du kannst bei deinem nächsten Zauber teilweise oder ganz auf die AsP des Ziels zugreifen. Der nächste Zauber kann nicht auf das Ziel des Raubs der Geisterkraft gewirkt werden.\newline Probenschwierigkeit: Magieresistenz\newline Vorbereitungszeit: 16 Aktionen\newline Ziel: Einzelperson\newline Reichweite: Berührung\newline Wirkungsdauer: 4 Stunden\newline Kosten: 4 AsP\newline Fertigkeiten: Geister rufen, Kraft, Verständigung\newline Erlernen: Smn 18; 40 EP}
}


\newglossaryentry{rufdesSchamanen_Talent}
{
    name={Ruf des Schamanen},
    description={Du sendest ein geistiges Signal an alle Stammesmitglieder im Radius von 4 Meilen, dass sie zu dir kommen sollen. Sie wissen, in welcher Richtung du dich befindest.\newline Mächtige Magie: Verdoppelt den Radius\newline Probenschwierigkeit: 12\newline Modifikationen: Ruf der Gefährten (–8; du rufst alle engen Gefährten zu dir)\newline Vorbereitungszeit: 16 Aktionen\newline Ziel: Zone\newline Reichweite: Berührung\newline Wirkungsdauer: augenblicklich\newline Kosten: 4 AsP\newline Fertigkeiten: Geister rufen, Verständigung\newline Erlernen: Smn 14; 40 EP}
}


\newglossaryentry{stimmedesNipakau_Talent}
{
    name={Stimme des Nipakau},
    description={Du erforschst die Erinnerung eines Gegenstandes, einer Pflanze oder eines Tieres. Dabei beantwortet dir das Ziel 4 Ja/Nein-Fragen. Beachte, dass die meisten Gegenstände und Pflanzen nicht gerade intelligent sind und meist nur Dinge wahrnehmen, mit denen sie direkt in Kontakt stehen. Sollte das Ziel tatsächlich durch eine intelligente Wesenheit beseelt sein, beantwortet diese Wesenheit die Fragen.\newline Mächtige Magie: Beantwortet 2 weitere Fragen.\newline Probenschwierigkeit: 12/Magieresistenz (bei Tieren)\newline Vorbereitungszeit: 16 Aktionen\newline Ziel: Einzelobjekt/Tier\newline Reichweite: Berührung\newline Wirkungsdauer: 4 Minuten\newline Kosten: 4 AsP\newline Fertigkeiten: Hellsicht, Geister rufen, Verständigung\newline Erlernen: Smn 16; 40 EP}
}


\newglossaryentry{tierischerHelfer_Talent}
{
    name={Tierischer Helfer},
    description={Du siehst die Gedanken des Tiers als verschwommene Bilder.\newline Mächtige Magie: Zusätzlich kannst du das Tier um einen kleineren/größeren/gefährlichen Gefallen bitten. Ob das Tier den Gefallen ausführen kann und will, ist Spielleiterentscheid.\newline Probenschwierigkeit: Magieresistenz\newline Vorbereitungszeit: 4 Aktionen\newline Ziel: Tier\newline Reichweite: 8 Schritt\newline Wirkungsdauer: 4 Minuten\newline Kosten: 8 AsP\newline Fertigkeiten: Hellsicht, Geister rufen, Verständigung\newline Erlernen: Smn 18; 40 EP}
}


\newglossaryentry{geisterbann_Talent}
{
    name={Geisterbann},
    description={Der Zauber gilt als Konterprobe (12) gegen einen Zauber. Gelingt sie, wird der Zauber aufgehoben.\newline Probenschwierigkeit: 12\newline Vorbereitungszeit: 4 Minuten\newline Ziel: Einzelperson\newline Reichweite: Berührung\newline Wirkungsdauer: augenblicklich\newline Kosten: halbe Basiskosten des Zaubers\newline Fertigkeiten: Antimagie, Geister vertreiben\newline Erlernen: Smn 14; 60 EP}
}


\newglossaryentry{geistheilung_Talent}
{
    name={Geistheilung},
    description={Dein Ziel erhält 2W6+4 Heilpunkte, für jede Überschreitung der WS wird eine Wunde geheilt.\newline Mächtige Magie: Erhöht die Heilpunkte um 4.\newline Probenschwierigkeit: 12+Wund-Mod. des Ziels\newline Modifikationen: Siechtum heilen (Du stoppst die Wirkung eines Giftes oder einer Krankheit bis Stufe 16 sofort. Mächtige Magie erhöht die maximal aufgehobene Stufe um +4.)\newline Vorbereitungszeit: 1 Stunde\newline Ziel: Einzelperson\newline Reichweite: Berührung\newline Wirkungsdauer: augenblicklich\newline Kosten: 8 AsP\newline Fertigkeiten: Geister vertreiben, Humus\newline Erlernen: Smn 12; 60 EP}
}


\newglossaryentry{großerGeisterbann_Talent}
{
    name={Großer Geisterbann},
    description={Hiermit bannst du beschworene Wesenheiten, in die keine gAsP geflossen sind (wie nicht gebundene Elementare und Dämonen).\newline Probenschwierigkeit: Beschwörungsschwierigkeit des zu bannenden Wesens\newline Modifikationen: Geisterkerker (+4; das Wesen wird nicht verbannt, sondern permanent in ein Gefäß gesperrt. Wird das Gefäß geöffnet oder zerstört, ist das Wesen frei und meist sehr wütend auf den Schamanen und/oder den Befreier.)\newline Gebundene Geister (Entspricht dem allgemeinen Zauber Destructibo (S. 126))\newline Geister vertreiben (Entspricht dem allgemeinen Zauber Geister austreiben (S. 127))\newline Vorbereitungszeit: 8 Aktionen\newline Ziel: Beschworenes Wesen\newline Reichweite: 4 Schritt\newline Wirkungsdauer: augenblicklich\newline Kosten: halbe Basiskosten der Beschwörung\newline Fertigkeiten: Antimagie, Geister vertreiben\newline Erlernen: Smn 16; 80 EP}
}


\newglossaryentry{großeGeistheilung_Talent}
{
    name={Große Geistheilung},
    description={Du erfüllst ein großes humusaffines Objekt wie einen Baum mit elementarer Lebenskraft. Wer im Radius von 8 Schritt eine Ruhepause verbringt, regeneriert eine zusätzliche Wunde.\newline Probenschwierigkeit: 12\newline Modifikationen: Heiliger Hain (–16, Wirkungsdauer bis die Bindung gelöst wird, 64 AsP, davon 16 gAsP)\newline Vorbereitungszeit: 1 Stunde\newline Ziel: Einzelwesen\newline Reichweite: Berührung\newline Wirkungsdauer: 1 Tag\newline Kosten: 16 AsP\newline Fertigkeiten: Geister vertreiben, Humus\newline Erlernen: Smn 18; 40 EP}
}


\newglossaryentry{ängstemehren_Talent}
{
    name={Ängste mehren},
    description={Das Opfer leidet an einer Angst deiner Wahl. Wird die Angst ausgelöst, steht es unter einem Furcht-Effekt Stufe 1. Litt das Opfer bereits zuvor an dieser Angst, steigt die Stufe des Furcht-Effekts um 1.\newline Mächtige Magie: Der Furcht-Effekt steigt um eine Stufe.\newline Probenschwierigkeit: Magieresistenz\newline Vorbereitungszeit: 4 Aktionen\newline Ziel: Einzelperson\newline Reichweite: 8 Schritt\newline Wirkungsdauer: 1 Woche\newline Kosten: 8 AsP\newline Fertigkeiten: Einfluss, Hexenflüche\newline Erlernen: Hex 12; 20 EP}
}


\newglossaryentry{beute!_Talent}
{
    name={Beute!},
    description={Tiere reagieren aggressiv auf dein Opfer. Proben zum Umgang mit Nutztieren sind um –4 erschwert und Raubtiere greifen stets das Opfer an, wobei sie +1 TP anrichten.\newline Mächtige Magie: Der Malus steigt um –2, die TP um +1.\newline Probenschwierigkeit: Magieresistenz\newline Vorbereitungszeit: 4 Aktionen\newline Ziel: Einzelperson\newline Reichweite: 8 Schritt\newline Wirkungsdauer: 1 Woche\newline Kosten: 8 AsP\newline Fertigkeiten: Eigenschaften, Hexenflüche\newline Erlernen: Hex 16; 20 EP}
}


\newglossaryentry{hexenschuss_Talent}
{
    name={Hexenschuss},
    description={Das Opfer wird vom Hexenschuss geplagt, der entweder alle körperlichen oder alle geistigen Tätigkeiten um –2 erschwert.\newline Mächtige Magie: Der Malus steigt um –1.\newline Probenschwierigkeit: Magieresistenz\newline Vorbereitungszeit: 2 Aktionen\newline Ziel: Einzelperson\newline Reichweite: 8 Schritt\newline Wirkungsdauer: 4 Stunden\newline Kosten: 8 AsP\newline Fertigkeiten: Eigenschaften, Hexenflüche\newline Erlernen: Hex 8; 40 EP}
}


\newglossaryentry{krötenkuss_Talent}
{
    name={Krötenkuss},
    description={Dein Ziel wird von seiner Umgebung furchtsam gemieden. Proben auf Überreden und Rhetorik sind um –4 erschwert, Proben auf Betören sogar um –8.\newline Mächtige Magie: Der Malus steigt um –2/–4.\newline Probenschwierigkeit: Magieresistenz\newline Vorbereitungszeit: 4 Aktionen\newline Ziel: Einzelperson\newline Reichweite: 8 Schritt\newline Wirkungsdauer: 1 Tag\newline Kosten: 8 AsP\newline Fertigkeiten: Einfluss, Hexenflüche, Illusion\newline Erlernen: Hex 16; 40 EP}
}


\newglossaryentry{pechandenHalswünschen_Talent}
{
    name={Pech an den Hals wünschen},
    description={Dein Opfer wird vom Unglück verfolgt. Seine Chance auf einen Patzer steigt um 1 auf dem W20 (zum Beispiel von 1 auf 1–2).\newline Mächtige Magie: Je zwei Stufen steigern die Chance auf einen Patzer um einen weiteren Punkt.\newline Probenschwierigkeit: Magieresistenz\newline Vorbereitungszeit: 4 Aktionen\newline Ziel: Einzelperson\newline Reichweite: 8 Schritt\newline Wirkungsdauer: 1 Tag\newline Kosten: 8 AsP\newline Fertigkeiten: Eigenschaften, Hexenflüche\newline Erlernen: Hex 18; 40 EP}
}


\newglossaryentry{sinntrüben_Talent}
{
    name={Sinn trüben},
    description={Bei deinem Opfer trübt sich ein Sinn deiner Wahl. Alle Tätigkeiten, die den Einsatz dieses Sinnes benötigen, sind um –4 erschwert.\newline Mächtige Magie: Der Malus steigt um –4.\newline Probenschwierigkeit: Magieresistenz\newline Vorbereitungszeit: 4 Aktionen\newline Ziel: Einzelperson\newline Reichweite: 8 Schritt\newline Wirkungsdauer: 1 Tag\newline Kosten: 16 AsP\newline Fertigkeiten: Eigenschaften, Hexenflüche\newline Erlernen: Hex 12; 20 EP}
}


\newglossaryentry{warzensprießen_Talent}
{
    name={Warzen sprießen},
    description={Eklige Warzen entstellen das Gesicht deines Ziels. Die Vorteile Eindrucksvoll I und II werden aufgehoben und alle Proben auf Autorität, Beeinflussung  und Gebräuche sind um –2 erschwert.\newline Mächtige Magie: Der Malus steigt um –1.\newline Probenschwierigkeit: Magieresistenz\newline Vorbereitungszeit: 4 Aktionen\newline Ziel: Einzelperson\newline Reichweite: 8 Schritt\newline Wirkungsdauer: 1 Tag\newline Kosten: 8 AsP\newline Fertigkeiten: Hexenflüche, Verwandlung\newline Erlernen: Hex 8; 20 EP}
}


\newglossaryentry{zungenschwellung_Talent}
{
    name={Zungenschwellung},
    description={Es fällt dem Ziel schwer, artikuliert zu sprechen. Die Zuhörer müssen eine Konterprobe (Sinnenschärfe, 16) ablegen, um den Sprecher zu verstehen. Laut gesprochene oder gesungene Zauberformeln sind dadurch unmöglich.\newline Probenschwierigkeit: Magieresistenz\newline Vorbereitungszeit: 4 Aktionen\newline Ziel: Einzelperson\newline Reichweite: 8 Schritt\newline Wirkungsdauer: 1 Tag\newline Kosten: 8 AsP\newline Fertigkeiten: Eigenschaften, Hexenflüche\newline Erlernen: Hex 16; 20 EP}
}


\newglossaryentry{weihederKeule_Talent}
{
    name={Weihe der Keule},
    description={Du stellst eine enge magische Bindung zu deinem Ritualgegenstand her, welche die Voraussetzung für alle weiteren Talente dieser Fertigkeit ist. Außerdem wird der Ritualgegenstand unzerbrechlich und zählt als magische Waffe.\newline Probenschwierigkeit: 12\newline Modifikationen: Geerbte Keule (Du bindest die Keule deines Lehrmeisters. Von ihm auf die Keule gewirkte Rituale verfliegen, aber die Keule gilt als Lehrmeister für diese Rituale.)\newline Vorbereitungszeit: 8 Stunden\newline Ziel: Ritualgegenstand\newline Reichweite: Berührung\newline Wirkungsdauer: permanent\newline Kosten: 16 AsP\newline Fertigkeiten: Keulenrituale, Kraft\newline Erlernen: Smn 8; 20 EP}
}


\newglossaryentry{apportderKeule_Talent}
{
    name={Apport der Keule},
    description={Der Ritualgegenstand kehrt fliegend mit einer Geschwindigkeit von 10 Meilen pro Stunde zu dir zurück. Er weicht Hindernissen aus oder durchbricht sie notfalls, wenn die KK von PW Keulenrituale/2 oder Umwelt/2 dafür ausreicht.\newline Probenschwierigkeit: 12\newline Vorbereitungszeit: 0 Aktionen\newline Ziel: Zone\newline Reichweite: 1 Meile\newline Wirkungsdauer: augenblicklich\newline Kosten: 1 AsP\newline Fertigkeiten: Keulenrituale, Umwelt\newline Erlernen: Smn 18; 20 EP}
}


\newglossaryentry{bannderKeule(passiv)_Talent}
{
    name={Bann der Keule (passiv)},
    description={Du lässt 2/4/6/8 gAsP in die Keule fließen. Fortan sind alle Beherrschungsproben für Elementare oder für Untote und Dämonen um +1/2/3/4 erleichtert. Du kannst die gAsP und gewählte Kreaturenklasse in einem einwöchigen Ritual, das 16 AsP kostet, verändern.\newline Erlernen: Smn 14; 40 EP}
}


\newglossaryentry{geistderKeule_Talent}
{
    name={Geist der Keule},
    description={Du bindest einen Totengeist in die Keule. Solange der Geist an die Keule gebunden ist, erfüllt er dir eine Bitte, wenn dir eine CH-Probe (16) gelingt. Die erste Bitte seit dem letzten Neumond ist nicht erschwert, aber bei jeder zusätzlichen Bitte erleidest du einen kumulativen Malus von –4. Es kann immer nur ein Wesen in der Keule gebunden sein.\newline Probenschwierigkeit: 12\newline Vorbereitungszeit: 1 Stunde\newline Ziel: Knochenkeule\newline Reichweite: Berührung\newline Wirkungsdauer: bis die Bindung gelöst wird\newline Kosten: 4 AsP, davon 2 gAsP\newline Fertigkeiten: Keulenrituale, Verständigung\newline Erlernen: Smn 12; 40 EP}
}


\newglossaryentry{elementarderKeule(passiv)_Talent}
{
    name={Elementar der Keule (passiv)},
    description={Du lässt 2 gAsP in die Keule fließen. Fortan können in deiner Keule gebundene Elementare (siehe S. 99) unbegrenzt lange gebunden bleiben. Das Binden in die Keule gilt immer als gewöhnlicher Dienst. Es kann immer nur ein Wesen in der Keule gebunden sein.\newline Erlernen: Smn 16, 20 EP\newline Anmerkung: Zusammen mit dem Vorteil Meister der Wünsche eine permanente Besessenheit möglich, wenn du das Elementar nach jedem Dienst sofort wieder bindest.}
}


\newglossaryentry{dämonderKeule(passiv)_Talent}
{
    name={Dämon der Keule (passiv)},
    description={Du lässt 2 gAsP in die Keule fließen. Fortan können in deiner Keule gebundene Dämonen (siehe S. 99) unbegrenzt lange gebunden bleiben. Das Binden in die Keule gilt immer als gewöhnlicher Dienst. Es kann immer nur ein Wesen in der Keule gebunden sein.\newline Erlernen: Smn 18, 20 EP\newline Anmerkung: Zusammen mit dem Vorteil Meister der Seelenlosen ist eine permanente Besessenheit möglich, wenn du das Wesen nach jedem Dienst sofort wieder bindest.}
}


\newglossaryentry{gespürderKeule_Talent}
{
    name={Gespür der Keule},
    description={Die Keule zeigt dir die Gegenwart von Magie in einem Radius von 1 Schritt an. Das entspricht einem Analysegrad von 1 für die Intensitätsanalyse (mehr dazu S. 80).\newline Mächtige Magie: Der Analysegrad steigt um 1.\newline Probenschwierigkeit: 12\newline Vorbereitungszeit: 16 Aktionen\newline Ziel: Zone\newline Reichweite: Berührung\newline Wirkungsdauer: augenblicklich\newline Kosten: 1 AsP\newline Fertigkeiten: Geister rufen, Keulenrituale, Kraft\newline Erlernen: Smn 12; 40 EP}
}


\newglossaryentry{kraftderKeule_Talent}
{
    name={Kraft der Keule},
    description={Gegen "Geister" – also Wesen der Kategorie Dämon, Elementar, Untoter, Feenwesen – steigt der Waffenschaden der Keule um +2.\newline Mächtige Magie: Der Bonus steigt um +1.\newline Probenschwierigkeit: 12\newline Modifikationen: Permanenz (Wirkungsdauer bis die Bindung gelöst wird, Kosten 8 AsP, davon 1 gAsP)\newline Vorbereitungszeit: 16 Aktionen\newline Ziel: Knochenkeule\newline Reichweite: Berührung\newline Wirkungsdauer: 1 Stunde\newline Kosten: 8 AsP\newline Fertigkeiten: Keulenrituale, Kraft\newline Erlernen: Smn 14; 40 EP}
}


\newglossaryentry{bindungdesSchuppenbeutels_Talent}
{
    name={Bindung des Schuppenbeutels},
    description={Du stellst eine enge magische Bindung zu deinem Ritualgegenstand her, welche die Voraussetzung für alle weiteren Talente dieser Fertigkeit ist. Außerdem wird der Ritualgegenstand unzerbrechlich.\newline Probenschwierigkeit: 12\newline Vorbereitungszeit: 8 Stunden\newline Ziel: Ritualgegenstand\newline Reichweite: Berührung\newline Wirkungsdauer: permanent\newline Kosten: 16 AsP\newline Fertigkeiten: Kraft, Kristallmagie\newline Erlernen: Ach 8; 20 EP}
}


\newglossaryentry{apportdesSchuppenbeutels_Talent}
{
    name={Apport des Schuppenbeutels},
    description={Der Ritualgegenstand kehrt fliegend mit einer Geschwindigkeit von 10 Meilen pro Stunde zu dir zurück. Er weicht Hindernissen aus oder durchbricht sie notfalls, wenn die KK von PW Kristallmagie/2 oder Umwelt/2 dafür ausreicht.\newline Probenschwierigkeit: 12\newline Vorbereitungszeit: 0 Aktionen\newline Ziel: Zone\newline Reichweite: 1 Meile\newline Wirkungsdauer: augenblicklich\newline Kosten: 1 AsP\newline Fertigkeiten: Kristallmagie, Umwelt\newline Erlernen: Ach 16; 20 EP}
}


\newglossaryentry{ewigeWegzehrung(passiv)_Talent}
{
    name={Ewige Wegzehrung (passiv)},
    description={Im Schuppenbeutel aufbewahrte Gegenstände altern nicht.\newline Erlernen: Ach 12; 20 EP}
}


\newglossaryentry{federleichterBeutel(passiv)_Talent}
{
    name={Federleichter Beutel (passiv)},
    description={Pro 4 volle Punkte PW Kristallmagie halbiert sich das Gewicht der Gegenstände im Schuppenbeutel (z.B. bei einem PW von 9 auf ein Viertel).\newline Erlernen: Ach 16; 20 EP}
}


\newglossaryentry{geräumigerSchuppenbeutel(passiv)_Talent}
{
    name={Geräumiger Schuppenbeutel (passiv)},
    description={Pro 4 volle Punkte PW Kristallmagie steigt das Volumen des Schuppenbeutels um 2 Liter. Von außen betrachtet bleibt die Größe des Beutels jedoch gleich.\newline Erlernen: Ach 18; 20 EP}
}


\newglossaryentry{suchendeFinger_Talent}
{
    name={Suchende Finger},
    description={Du hältst einen im Schuppenbeutel aufbewahrten Gegenstand sofort in der Hand.\newline Probenschwierigkeit: 12\newline Vorbereitungszeit: 0 Aktionen\newline Ziel: Zone\newline Reichweite: Berührung\newline Wirkungsdauer: augenblicklich\newline Kosten: 1 AsP\newline Fertigkeiten: Hellsicht, Kristallmagie, Umwelt\newline Erlernen: Ach 12; 20 EP}
}


\newglossaryentry{kristallbindung_Talent}
{
    name={Kristallbindung},
    description={Du bindest einen mindestens daumennagelgroßen Kristall an dich und bestimmst eine Fertigkeit. Der Kristall gilt als passender gebundener Kristall für diese Fertigkeit und erfüllt damit die Bedingung der Kristallomantischen Tradition. Außerdem ist der Kristall unzerstörbar. Du kannst mehrere solche Kristalle besitzen.\newline Probenschwierigkeit: 12\newline Modifikationen: Mächtiger Kristall (16 AsP, davon 2/4/6/8 gAsP; Zauber mit dem entsprechenden Merkmal sind um +1/2/3/4 erleichtert.)\newline Vorbereitungszeit: 1 Stunde\newline Ziel: Einzelobjekt\newline Reichweite: Berührung\newline Wirkungsdauer: bis die Bindung gelöst wird\newline Kosten: 8 AsP, davon 1 gAsP\newline Fertigkeiten: Kraft, Kristallmagie\newline Erlernen: Ach 4; 20 EP\newline Anmerkung: Die meisten Kristallomanten verwenden Kristalle, die affin zu der entsprechenden Fertigkeit sind. Eine entsprechende Tabelle findest du beispielsweise im Wege der Zauberei, S. 325. Ihr könnt aber auch einfach darauf verzichten, penibel die Art jedes einzelnen Kristalls festzulegen – entscheidend ist, dass der Kristall für die richtige Fertigkeit gebunden wurde.}
}


\newglossaryentry{rundschliff_Talent}
{
    name={Rundschliff},
    description={Du versiehst einen gebundenen Kristall mit einem Rundschliff. Wann immer du mit Hilfe dieses Kristalls einen Zauber wirkst, kannst du die spontane Modifikation Mehrere Ziele ohne Erschwernis ausführen. Nicht mit anderen Schliffen kombinierbar.\newline Probenschwierigkeit: 12\newline Vorbereitungszeit: 4 Stunden\newline Ziel: gebundener Kristall\newline Reichweite: Berührung\newline Wirkungsdauer: bis die Bindung gelöst wird\newline Kosten: 8 AsP, davon 2 gAsP\newline Fertigkeiten: Erz, Kristallmagie, Kraft\newline Erlernen: Ach 14; 40 EP}
}


\newglossaryentry{tafelschliff_Talent}
{
    name={Tafelschliff},
    description={Der Zauber gleicht weitgehend dem Rundschliff, aber du kannst die spontane Modifikation Reichweite erhöhen einmal ohne Erschwernis ausführen. Nicht mit anderen Schliffen kombinierbar.\newline Fertigkeiten: Erz, Kristallmagie, Kraft\newline Erlernen: Ach 14; 40 EP}
}


\newglossaryentry{tropfenschliff_Talent}
{
    name={Tropfenschliff},
    description={Der Zauber gleicht weitgehend dem Rundschliff, aber du kannst die spontane Modifikation Wirkungsdauer verlängern einmal ohne Erschwernis ausführen. Nicht mit anderen Schliffen kombinierbar.\newline Fertigkeiten: Erz, Kristallmagie, Kraft\newline Erlernen: Ach 14; 40 EP}
}


\newglossaryentry{smaragdschliff_Talent}
{
    name={Smaragdschliff},
    description={Der Zauber gleicht weitgehend dem Rundschliff, aber du kannst die spontane Modifikation Kosten sparen einmal ohne Erschwernis ausführen. Du musst die Modifikation auch ohne den Schliff beherrschen. Nicht mit anderen Schliffen kombinierbar.\newline Kosten: 8 AsP, davon 4 gAsP\newline Fertigkeiten: Erz, Kristallmagie, Kraft\newline Erlernen: Ach 18; 40 EP}
}


\newglossaryentry{h‘SzintsAuge(passiv)_Talent}
{
    name={H‘Szints Auge (passiv)},
    description={Du kannst einen gebundenen Kristall als Hilfsmittel für eine magische Analyse (S. 80) verwenden. Gehört der Kristall zur selben Fertigkeit wie die wirkende Magie (bei Artefakten nur wirkende Sprüche), steigt der AG um +1.\newline Erlernen: Ach 16; 20 EP}
}


\newglossaryentry{kristallkraftbündeln(passiv)_Talent}
{
    name={Kristallkraft bündeln (passiv)},
    description={Du kannst einen Zauber mit der Essenz eines gebundenen Kristalls stärken. Der Zauber ist um +4 erleichtert und seine Kosten sinken um die gAsP des Kristalls. Nach dem Zauber zerfällt der Kristall zu Staub.\newline Erlernen: Ach 12; 40 EP}
}


\newglossaryentry{orbitarium_Talent}
{
    name={Orbitarium},
    description={Über dem Kristall entsteht ein Abbild des aktuellen Sternenhimmels, von dem du einzelne Teile hervortreten lassen kannst. Das Abbild leuchtet nur schwach und ist im Sonnenlicht nicht zu sehen.\newline Probenschwierigkeit: 12\newline Vorbereitungszeit: 8 Aktionen\newline Ziel: gebundener Kristall\newline Reichweite: Berührung\newline Wirkungsdauer: 1 Stunde\newline Kosten: 4 AsP\newline Fertigkeiten: Hellsicht, Kristallmagie\newline Erlernen: Ach 14; 20 EP}
}


\newglossaryentry{optikstein_Talent}
{
    name={Optikstein},
    description={Der Kristall kann seine Brennweite nach Belieben verändern und so als Vergrößerungsglas oder Prisma verwendet werden.\newline Probenschwierigkeit: 12\newline Vorbereitungszeit: 8 Aktionen\newline Ziel: gebundener Kristall\newline Reichweite: Berührung\newline Wirkungsdauer: 1 Stunde\newline Kosten: 1 AsP\newline Fertigkeiten: Kristallmagie, Verwandlung\newline Erlernen: Ach 12; 20 EP}
}


\newglossaryentry{thesiskristall_Talent}
{
    name={Thesiskristall},
    description={Du speicherst die Struktur eines Zaubers für 20 EP in einen Kristall. Andere Zauberer können den Zauber aus dieser Struktur erlernen, sofern sie über die Kristallomantische Tradition I verfügen.\newline Mächtige Magie: Du kannst auch Zauber für 40/60/80/beliebig viele EP speichern.\newline Probenschwierigkeit: 12\newline Modifikationen: Permanenz (Wirkungsdauer bis die Bindung gelöst wird, 8 AsP, davon 1 gAsP)\newline Vorbereitungszeit: 1 Stunde\newline Ziel: Einzelobjekt\newline Reichweite: Berührung\newline Wirkungsdauer: 1 Jahr\newline Kosten: 8 AsP\newline Fertigkeiten: Kraft, Kristallmagie\newline Erlernen: Ach 16; 20 EP}
}


\newglossaryentry{wachenderStein_Talent}
{
    name={Wachender Stein},
    description={Du kannst einen Blick durch einen gebundenen Kristall werfen. Dem Kristall ist dabei nichts anzumerken.\newline Probenschwierigkeit: 12\newline Vorbereitungszeit: 8 Aktionen\newline Ziel: gebundener Kristall\newline Reichweite: 1 Meile\newline Wirkungsdauer: 1 Stunde\newline Kosten: 8 AsP\newline Fertigkeiten: Hellsicht, Kristallmagie\newline Erlernen: Ach 18; 40 EP}
}


\newglossaryentry{warnenderStein_Talent}
{
    name={Warnender Stein},
    description={Der Kristall leuchtet auf, falls sich eine feindlich gesinnte Person in weniger als 16 Schritt Entfernung befindet und ihr eine Konterprobe (MR, 12) misslingt.\newline Probenschwierigkeit: 12\newline Vorbereitungszeit: 4 Aktionen\newline Ziel: gebundener Kristall\newline  Reichweite: Berührung\newline Wirkungsdauer: 1 Stunde\newline Kosten: 4 AsP\newline Fertigkeiten: Hellsicht, Kristallmagie\newline Erlernen: Ach 16; 40 EP}
}


\newglossaryentry{bindungderKugel_Talent}
{
    name={Bindung der Kugel},
    description={Du stellst eine enge magische Bindung zu deinem Ritualgegenstand her, welche die Voraussetzung für alle weiteren Talente dieser Fertigkeit ist. Außerdem wird der Ritualgegenstand unzerbrechlich.\newline Probenschwierigkeit: 12\newline Vorbereitungszeit: 8 Stunden\newline Ziel: Ritualgegenstand\newline Reichweite: Berührung\newline Wirkungsdauer: permanent\newline Kosten: 16 AsP\newline Fertigkeiten: Kraft, Kugelzauber\newline Erlernen: Srl 8; Mag 14; 20 EP}
}


\newglossaryentry{apportderKugel_Talent}
{
    name={Apport der Kugel},
    description={Der Ritualgegenstand kehrt fliegend mit einer Geschwindigkeit von 10 Meilen pro Stunde zu dir zurück. Er weicht Hindernissen aus oder durchbricht sie notfalls, wenn die KK von PW Kugelzauber/2 oder Umwelt/2 dafür ausreicht.\newline Probenschwierigkeit: 12\newline Vorbereitungszeit: 0 Aktionen\newline Ziel: Zone\newline Reichweite: 1 Meile\newline Wirkungsdauer: augenblicklich\newline Kosten: 1 AsP\newline Fertigkeiten: Kugelzauber, Umwelt\newline Erlernen: Srl 12; Mag 18; 20 EP}
}


\newglossaryentry{bildergalerie_Talent}
{
    name={Bildergalerie},
    description={Du speicherst einen soeben gewirkten Illusionszauber in die Kugel. Maximal kannst du PW Kugelzauber Illusionen speichern. Wenn du die Illusion abspielen möchtest, musst du eine Probe auf Kugelzauber (12) ablegen, 1 Aktion Konflikt und 1 AsP aufwenden. Beim Abspielen befindet sich die Kugel im Zentrum der Illusion.\newline Probenschwierigkeit: 12\newline Vorbereitungszeit: 4 Aktionen\newline Ziel: Kristallkugel\newline Reichweite: Berührung\newline Wirkungsdauer: bis die Bindung gelöst wird\newline Kosten: 4 AsP, davon 1 gAsP\newline Fertigkeiten: Illusion, Kraft, Kugelzauber\newline Erlernen: Srl 16; 20 EP}
}


\newglossaryentry{bilderspiel_Talent}
{
    name={Bilderspiel},
    description={Du kannst in der Kugel eine beliebige Illusion (Sicht) erscheinen lassen. Die Illusion kann sich auch bewegen, was Konzentration erfordert.\newline Probenschwierigkeit: 12\newline Vorbereitungszeit: 4 Aktionen\newline Ziel: Kristallkugel\newline Reichweite: 2 Schritt\newline Wirkungsdauer: 16 Initiativphasen\newline Kosten: 4 AsP\newline Fertigkeiten: Illusion, Kugelzauber\newline Erlernen: Srl 12; 20 EP}
}


\newglossaryentry{brennglasundPrisma_Talent}
{
    name={Brennglas und Prisma},
    description={Die Kugel kann ihre Brennweite nach Belieben verändern und so als Vergrößerungsglas oder Prisma verwendet werden.\newline Probenschwierigkeit: 12\newline Vorbereitungszeit: 8 Aktionen\newline Ziel: Kristallkugel\newline Reichweite: Berührung\newline Wirkungsdauer: 1 Stunde\newline Kosten: 1 AsP\newline Fertigkeiten: Kugelzauber, Verwandlung\newline Erlernen: Srl 12; Mag 16; 20 EP}
}


\newglossaryentry{wachendeKugel_Talent}
{
    name={Wachende Kugel},
    description={Du kannst einen Blick durch eine gebundene Kugel werfen. Der Kugel ist dabei nichts anzumerken.\newline Probenschwierigkeit: 12\newline Vorbereitungszeit: 8 Aktionen\newline Ziel: Kristallkugel\newline Reichweite: 1 Meile\newline Wirkungsdauer: 1 Stunde\newline Kosten: 8 AsP\newline Fertigkeiten: Hellsicht, Kugelzauber\newline Erlernen: Srl 16; 40 EP}
}


\newglossaryentry{warnendeKugel_Talent}
{
    name={Warnende Kugel},
    description={Die Kugel leuchtet auf, falls sich eine feindlich gesinnte Person in weniger als 16 Schritt Entfernung befindet und ihr eine Konterprobe (MR, 12) misslingt.\newline Probenschwierigkeit: 12\newline Vorbereitungszeit: 4 Aktionen\newline Ziel: Kristallkugel\newline Reichweite: Berührung\newline Wirkungsdauer: 1 Stunde\newline Kosten: 4 AsP\newline Fertigkeiten: Hellsicht, Kugelzauber\newline Erlernen: Srl 14; Mag 20; 40 EP}
}


\newglossaryentry{kugeldesAstrologen_Talent}
{
    name={Kugel des Astrologen},
    description={Über der Kugel entsteht ein Abbild des aktuellen Sternenhimmels, von dem du einzelne Teile hervortreten lassen kannst. Das Abbild leuchtet nur schwach und ist im Sonnenlicht nicht zu sehen.\newline Probenschwierigkeit: 12\newline Vorbereitungszeit: 8 Aktionen\newline Ziel: Kristallkugel\newline Reichweite: Berührung\newline Wirkungsdauer: 1 Stunde\newline Kosten: 4 AsP\newline Fertigkeiten: Hellsicht, Kugelzauber\newline Erlernen: Srl 14; Mag 16; 20 EP}
}


\newglossaryentry{kugeldesHellsehers(passiv)_Talent}
{
    name={Kugel des Hellsehers (passiv)},
    description={Du lässt 2 gAsP in die Kugel fließen. Fortan sind Hellsichtszauber um +1 erleichtert.\newline Modifikationen: Mächtige Kugel (Hellsichtszauber sind um +2 erleichtert, du musst 4 gAsP in die Kugel fließen lassen.)\newline Fertigkeiten: Hellsicht, Kugelzauber\newline Erlernen: Mag, Srl 16; 20 EP}
}


\newglossaryentry{kugeldesIllusionisten(passiv)_Talent}
{
    name={Kugel des Illusionisten (passiv)},
    description={Du lässt 2 gAsP in die Kugel fließen. Fortan sind Illusionszauber um +1 erleichtert\newline Modifikationen: Mächtige Kugel (Illusionszauber sind um +2 erleichtert, du musst 4 gAsP in die Kugel fließen lassen.)\newline Fertigkeiten: Illusion, Kugelzauber\newline Erlernen: Srl 12; Mag 18; 20 EP}
}


\newglossaryentry{schutzgegenUntote_Talent}
{
    name={Schutz gegen Untote},
    description={Eine silbrige Kuppel mit 2 Schritt Radius umgibt dich und bewegt sich mit dir. Untote mit einer Beschwörungsschwierigkeit von maximal 16 können sie nicht durchschreiten. Du kannst Untote mit der Kugel zurückdrängen, wenn dir eine vergleichende KK-Probe gelingt. Umgekehrt ist das nicht möglich.\newline Mächtige Magie: Erhöht die Beschwörungsschwierigkeit um 4.\newline Probenschwierigkeit: 12\newline Vorbereitungszeit: 8 Aktionen\newline Ziel: Zone\newline Reichweite: Berührung\newline Wirkungsdauer: 1 Stunde\newline Kosten: 8 AsP\newline Fertigkeiten: Antimagie, Kugelzauber\newline Erlernen: Mag 18; Srl 20; 40 EP}
}


\newglossaryentry{bindungdesRinges_Talent}
{
    name={Bindung des Ringes},
    description={Du stellst eine enge magische Bindung zu deinem Ritualgegenstand her, welche die Voraussetzung für alle weiteren Talente dieser Fertigkeit ist. Außerdem wird der Ritualgegenstand unzerbrechlich.\newline Probenschwierigkeit: 12\newline Vorbereitungszeit: 8 Stunden\newline Ziel: Ritualgegenstand\newline Reichweite: Berührung\newline Wirkungsdauer: permanent\newline Kosten: 16 AsP\newline Fertigkeiten: Kraft, Ringrituale\newline Erlernen: Geo 4; 20 EP}
}


\newglossaryentry{herrderFlammen_Talent}
{
    name={Herr der Flammen},
    description={Du kannst ein bestehendes Feuer bis zur Größe einer Fackel formen oder verlöschen lassen. Die Verformung geschieht mit einer GS von 8 Schritt pro Initiativephase und benötigt Konzentration.\newline Mächtige Magie: Du kannst ein Feuer bis zur Größe eines Kaminfeuers/eines großen Lagerfeuers/eines Scheiterhaufens/eines brennenden Schiffes manipulieren.\newline Probenschwierigkeit: 12\newline Vorbereitungszeit: 4 Aktionen\newline Ziel: Einzelobjekt\newline Reichweite: 16 Schritt\newline Wirkungsdauer: 8 Initiativphasen\newline Kosten: 8 AsP\newline Fertigkeiten: Feuer, Ringrituale\newline Erlernen: Geo 14; 20 EP}
}


\newglossaryentry{kräftederNatur_Talent}
{
    name={Kräfte der Natur},
    description={Die Heilkraft Sumus beendet ein Gift oder eine Krankheit bis maximal Stufe 16.\newline Mächtige Magie: Die maximal aufgehobene Gift-/Krankheitsstufe steigt um 4.\newline Probenschwierigkeit: 12\newline Vorbereitungszeit: 16 Aktionen\newline Ziel: Einzelperson\newline Reichweite: Berührung\newline Wirkungsdauer: augenblicklich\newline Kosten: 8 AsP\newline Fertigkeiten: Humus, Ringrituale\newline Erlernen: Geo 12; 40 EP}
}


\newglossaryentry{launendesWindes_Talent}
{
    name={Launen des Windes},
    description={Dein Körper wird leicht wie eine Feder, die emporschwebt und vom Wind fortgetragen wird. Du kannst die Flugrichtung nur durch das Abstoßen von Gegenständen beeinflussen. Gegen Ende der Wirkungsdauer nimmt dein Gewicht langsam wieder zu.\newline Probenschwierigkeit: 12\newline Vorbereitungszeit: 1 Aktion\newline Ziel: selbst\newline Reichweite: Berührung\newline Wirkungsdauer: 4 Minuten\newline Kosten: 8 AsP\newline Fertigkeiten: Eigenschaften, Luft, Ringrituale\newline Erlernen: Geo 16; 40 EP}
}


\newglossaryentry{machtdesLebens_Talent}
{
    name={Macht des Lebens},
    description={Die strahlende Macht des Lebens schreckt Untote und Daimoniden ab. Liegt die Beschwörungsschwierigkeit eines dieser Wesen bei maximal 16, kann es dir nicht näher als 4 Schritt kommen.\newline Mächtige Magie: Erhöht die Beschwörungsschwierigkeit um 4.\newline Probenschwierigkeit: 12\newline Vorbereitungszeit: 4 Aktionen\newline Ziel: Zone\newline Reichweite: Berührung\newline Wirkungsdauer: 4 Minuten\newline Kosten: 8 AsP\newline Fertigkeiten: Antimagie, Ringrituale\newline Erlernen: Geo 18; 40 EP}
}


\newglossaryentry{machtüberdenRegen_Talent}
{
    name={Macht über den Regen},
    description={Das Wetter in einer Zone von bis zu 1 Meile Radius verändert sich nach deinem Willen. Du kannst folgende Skalen um insgesamt zwei Stufen verändern:\newline Niederschlag: trocken/Nieselregen/Regen/starker Regen/Wolkenbruch\newline Wind: windstill/leichte Brise/steife Brise/Sturm/Orkan\newline Temperatur: heiß/warm/mittel/kühl/kalt.\newline Mächtige Magie: Du kannst die Skalen um eine weitere Stufe verändern.\newline Probenschwierigkeit: 12\newline Vorbereitungszeit: 4 Minuten\newline Ziel: Zone\newline Reichweite: 1 Meile\newline Wirkungsdauer: 1 Stunde\newline Kosten: 32 AsP\newline Fertigkeiten: Luft, Ringrituale, Umwelt, Wasser\newline Erlernen: Geo 14; 40 EP}
}


\newglossaryentry{magnetismus_Talent}
{
    name={Magnetismus},
    description={Der Erdboden im Radius von 8 Schritt wirkt wie ein starker Magnet. Wenn eine Konterprobe (KK, 16) misslingt, werden Metallwaffen aus der Hand gerissen und kleben am Boden fest, während Kämpfer in Metallrüstungen zu Boden gehen und nicht mehr aufstehen können. Selbst bei einer gelungenen Konterprobe sind Kampfaktionen mit Metallwaffen um –4 erschwert.\newline Probenschwierigkeit: 12\newline Vorbereitungszeit: 2 Aktionen\newline Ziel: Zone\newline Reichweite: Berührung\newline Wirkungsdauer: 16 Initiativphasen\newline Kosten: 16 AsP\newline Fertigkeiten: Erz, Ringrituale, Umwelt\newline Erlernen: Geo 16; 60 EP}
}


\newglossaryentry{seelenfeuer_Talent}
{
    name={Seelenfeuer},
    description={Kalt leuchtendes Licht schießt in einem Radius von 4 Schritt aus dem Boden. Es erstarrt in der von dir gewünschten Form und verbrennt alle Humanoiden, die es berühren, mit 2W6 SP pro Initiativephase.\newline Mächtige Magie: Erhöht den Schaden um +4 SP und den Radius um 2 Schritt.\newline Probenschwierigkeit: 12\newline Vorbereitungszeit: 1 Aktion\newline Ziel: Zone\newline Reichweite: Berührung\newline Wirkungsdauer: 4 Initiativphasen\newline Kosten: 16 AsP\newline Fertigkeiten: Kraft, Ringrituale\newline Erlernen: Geo 16; 60 EP}
}


\newglossaryentry{wasserbann_Talent}
{
    name={Wasserbann},
    description={Eine Luftschicht schützt dich vor jeglichem Wasser. Du kannst unter Wasser atmen und hältst Druck und Strömungen stand, dafür sinkst du zu Boden und bewegst dich gehend fort. Erlaubt Aufrechterhalten.\newline Mächtige Magie: Du kannst den Schutz auf eine Person oder einen ähnlich großen Gegenstand ausdehnen.\newline Probenschwierigkeit: 12\newline Vorbereitungszeit: 4 Aktionen\newline Ziel: selbst\newline Reichweite: Berührung\newline Wirkungsdauer: 4 Minuten\newline Kosten: 8 AsP\newline Fertigkeiten: Luft, Ringrituale, Wasser\newline Erlernen: Geo 12; 40 EP}
}


\newglossaryentry{wegdurchSumusLeib_Talent}
{
    name={Weg durch Sumus Leib},
    description={Du bewegst dich innerhalb 1 Minute durch Erde und Stein an einen Ort, der maximal 4 Meilen entfernt sein darf. Alle Metallgegenstände, die nicht kaltgeschmiedet sind (wie der Schlangenreif), verbleiben in der Erde.\newline Mächtige Magie: Vervierfacht die Distanz\newline Modifikationen: Anhalter (–8; du kannst eine Person mitnehmen. Mehrfach wählbar.)\newline Probenschwierigkeit: 12\newline Vorbereitungszeit: 4 Aktionen\newline Ziel: selbst\newline Reichweite: Berührung\newline Wirkungsdauer: augenblicklich\newline Kosten: 16 AsP\newline Fertigkeiten: Erz, Humus, Ringrituale\newline Erlernen: Geo 18; 40 EP}
}


\newglossaryentry{wirbelnderLuftschild_Talent}
{
    name={Wirbelnder Luftschild},
    description={Unberechenbare Winde lenken alle auf dich gerichteten Geschosse bis zur Größe eines Wurfspeers ab. Ungezielte Geschosse (wie in einem Pfeilhagel) treffen dich nicht, Fernkampfangriffe auf dich sind um –4 erschwert.\newline Mächtige Magie: Der Malus steigt um –2.\newline Probenschwierigkeit: 12\newline Vorbereitungszeit: 2 Aktionen\newline Ziel: selbst\newline Reichweite: Berührung\newline Wirkungsdauer: 16 Initiativphasen\newline Kosten: 8 AsP\newline Fertigkeiten: Luft, Ringrituale\newline Erlernen: Geo 14; 40 EP}
}


\newglossaryentry{bindungderSchale_Talent}
{
    name={Bindung der Schale},
    description={Du stellst eine enge magische Bindung zu deinem Ritualgegenstand her, welche die Voraussetzung für alle weiteren Talente dieser Fertigkeit ist. Außerdem wird der Ritualgegenstand unzerbrechlich.\newline Probenschwierigkeit: 12\newline Vorbereitungszeit: 8 Stunden\newline Ziel: Ritualgegenstand\newline Reichweite: Berührung\newline Wirkungsdauer: permanent\newline Kosten: 16 AsP\newline Fertigkeiten: Kraft, Schalenzauber\newline Erlernen: Alch 8; Hex 14; Mag 16; 20 EP}
}


\newglossaryentry{allegorischeAnalyse(passiv)_Talent}
{
    name={Allegorische Analyse (passiv)},
    description={Du lässt einen 1 gAsP in die Schale fließen. Fortan leuchten bei der Analyse einer Astralstruktur in deiner Schale arkane Symbole auf der Schale auf. Dadurch steigt dein Analysegrad bei Intensitätsanalysen um 1 (S. 80).\newline Erlernen: Alch 12; Hex, Mag 16; 20 EP}
}


\newglossaryentry{apportderSchale_Talent}
{
    name={Apport der Schale},
    description={Der Ritualgegenstand kehrt fliegend mit einer Geschwindigkeit von 10 Meilen pro Stunde zu dir zurück. Er weicht Hindernissen aus oder durchbricht sie notfalls, wenn die KK von PW Schalenzauber/2 oder Umwelt/2 dafür ausreicht.\newline Probenschwierigkeit: 12\newline Vorbereitungszeit: 0 Aktionen\newline Ziel: Zone\newline Reichweite: 1 Meile\newline Wirkungsdauer: augenblicklich\newline Kosten: 1 AsP\newline Fertigkeiten: Schalenzauber, Umwelt\newline Erlernen: Alch 12; Hex 18; 20 EP}
}


\newglossaryentry{astraleAufladung_Talent}
{
    name={Astrale Aufladung},
    description={Bei der Herstellung eines Elixiers lässt du durch die Schale zusätzliche Astralenergie einfließen. Alle Proben zur Herstellung dieses Elixiers sind um +4 erleichtert.\newline Mächtige Magie: Der Bonus steigt um +2.\newline Probenschwierigkeit: 12\newline Vorbereitungszeit: 0 Aktionen\newline Ziel: Elixier\newline Reichweite: Berührung\newline Wirkungsdauer: bis zur Fertigstellung\newline Kosten: 8 AsP\newline Fertigkeiten: Schalenzauber, Verwandlung\newline Erlernen: Alch 12; Hex 16; Mag 18; 60 EP}
}


\newglossaryentry{chymischeHochzeit(passiv)_Talent}
{
    name={Chymische Hochzeit (passiv)},
    description={Du lässt 2 gAsP in die Schale fließen. Fortan sinken die Abzüge durch improvisiertes Werkzeug für je 4 volle Punkte PW Schale des Alchemisten um 1. Dieser Bonus ist kumulativ zum Vorteil Improvisation (S. 58).\newline Erlernen: Alch 16; Hex 18; 60 EP}
}


\newglossaryentry{essenzkonzentration_Talent}
{
    name={Essenzkonzentration},
    description={Du konzentrierst zwei alchemistische Produkte der gleichen Art zu einem. Das entstehende Produkt verfügt über die freiwilligen Modifikationen beider Ausgangsprodukte, höchstens aber 4 Modifikationen. Darüber hinausgehende Modifikationen verfallen - du entscheidest welche.\newline Mächtige Magie: Verzehnfacht die (verbesserte) Haltbarkeit und erlaubt eine weitere Modifikation.\newline Probenschwierigkeit: 12\newline Vorbereitungszeit: 4 Stunden\newline Ziel: zwei Elixiere\newline Reichweite: Berührung\newline Wirkungsdauer: augenblicklich\newline Kosten: 8 AsP\newline Fertigkeiten: Schalenzauber, Verwandlung\newline Erlernen: Alch 18; 40 EP\newline Anmerkung: Die Essenzkonzentration erklärt, warum in uralten Gemäuern immer noch wirksame Alchemika zu finden sind.}
}


\newglossaryentry{feuerundEis_Talent}
{
    name={Feuer und Eis},
    description={Du kannst die Schale auf eine Temperatur zwischen niederhöllische Kälte und der Hitze einer Eisenschmelze einstellen.\newline Probenschwierigkeit: 12\newline Vorbereitungszeit: 4 Aktionen\newline Ziel: Schale der Alchemie\newline Reichweite: 2 Schritt\newline Wirkungsdauer: 1 Stunde\newline Kosten: 1 AsP\newline Fertigkeiten: Eis, Feuer, Schalenzauber\newline Erlernen: Alch 16; Mag 18; 20 EP}
}


\newglossaryentry{sichereAufbewahrung(passiv)_Talent}
{
    name={Sichere Aufbewahrung (passiv)},
    description={In der Schale aufbewahrte Substanzen verderben oder zersetzen sich nicht. Du kannst so auch explosive Materialien sicher transportieren.\newline Erlernen: Alch 14; Hex 16; 20 EP}
}


\newglossaryentry{transmutationderElemente_Talent}
{
    name={Transmutation der Elemente},
    description={Du beschwörst eine etwa faustgroße Menge oder kleine Manifestation des Elements herbei, mit dem du den Zauber wirkst. Sie verschwindet nach dem Ende der Wirkungsdauer.\newline Fertigkeit Eis: Du kannst zum Beispiel einen Schneeball oder einen Eiszapfen erscheinen lassen, ein Getränk kühlen, oder deine Körpertemperatur etwas senken.\newline Fertigkeit Erz: Du kannst zum Beispiel eine Hand voll Sand oder einen Stein erscheinen lassen oder das Gewicht eines Gegenstandes kurzzeitig etwas erhöhen.\newline Fertigkeit Feuer: Du kannst zum Beispiel eine kleine Flamme über deiner Hand tanzen lassen, deine Körpertemperatur etwas erhöhen, eine Pfeife anzünden  oder einen Funkenschwarm erzeugen.\newline Fertigkeit Humus: Du kannst zum Beispiel eine Knospe erblühen lassen, eine Hand voll Kuhmist herbeizaubern, ein Gänseblümchen erscheinen lassen, oder einen kleinen Kratzer schließen.\newline Fertigkeit Luft: Du kannst zum Beispiel einen frischen Windhauch rufen, einen Wohlgeruch erzeugen, oder einen kleinen Luftwirbel erschaffen.\newline Fertigkeit Wasser: Du kannst zum Beispiel einen Becher mit Wasser füllen, deine Hände abspülen, eine Pfütze bilden oder tagsüber einen kleinen Regenbogen glitzern lassen.\newline Probenschwierigkeit: 12\newline Modifikationen: Demanifesto (–4; du kannst die Manifestation nach Belieben verschwinden und wieder auftauchen lassen.)\newline Vorbereitungszeit: 0 Aktionen\newline Ziel: Zone\newline Reichweite: Berührung\newline Wirkungsdauer: 1 Stunde\newline Kosten: 1 AsP\newline Fertigkeiten: Eis, Erz, Feuer, Humus, Luft, Wasser, Schalenzauber\newline Erlernen: Alch 12; 20 EP}
}


\newglossaryentry{bindungdesStabs_Talent}
{
    name={Bindung des Stabs},
    description={Du stellst eine enge magische Bindung zu deinem Ritualgegenstand her, welche die Voraussetzung für alle weiteren Talente dieser Fertigkeit ist. Außerdem wird der Ritualgegenstand unzerbrechlich und gilt als magische Waffe.\newline Probenschwierigkeit: 12\newline Vorbereitungszeit: 8 Stunden\newline Ziel: Ritualgegenstand\newline Reichweite: Berührung\newline Wirkungsdauer: permanent\newline Kosten: 16 AsP\newline Fertigkeiten: Kraft, Stabzauber\newline Erlernen: Mag 8; 20 EP}
}


\newglossaryentry{apportdesStabs_Talent}
{
    name={Apport des Stabs},
    description={Der Ritualgegenstand kehrt fliegend mit einer Geschwindigkeit von 10 Meilen pro Stunde zu dir zurück. Er weicht Hindernissen aus oder durchbricht sie notfalls, wenn die KK von PW Stabzauber/2 oder Umwelt/2 dafür ausreicht.\newline Probenschwierigkeit: 12\newline Vorbereitungszeit: 0 Aktionen\newline Ziel: Zone\newline Reichweite: 1 Meile\newline Wirkungsdauer: augenblicklich\newline Kosten: 1 AsP\newline Fertigkeiten: Stabzauber, Umwelt\newline Erlernen: Mag 16; 20 EP}
}


\newglossaryentry{astralspeicher(passiv)_Talent}
{
    name={Astralspeicher (passiv)},
    description={Du lässt 8 gAsP in den Stab fließen. Fortan wird der Zauberstab zu einem Speicher für Astralenergie. In einem einstündigen Ritual kannst du eigene Astralenergie in den Stab einspeichern, wobei 2 zusätzliche AsP verloren gehen. Zusätzlich wählst du einen Zauber pro 4 volle Punkte PW Stabzauber. Bis zum nächsten Aufladen kannst du die Kosten für diese Zauber teilweise oder ganz aus dem Astralspeicher bezahlen. Der Astralspeicher kann maximal 32 AsP fassen.\newline Erlernen: Mag 18; 80 EP}
}


\newglossaryentry{auradesStabes(passiv)_Talent}
{
    name={Aura des Stabes (passiv)},
    description={Wenn du bei einer Verwandlung deines Körpers (egal ob freiwillig oder unfreiwillig) deinen Stab bei dir trägst, hast du ihn auch nach der Rückverwandlung wieder bei dir.\newline Erlernen: Mag 18; 20 EP}
}


\newglossaryentry{ewigeFlamme_Talent}
{
    name={Ewige Flamme},
    description={Die Spitze des Stabes bricht in Flammen aus, die in Leuchtkraft und Eigenschaften einer gewöhnlichen Fackel entsprechen. Der Zauber endet vorzeitig, wenn du den Stab nicht mehr berührst.\newline Probenschwierigkeit: 12\newline Vorbereitungszeit: 0 Aktionen\newline Ziel: Zauberstab\newline Reichweite: Berührung\newline Wirkungsdauer: 8 Stunden\newline Kosten: 1 AsP\newline Fertigkeiten: Feuer, Stabzauber\newline Erlernen: Mag 12; 20 EP}
}


\newglossaryentry{flammenschwert_Talent}
{
    name={Flammenschwert},
    description={Dein Stab verwandelt sich in eine feurige Klinge. Die Kampfwerte entsprechen einem gewöhnlichen Schwert, aber es verursacht Feuerschaden und Nachbrennen (S. 98). Der Zauber endet vorzeitig, wenn du den Stab nicht mehr berührst.\newline Mächtige Magie: Erhöht die TP um +2.\newline Probenschwierigkeit: 12\newline Modifikationen: Schwebendes Schwert (16 Schritt, 16 AsP; du kannst das Schwert mit 4 Schritt pro Initiativephase fernsteuern. Der Angriffswert entspricht dem PW der benutzten Fertigkeit. Das Schwert kann nur Aktionen Konflikt und nur Basismanöver ausführen. Du musst das Schwert nicht berühren, aber der Zauber benötigt Konzentration.)\newline Vorbereitungszeit: 1 Aktion\newline Ziel: Zauberstab\newline Reichweite: Berührung\newline Wirkungsdauer: 16 Initiativphasen\newline Kosten: 8 AsP\newline Fertigkeiten: Feuer, Stabzauber, Verwandlung, Umwelt (schwebendes Schwert)\newline Erlernen: Mag 16; 40 EP}
}


\newglossaryentry{hammerdesMagus_Talent}
{
    name={Hammer des Magus},
    description={Das mit dem Stab berührte Objekt erleidet 4W6 SP. An Lebewesen richtet der Hammer keinen Schaden an.\newline Mächtige Magie: Erhöht die SP um +4.\newline Probenschwierigkeit: 12\newline Vorbereitungszeit: 1 Aktion\newline Ziel: Einzelobjekt\newline Reichweite: Berührung\newline Wirkungsdauer: augenblicklich\newline Kosten: 4 AsP\newline Fertigkeiten: Stabzauber, Umwelt\newline Erlernen: Mag 18; 20 EP}
}


\newglossaryentry{langerArm_Talent}
{
    name={Langer Arm},
    description={An der Spitze des Stabs entsteht eine astrale Hand, die du steuern kannst. Mit ihr kannst du gefährliche Versuche durchführen und Zauber mit der Reichweite Berührung wirken. Der Zauber endet vorzeitig, wenn du den Stab nicht mehr berührst.\newline Probenschwierigkeit: 12\newline Vorbereitungszeit: 0 Aktionen\newline Ziel: Zauberstab\newline Reichweite: Berührung\newline Wirkungsdauer: 1 Stunde\newline Kosten: 1 AsP\newline Fertigkeiten: Stabzauber, Umwelt\newline Erlernen: Mag 14; 20 EP}
}


\newglossaryentry{matrixstabilisierung(passiv)_Talent}
{
    name={Matrixstabilisierung (passiv)},
    description={Der Zauberstab stabilisiert die Zauberei bestimmter Fertigkeiten. In einem einstündigen Ritual, das 16 AsP kostet, kannst du eine Fertigkeit pro 4 volle Punkte PW Stabzauber wählen. Bei Proben auf diese Fertigkeiten patzt du nur, wenn der gewertete Würfel eine 1 zeigt und der nächsthöhere Würfel eine 1–8.\newline Erlernen: Mag 14; 40 EP}
}


\newglossaryentry{reinigung_Talent}
{
    name={Reinigung},
    description={Deine Kleidung ist so sauber, als wäre sie frisch gewaschen.\newline Probenschwierigkeit: 12\newline Modifikationen: Läusekamm (–4; du wirst auch von Kleintieren und Ungeziefer befreit.)\newline Vorbereitungszeit: 0 Aktionen\newline Ziel: selbst\newline Reichweite: Berührung\newline Wirkungsdauer: augenblicklich\newline Kosten: 1 AsP\newline Fertigkeiten: Luft, Stabzauber, Umwelt, Wasser\newline Erlernen: Mag 12; 20 EP}
}


\newglossaryentry{wandlungdesStabs_Talent}
{
    name={Wandlung des Stabs},
    description={Du kannst den Stab auf die doppelte Länge ausdehnen oder auf die halbe Länge stauchen. Der Zauber endet vorzeitig, wenn du den Stab (oder seine verwandelte Form) nicht mehr berührst.\newline Probenschwierigkeit: 12\newline Modifikationen: Seil des Adepten (–8, Wirkungsdauer 4 Minuten; du verwandelst den Stab in ein Seil, das sich selbstständig verknoten und lösen kann.)\newline Vorbereitungszeit: 0 Aktionen\newline Ziel: Zauberstab\newline Reichweite: Berührung\newline Wirkungsdauer: 1 Stunde\newline Kosten: 1 AsP\newline Fertigkeiten: Stabzauber, Umwelt, Verwandlung\newline Erlernen: Mag 12; 20 EP}
}


\newglossaryentry{zaubersiegel_Talent}
{
    name={Zaubersiegel},
    description={Auf dem mit dem Stab berührten Objekt entsteht ein handtellergroßes Siegel. Diese Illusion (Sicht) leuchtet bei jeder weiteren Berührung mit deinem Stab auf.\newline Probenschwierigkeit: 12\newline Modifikationen: Permanenz (–4, Wirkungsdauer bis die Bindung gelöst wird, Kosten 1 AsP, davon 1 gAsP)\newline Vorbereitungszeit: 0 Aktionen\newline Ziel: Einzelobjekt\newline Reichweite: Berührung\newline Wirkungsdauer: 1 Monat\newline Kosten: 1 AsP\newline Fertigkeiten: Illusion, Stabzauber\newline Erlernen: Mag 18; 20 EP}
}


\newglossaryentry{rhythmenderErmutigung_Talent}
{
    name={Rhythmen der Ermutigung},
    description={Die Rhythmen schenken Zuversicht. Alle Zuhörer in einem Radius von 4 Schritt erhalten eine Erleichterung +2 auf Anführen und MU.\newline Mächtige Magie: Der Bonus steigt um +1.\newline Probenschwierigkeit: 12\newline Vorbereitungszeit: 1 Stunde\newline Ziel: Zone\newline Reichweite: Berührung\newline Wirkungsdauer: 1 Tag\newline Kosten: 16 AsP\newline Fertigkeiten: Eigenschaften, Einfluss, Trommelrituale\newline Erlernen: Der 10; 40 EP}
}


\newglossaryentry{rhythmenderGüte_Talent}
{
    name={Rhythmen der Güte},
    description={Deine Rhythmen verschaffen allen Humanoiden im Radius von 4 Schritt einen erholsamen Schlaf. In ihrer nächsten Ruhepause regenerieren sie eine zusätzliche Wunde.\newline Probenschwierigkeit: 12\newline Vorbereitungszeit: 1 Stunde\newline Ziel: Zone\newline Reichweite: Berührung\newline Wirkungsdauer: augenblicklich\newline Kosten: 16 AsP\newline Fertigkeiten: Humus, Trommelrituale\newline Erlernen: Der 14; 40 EP}
}


\newglossaryentry{rhythmenderJagd_Talent}
{
    name={Rhythmen der Jagd},
    description={Die Rhythmen schärfen die Jagdinstinkte der Zuhörer. Alle Zuhörer in einem Radius von 4 Schritt erhalten eine Erleichterung +2 auf Pirschen, Überleben, Tierkunde und GE-Proben.\newline Mächtige Magie: Der Bonus steigt um +1.\newline Probenschwierigkeit: 12\newline Vorbereitungszeit: 1 Stunde\newline Ziel: Zone\newline Reichweite: Berührung\newline Wirkungsdauer: 1 Tag\newline Kosten: 16 AsP\newline Fertigkeiten: Eigenschaften, Trommelrituale\newline Erlernen: Der 12; 40 EP}
}


\newglossaryentry{rhythmendesgottgefälligenZornes_Talent}
{
    name={Rhythmen des gottgefälligen Zornes},
    description={Die Waffen der Zuhörer im Radius von 4 Schritt gelten während der Wirkungsdauer als magisch.\newline Probenschwierigkeit: 12\newline Vorbereitungszeit: 1 Stunde\newline Ziel: Zone\newline Reichweite: Berührung\newline Wirkungsdauer: 1 Tag\newline Kosten: 16 AsP\newline Fertigkeiten: Kraft, Trommelrituale\newline Erlernen: Der 18; 60 EP}
}


\newglossaryentry{rhythmendesKrieges_Talent}
{
    name={Rhythmen des Krieges},
    description={Die aufpeitschenden Rhythmen stärken den Kampfesmut der Zuhörer. Alle Zuhörer in einem Radius von 4 Schritt erhalten eine Erleichterung +1 auf alle Nahkampffertigkeiten, Zähigkeit und KK.\newline Mächtige Magie: Je zwei Stufen erhöhen den Bonus um +1.\newline Probenschwierigkeit: 12\newline Vorbereitungszeit: 1 Stunde\newline Ziel: Zone\newline Reichweite: Berührung\newline Wirkungsdauer: 1 Tag\newline Kosten: 16 AsP\newline Fertigkeiten: Eigenschaften, Trommelrituale\newline Erlernen: Der 14; 60 EP}
}


\newglossaryentry{rhythmenderReinigung_Talent}
{
    name={Rhythmen der Reinigung},
    description={Die Rhythmen gelten als Konterprobe (12) gegen sämtliche Einflusszauber, Hexenflüche und Druidenrituale auf dem Ziel. Bei gelungener Probe werden die Zauber aufgehoben.\newline Probenschwierigkeit: 12\newline Vorbereitungszeit: 4 Stunden\newline Ziel: Einzelperson\newline Reichweite: 4 Schritt\newline Wirkungsdauer: augenblicklich\newline Kosten: 8 AsP\newline Fertigkeiten: Antimagie, Trommelrituale\newline Erlernen: Der 18; 40 EP}
}


\newglossaryentry{rhythmendesSchutzes_Talent}
{
    name={Rhythmen des Schutzes},
    description={Die sanften Rhythmen erhöhen die Magieresistenz. Alle Zuhörer im Radius von 4 Schritt erhalten eine Erleichterung von +4 auf die MR.\newline Mächtige Magie: Der Bonus steigt um +1.\newline Probenschwierigkeit: 12\newline Vorbereitungszeit: 1 Stunde\newline Ziel: Zone\newline Reichweite: Berührung\newline Wirkungsdauer: 1 Tag\newline Kosten: 16 AsP\newline Fertigkeiten: Eigenschaften, Trommelrituale\newline Erlernen: Der 16; 40 EP}
}


\newglossaryentry{rhythmendesSturmes_Talent}
{
    name={Rhythmen des Sturmes},
    description={Die sanften Rhythmen erhöhen die Reitkünste. Alle Zuhörer im Radius von 4 Schritt erhalten eine Erleichterung von +2 auf Lanzenreiten, Reiten und andere Proben im Umgang mit Pferden.\newline Mächtige Magie: Der Bonus steigt um +1.\newline Probenschwierigkeit: 12\newline Vorbereitungszeit: 1 Stunde\newline Ziel: Zone\newline Reichweite: Berührung\newline Wirkungsdauer: 1 Tag\newline Kosten: 16 AsP\newline Fertigkeiten: Eigenschaften, Trommelrituale\newline Erlernen: Der 8; 40 EP}
}


\newglossaryentry{bindungdesVertrauten_Talent}
{
    name={Bindung des Vertrauten},
    description={Du stellst eine enge magische Bindung zu deinem Vertrautentier her. Das Tier kann andere Vertrautenzauber wirken, versteht deine Muttersprache und kann sich mit dir auf telepathischem Wege unterhalten.\newline Probenschwierigkeit: 12\newline Vorbereitungszeit: 8 Stunden\newline Ziel: Vertrautentier\newline Reichweite: Berührung\newline Wirkungsdauer: permanent\newline Kosten: 16 AsP\newline Fertigkeiten: Kraft, Verständigung, Vertrautenmagie\newline Erlernen: Hex 4, Geo 8; 20 EP}
}


\newglossaryentry{dingeaufspüren_Talent}
{
    name={Dinge aufspüren},
    description={Dein Vertrautentier teilt dir mit, in welcher Richtung sich ein Gegenstand aus deinem Besitz befindet.\newline Probenschwierigkeit: 12\newline Vorbereitungszeit: 0 Aktionen\newline Ziel: Einzelobjekt\newline Reichweite: 16 Meilen\newline Wirkungsdauer: augenblicklich\newline Kosten: 1 AsP\newline Fertigkeiten: Hellsicht, Vertrautenmagie\newline Erlernen: Geo, Hex 12; 20 EP}
}


\newglossaryentry{ersteruntergleichen_Talent}
{
    name={Erster unter gleichen},
    description={Dein Vertrautentier schüchtert Tiere aus seiner Gattung ein. Sie erleiden einen Furcht-Effekt Stufe 2.\newline Probenschwierigkeit: 12\newline Vorbereitungszeit: 0 Aktionen\newline Ziel: Zone\newline Reichweite: 16 Schritt\newline Wirkungsdauer: 1 Stunde\newline Kosten: 1 AsP\newline Fertigkeiten: Einfluss, Vertrautenmagie\newline Erlernen: Geo, Hex 16; 20 EP}
}


\newglossaryentry{bindungspartnerfinden_Talent}
{
    name={Bindungspartner finden},
    description={Das Vertrautentier erspürt deinen ungefähren Aufenthaltsort.\newline Probenschwierigkeit: 12\newline Vorbereitungszeit: 0 Aktionen\newline Ziel: Bindungspartner\newline Reichweite: 16 Meilen\newline Wirkungsdauer: 1 Stunde\newline Kosten: 1 AsP\newline Fertigkeiten: Einfluss, Vertrautenmagie\newline Erlernen: Geo, Hex 8; 20 EP}
}


\newglossaryentry{letzterAusweg_Talent}
{
    name={Letzter Ausweg},
    description={Dein Vertrautentier mobilisiert alle Kräfte, um dich oder sich zu retten. Alle Feinde in 4 Schritt Radius erleiden 2W6 SP. Das Vertrautentier erleidet sofort 4 Wunden.\newline Mächtige Magie: Die SP steigen um +4.\newline Probenschwierigkeit: 12\newline Vorbereitungszeit: 0 Aktionen\newline Ziel: Zone\newline Reichweite: Berührung\newline Wirkungsdauer: augenblicklich\newline Kosten: 1 AsP\newline Fertigkeiten: Kraft, Vertrautenmagie\newline Erlernen: Geo 14, Hex 16; 40 EP}
}


\newglossaryentry{schlafrauben_Talent}
{
    name={Schlaf rauben},
    description={Das von deinem Vertrautentier beobachtete Opfer regeneriert in dieser Nacht keine AsP oder Wunden. Stattdessen erscheint ihm das Vertrautentier in fürchterlichen Albträumen.\newline Probenschwierigkeit: Magieresistenz\newline Vorbereitungszeit: 0 Aktionen\newline Ziel: Einzelperson\newline Reichweite: 16 Schritt\newline Wirkungsdauer: augenblicklich\newline Kosten: 1 AsP\newline Fertigkeiten: Einfluss, Vertrautenmagie\newline Erlernen: Hex 16; 20 EP}
}


\newglossaryentry{stimmungssinn_Talent}
{
    name={Stimmungssinn},
    description={Dein Vertrautentier starrt das Ziel an und gibt dir seine Eindrücke weiter. Menschenkenntnis-Proben gegen das Ziel sind um +4 erleichtert.\newline Mächtige Magie: Der Bonus steigt um +2.\newline Probenschwierigkeit: Magieresistenz\newline Vorbereitungszeit: 0 Aktionen\newline Ziel: Einzelperson\newline Reichweite: 2 Schritt\newline Wirkungsdauer: 1 Stunde\newline Kosten: 8 AsP\newline Fertigkeiten: Hellsicht, Vertrautenmagie\newline Erlernen: Hex 12, Geo 16; 40 EP}
}


\newglossaryentry{tarnung_Talent}
{
    name={Tarnung},
    description={Dein Vertrautentier erhält die Eigenschaft Tarnung (S. 98).\newline Probenschwierigkeit: 12\newline Vorbereitungszeit: 0 Aktionen\newline Ziel: selbst\newline Reichweite: Berührung\newline Wirkungsdauer: 1 Stunde\newline Kosten: 1 AsP\newline Fertigkeiten: Illusion, Vertrautenmagie\newline Erlernen: Hex 14, Geo 18; 20 EP}
}


\newglossaryentry{tiersinne_Talent}
{
    name={Tiersinne},
    description={Dein Vertrautentier schließt seine Augen und leiht dir seine Sinne. Deine Wahrnehmung steigt um +4, zusätzlich erhältst du einen Bonus je nach Tier (Katze und Eule  Angepasst II (Dunkelheit), Rabe und Falke Sehsinn +4, Hund und Schlange Geruchssinn +4, Kröte Magiegespür, Affe Gehör +4, Spinne Tastsinn +4). Der Zauber endet, wenn der Vertraute seine Augen öffnet.\newline Mächtige Magie: Der Bonus steigt um +2.\newline Probenschwierigkeit: 12\newline Vorbereitungszeit: 2 Aktionen\newline Ziel: selbst\newline Reichweite: Berührung\newline Wirkungsdauer: 1 Stunde\newline Kosten: 8 AsP\newline Fertigkeiten: Eigenschaften, Verständigung, Vertrautenmagie\newline Erlernen: Geo, Hex 16; 40 EP}
}


\newglossaryentry{ungesehenerBeobachter_Talent}
{
    name={Ungesehener Beobachter},
    description={Du blickst in Trance durch die Augen deines Vertrautentiers, das ein Ziel ausspioniert. Der Zauber endet mit der Rückkehr des Vertrautentiers und benötigt Konzentration.\newline Mächtige Magie: Du teilst einen zusätzlichen Sinn mit dem Vertrauten.\newline Probenschwierigkeit: 12\newline Vorbereitungszeit: 2 Aktionen\newline Ziel: Bindungspartner\newline Reichweite: 8 Meilen\newline Wirkungsdauer: 8 Stunden\newline Kosten: 8 AsP\newline Fertigkeiten: Verständigung, Vertrautenmagie\newline Erlernen: Hex 12; 40 EP}
}


\newglossaryentry{wachsameAugen_Talent}
{
    name={Wachsame Augen},
    description={Dein Vertrautentier ruft Tiere seiner Gattung  aus bis zu 1 Meile herbei, die über deinen Schlaf wachen und dich vor Gefahren warnen.\newline Probenschwierigkeit: 12\newline Vorbereitungszeit: 0 Aktionen\newline Ziel: Zone\newline Reichweite: Berührung\newline Wirkungsdauer: 8 Stunden\newline Kosten: 1 AsP\newline Fertigkeiten: Verständigung, Vertrautenmagie\newline Erlernen: Geo 14; Hex 18; 20 EP}
}


\newglossaryentry{aufpeitschenderKlang_Talent}
{
    name={Aufpeitschender Klang},
    description={Du entlockst deinem Instrument einen elektrisierenden Ton. Wenn das Ziel die nächste Probe auf eine körperliche Aktivität (z.B. Laufen, Nahkampfangriff) ablegt, ist diese um +2 erleichtert.\newline Mächtige Magie: Der Bonus steigt um +1\newline Probenschwierigkeit: 12\newline Vorbereitungszeit: 0 Aktionen\newline Ziel: Einzelperson\newline Reichweite: 8 Schritt\newline Wirkungsdauer: 4 Initiativphasen\newline Kosten: 4 AsP\newline Fertigkeiten: Eigenschaften, Zaubermelodien\newline Erlernen: Bard 12; 40 EP}
}


\newglossaryentry{disharmonischerKlang_Talent}
{
    name={Disharmonischer Klang},
    description={Der Klang stört einen anderen Zauberer während dessen Zaubervorbereitung. Das Ziel muss eine Konterprobe (Willenskraft, 16) ablegen, um den Zauber fortzusetzen.\newline Probenschwierigkeit: 12\newline Vorbereitungszeit: 0 Aktionen\newline Ziel: Einzelperson\newline Reichweite: 8 Schritt\newline Wirkungsdauer: augenblicklich\newline Kosten: 4 AsP\newline Fertigkeiten: Antimagie, Einfluss, Zaubermelodien\newline Erlernen: Bard 18; 40 EP}
}


\newglossaryentry{liedderFeen_Talent}
{
    name={Lied der Feen},
    description={Du rufst ein Feenwesen zur Hilfe. Befinden sich Feenwesen in einem Radius von 1 Meile, eilt eines von ihnen herbei. Du kannst das Feenwesen um einen Gefallen bitten, aber es entscheidet selbst, ob es den Gefallen erfüllt.\newline Mächtige Magie: Verdoppelt den Radius.\newline Probenschwierigkeit: 12\newline Modifikationen: Namensruf (–4; du rufst ein dir bereits bekanntes Feenwesen in einem Radius von 8 Meilen herbei.)\newline Vorbereitungszeit: 4 Minuten\newline Ziel: Zone\newline Reichweite: Berührung\newline Wirkungsdauer: augenblicklich\newline Kosten: 8 AsP\newline Fertigkeiten: Verständigung, Zaubermelodien\newline Erlernen: Bard 16; 20 EP}
}


\newglossaryentry{liedderFlammen_Talent}
{
    name={Lied der Flammen},
    description={Flammen im Radius von 4 Schritt lodern heller und spenden mehr Wärme. In ihrer Umgebung steigt die Temperaturstufe um eins (siehe S. 35).\newline Probenschwierigkeit: 12\newline Modifikationen: Innere Wärme (–8; auch ohne Feuer steigt die Temperaturstufe für alle Zuhörer um eins.)\newline Vorbereitungszeit: 16 Aktionen\newline Ziel: Zone\newline Reichweite: Berührung\newline Wirkungsdauer: 1 Stunde\newline Kosten: 8 AsP\newline Fertigkeiten: Feuer, Verwandlung, Zaubermelodien\newline Erlernen: Bard 16; 20 EP}
}


\newglossaryentry{liedderHeilung_Talent}
{
    name={Lied der Heilung},
    description={Das Lied stärkt die Lebenskraft der Verwundeten. Der am schwersten verletze Zuhörer in 4 Schritt Radius regeneriert in der nächsten Ruhephase zwei zusätzliche Wunden. Bei Gleichstand entscheidet der Spieler.\newline Mächtige Magie: Mit zwei/vier Stufen erstreckt sich die Wirkung auch auf den am zweit-/drittstärksten Verletzten.\newline Probenschwierigkeit: 12\newline Vorbereitungszeit: 1 Stunde\newline Ziel: Zone\newline Reichweite: Berührung\newline Wirkungsdauer: augenblicklich\newline Kosten: 8 AsP\newline Fertigkeiten: Humus, Zaubermelodien\newline Erlernen: Bard 14; 40 EP}
}


\newglossaryentry{lieddesWanderers_Talent}
{
    name={Lied des Wanderers},
    description={Das Lied stärkt das Durchhaltevermögen eines müden Wanderers. Der Zuhörer mit dem geringsten DH (siehe S. 34) in 4 Schritt Radius erhält ein verdoppeltes DH* und der Intervall, in dem er durch körperliche Anstrengung Erschöpfung verursacht, verdoppelt sich.\newline Probenschwierigkeit: 12\newline Vorbereitungszeit: 16 Aktionen\newline Ziel: Zone\newline Reichweite: 4 Schritt\newline Wirkungsdauer: 8 Stunden\newline Kosten: 4 AsP\newline Fertigkeiten: Eigenschaften, Zaubermelodien\newline Erlernen: Bard 12; 20 EP}
}


\newglossaryentry{melodiederBeruhigung_Talent}
{
    name={Melodie der Beruhigung},
    description={Deine Melodie nimmt Zuhörern im Radius von 4 Schritt die Angst. Auf ihnen lastende Furcht-Effekte sinken um eine Stufe. Erfordert Konzentration, erlaubt Aufrechterhalten.\newline Probenschwierigkeit: 12\newline Vorbereitungszeit: 1 Aktion\newline Ziel: Zone\newline Reichweite: Berührung\newline Wirkungsdauer: 16 Initiativphasen\newline Kosten: 16 AsP\newline Fertigkeiten: Einfluss, Zaubermelodien\newline Erlernen: Bard 14; 60 EP}
}


\newglossaryentry{melodiederBesänftigung_Talent}
{
    name={Melodie der Besänftigung},
    description={Die Melodie besänftigt Tiere im Radius von 32 Schritt. Wenn ihnen eine Konterprobe (Magieresistenz, 16) misslingt, verlieren sie ihre Angriffslust, solange sie nicht angegriffen oder gereizt werden. Erfordert Konzentration, erlaubt Aufrechterhalten.\newline Probenschwierigkeit: 12\newline Vorbereitungszeit: 1 Aktion\newline Ziel: Zone\newline Reichweite: Berührung\newline Wirkungsdauer: 4 Minuten\newline Kosten: 8 AsP\newline Fertigkeiten: Einfluss, Zaubermelodien\newline Erlernen: Bard 16; 20 EP}
}


\newglossaryentry{melodiederErmutigung_Talent}
{
    name={Melodie der Ermutigung},
    description={Die Melodie schenkt den Menschen Zuversicht. Alle Zuhörer in einem Radius von 4 Schritt erhalten eine Erleichterung +2 auf Anführen und MU. Erfordert Konzentration, erlaubt Aufrechterhalten.\newline Mächtige Magie: Der Bonus steigt um +1\newline Probenschwierigkeit: 12\newline Vorbereitungszeit: 4 Aktionen\newline Ziel: Zone\newline Reichweite: Berührung\newline Wirkungsdauer: 16 Initiativphasen\newline Kosten: 8 AsP\newline Fertigkeiten: Eigenschaften, Einfluss, Zaubermelodien\newline Erlernen: Bard 12; 40 EP}
}


\newglossaryentry{melodiederVersöhnung_Talent}
{
    name={Melodie der Versöhnung},
    description={Die Melodie schafft in 4 Schritt Radius eine versöhnliche Atmosphäre, wenn den Zuhörern keine Konterprobe (Magieresistenz, 16) gelingt. Proben auf Betören und Rhetorik sind um +4 erleichtert, Proben auf Überreden und Einschüchtern um –4 erschwert. Erfordert Konzentration, erlaubt Aufrechterhalten.\newline Probenschwierigkeit: 12\newline Vorbereitungszeit: 4 Aktionen\newline Ziel: Zone\newline Reichweite: Berührung\newline Wirkungsdauer: 1 Stunde\newline Kosten: 8 AsP\newline Fertigkeiten: Einfluss, Zaubermelodien\newline Erlernen: Bard 12; 40 EP}
}


\newglossaryentry{melodiederVerwirrung_Talent}
{
    name={Melodie der Verwirrung},
    description={Die Melodie verwirrt die Zuhörer in 4 Schritt Radius. Wenn ihnen eine Gegenprobe (Magieresistenz, 16) misslingt, sind alle Proben auf KL und IN um –2 erschwert, Proben auf Fertigkeiten mit KL und IN um –1. Erfordert Konzentration, erlaubt Aufrechterhalten.\newline Mächtige Magie: Der Malus steigt um –2/–1.\newline Probenschwierigkeit: Magieresistenz\newline Vorbereitungszeit: 4 Aktionen\newline Ziel: Zone\newline Reichweite: 4 Schritt\newline Wirkungsdauer: 16 Initiativphasen\newline Kosten: 4 AsP\newline Fertigkeiten: Eigenschaften, Zaubermelodien\newline Erlernen: Bard 16; 20 EP}
}


\newglossaryentry{melodiedesEinlullens_Talent}
{
    name={Melodie des Einlullens},
    description={Alle Humanoide in Hörreichweite sind von deiner immer berauschenderen Musik völlig gefesselt. Nur mit einer Konterprobe (Willenskraft, 16) kann sich ein Zuhörer für 16 Initiativphasen dem Zauber des Liedes entziehen. Ansonsten erwacht er, wenn du aufhörst zu spielen oder er eine Wunde erleidet. Erfordert Konzentration, erlaubt Aufrechterhalten.\newline Probenschwierigkeit: 12\newline Vorbereitungszeit: 16 Aktionen\newline Ziel: Zone\newline Reichweite: Berührung\newline Wirkungsdauer: 1 Stunde\newline Kosten: 8 AsP\newline Fertigkeiten: Einfluss, Zaubermelodien\newline Erlernen: Bard 16; 40 EP}
}


\newglossaryentry{melodiedesZauberschutzes_Talent}
{
    name={Melodie des Zauberschutzes},
    description={Magieresistenz-Proben aller Zuhörer im Radius von 4 Schritt sind um +2 erleichtert. Erfordert Konzentration, erlaubt Aufrechterhalten.\newline Mächtige Magie: Erhöht den Bonus um +1.\newline Probenschwierigkeit: 12\newline Vorbereitungszeit: 1 Aktion\newline Ziel: Zone\newline Reichweite: 4 Schritt\newline Wirkungsdauer: 16 Initiativphasen\newline Kosten: 4 AsP\newline Fertigkeiten: Antimagie, Eigenschaften, Zaubermelodien\newline Erlernen: Bard 16; 40 EP}
}


\newglossaryentry{schrillerKlang_Talent}
{
    name={Schriller Klang},
    description={Du entlockst deinem Instrument einen schrillen Klang, der für dein Ziel hundertfach verstärkt ist. Bis zu deiner nächsten Intitiativephase sind alle Proben des Ziels um –4 erschwert. Der schrille Klang wirkt nicht auf unheilige Wesen und Elementare.\newline Mächtige Magie: Der Malus steigt um –2.\newline Probenschwierigkeit: Magieresistenz\newline Vorbereitungszeit: 0 Aktionen\newline Ziel: Einzelwesen\newline Reichweite: 8 Schritt\newline Wirkungsdauer: augenblicklich\newline Kosten: 2 AsP\newline Fertigkeiten: Einfluss, Zaubermelodien\newline Erlernen: Bard 14; 40 EP}
}


\newglossaryentry{warnenderKlang_Talent}
{
    name={Warnender Klang},
    description={Du entlockst deinem Instrument einen alarmierenden Klang. Alle Zuhörer in 8 Meter Radius schrecken sofort auf und sind hellwach. Gegen folgende Angriffe gelten sie nicht als überrascht.\newline Probenschwierigkeit: 12\newline Vorbereitungszeit: 0 Aktionen\newline Ziel: Zone\newline Reichweite: Berührung\newline Wirkungsdauer: augenblicklich\newline Kosten: 4 AsP\newline Fertigkeiten: Eigenschaften, Zaubermelodien\newline Erlernen: Bard 14; 20 EP}
}


\newglossaryentry{tanzderBilder_Talent}
{
    name={Tanz der Bilder},
    description={Alle humanoiden Zuseher im Radius von 4 Schritt werden von einer raschen Abfolge von Bildern mit einer beruhigenden, heiteren oder alptraumhaften Grundstimmung (nach deiner Wahl) gefesselt. Die Wirkung endet, wenn du aufhörst zu tanzen. Erlaubt Aufrechterhalten, erfordert Konzentration.\newline Probenschwierigkeit: 12\newline Vorbereitungszeit: 16 Aktionen\newline Ziel: Zone\newline Reichweite: Berührung\newline Wirkungsdauer: 1 Stunde\newline Kosten: 4 AsP\newline Fertigkeiten: Illusion, Zaubertänze\newline Erlernen: Ztz 8; 0 EP}
}


\newglossaryentry{tanzderErholung_Talent}
{
    name={Tanz der Erholung},
    description={Dein Tanz verschafft allen Humanoiden im Radius von 4 Schritt einen erholsamen Schlaf. In ihrer nächsten Ruhepause regenerieren sie eine zusätzliche Wunde.\newline Probenschwierigkeit: 12\newline Vorbereitungszeit: 1 Stunde\newline Ziel: Zone\newline Reichweite: Berührung\newline Wirkungsdauer: augenblicklich\newline Kosten: 16 AsP\newline Fertigkeiten: Humus, Zaubertänze\newline Erlernen: Ztz 12; 40 EP}
}


\newglossaryentry{tanzderErlösung_Talent}
{
    name={Tanz der Erlösung},
    description={Der Tanz gilt als Konterprobe (12) gegen sämtliche Einflusszauber, Hexenflüche und Druidenrituale auf dem Ziel. Bei gelungener Probe werden die Zauber aufgehoben.\newline Probenschwierigkeit: 12\newline Vorbereitungszeit: 4 Stunden\newline Ziel: Einzelperson\newline Reichweite: 4 Schritt\newline Wirkungsdauer: augenblicklich\newline Kosten: 8 AsP\newline Fertigkeiten: Antimagie, Zaubertänze\newline Erlernen: Ztz 12; 20 EP}
}


\newglossaryentry{tanzderErmutigung_Talent}
{
    name={Tanz der Ermutigung},
    description={Der Tanz schenkt Zuversicht. Alle Zuhörer in einem Radius von 4 Schritt erhalten eine Erleichterung +2 auf Anführen und MU.\newline Mächtige Magie: Der Bonus steigt um +1.\newline Probenschwierigkeit: 12\newline Vorbereitungszeit: 1 Stunde\newline Ziel: Zone\newline Reichweite: Berührung\newline Wirkungsdauer: 1 Tag\newline Kosten: 16 AsP\newline Fertigkeiten: Eigenschaften, Einfluss, Zaubertänze\newline Erlernen: Ztz 10; 40 EP}
}


\newglossaryentry{tanzderFreude_Talent}
{
    name={Tanz der Freude},
    description={Der Tanz stärkt die Lebenskräfte und beendet ein Gift oder eine Krankheit bis maximal Stufe 16.\newline Mächtige Magie: Die maximal aufgehobene Gift-/Krankheitsstufe steigt um 4.\newline Probenschwierigkeit: 12\newline Vorbereitungszeit: 1 Stunde\newline Ziel: Einzelperson\newline Reichweite: 4 Schritt\newline Wirkungsdauer: augenblicklich\newline Kosten: 8 AsP\newline Fertigkeiten: Humus, Zaubertänze\newline Erlernen: Ztz 14; 20 EP}
}


\newglossaryentry{tanzderJagd_Talent}
{
    name={Tanz der Jagd},
    description={Der Tanz schärft die Jagdinstinkte der Zuhörer. Alle Zuhörer in einem Radius von 4 Schritt erhalten eine Erleichterung +2 auf Pirschen, Überleben, Tierkunde und GE-Proben.\newline Mächtige Magie: Der Bonus steigt um +1.\newline Probenschwierigkeit: 12\newline Vorbereitungszeit: 1 Stunde\newline Ziel: Zone\newline Reichweite: Berührung\newline Wirkungsdauer: 1 Tag\newline Kosten: 16 AsP\newline Fertigkeiten: Eigenschaften, Zaubertänze\newline Erlernen: Ztz 16; 40 EP}
}


\newglossaryentry{tanzderBetörung_Talent}
{
    name={Tanz der Betörung},
    description={Das Ziel entbrennt in heißer Leidenschaft zu dir. Auf einer Skala von abstoßend/uninteressant/neutral/begehrenswert/unwiderstehlich steigt seine Einstellung dir gegenüber um eine Stufe.\newline Probenschwierigkeit: Magieresistenz\newline Modifikationen: Tanz des Begehrens (–8, Zone, Wirkungsdauer 1 Stunde, alle Zuseher im Radius von 4 Schritt, denen eine Konterprobe (MR, 12) misslingt, entbrennen in Leidenschaft zu dir)\newline Vorbereitungszeit: 1 Stunde\newline Ziel: Einzelperson\newline Reichweite: 4 Schritt\newline Wirkungsdauer: 1 Tag\newline Kosten: 8 AsP\newline Fertigkeiten: Einfluss, Verständigung, Zaubertänze\newline Erlernen: Ztz 14; 20 EP}
}


\newglossaryentry{tanzderUnantastbarkeit_Talent}
{
    name={Tanz der Unantastbarkeit},
    description={Der Tanz verbessert deinen Gleichgewichtssinn und deine Körperbeherrschung. Du erhältst den Vorteil Körperbeherrschung (S. 58). Besitzt du den Vorteil bereits, erleidest du beim Einsatz des Vorteils keine Erschöpfung.\newline Probenschwierigkeit: 12\newline Vorbereitungszeit: 4 Minuten\newline Ziel: selbst\newline Reichweite: Berührung\newline Wirkungsdauer: 1 Stunde\newline Kosten: 8 AsP\newline Fertigkeiten: Eigenschaften, Zaubertänze\newline Erlernen: Ztz 18; 40 EP}
}


\newglossaryentry{tanzderWahrheit_Talent}
{
    name={Tanz der Wahrheit},
    description={Du siehst die Gedanken deines Ziels als verschwommene Bilder vor dir. Weiß das Ziel, dass seine Gedanken gelesen werden, kann es dich mit einer Konterprobe (Willenskraft, 16) in die Irre führen. Die Wirkung endet, wenn du aufhörst zu tanzen. Erlaubt Aufrechterhalten, erfordert Konzentration.\newline Mächtige Magie: Du erhältst einen deutlichen/klaren Eindruck in die Gedanken deines Ziels.\newline Probenschwierigkeit: Magieresistenz\newline Modifikationen: Tanz des Wahrheitssinns (–8, Zone, 16 AsP; du liest die Gedanken aller Humanoiden im Radius von 4 Schritt, denen eine Konterprobe (MR, 12) misslingt.)\newline Vorbereitungszeit: 16 Aktionen\newline Ziel: Einzelperson\newline Reichweite: 4 Schritt\newline Wirkungsdauer: 16 Initiativphasen\newline Kosten: 8 AsP\newline Fertigkeiten: Hellsicht, Zaubertänze\newline Erlernen: Ztz 16; 40 EP}
}


\newglossaryentry{tanzderWeisheit_Talent}
{
    name={Tanz der Weisheit},
    description={Der Tanz schärft den Geist der Zuseher. Alle Zuseher in einem Radius von 4 Schritt erhalten eine Erleichterung +2 auf Derekunde, Heilkunde, Mythenkunde und KL.\newline Mächtige Magie: Der Bonus steigt um +1.\newline Probenschwierigkeit: 12\newline Vorbereitungszeit: 1 Stunde\newline Ziel: Zone\newline Reichweite: Berührung\newline Wirkungsdauer: 1 Tag\newline Kosten: 16 AsP\newline Fertigkeiten: Eigenschaften, Zaubertänze\newline Erlernen: Ztz 18; 40 EP}
}


\newglossaryentry{tanzdesMondes_Talent}
{
    name={Tanz des Mondes},
    description={Mit diesem Tanz kannst du in die Träume eines schlafenden Zieles eindringen und ihm dort Botschaften überbringen. Woran sich das Ziel erinnern kann, ist Spielleiterentscheid. Du musst eine emotionale Beziehung zum Ziel haben.\newline Probenschwierigkeit: Magieresistenz\newline Vorbereitungszeit: 1 Stunde\newline Ziel: Einzelperson\newline Reichweite: 100 Meilen\newline Wirkungsdauer: 1 Stunde\newline Kosten: 8 AsP\newline Fertigkeiten: Verständigung, Zaubertänze\newline Erlernen: Ztz 18; 40 EP}
}


\newglossaryentry{tanzdesUngehorsams_Talent}
{
    name={Tanz des Ungehorsams},
    description={MR-Proben des Ziels sind um +4 erleichtert.\newline Mächtige Magie: Erhöht den Bonus um +2\newline Probenschwierigkeit: 12\newline Vorbereitungszeit: 1 Stunde\newline Ziel: Einzelperson\newline Reichweite: 4 Schritt\newline Wirkungsdauer: 1 Tag\newline Kosten: 8 AsP\newline Fertigkeiten: Antimagie, Eigenschaften, Zaubertänze\newline Erlernen: Ztz 18; 40 EP}
}


\newglossaryentry{tanzohneEnde_Talent}
{
    name={Tanz ohne Ende},
    description={Alle Humanoide in Hörreichweite sind von deinem immer berauschenderen Tanz völlig gefesselt. Nur mit einer Konterprobe (Willenskraft, 16) kann sich ein Zuhörer für 16 Initiativphasen dem Zauber entziehen. Ansonsten erwacht er, wenn du aufhörst zu spielen oder er eine Wunde erleidet. Erfordert Konzentration, erlaubt Aufrechterhalten.\newline Probenschwierigkeit: 12\newline Vorbereitungszeit: 16 Aktionen\newline Ziel: Zone\newline Reichweite: Berührung\newline Wirkungsdauer: 1 Stunde\newline Kosten: 8 AsP\newline Fertigkeiten: Einfluss, Zaubertänze\newline Erlernen: Ztz 14; 40 EP}
}


\newglossaryentry{eidsegen_Talent}
{
    name={Eidsegen},
    description={Ein freiwilliges und aufrichtiges Gelübde der Zielperson wird gesegnet. Alle Versuche, den Schwörenden zum Eidbruch zu bewegen (etwa über Betören, Überreden oder einen Imperavi), sind um –4 erschwert. Bricht der Schwörende den Eid im Vollbesitz seiner geistigen Kräfte, endet der Segen.\newline Probenschwierigkeit: 12\newline Vorbereitungszeit: 16 Aktionen\newline Ziel: Einzelperson\newline Reichweite: Berührung\newline Wirkungsdauer: 1 Jahr\newline Kosten: 8 KaP\newline Fertigkeiten: Ordnung, Zwölfgöttlicher Ritus\newline Erlernen: Pra 4; Ang, Aves, Eff, Brn, Fir, Hes, Ifi, Ing, Kor, Nan, Per, Phe, Rah, Ron, Swa, Tra, Tsa 8; 0 EP}
}


\newglossaryentry{feuersegen_Talent}
{
    name={Feuersegen},
    description={Eine kleine Flamme entsteht in deiner Hand. Die Flamme und ein mit ihr entzündetes Feuer wird durch Wind oder Regen nicht gelöscht.\newline Probenschwierigkeit: 12\newline Vorbereitungszeit: 2 Aktionen\newline Ziel: Zone\newline Reichweite: Berührung\newline Wirkungsdauer: 1 Stunde\newline Kosten: 1 KaP\newline Fertigkeiten: Heiliges Feuer, Zwölfgöttlicher Ritus\newline Erlernen: Ang, Ing 4; Aves, Brn, Fir, Hes, Ifi, Kor, Nan, Per, Phe, Pra, Rah, Ron, Swa, Tra, Tsa 8; Eff 12; 0 EP}
}


\newglossaryentry{geburtssegen_Talent}
{
    name={Geburtssegen},
    description={Das gesegnete Kind ist immun gegen Einflüsterungen von Dämonen und Kobolden.\newline Probenschwierigkeit: 12\newline Vorbereitungszeit: 16 Aktionen\newline Ziel: Einzelperson\newline Reichweite: Berührung\newline Wirkungsdauer: bis zum 12. Lebensjahr\newline Kosten: 2 KaP\newline Fertigkeiten: Zwölfgöttlicher Ritus, Neubeginn\newline Erlernen: Tsa 4; Aves, Brn, Eff, Fir, Hes, Ifi, Ing, Nan, Per, Phe, Pra, Rah, Ron, Swa, Tra 8; Kor 12; 0 EP}
}


\newglossaryentry{glückssegen_Talent}
{
    name={Glückssegen},
    description={Der Gesegnete darf eine Probe wiederholen und das bessere Ergebnis wählen. Der Segen kann nur für Handlungen eingesetzt werden, denen deine Gottheit zumindest neutral gegenübersteht.\newline Probenschwierigkeit: 12\newline Vorbereitungszeit: 1 Aktion\newline Ziel: Einzelperson\newline Reichweite: Berührung\newline Wirkungsdauer: 1 Stunde\newline Kosten: 4 KaP\newline Fertigkeiten: List, Zwölfgöttlicher Ritus\newline Erlernen: Aves, Phe 4; Ang, Brn, Eff, Fir, Hes, Ifi, Ing, Kor, Nan, Per, Phe, Pra, Rah, Ron, Swa, Tra, Tsa 8; 40 EP}
}


\newglossaryentry{göttlicheVerständigung_Talent}
{
    name={Göttliche Verständigung},
    description={Du kannst der Vorsteherin deines Heimattempels eine Nachricht mit bis zu 8 Worten schicken.\newline Mächtige Liturgie: Die Nachricht kann 8 weitere Worte lang sein.\newline Probenschwierigkeit: 12\newline Modifikationen: Bekannter Empfänger (–4; die Nachricht kann an einen beliebigen, dir bekannten Geweihten gehen.)\newline Unbekannter Empfänger (–8, 16 KaP; die Nachricht kann an einen beliebigen Tempel geschickt werden.)\newline Vorbereitungszeit: 4 Aktionen\newline Ziel: selbst\newline Reichweite: dereweit\newline Wirkungsdauer: augenblicklich\newline Kosten: 8 KaP\newline Fertigkeiten: Zwölfgöttlicher Ritus\newline Erlernen: Ang, Aves, Brn, Eff, Fir, Hes, Ifi, Ing, Kor, Nan, Per, Phe, Pra, Rah, Ron, Tra, Tsa, Swa 8; 40 EP}
}


\newglossaryentry{grabsegen_Talent}
{
    name={Grabsegen},
    description={Du segnest ein Grab, das fortan als geweihter Boden gilt. Beschwörungen von unheiligen Wesenheiten sind um –8 erschwert.\newline Mächtige Liturgie: Der Malus steigt um –4.\newline Probenschwierigkeit: 12\newline Vorbereitungszeit: 16 Aktionen\newline Ziel: Zone\newline Reichweite: Berührung\newline Wirkungsdauer: permanent\newline Kosten: 2 KaP\newline Fertigkeiten: Tod, Zwölfgöttlicher Ritus\newline Erlernen: Brn 4; Aves, Eff, Fir, Hes, Ifi, Ing, Kor, Nan, Per, Phe, Pra, Rah, Ron, Tra, Tsa, Swa 8; 0 EP}
}


\newglossaryentry{großerEidsegen_Talent}
{
    name={Großer Eidsegen},
    description={Du segnest einen freiwillig geschworenen Eid, wie einen Traviabund oder einen Lehens­eid. Alle Versuche, einen Schwörenden zum Eidbruch zu bewegen (etwa über Betören, Überreden oder einen Imperavi), sind um –4 erschwert. Bricht ein Schwörender den Eid im Vollbesitz seiner geistigen Kräfte, ruht die Wirkung des Segens, bis er Buße geübt hat.\newline Mächtige Liturgie: Erhöht den Malus um –2.\newline Probenschwierigkeit: 12\newline Vorbereitungszeit: 1 Stunde\newline Ziel: zwei Personen\newline Reichweite: Berührung\newline Wirkungsdauer: permanent\newline Kosten: 8 KaP\newline Fertigkeiten: Ehre, Heim und Herd, Ordnung, Zwölfgöttlicher Ritus\newline Erlernen: Tra 8; Ang, Hes, Per, Pra, Rah, Swa, Tsa 14; Aves, Kor, Ron 18; 0 EP}
}


\newglossaryentry{harmoniesegen_Talent}
{
    name={Harmoniesegen},
    description={Der Gesegnete empfindet tiefe Harmonie und strahlt diese auch aus. Proben um Frieden und Harmonie zu stiften sind um +4 erleichtert.\newline Probenschwierigkeit: 12\newline Vorbereitungszeit: 4 Aktionen\newline Ziel: Einzelperson\newline Reichweite: Berührung\newline Wirkungsdauer: 1 Stunde\newline Kosten: 4 KaP\newline Fertigkeiten: Harmonie, Zwölfgöttlicher Ritus\newline Erlernen: Rah 4; Aves, Brn, Eff, Fir, Hes, Ifi, Ing, Nan, Per, Phe, Pra, Ron, Tra, Tsa. Swa 8; Kor 18; 20 EP}
}


\newglossaryentry{heilungssegen_Talent}
{
    name={Heilungssegen},
    description={Der Gesegnete regeneriert eine Wunde.\newline Mächtige Liturgie: Für je zwei Stufen regeneriert der Gesegnete eine weitere Wunde.\newline Probenschwierigkeit: 12\newline Vorbereitungszeit: 16 Minuten\newline Ziel: Einzelperson\newline Reichweite: Berührung\newline Wirkungsdauer: augenblicklich\newline Kosten: 2 KaP\newline Fertigkeiten: Heilung, Zwölfgöttlicher Ritus\newline Erlernen: Ifi, Per 4; Ang, Aves, Brn, Eff, Fir, Hes, Ing, Nan, Phe, Pra, Rah, Ron, Tra, Tsa, Swa 8; Kor 14; 40 EP}
}


\newglossaryentry{liturgischeBindung_Talent}
{
    name={Liturgische Bindung},
    description={Mit dieser Liturgie kannst du Liturgien in einem Gegenstand speichern. Die Regeln entsprechen jenen zur Erschaffung magischer Artefakte (S. 78), allerdings können nur ladungsbasierte Artefakte erschaffen werden.\newline Probenschwierigkeit: 12\newline Vorbereitungszeit: 8 Stunden\newline Ziel: Einzelobjekt\newline Reichweite: Berührung\newline Wirkungsdauer: nach Artefakt\newline Kosten: 8 KaP\newline Fertigkeiten: Zwölfgöttlicher Ritus\newline Erlernen: Ang, Ing 18; Aves, Brn, Eff, Fir, Hes, Ifi, Kor, Nan, Per, Phe, Pra, Rah, Ron, Tsa, Tra, Swa 20; 60 EP\newline Anmerkung: Geweihte Artefakte werden nur sehr selten an Laien vergeben. Am ehesten kommen Gläubige auf einer gottgefälligen Queste in Frage.}
}


\newglossaryentry{märtyrersegen_Talent}
{
    name={Märtyrersegen},
    description={Der Gesegnete gewinnt an Tapferkeit. Auf ihm lastende Furcht-Effekte gelten als 1 Stufe niedriger.\newline Probenschwierigkeit: 12\newline Vorbereitungszeit: 4 Aktionen\newline Ziel: Einzelperson\newline Reichweite: Berührung\newline Wirkungsdauer: 1 Stunde\newline Kosten: 4 KaP\newline Fertigkeiten: Winter, Zwölfgöttlicher Ritus, Guter Kampf\newline Erlernen: Fir, Kor 4; Ang, Aves, Brn, Eff, Hes, Ifi, Ing, Nan, Per, Phe, Pra, Rah, Ron, Tra, Tsa, Swa 8; 20 EP}
}


\newglossaryentry{objektsegen_Talent}
{
    name={Objektsegen},
    description={Du segnest ein Objekt bis zur Größe eines Rucksacks oder eine Substanz mit bis zu 8 Litern Volumen. Das Objekt gilt als geweiht.\newline Mächtige Liturgie: Verdoppelt die Größe des Objekts bzw. das Volumen.\newline Probenschwierigkeit: 12\newline Modifikationen: Bauwerk (–8; du kannst größere Objekte wie Brücken oder Stollen segnen. Kosten und Wirkungsdauer sind deutlich höher und Spielleiterentscheid.)\newline Vorbereitungszeit: 4 Minuten\newline Ziel: Einzelobjekt\newline Reichweite: Berührung\newline Wirkungsdauer: 1 Tag\newline Kosten: 4 KaP\newline Fertigkeiten: Zwölfgöttlicher Ritus\newline Erlernen: Ang, Aves, Brn, Eff, Fir, Hes, Ifi, Ing,  Kor, Nan, Per, Phe, Pra, Rah, Ron, Tra, Tsa, Swa 8; 20 EP}
}


\newglossaryentry{schutzsegen_Talent}
{
    name={Schutzsegen},
    description={Du ziehst einen Kreis von maximal 8 Schritt Radius und nennst eine Kreaturenklasse (Dämon, Daimonoid oder Untoter). Liegt die Beschwörungsschwierigkeit eines dieser Wesen bei maximal 20, kann es die Zone nicht betreten. Der Boden innerhalb des Schutzkreises gilt als geweihter Boden.\newline Mächtige Liturgie: Erhöht die Beschwörungsschwierigkeit um 4.\newline Probenschwierigkeit: 12\newline Modifikationen: Universeller Schutz (–4; der Segen wirkt gegen alle unheiligen Wesen.)\newline Vorbereitungszeit: 4 Aktionen\newline Ziel: Zone\newline Reichweite: Berührung\newline Wirkungsdauer: 1 Stunde\newline Kosten: 4 KaP\newline Fertigkeiten: Schutz der Gläubigen, Zwölfgöttlicher Ritus\newline Erlernen: Ron 4; Ang, Brn, Eff, Fir, Ifi, Kor, Pra, Swa 8; Aves, Hes, Ing, Nan, Per, Phe, Rah, Tsa, Tra 12; 40 EP}
}


\newglossaryentry{seelenprüfung_Talent}
{
    name={Seelenprüfung},
    description={Du erkennst, ob eine Person oder ein Ort geweiht (auch dem Namenlosen), profan oder dämonisch verzerrt ist.\newline Mächtige Liturgie: Du kannst den Dämonen oder Gott/den Grad der Verdammnis oder der Weihe erkennen.\newline Probenschwierigkeit: 12\newline Vorbereitungszeit: 1 Stunde\newline Ziel: Einzelperson\newline Reichweite: Berührung\newline Wirkungsdauer: augenblicklich\newline Kosten: 8 KaP\newline Fertigkeiten: Zwölfgöttlicher Ritus\newline Erlernen: Pra 8; Ang, Aves, Brn, Eff, Fir, Hes, Ifi, Ing, Kor, Nan, Per, Phe, Rah, Ron, Tra, Tsa, Swa 12; 40 EP}
}


\newglossaryentry{speisesegen_Talent}
{
    name={Speisesegen},
    description={Du reinigst eine Mahlzeit für bis zu 16 Personen von Schmutz, Fäulnis und Krankheiten.\newline Probenschwierigkeit: 12\newline Vorbereitungszeit: 4 Aktionen\newline Ziel: Zone\newline Reichweite: Berührung\newline Wirkungsdauer: augenblicklich\newline Kosten: 2 KaP\newline Fertigkeiten: Heim und Herd, Zwölfgöttlicher Ritus\newline Erlernen: Tra 4; Ang, Aves, Eff, Brn, Fir, Hes, Ifi, Ing, Kor, Nan, Per, Phe, Pra, Rah, Ron, Tsa, Swa 8; 0 EP}
}


\newglossaryentry{tranksegen_Talent}
{
    name={Tranksegen},
    description={Du reinigst 4 Liter Wasser von Schmutz, Fäulnis, Meersalz oder Krankheiten.\newline Probenschwierigkeit: 12\newline Vorbereitungszeit: 4 Aktionen\newline Ziel: Einzelobjekt\newline Reichweite: Berührung\newline Wirkungsdauer: augenblicklich\newline Kosten: 2 KaP\newline Fertigkeiten: Seefahrt, Zwölfgöttlicher Ritus\newline Erlernen: Eff, Swa 4; Aves, Brn, Fir, Hes, Ifi, Ing, Kor, Nan, Per, Phe, Pra, Rah, Ron, Tra, Tsa 8; 0 EP}
}


\newglossaryentry{weisheitssegen_Talent}
{
    name={Weisheitssegen},
    description={Proben des Gesegneten auf KL sind um +4 erleichtert, alle Fertigkeiten mit KL um +2.\newline Probenschwierigkeit: 12\newline Vorbereitungszeit: 4 Aktionen\newline Ziel: Einzelperson\newline Reichweite: Berührung\newline Wirkungsdauer: 1 Stunde\newline Kosten: 4 KaP\newline Fertigkeiten: Wissen, Zwölfgöttlicher Ritus\newline Erlernen: Hes, Nan 4; Ang, Aves, Brn, Eff, Fir, Ifi, Ing, Kor, Per, Phe, Pra, Rah, Ron, Tra, Tsa Swa 8; 20 EP}
}


\newglossaryentry{augedesMondes_Talent}
{
    name={Auge des Mondes},
    description={Das Ziel ignoriert eine Stufe Dunkelheit.\newline Mächtige Liturgie: Das Ziel ignoriert 2 Stufen/3 Stufen/absolute Dunkelheit.\newline Probenschwierigkeit: 12\newline Vorbereitungszeit: 16 Aktionen\newline Ziel: Einzelperson\newline Reichweite: Berührung\newline Wirkungsdauer: 1 Stunde\newline Kosten: 4 KaP\newline Fertigkeiten: Nächtlicher Schatten, Schlaf\newline Erlernen: Phe 12, Brn 14; 40 EP}
}


\newglossaryentry{bishdarielsAuge_Talent}
{
    name={Bishdariels Auge},
    description={Du erhältst einen kurzen Einblick in den Traum deines Ziels. Über den Inhalt musst du anderen gegenüber schweigen.\newline Probenschwierigkeit: 12\newline Modifikationen: Traumreise (–4, 8 KaP; du kannst in den Traum des Schlafenden reisen.)\newline Große Traumreise (–8, 16 KaP; du kannst mit einigen Gefährten in den Traum reisen.)\newline Vorbereitungszeit: 16 Aktionen\newline Ziel: Einzelperson\newline Reichweite: Berührung\newline Wirkungsdauer: 1 Stunde\newline Kosten: 4 KaP\newline Fertigkeiten: Schlaf\newline Erlernen: Brn 12; 20 EP}
}


\newglossaryentry{bishdarielsWarnung_Talent}
{
    name={Bishdariels Warnung},
    description={Dein Ziel wird Nacht für Nacht von der Albtraumgestalt Bishdariels heimgesucht, die Warnungen und Mahnungen überbringt. Es regeneriert in dieser Zeit nicht.\newline Mächtige Liturgie: Ab dem folgenden Tag sind alle Proben des Ziels sind um –1/–2/ –3/–4 erschwert.\newline Probenschwierigkeit: 12\newline Vorbereitungszeit: 1 Stunde\newline Ziel: Einzelperson\newline Reichweite: dereweit\newline Wirkungsdauer: 1 Woche\newline Kosten: 16 KaP\newline Fertigkeiten: Schlaf\newline Erlernen: Brn 14; 40 EP}
}


\newglossaryentry{hauchBorons_Talent}
{
    name={Hauch Borons},
    description={Deine Umgebung wird im Radius von 8 Schritt von absoluter Dunkelheit erfüllt, die dich selbst nicht einschränkt. Das Gebiet gilt als geweihter Boden. Die Liturgie gilt zusätzlich als Konterprobe (12) gegen Lichtzauber, die bei Gelingen aufgehoben werden.\newline Mächtige Liturgie: Mit zwei Stufen gilt der Boden als heilig.\newline Probenschwierigkeit: 12\newline Modifikationen: Begleiter (–4, Wirkungsdauer 4 Minuten; die Dunkelheit bewegt sich mit dir.)\newline Vorbereitungszeit: 16 Aktionen\newline Ziel: Zone\newline Reichweite: Berührung\newline Wirkungsdauer: 1 Stunde\newline Kosten: 8 KaP\newline Fertigkeiten: Schlaf\newline Erlernen: Brn 14; 40 EP}
}


\newglossaryentry{rauschsegen_Talent}
{
    name={Rauschsegen},
    description={Die gesegneten Rauschmittel entfalten stärkere Wirkung bei geringeren Nebenwirkungen und machen nicht mehr abhängig.\newline Probenschwierigkeit: 12\newline Vorbereitungszeit: 16 Aktionen\newline Ziel: Einzelobjekt\newline Reichweite: Berührung\newline Wirkungsdauer: 1 Tag\newline Kosten: 1 KaP\newline Fertigkeiten: Rausch, Schlaf\newline Erlernen: Brn, Rah 12; 20 EP}
}


\newglossaryentry{rufinBoronsArme_Talent}
{
    name={Ruf in Borons Arme},
    description={Das Ziel fällt in einen tiefen Schlaf, während dem Gifte und Krankheiten nicht wirken und es nicht von Albträumen geplagt wird. Alle Versuche, in den Traum des Ziels einzudringen oder sie zu manipulieren, sind um –4 erschwert.\newline Mächtige Liturgie: Der Malus steigt um –2.\newline Probenschwierigkeit: 12\newline Vorbereitungszeit: 8 Aktionen\newline Ziel: Einzelperson\newline Reichweite: Berührung\newline Wirkungsdauer: 8 Stunden\newline Kosten: 2 KaP\newline Fertigkeiten: Schlaf\newline Erlernen: Brn 4; 20 EP}
}


\newglossaryentry{schlafdesGesegneten_Talent}
{
    name={Schlaf des Gesegneten},
    description={Dein Ziel sinkt in einen tiefen Schlaf, während dem es 2 zusätzliche Wunden regeneriert.\newline Mächtige Liturgie: Das Ziel regeneriert 1 weitere Wunde.\newline Probenschwierigkeit: 12\newline Vorbereitungszeit: 16 Aktionen\newline Ziel: Einzelperson\newline Reichweite: Berührung\newline Wirkungsdauer: 8 Stunden\newline Kosten: 4 KaP\newline Fertigkeiten: Harmonie, Heilung, Schlaf, Stiller Wanderer\newline Erlernen: Brn, Per 12; Aves, Rah 16; 40 EP}
}


\newglossaryentry{bannfluchdesHeiligenKhalid_Talent}
{
    name={Bannfluch des Heiligen Khalid},
    description={Während der Wirkungsdauer zerfallen die ersten 4 Untoten, die du mit einem geweihten Gegenstand (wie Graberde, Weihrauch oder deiner Hand) berührst, zu Staub. Geister werden gebannt. Wirkt nur auf Wesen, deren Beschwörungsschwierigkeit bei maximal 12 liegt.\newline Mächtige Liturgie: Zwei weitere Untote oder Geister können gebannt werden und die Beschwörungsschwierigkeit steigt um 4.\newline Probenschwierigkeit: 12\newline Vorbereitungszeit: 8 Aktionen\newline Ziel: selbst\newline Reichweite: Berührung\newline Wirkungsdauer: 8 Initiativphasen\newline Kosten: 16 KaP\newline Fertigkeiten: Tod\newline Erlernen: Brn 14; 40 EP}
}


\newglossaryentry{etiliasGnade_Talent}
{
    name={Etilias Gnade},
    description={Ein Sterbender leidet keine Schmerzen mehr.\newline Probenschwierigkeit: 12\newline Modifikationen: Marbos Aufschub (–8, 1 Stunde, 16 KaP; ein langsamer Tod durch Siechtum oder Schwäche verzögert sich zusätzlich um 1 Woche. Der Gesegnete ist jedoch nicht immun gegen einen gewaltsamen Tod.)\newline Vorbereitungszeit: 16 Aktionen\newline Ziel: Einzelperson\newline Reichweite: Berührung\newline Wirkungsdauer: 1 Woche\newline Kosten: 4 KaP\newline Fertigkeiten: Tod\newline Erlernen: Brn 8; 20 EP}
}


\newglossaryentry{exorzismus_Talent}
{
    name={Exorzismus},
    description={Du verbannst einen Dämon.\newline Probenschwierigkeit: Beschwörungsschwierigkeit des Dämons -4\newline Vorbereitungszeit: 8 Aktionen\newline Ziel: Einzelwesen\newline Reichweite: 32 Schritt\newline Wirkungsdauer: augenblicklich\newline Kosten: halbe Basiskosten d. Beschwörung in KaP\newline Fertigkeiten: Abu al'Mada, Heiliges Feuer, Magie, Magiebann, Schutz der Gläubigen, Tod, Wind und Wogen\newline Erlernen: Hes, Pra, Ron 8; Eff 12; Brn, Phe 14; Ang, Ing 16; 60 EP}
}


\newglossaryentry{marbosGeleit_Talent}
{
    name={Marbos Geleit},
    description={Deine Seele verlässt kurzzeitig deinen Körper und begleitet und beschützt die des Verstorbenen auf dem Weg über das Nirgendmeer.\newline Probenschwierigkeit: 12\newline Vorbereitungszeit: 1 Stunde\newline Ziel: Einzelperson\newline Reichweite: Berührung\newline Wirkungsdauer: bis zu 1 Tag\newline Kosten: 16 KaP\newline Fertigkeiten: Tod\newline Erlernen: Brn 18; 0 EP}
}


\newglossaryentry{nemekathsGeisterblick_Talent}
{
    name={Nemekaths Geisterblick },
    description={Du siehst unsichtbare Geister, Dämonen oder Elementarwesen.\newline Mächtige Liturgie: Du kannst die Geister hören/mit Geistern sprechen/Geister um Hilfe bitten (die sie normalerweise gewähren).\newline Probenschwierigkeit: 12\newline Vorbereitungszeit: 16 Aktionen\newline Ziel: selbst\newline Reichweite: Berührung\newline Wirkungsdauer: 1 Stunde\newline Kosten: 8 KaP\newline Fertigkeiten: Tod\newline Erlernen: Brn 16; 20 EP}
}


\newglossaryentry{tiergestalt_Talent}
{
    name={Tiergestalt},
    description={Du verwandelst dich in das heilige Tier deiner Gottheit, wobei du deine geistigen Fähigkeiten behältst. Deine körperlichen Fähigkeiten entsprechen denen des Tiers und du kannst in Tiergestalt keine Liturgien wirken. Erlaubt Aufrechterhalten.\newline Mächtige Liturgie: Das Tier ist ein überdurchschnittlicher/außergewöhnlicher/herausragender/einzigartiger Vertreter seiner Art. Die Werte des Tieres sind nach Spielleiterentscheid erhöht.\newline Probenschwierigkeit: 12\newline Vorbereitungszeit: 16 Aktionen\newline Ziel: selbst\newline Reichweite: Berührung\newline Wirkungsdauer: 4 Stunden\newline Kosten: 8 KaP\newline Fertigkeiten: Ehre, Guter Kampf, Jagd, List, Magie, Tod, Wind und Wogen\newline Erlernen: Fir, Ifi 14; Eff, Ron 16; Bor, Hes, Kor, Phe 18; normalerweise 40 EP\newline Anmerkungen: Nur herausragende Geweihte können sich in Tiere verwandeln, die nicht zu den Größenklassen klein oder mittel gehören. Solche Verwandlungen sind prinzipiell um –8 erschwert. Die Lernkosten des Zaubers orientieren sich an der Tierart, wobei fliegende, giftige, sehr starke usw. Tiere teurer sind.\newline Sephrasto: Trage das entsprechende Tier in das Kommentarfeld ein.}
}


\newglossaryentry{waffenweihe_Talent}
{
    name={Waffenweihe},
    description={Du segnest eine Waffe. Diese gilt als geweiht.\newline Probenschwierigkeit: 12\newline Vorbereitungszeit: 1 Stunde\newline Ziel: Einzelobjekt\newline Reichweite: Berührung\newline Wirkungsdauer: bis die Bindung gelöst wird\newline Kosten: 8 KaP, davon 2 gKaP\newline Fertigkeiten: Gutes Gold, Magiebann, Heiliges Feuer, Heerführung, Tod\newline Erlernen: Ron 12; Ang, Ing, Kor 14; Brn, Pra 18; 60 EP\newline Anmerkung: Geweihte Waffen werden nur in seltenen Ausnahmefällen und nach sorgfältiger Überprüfung an Nicht-Geweihte ausgegeben. }
}


\newglossaryentry{weihederletztenRuhestatt_Talent}
{
    name={Weihe der letzten Ruhestatt},
    description={Du segnest einen Boronsanger, der fortan als geweihter Boden gilt und Platz für etwa 32 Gräber bietet. Beschwörungen von unheiligen Wesenheiten sind um –8 erschwert.\newline Mächtige Liturgie: Verdoppelt die Größe.\newline Vorbereitungszeit: 1 Stunde\newline Ziel: Zone\newline Reichweite: Berührung\newline Wirkungsdauer: permanent\newline Kosten: 16 KaP\newline Fertigkeiten: Tod\newline Erlernen: Brn 12; 20 EP}
}


\newglossaryentry{boronssüßeGnade_Talent}
{
    name={Borons süße Gnade},
    description={Du stellst verlorene Erinnerungen wieder her oder nimmst deinem Ziel die Erinnerung an bestimmte Ereignisse, wenn ihm eine Konterprobe (KL, 20) misslingt. Du kannst Erinnerungen an maximal einen Zeitraum von einer Woche beeinflussen.\newline Mächtige Liturgie: Der Zeitraum beträgt maximal 1 Monat/Jahr/Jahrzehnt/unbegrenzt.\newline Probenschwierigkeit: 12\newline Vorbereitungszeit: 1 Stunde\newline Ziel: Einzelperson\newline Reichweite: Berührung\newline Wirkungsdauer: permanent\newline Kosten: 8 KaP\newline Fertigkeiten: Vergessen\newline Erlernen: Bor 18; 40 EP}
}


\newglossaryentry{heiligerBefehl_Talent}
{
    name={Heiliger Befehl},
    description={Das Ziel befolgt einen gottgefälligen Befehl, wenn ihm keine Konterprobe (Willenskraft, 20) gelingt.\newline Probenschwierigkeit: 12\newline Vorbereitungszeit: 4 Aktionen\newline Ziel: Einzelperson\newline Reichweite: 32 Schritt\newline Wirkungsdauer: 1 Tag\newline Kosten: 8 KaP\newline Fertigkeiten: Ehre, Gutes Gold, Ordnung, Rausch, Vergessen\newline Erlernen: Pra 8; Brn, Ron 12; Kor, Rah 14; 60 EP}
}


\newglossaryentry{innereRuhe_Talent}
{
    name={Innere Ruhe},
    description={Du besinnst dich auf deine innere Kraft. Proben auf Selbstbeherrschung sind um +4 erleichtert.\newline Mächtige Liturgie: Der Bonus steigt um +2. Probenschwierigkeit: 12\newline Vorbereitungszeit: 4 Aktionen\newline Ziel: selbst\newline Reichweite: Berührung\newline Wirkungsdauer: 8 Stunden\newline Kosten: 8 KaP\newline Fertigkeiten: Ordnung, Vergessen\newline Erlernen: Brn, Pra 12; 40 EP}
}


\newglossaryentry{segenderHeiligenNoiona_Talent}
{
    name={Segen der Heiligen Noiona},
    description={Du erhältst während der Wirkungsdauer die Möglichkeit, mit deinem Probenwert in Heilkunde Gifte/Krankheiten auch Sucht- und Geisteskrankheiten zu heilen.\newline Probenschwierigkeit: 12\newline Vorbereitungszeit: 1 Stunde\newline Ziel: selbst\newline Reichweite: Berührung\newline Wirkungsdauer: 1 Woche\newline Kosten: 8 KaP\newline Fertigkeiten: Heilung, Heim und Herd, Vergessen\newline Erlernen: Brn 14; Per, Tra 16; 20 EP}
}


\newglossaryentry{segenderHeiligenVelvenya_Talent}
{
    name={Segen der Heiligen Velvenya},
    description={Mit dem Segen Borons ist dein Ziel frei von Schlafstörungen, Albträumen, Todesängsten und quälenden Erinnerungen. Alle Furcht-Effekte gelten um eine Stufe niedriger.\newline Probenschwierigkeit: 12\newline Modifikationen: Permanenz (–4, Wirkungsdauer bis die Bindung gelöst wird, 8 KaP, davon 2 gKaP)\newline Vorbereitungszeit: 1 Stunde\newline Ziel: Einzelperson\newline Reichweite: Berührung\newline Wirkungsdauer: 1 Monat\newline Kosten: 8 KaP\newline Fertigkeiten: Vergessen\newline Erlernen: Brn 14; 40 EP}
}


\newglossaryentry{siegelBorons_Talent}
{
    name={Siegel Borons},
    description={Dein Ziel kann über ein bestimmtes Wissen oder Ereignis nicht sprechen oder es auf irgendeine andere Art mitteilen, außer es gelingt ihm eine Konterprobe (Willenskraft, 20).\newline Probenschwierigkeit: 12\newline Modifikationen: Umfassendes Siegel (–8, 16 KaP; das Ziel darf über einen großen Bereich seines Lebens nicht sprechen.)\newline Vorbereitungszeit: 16 Aktionen\newline Ziel: Einzelperson\newline Reichweite: Berührung\newline Wirkungsdauer: permanent\newline Kosten: 8 KaP\newline Fertigkeiten: Vergessen\newline Erlernen: Brn 18; 40 EP}
}


\newglossaryentry{rufzurRuhe_Talent}
{
    name={Ruf zur Ruhe},
    description={Dein Ziel verliert das Bedürfnis zu sprechen, solange du nicht sprichst, außer es gelingt ihm eine Konterprobe (Willenskraft, 20).\newline Probenschwierigkeit: 12\newline Vorbereitungszeit: 4 Aktionen\newline Ziel: Einzelperson\newline Reichweite: 32 Schritt\newline Wirkungsdauer: 1 Woche\newline Kosten: 4 KaP\newline Fertigkeiten: Vergessen\newline Erlernen: Brn 12; 40 EP}
}


\newglossaryentry{erwachendesStromes_Talent}
{
    name={Erwachen des Stromes},
    description={Innerhalb weniger Minuten schwillt der Fluss an und führt so viel Wasser wie zur stärksten Regenzeit. Furten und manche Brücken werden unpassierbar und das Wasser reißt lose Gegenstände am Ufer mit.\newline Mächtige Liturgie: Der Fluss führt um ein/zwei/drei/vier Viertel mehr Wasser als zur stärksten Regenzeit.\newline Probenschwierigkeit: 12\newline Modifikationen: Flutwelle (–8, 8 Aktionen, Wirkungsdauer augenblicklich; das Wasser erscheint in einer plötzlichen, zerstörerischen Flutwelle.)\newline Vorbereitungszeit: 4 Minuten\newline Ziel: Zone\newline Reichweite: 32 Schritt\newline Wirkungsdauer: 1 Stunde\newline Kosten: 16 KaP\newline Fertigkeiten: Flüsse und Quellen\newline Erlernen: Eff 16; 20 EP}
}


\newglossaryentry{flüsternderFluten_Talent}
{
    name={Flüstern der Fluten},
    description={Du nimmst vage Eindrücke von Fischen und Meereslebewesen, schwimmenden und versunkenen Gegenständen oder Verschmutzungen im Radius von 4 Meilen (bei Flüssen halb so weit mit- und doppelt so weit gegen die Strömung) wahr. Erfordert Konzentration.\newline Probenschwierigkeit: 12\newline Vorbereitungszeit: 16 Aktionen\newline Ziel: selbst\newline Reichweite: Berührung\newline Wirkungsdauer: 1 Stunde\newline Kosten: 1 KaP\newline Fertigkeiten: Flüsse und Quellen, Wind und Wogen\newline Erlernen: Eff 16; 20 EP}
}


\newglossaryentry{gesegneterFang_Talent}
{
    name={Gesegneter Fang},
    description={Die Ausbeute der gesegneten Netze oder Reusen ist verdoppelt.\newline Mächtige Liturgie: Die Ausbeute steigt auf das Drei-/Vier-/Fünf-/Sechsfache.\newline Probenschwierigkeit: 12\newline Vorbereitungszeit: 4 Minuten\newline Ziel: Einzelobjekt\newline Reichweite: Berührung\newline Wirkungsdauer: 8 Stunden oder bis der Fang eingeholt wird\newline Kosten: 8 KaP\newline Fertigkeiten: Flüsse und Quellen, Seefahrt\newline Erlernen: Eff 8; Swa 12; 20 EP}
}


\newglossaryentry{hashnabithsFlehen_Talent}
{
    name={Hashnabiths Flehen},
    description={Du findest den kürzesten begehbaren Weg zur nächsten Wasserquelle.\newline Probenschwierigkeit: 12\newline Modifikationen: Quellsegen (–4, Zone; verdoppelt die Wassermenge einer Quelle.)\newline Azilas Quellgesang (–8, 16 KaP, Zone; aus dem Boden strömt ausreichend reines Trinkwasser, um eine kleine Karawane zu versorgen.)\newline Vorbereitungszeit: 4 Minuten\newline Ziel: selbst\newline Reichweite: Berührung\newline Wirkungsdauer: 8 Stunden\newline Kosten: 4 KaP\newline Fertigkeiten: Stiller Wanderer, Flüsse und Quellen\newline Erlernen: Eff 16; Aves 18; 20 EP}
}


\newglossaryentry{tränendesMilden_Talent}
{
    name={Tränen des Milden},
    description={Innerhalb des nächsten Tages regnet es auf die Felder im Radius von 1 Meile und das Saatgut kann wachsen.\newline Mächtige Liturgie: Verdoppelt den Radius.\newline Probenschwierigkeit: 12\newline Modifikationen: Tränen des Zürnenden (–8, 8 Aktionen, Wirkungsdauer 1 Stunde, 16 KaP; statt einem sanften Regen bricht nach 1 Minute ein heftiger Platzregen los, der Wege in Schlammpisten verwandelt und Fernkampfangriffe um –8 erschwert.)\newline Vorbereitungszeit: 1 Stunde\newline Ziel: Zone\newline Reichweite: Berührung\newline Wirkungsdauer: 1 Tag\newline Kosten: 8 KaP\newline Fertigkeiten: Flüsse und Quellen\newline Erlernen: Eff 12; 20 EP}
}


\newglossaryentry{segendesFlussvaters_Talent}
{
    name={Segen des Flussvaters},
    description={Du wirst eins mit einem Fluss. Du kannst dich durch den Fluss bewegen, ohne zu ertrinken, erschöpfen oder abgetrieben zu werden. Mit dem Strom bewegst du dich mit einer Geschwindigkeit von 8 Meilen pro Stunde, gegen den Strom mit einer Geschwindigkeit von 4 Meilen pro Stunde. Selbstverständlich kannst du auch einfach einen Fluss überqueren.\newline Mächtige Liturgie: Die Geschwindigkeit steigt um 4/2 Meilen pro Stunde.\newline Probenschwierigkeit: 12\newline Vorbereitungszeit: 4 Minuten\newline Ziel: selbst\newline Reichweite: Berührung\newline Wirkungsdauer: 1 Stunde\newline Kosten: 8 KaP\newline Fertigkeiten: Flüsse und Quellen\newline Erlernen: Eff 18; 40 EP}
}


\newglossaryentry{segensreichesWasser_Talent}
{
    name={Segensreiches Wasser},
    description={Efferdgeweihte rufen Efferds Wasserkrug, Perainegeweihte den Krug der Heiligen Lindegard. Der Krug ist mit 8 Schlucken heiligen Wassers gefüllt. Jeder Schluck heilt 1 Wunde oder stillt den Durst für einen Tag (Efferd) oder stärkt das Wachstum einer Nutzpflanze, die dann innerhalb eines Tages eine Anwendung Früchte/Blüten/Blätter hervorbringt (Peraine).\newline Mächtige Liturgie: Der Krug enthält 4 zusätzliche Schlucke.\newline Probenschwierigkeit: 12\newline Vorbereitungszeit: 4 Minuten\newline Ziel: Einzelobjekt\newline Reichweite: dereweit\newline Wirkungsdauer: 1 Stunde\newline Kosten: 32 KaP\newline Fertigkeiten: Flüsse und Quellen, Wachstum\newline Erlernen: Eff, Per 18; 60 EP}
}


\newglossaryentry{aitheokles‘Klageruf_Talent}
{
    name={Aitheokles‘ Klageruf},
    description={Du löschst ein Feuer bis zur Größe eines großen Lagerfeuers.\newline Mächtige Liturgie: Du kannst ein Feuer bis zur Größe eines Scheiterhaufens/brennenden Schiffes/einer brennenden Häusergruppe/eines brennenden Stadtviertels löschen.\newline Probenschwierigkeit: 12\newline Modifikationen: Kleiner Feuerbann (0 Aktionen, 1 KaP; ein Feuer bis zur Größe einer Fackel erlischt.)\newline Selbstlöschung (0 Aktionen, 2 KaP; du löschst dich selbst.)\newline Vorbereitungszeit: 2 Aktionen\newline Ziel: Zone\newline Reichweite: 32 Schritt\newline Wirkungsdauer: augenblicklich\newline Kosten: 8 KaP\newline Fertigkeiten: Seefahrt\newline Erlernen: Eff 16; 40 EP\newline Anmerkung: Alternativ wirkt die Liturgie wie Antimagie gegen Feuerzauber (S. 126). Die Randbedingungen entsprechen den dort angegebenen, außer dass die Kosten mit KaP bezahlt werden und die Liturgie um +4 erleichtert ist.}
}


\newglossaryentry{anrufungderWinde_Talent}
{
    name={Anrufung der Winde},
    description={Im Radius von 1 Meile kannst du den Wind auf einer Skala von windstill/leichte Brise/steife Brise/Sturm/Orkan um zwei Stufen verändern und seine Richtung lenken.\newline Mächtige Liturgie: Verändert den Wind um eine weitere Stufe und verdoppelt den Radius.\newline Probenschwierigkeit: 12\newline Modifikationen: Windhauch (0 Aktionen, Wirkungsdauer augenblicklich, 1 KaP; ein kurzer Windstoß weht giftige Dämpfe fort oder löscht ein kleines Feuer.)\newline Vorbereitungszeit: 4 Minuten\newline Ziel: Zone\newline Reichweite: Berührung\newline Wirkungsdauer: 8 Stunden\newline Kosten: 16 KaP\newline Fertigkeiten: Seefahrt, Wind und Wogen\newline Erlernen: Eff, Swa 14; 40 EP}
}


\newglossaryentry{bootssegen_Talent}
{
    name={Bootssegen},
    description={Die Besatzung des gesegneten Schiffes geht  dem Glauben nach bei einem Unglück in Efferds Wasserreich ein. Nach Spielleiterentscheid sind Proben, um widernatürlichen Gefahren zu entgehen, um +2 erleichtert.\newline Probenschwierigkeit: 12\newline Vorbereitungszeit: 1 Stunde\newline Ziel: Zone\newline Reichweite: Berührung\newline Wirkungsdauer: bis das Schiff deutlich umgebaut wird\newline Kosten: 16 KaP\newline Fertigkeiten: Seefahrt\newline Erlernen: Eff, Swa 8; 20 EP}
}


\newglossaryentry{conagasRuf_Talent}
{
    name={Conagas Ruf},
    description={Alle Meeresungeheuer im Radius von 100 Schritt erleiden einen Furcht-Effekt Stufe 2.\newline Mächtige Liturgie: Verdoppelt den Radius und je zwei Stufen erhöhen den Furcht-Effekt um eine Stufe.\newline Probenschwierigkeit: 12\newline Vorbereitungszeit: 1 Aktion\newline Ziel: Zone\newline Reichweite: Berührung\newline Wirkungsdauer: 1 Stunde\newline Kosten: 16 KaP\newline Fertigkeiten: Seefahrt\newline Erlernen: Swa 12; Eff 18; 20 EP}
}


\newglossaryentry{liaiellasOrakel_Talent}
{
    name={Liaiellas Orakel},
    description={Du erfährst, ob die Seele eines Verschollenen von Liaiella in Borons Hallen (oder an Swafnirs Tafel) getragen wurde oder nicht. Letzteres ist auch der Fall, wenn der Verschollene lebt oder an Land, oder dämonisch verseuchtem Gewässer, starb.\newline Probenschwierigkeit: 12\newline Vorbereitungszeit: 4 Minuten\newline Ziel: selbst\newline Reichweite: Berührung\newline Wirkungsdauer: augenblicklich\newline Kosten: 4 KaP\newline Fertigkeiten: Seefahrt\newline Erlernen: Eff, Swa 12; 0 EP}
}


\newglossaryentry{mannschaftssegen_Talent}
{
    name={Mannschaftssegen},
    description={Die gesamte Schiffsmannschaft erhält eine Erleichterung von +2 auf MU-Proben und ignoriert eine Stufe jedes Furchteffekts, solange sie sich an Bord eines Schiffes aufhält.\newline Probenschwierigkeit: 12\newline Vorbereitungszeit: 1 Stunde\newline Ziel: Zone\newline Reichweite: Berührung\newline Wirkungsdauer: bis die Bindung gelöst wird\newline Kosten: 16 KaP, davon 2 gKaP\newline Fertigkeiten: Seefahrt\newline Erlernen: Swa 8; Eff 12; 20 EP}
}


\newglossaryentry{segendesPlättlings_Talent}
{
    name={Segen des Plättlings},
    description={Du verwandelst 100 Liter Salzwasser in Trinkwasser.\newline Mächtige Liturgie: Du verwandelst 200/300/400/500 Liter.\newline Probenschwierigkeit: 12\newline Vorbereitungszeit: 4 Minuten\newline Ziel: Einzelobjekt\newline Reichweite: Berührung\newline Wirkungsdauer: augenblicklich\newline Kosten: 8 KaP\newline Fertigkeiten: Seefahrt\newline Erlernen: Eff, Swa 14; 20 EP}
}


\newglossaryentry{sternefunkelnimmerfort_Talent}
{
    name={Sterne funkeln immerfort},
    description={Dein Ziel kann selbst durch Wolken und Nebel die Sterne erkennen. Proben zur Orientierung sind um +4 erleichtert.\newline Probenschwierigkeit: 12\newline Vorbereitungszeit: 16 Aktionen\newline Ziel: Einzelperson\newline Reichweite: Berührung\newline Wirkungsdauer: 4 Stunden\newline Kosten: 4 KaP\newline Fertigkeiten: Licht, Nächtlicher Schatten, Seefahrt, Stiller Wanderer, Wildnis, Wissen\newline Erlernen: Aves, Fir, Ifi, Pra 8; Phe 12; Eff, Hes, Nan, Swa 14; 20 EP}
}


\newglossaryentry{anrufungKaucas_Talent}
{
    name={Anrufung Kaucas},
    description={Tosende Winde umgeben dich und wehren Fernkampfangriffe bis zur Größe eines Wurfspeeres ab. Ungezielte Geschosse (wie in einem Pfeilhagel) treffen dich nicht, Fernkampfangriffe auf dich sind um –4 erschwert.\newline Mächtige Liturgie: Der Malus steigt um –2.\newline Probenschwierigkeit: 12\newline Modifikationen: Im Auge des Sturms (–8, Zone, 16 KaP; die Winde umgeben ein ganzes Schiff und wehren Geschosse bis zur Größe einer Rotzenkugel ab.)\newline Vorbereitungszeit: 2 Aktionen\newline Ziel: selbst\newline Reichweite: Berührung\newline Wirkungsdauer: 4 Minuten\newline Kosten: 8 KaP\newline Fertigkeiten: Wind und Wogen\newline Erlernen: Swa 14; Eff 16; 40 EP}
}


\newglossaryentry{anrufungNuiannas_Talent}
{
    name={Anrufung Nuiannas},
    description={In einem Bereich von 100 Schritt Radius entsteht eine dichte Nebelfront und breitet sich in die von dir bestimmte Richtung aus.\newline Mächtige Liturgie: Verdoppelt den Radius.\newline Probenschwierigkeit: 12\newline Modifikationen: Begleiter (–4; die Nebelfront bewegt sich mit dir.)\newline Vorbereitungszeit: 4 Minuten\newline Ziel: Zone\newline Reichweite: Berührung\newline Wirkungsdauer: 8 Stunden\newline Kosten: 8 KaP\newline Fertigkeiten: Wind und Wogen\newline Erlernen: Eff 14; Swa 18; 40 EP}
}


\newglossaryentry{begehenderheiligenWasser_Talent}
{
    name={Begehen der heiligen Wasser},
    description={Du kannst auf der Wasseroberfläche gehen, als würdest du von einer Welle getragen. Du erleidest keine Abzüge durch ungünstige Position und wirst nicht durch Wellen und Strömungen beeinflusst. Erlaubt Aufrechterhalten.\newline Probenschwierigkeit: 12\newline Vorbereitungszeit: 8 Aktionen\newline Ziel: selbst\newline Reichweite: Berührung\newline Wirkungsdauer: 1 Stunde\newline Kosten: 4 KaP\newline Fertigkeiten: Wind und Wogen\newline Erlernen: Eff 12; 20 EP}
}


\newglossaryentry{efferdanesReinigung_Talent}
{
    name={Efferdanes Reinigung},
    description={Dein Gebet schwächt charyptide Einflüsse in der Umgebung. Auf der Skala normal/leicht verseucht/deutlich verseucht/stark verseucht/kleines Unheiligtum/mittleres Unheiligtum/mächtiges Unheiligtum verschiebt sich das Gebiet um eine Stufe nach vorne.\newline Mächtige Liturgie: Das Gebiet sinkt um eine weitere Stufe.\newline Probenschwierigkeit: 12\newline Vorbereitungszeit: 1 Stunde\newline Ziel: Zone\newline Reichweite: 32 Schritt\newline Wirkungsdauer: 1 Jahr\newline Kosten: 8 KaP\newline Fertigkeiten: Wind und Wogen\newline Erlernen: Eff, Swa 16; 40 EP\newline Anmerkung: Verschiebst du die Skala auf normal, kehren die dämonischen Einflüsse nicht von selbst zurück – außer die Ursache der Verseuchung existiert weiterhin. Alternativ kann die Liturgie wie Antimagie gegen charyptide Magie wirken (S. 126). Die Randbedingungen entsprechen den dort angegebenen, außer dass die Kosten mit KaP bezahlt werden und die Liturgie um +4 erleichtert ist.}
}


\newglossaryentry{gesangderDelphine_Talent}
{
    name={Gesang der Delphine},
    description={Die von dir gesegnete Person erhält den Vorteil Tierempathie (Delphine für Eff oder Pottwale für Swa) und erleidet keine Erschöpfung durch den Einsatz dieser Gabe. Beherrscht sie die Gabe bereits, sind damit verbundene Proben um +4 erleichtert. Segnest du dich selbst, kannst du notwendige Proben auf Wind und Wogen statt auf Überleben ablegen.\newline Mächtige Liturgie: Der Vorteil gilt für alle Walartigen/Fische/Meeresbewohner.\newline Probenschwierigkeit: 12\newline Vorbereitungszeit: 16 Aktionen\newline Ziel: Einzelperson\newline Reichweite: Berührung\newline Wirkungsdauer: 1 Stunde\newline Kosten: 4 KaP\newline Fertigkeiten: Wind und Wogen\newline Erlernen: Eff, Swa 16; 20 EP}
}


\newglossaryentry{gleichklangdesGeistes_Talent}
{
    name={Gleichklang des Geistes},
    description={Du erhältst einen Einblick in die Gefühle deines Gegenübers.\newline Probenschwierigkeit: 12\newline Vorbereitungszeit: 8 Aktionen\newline Ziel: Einzelperson\newline Reichweite: Berührung\newline Wirkungsdauer: 1 Stunde\newline Kosten: 1 KaP\newline Fertigkeiten: Friede, Harmonie, Heilung, Wind und Wogen\newline Erlernen: Swa 8; Eff, Rah, Tsa 12, Per 14; 20 EP}
}


\newglossaryentry{neckeratem_Talent}
{
    name={Neckeratem},
    description={Das Ziel kann im Wasser atmen und sieht durch Wasser wie durch Luft. Erlaubt Aufrechterhalten.\newline Probenschwierigkeit: 12\newline Modifikationen: Gruß des Versunkenen (–8, Zone, 16 KaP; der Zauber betrifft alle im Radius von 4 Schritt.)\newline Vorbereitungszeit: 8 Aktionen\newline Ziel: Einzelperson\newline Reichweite: Berührung\newline Wirkungsdauer: 2 Stunden\newline Kosten: 8 KaP\newline Fertigkeiten: Wind und Wogen\newline Erlernen: Eff 12; Swa 16; 20 EP}
}


\newglossaryentry{rufderGefährten_Talent}
{
    name={Ruf der Gefährten},
    description={Du rufst einen Delphin aus einem Radius von 4 Meilen herbei, der sich so schnell wie möglich nähert, und kannst ihn um einen Gefallen bitten. Ist kein Tier in Reichweite, passiert nichts.\newline Mächtige Liturgie: Du rufst 2/4/8/16 Tiere.\newline Probenschwierigkeit: 12\newline Vorbereitungszeit: 2 Aktionen\newline Ziel: Zone\newline Reichweite: Berührung\newline Wirkungsdauer: 1 Stunde\newline Kosten: 8 KaP\newline Fertigkeiten: Wind und Wogen\newline Erlernen: Eff, Swa 16; 20 EP}
}


\newglossaryentry{swafnirsRuhelied_Talent}
{
    name={Swafnirs Ruhelied},
    description={Du stärkst die Selbstbeherrschung deines Zieles. Willenskraft-Proben, um sich zu beruhigen oder Zorn zu unterdrücken, sind um +4 erleichtert.\newline Mächtige Liturgie: Der Bonus steigt um +2.\newline Probenschwierigkeit: 12\newline Vorbereitungszeit: 4 Aktionen\newline Ziel: Einzelperson\newline Reichweite: 2 Schritt\newline Wirkungsdauer: 1 Stunde\newline Kosten: 4 KaP\newline Fertigkeiten: Wind und Wogen\newline Erlernen: Swa 12; 20 EP\newline Anmerkung: Alternativ kann die Liturgie wie Antimagie gegen Berserkerzauber wirken (S. 126). Die Randbedingungen entsprechen den dort angegebenen, außer dass die Kosten mit KaP bezahlt werden und die Liturgie um +4 erleichtert ist.}
}


\newglossaryentry{augedesJägers_Talent}
{
    name={Auge des Jägers},
    description={Du benennst eine Beute. Sinnenschärfe-, Überleben- und andere Proben, um die Beute zu finden sind um +2 erleichtert.\newline Mächtige Liturgie: Der Bonus steigt um +1.\newline Probenschwierigkeit: 12\newline Vorbereitungszeit: 16 Aktionen\newline Ziel: selbst\newline Reichweite: Berührung\newline Wirkungsdauer: 8 Stunden\newline Kosten: 8 KaP\newline Fertigkeiten: Jagd\newline Erlernen: Fir 16; Ifi 18; 40 EP}
}


\newglossaryentry{firunsFluch_Talent}
{
    name={Firuns Fluch},
    description={Du schlägst dein Ziel mit Firuns Fluch. Alle Fernkampfangriffe auf dieses Ziel sind um +2 erleichtert und die Chance auf einen Triumph steigt um 1 auf dem W20 (zum Beispiel von 20 auf 19–20).\newline Mächtige Liturgie: Der Bonus steigt um +1.\newline Modifikationen: Permanenz (–4, Kosten 8 KaP, davon 2 gKaP, Wirkungsdauer bis die Bindung gelöst wird)\newline Probenschwierigkeit: 12\newline Vorbereitungszeit: 1 Aktion\newline Ziel: Einzelwesen\newline Reichweite: 32 Schritt\newline Wirkungsdauer: 1 Woche\newline Kosten: 8 KaP\newline Fertigkeiten: Jagd\newline Erlernen: Fir 16; 40 EP}
}


\newglossaryentry{gemeinschafttreuerGefährten_Talent}
{
    name={Gemeinschaft treuer Gefährten},
    description={Du stärkst die Bindung zwischen einem Tier und seinem Besitzer. Proben im Umgang mit dem Tier sind um +4 erleichtert, Freie Fertigkeiten (wie Abrichter) zählen als eine Stufe höher.\newline Mächtige Liturgie: Der Bonus steigt um +2, je zwei Stufen erhöhen Freie Fertigkeiten um eine weitere Stufe.\newline Probenschwierigkeit: 12\newline Vorbereitungszeit: 1 Woche\newline Ziel: Tier und Einzelperson\newline Reichweite: Berührung\newline Wirkungsdauer: bis die Bindung gelöst wird\newline Kosten: 8 KaP, davon 2 gKaP\newline Fertigkeiten: Fröhlicher Wanderer, Jagd, Sichere Heimkehr\newline Erlernen: Ifi 14, Aves, Tra 16; 40 EP\newline Anmerkung: Besonders große und starke Tiere erhöhen die gKaP-Kosten auf 4 oder sogar 8.}
}


\newglossaryentry{jagdglück_Talent}
{
    name={Jagdglück},
    description={Im Laufe der Wirkungsdauer findest du das nächste Tier, das firungefällig erjagt werden kann. Dadurch sinkt die Schwierigkeit einer Jagd auf 12 (entsprechend eines wildreichen Gebiets, S. 68).\newline Mächtige Liturgie: Verdoppelt die Zahl der gefundenen Tiere.\newline Probenschwierigkeit: 12\newline Vorbereitungszeit: 16 Aktionen\newline Ziel: selbst\newline Reichweite: Berührung\newline Wirkungsdauer: 4 Stunden\newline Kosten: 8 KaP\newline Fertigkeiten: Jagd\newline Erlernen: Fir, Ifi 12; 20 EP}
}


\newglossaryentry{seelengefährte_Talent}
{
    name={Seelengefährte},
    description={Du bittest Firuns Wilde Jagd um einen Gefallen. Das Tier der Wilden Jagd, das diesen Gefallen am ehesten erfüllen kann und sich in einem Radius von 4 Meilen befindet, macht sich sofort auf den Weg zu dir und erfüllt den Gefallen. Ist diese Tierart in der Region nicht heimisch, erreicht der Ruf ein verwandtes Tier. Ist kein Tier in Reichweite, passiert nichts.\newline Mächtige Liturgie: Du rufst 2/4/8/16 Tiere.\newline Probenschwierigkeit: 12\newline Vorbereitungszeit: 8 Aktionen\newline Ziel: Zone\newline Reichweite: Berührung\newline Wirkungsdauer: 1 Stunde\newline Kosten: 8 KaP\newline Fertigkeiten: Jagd\newline Erlernen: Fir 16; 20 EP}
}


\newglossaryentry{segnungdesHeiligenMikail_Talent}
{
    name={Segnung des Heiligen Mikail},
    description={Du bittest den Heiligen Mikail um Treffsicherheit für den Schützen. Der nächste Fernkampfangriff ist um +4 erleichtert und ignoriert eine Stufe Deckung des Ziels (S. 46).\newline Mächtige Liturgie: Der Bonus steigt um +2.\newline Probenschwierigkeit: 12\newline Modifikationen: Langer Segen (–8, 8 KaP; die Liturgie wirkt auf alle Fernkampfangriffe während der Wirkungsdauer.)\newline Vorbereitungszeit: 0 Aktionen\newline Ziel: Einzelperson\newline Reichweite: Berührung\newline Wirkungsdauer: 8 Initiativphasen\newline Kosten: 4 KaP\newline Fertigkeiten: Jagd\newline Erlernen: Fir 12, Ifi 14; 20 EP}
}


\newglossaryentry{tierempathie_Talent}
{
    name={Tierempathie},
    description={Du erhältst den Vorteil Tierempathie und erleidest keine Erschöpfung durch den Einsatz dieser Gabe. Außerdem kannst du notwendige Proben auf Jagd (Ifi) oder Friede (Tsa) statt Überleben ablegen. Beherrschst du die Gabe bereits, sind damit verbundene Proben um +4 erleichtert.\newline Probenschwierigkeit: 12\newline Vorbereitungszeit: 16 Aktionen\newline Ziel: selbst\newline Reichweite: Berührung\newline Wirkungsdauer: 1 Stunde\newline Kosten: 8 KaP\newline Fertigkeiten: Friede, Jagd\newline Erlernen: Ifi 16; Tsa 18; 40 EP}
}


\newglossaryentry{trophäeerhalten_Talent}
{
    name={Trophäe erhalten},
    description={Die gesegnete Jagdbeute verdirbt während der Wirkungsdauer nicht und kann so verarbeitet werden, als wäre das Beutetier frisch erlegt worden.\newline Probenschwierigkeit: 12\newline Vorbereitungszeit: 8 Aktionen\newline Ziel: Einzelobjekt\newline Reichweite: Berührung\newline Wirkungsdauer: 1 Woche\newline Kosten: 4 KaP\newline Fertigkeiten: Jagd\newline Erlernen: Fir 12, Ifi 16; 20 EP}
}


\newglossaryentry{firunsEinsicht_Talent}
{
    name={Firuns Einsicht},
    description={Du rufst Firuns Ring herbei, der dir den Aufenthaltsort von Lebewesen im Radius von 16 Meilen offenbart. Du kannst außerdem Tiere von Humanoiden unterscheiden.\newline Probenschwierigkeit: 12\newline Vorbereitungszeit: 16 Aktionen\newline Ziel: Einzelobjekt\newline Reichweite: dereweit\newline Wirkungsdauer: 1 Stunde\newline Kosten: 8 KaP\newline Fertigkeiten: Wildnis\newline Erlernen: Fir 16; 40 EP}
}


\newglossaryentry{flüsternderWildnis_Talent}
{
    name={Flüstern der Wildnis},
    description={Die von dir gesegnete Person erhält den Vorteil Tierempathie (Eisbären) und erleidet keine Erschöpfung durch den Einsatz dieser Gabe. Beherrscht sie die Gabe bereits, sind damit verbundene Proben um +4 erleichtert. Segnest du dich selbst, kannst du notwendige Proben auf Wildnis statt auf Überleben ablegen.\newline Mächtige Liturgie: Der Vorteil gilt für alle Bären/Hundeartigen/Raubtiere.\newline Probenschwierigkeit: 12\newline Vorbereitungszeit: 16 Aktionen\newline Ziel: Einzelperson\newline Reichweite: Berührung\newline Wirkungsdauer: 1 Stunde\newline Kosten: 4 KaP\newline Fertigkeiten: Wildnis\newline Erlernen: Fir 14; 20 EP}
}


\newglossaryentry{geteiltesLeid_Talent}
{
    name={Geteiltes Leid},
    description={Du rufst den Mantel der Heiligen Matscha und schneidest für den Gesegneten ein Stück ab. Der Mantel verleiht ihm Immunität gegen Kälteschaden, bis er an angemessene Kleidung gelangen kann.\newline Probenschwierigkeit: 12\newline Vorbereitungszeit: 16 Aktionen\newline Ziel: Einzelperson\newline Reichweite: Berührung\newline Wirkungsdauer: 1 Monat\newline Kosten: 8 KaP\newline Fertigkeiten: Heim und Herd, Wildnis\newline Erlernen: Ifi 12; Tra 14; 20 EP}
}


\newglossaryentry{pfeildesHeiligenIsegrein_Talent}
{
    name={Pfeil des Heiligen Isegrein},
    description={Du rufst den Heiligen Isegrein an, um einen Pfeil zu segnen. Der Pfeil gilt als geweiht und seine Reichweite ist verdoppelt.\newline Mächtige Liturgie: Der Schaden des Pfeils erhöht sich um 1W6 TP.\newline Probenschwierigkeit: 12\newline Modifikationen: Permanenz (–4, Wirkungsdauer bis die Bindung gelöst oder der Pfeil verschossen wird, Kosten 8 KaP, davon 1 gKaP)\newline Vorbereitungszeit: 1 Aktion\newline Ziel: Einzelobjekt\newline Reichweite: Berührung\newline Wirkungsdauer: 8 Initiativphasen\newline Kosten: 8 KaP\newline Fertigkeiten: Wildnis\newline Erlernen: Fir 16; 40 EP}
}


\newglossaryentry{segendesHeiligenIsegrein_Talent}
{
    name={Segen des Heiligen Isegrein},
    description={Der Gesegnete wird eins mit seiner Umgebung. In der Wildnis wird er wie mit der Gabe Gefahreninstinkt vor natürlichen Gefahren gewarnt. Besitzt er die Gabe bereits, sind damit verbundene Proben um +4 erleichtert. Wirkt nicht unter der Erde, auf dem Meer oder in dämonisch pervertiertem Gebiet.\newline Mächtige Liturgie: Überleben-Proben sind um +2 erleichtert.\newline Probenschwierigkeit: 12\newline Vorbereitungszeit: 4 Minuten\newline Ziel: Einzelperson\newline Reichweite: Berührung\newline Wirkungsdauer: 2 Tage\newline Kosten: 8 KaP\newline Fertigkeiten: Stiller Wanderer, Wildnis\newline Erlernen: Fir, Ifi 14; Aves 16; 40 EP}
}


\newglossaryentry{sichereWanderungimSchnee_Talent}
{
    name={Sichere Wanderung im Schnee},
    description={Der Gesegnete kann in tiefstem Schnee und auf Eis laufen, als würde er über festen Boden gehen. Erlaubt Aufrechterhalten.\newline Probenschwierigkeit: 12\newline Modifikationen: Eiskletterer (–4; der Gesegnete kann an Eiswänden wie auf Fels klettern.)\newline Vorbereitungszeit: 16 Aktionen\newline Ziel: Einzelperson\newline Reichweite: Berührung\newline Wirkungsdauer: 4 Stunden\newline Kosten: 4 KaP\newline Fertigkeiten: Wildnis, Winter\newline Erlernen: Fir, Ifi 12; 20 EP}
}


\newglossaryentry{stärkedesWaldläufers_Talent}
{
    name={Stärke des Waldläufers},
    description={Verdoppelt dein DH* (S. 34) und das Intervall, in dem körperliche Anstrengung Erschöpfung verursacht.\newline Mächtige Liturgie: Verdreifacht/vervierfacht/verünffacht/versechsfacht dein DH* und das Intervall.\newline Probenschwierigkeit: 12\newline Vorbereitungszeit: 16 Aktionen\newline Ziel: selbst\newline Reichweite: Berührung\newline Wirkungsdauer: 2 Tage\newline Kosten: 8 KaP\newline Fertigkeiten: Stiller Wanderer, Wildnis\newline Erlernen: Fir 12; Aves 16; 20 EP}
}


\newglossaryentry{wegdesFuchses_Talent}
{
    name={Weg des Fuchses},
    description={Du kannst dich auch auf schlechten Pfaden, durch natürliche Hindernisse oder Menschenmassen wie auf einer normalen Straße bewegen. Proben in Verfolgungsjagden und unter schwierigen Bedingungen sind um +4 erleichtert.\newline Mächtige Liturgie: Der Bonus steigt um +2.\newline Probenschwierigkeit: 12\newline Vorbereitungszeit: 2 Aktionen\newline Ziel: selbst\newline Reichweite: Berührung\newline Wirkungsdauer: 1 Stunde\newline Kosten: 8 KaP\newline Fertigkeiten: Nächtlicher Schatten, Stiller Wanderer, Wildnis\newline Erlernen: Fir, Phe 14; Aves, Ifi 16; 40 EP}
}


\newglossaryentry{zufluchtfinden_Talent}
{
    name={Zuflucht finden},
    description={Du findest intuitiv den Weg zur nächsten sicheren Lagerstätte. Entsprechende Überleben-Proben sind um +4 erleichtert.\newline Mächtige Liturgie: Der Bonus steigt um +2\newline Probenschwierigkeit: 12\newline Vorbereitungszeit: 16 Aktionen\newline Ziel: selbst\newline Reichweite: Berührung\newline Wirkungsdauer: 1 Stunde\newline Kosten: 4 KaP\newline Fertigkeiten: Sichere Heimkehr, Wildnis\newline Erlernen: Fir, Ifi 12; Tra 14; 20 EP}
}


\newglossaryentry{eiskerker_Talent}
{
    name={Eiskerker},
    description={Das Ziel wird von einem Eiskerker mit einer Härte von 16 eingeschlossen, wenn ihm keine Konterprobe (KK, 20) gelingt.\newline Mächtige Liturgie: Die Härte steigt um +8.\newline Probenschwierigkeit: 12\newline Modifikationen: Eisgrab (–8, Wirkungsdauer bis die Bindung gelöst wird, Kosten 16 KaP, davon 2 gKaP; verdoppelt die Härte und erlaubt keine Konterprobe.)\newline Vorbereitungszeit: 2 Aktionen\newline Ziel: Einzelwesen\newline Reichweite: 16 Schritt\newline Wirkungsdauer: 1 Stunde\newline Kosten: 8 KaP\newline Fertigkeiten: Winter\newline Erlernen: Fir 16; 60 EP\newline Anmerkung: Wenige Hochgeweihte kennen eine Variante, die ganze Gebäude unter Eis begräbt. }
}


\newglossaryentry{rufdesAsainyf_Talent}
{
    name={Ruf des Asainyf},
    description={Du bist immun gegen Eis- und Kälteschaden.\newline Modifikationen: Erweiterter Schutz (–4; der Schutz betrifft auch deine Ausrüstung.)\newline Vorbereitungszeit: 16 Aktionen\newline Ziel: selbst\newline Reichweite: Berührung\newline Wirkungsdauer: 8 Stunden\newline Kosten: 8 KaP\newline Fertigkeiten: Winter\newline Erlernen: Fir 12; 20 EP}
}


\newglossaryentry{schneesturm_Talent}
{
    name={Schneesturm},
    description={Im Radius von 1 Meile entsteht ein Schneesturm. Der Wind steigt auf einer Skala von windstill/leichte Brise/steife Brise/Sturm/Orkan um eine Stufe. Die Temperatur sinkt auf einer Skala von kühl/kalt/eiskalt/Firunsfrost (S. 35) um eine Stufe, aber mindestens auf kalt. Es herrscht starker Schneefall. Eine Verfolgung durch den Schneesturm ist nur mit einer Konterprobe (Überleben, 16) möglich.\newline Mächtige Liturgie: Du kannst den Wind um eine weitere Stufe erhöhen und den Radius verdoppeln.\newline Probenschwierigkeit: 12\newline Vorbereitungszeit: 4 Minuten\newline Ziel: Zone\newline Reichweite: 1 Meile\newline Wirkungsdauer: 8 Stunden\newline Kosten: 16 KaP\newline Fertigkeiten: Winter\newline Erlernen: Fir 16; 40 EP\newline Anmerkung: Hohe Temperaturen können die Wirkungsdauer nach Spielleiterentscheid auf bis zu 1 Stunde senken. }
}


\newglossaryentry{winterschlaf_Talent}
{
    name={Winterschlaf},
    description={Das gesegnete Ziel fällt in einen tiefen Winterschlaf, in dem sie weder Nahrung noch Atemluft benötigt und jeglichen Schaden durch natürliche Hitze oder Kälte ignoriert. Die Effekte von Giften und Krankheiten werden gestoppt, allerdings regeneriert sie auch nur einmal pro Woche. Die Gesegnete erwacht, wenn sie eine Wunde erleidet.\newline Probenschwierigkeit: 12\newline Vorbereitungszeit: 1 Stunde\newline Ziel: Einzelperson\newline Reichweite: Berührung\newline Wirkungsdauer: 1 Monat\newline Kosten: 16 KaP\newline Fertigkeiten: Winter\newline Erlernen: Fir 14; 20 EP}
}


\newglossaryentry{zorndesHeiligenFirungald_Talent}
{
    name={Zorn des Heiligen Firungald},
    description={Du segnest eine Nahkampfwaffe. Diese gilt als geweiht und verursacht Erfrieren (S. 98).\newline Mächtige Liturgie: Erhöht den Waffenschaden um +1.\newline Probenschwierigkeit: 12\newline Vorbereitungszeit: 8 Aktionen\newline Ziel: Einzelobjekt\newline Reichweite: Berührung\newline Wirkungsdauer: 1 Tag\newline Kosten: 8 KaP\newline Fertigkeiten: Winter\newline Erlernen: Fir 14; 40 EP}
}


\newglossaryentry{argelionsbannendeHand_Talent}
{
    name={Argelions bannende Hand},
    description={Die Liturgie wirkt als Konterprobe (12) gegen einen Zauber, in den keine gAsP geflossen sind. Gelingt sie, wird der Zauber aufgehoben.\newline Probenschwierigkeit: 12\newline Vorbereitungszeit: 4 Aktionen\newline Ziel: wirkender Zauber\newline Reichweite: 8 Schritt\newline Wirkungsdauer: augenblicklich\newline Kosten: halbe Basiskosten des Zaubers in KaP\newline Fertigkeiten: Magie, Magiebann\newline Erlernen: Hes, Pra 12; 60 EP}
}


\newglossaryentry{argelionsMantel_Talent}
{
    name={Argelions Mantel},
    description={Du ignorierst bei allen auf dich gewirkten, schädlichen Zaubern zwei Stufen Mächtige Magie. Schädliche Zauber ohne Mächtige Magie haben auf dich keine Wirkung.\newline Mächtige Liturgie: Du ignorierst eine weitere Stufe Mächtige Magie.\newline Probenschwierigkeit: 12\newline Vorbereitungszeit: 16 Aktionen\newline Ziel: selbst\newline Reichweite: Berührung\newline Wirkungsdauer: 1 Stunde\newline Kosten: 8 KaP\newline Fertigkeiten: Abu al'Mada, Magie, Magiebann\newline Erlernen: Pra 14; Hes 16; Phe 18; 60 EP}
}


\newglossaryentry{argelionsSpiegel_Talent}
{
    name={Argelions Spiegel},
    description={Der nächste auf dich gewirkte Zauber wird auf den Zauberer zurückgeworfen.\newline Mächtige Liturgie: Ein weiterer Zauber wird zurückgeworfen.\newline Probenschwierigkeit: 12\newline Vorbereitungszeit: 0 Aktionen\newline Ziel: selbst\newline Reichweite: Berührung\newline Wirkungsdauer: 1 Stunde\newline Kosten: 16 KaP\newline Fertigkeiten: Magie\newline Erlernen: Hes 18; 60 EP}
}


\newglossaryentry{blickderWeberin_Talent}
{
    name={Blick der Weberin},
    description={Du analysierst die Kraftfäden eines arkanen Artefakts oder eines magischen Wesens. Das entspricht einem AG von 2 für die Strukturanalyse (mehr dazu siehe S. 80).\newline Mächtige Liturgie: Der AG steigt um 1.\newline Probenschwierigkeit: 16\newline Vorbereitungszeit: 1 Stunde\newline Ziel: Einzelobjekt, Einzelwesen\newline Reichweite: Berührung\newline Wirkungsdauer: augenblicklich\newline Kosten: 8 KaP\newline Fertigkeiten: Magie\newline Erlernen: Hes 12; 40 EP}
}


\newglossaryentry{ingalfsAlchemie_Talent}
{
    name={Ingalfs Alchemie},
    description={Du rufst den Heiligen Trichter Hesindes herbei, der jede alchemistische Substanz in ihre Bestandteile zerlegen kann. Bei der alchemistischen Analyse erhältst du +2 AG. Außerdem können so seltene Zutaten gewonnen werden.\newline Mächtige Liturgie: Der AG steigt um 1.\newline Probenschwierigkeit: 12\newline Vorbereitungszeit: 1 Stunde\newline Ziel: Einzelobjekt\newline Reichweite: dereweit\newline Wirkungsdauer: 8 Stunden\newline Kosten: 8 KaP\newline Fertigkeiten: Magie\newline Erlernen: Hes 18; 40 EP}
}


\newglossaryentry{purgation_Talent}
{
    name={Purgation},
    description={Du bannst einen Zauber, in den gAsP geflossen sind (wie ein Artefakt oder ein permanenter Fluch), oder ein Wesen, in das gAsP geflossen sind (wie eine Chimäre oder einen gebundenen Dämon).\newline Probenschwierigkeit: Probenschwierigkeit –4 bzw. Beschwörungsschwierigkeit –4\newline Modifikationen: Praios Gnade (Probenschwierigkeit 12, 32 KaP; du brennst einem Magiebegabten die astrale Kraft aus, wenn ihm eine Konterprobe (KO, 16) misslingt. Er verliert den Vorteil Zauberer und seine komplette Astralenergie.)\newline Vorbereitungszeit: 1 Stunde\newline Ziel: Zauber oder Einzelwesen\newline Reichweite: 2 Schritt\newline Wirkungsdauer: augenblicklich\newline Kosten: halbe Basiskosten des Zaubers oder der Beschwörung in KaP\newline Fertigkeiten: Heiliges Handwerk, Magie, Magiebann, Schutz der Gläubigen\newline Erlernen: Pra, Hes 14; Ing 16; Ang, Ron 18; 40 EP}
}


\newglossaryentry{sichtaufMadasWelt_Talent}
{
    name={Sicht auf Madas Welt},
    description={Du nimmst magische Kraft in deiner Umgebung als silbernen Schimmer wahr. Das entspricht einem Analysegrad von 1 für die Intensitätsanalyse.\newline Mächtige Liturgie: Der Anyalsegrad steigt um 1.\newline Probenschwierigkeit: 12\newline Vorbereitungszeit: 1 Aktion\newline Ziel: selbst\newline Reichweite: 16 Schritt\newline Wirkungsdauer: 4 Initiativphasen\newline Kosten: 4 KaP\newline Fertigkeiten: Abu al'Mada, Magie, Magiebann\newline Erlernen: Hes 8; Pra 12; Phe 18; 20 EP}
}


\newglossaryentry{unverstellterBlick_Talent}
{
    name={Unverstellter Blick},
    description={Du nimmst alle Illusionen als golden (Pra), silbern (Phe) oder grün (Hes) schillernd wahr.\newline Probenschwierigkeit: 12\newline Vorbereitungszeit: 8 Aktionen\newline Ziel: selbst\newline Reichweite: Berührung\newline Wirkungsdauer: 4 Minuten\newline Kosten: 8 KaP\newline Fertigkeiten: Abu al'Mada, Magie, Magiebann\newline Erlernen: Hes 12; Phe, Pra 16; 20 EP}
}


\newglossaryentry{augedesHändlers_Talent}
{
    name={Auge des Händlers},
    description={Mit untrüglicher Sicherheit findest du ein besonderes Stück auf einem Markt, in einem Geschäft oder einer Ansammlung von Gegenständen. Meistens handelt es sich um ein Schnäppchen, es kann aber auch ein Gegenstand mit einer besonderen Vorgeschichte oder einem verborgenen Geheimnis sein.\newline Probenschwierigkeit: 12\newline Vorbereitungszeit: 4 Aktionen\newline Ziel: selbst\newline Reichweite: Berührung\newline Wirkungsdauer: augenblicklich\newline Kosten: 4 KaP\newline Fertigkeiten: List, Wissen\newline Erlernen: Phe 14; Nan 18; 20 EP}
}


\newglossaryentry{auraderForm_Talent}
{
    name={Aura der Form},
    description={Du erhältst einen groben Eindruck in die Vorgeschichte eines Gegenstandes.\newline Mächtige Liturgie: Du erhältst einen ungefähren/guten/präzisen/vollständigen Eindruck.\newline Probenschwierigkeit: 12\newline Vorbereitungszeit: 1 Stunde\newline Ziel: Einzelobjekt\newline Reichweite: Berührung\newline Wirkungsdauer: augenblicklich\newline Kosten: 8 KaP\newline Fertigkeiten: List, Wissen\newline Erlernen: Phe, Hes 14; 40 EP}
}


\newglossaryentry{buchprüfung_Talent}
{
    name={Buchprüfung},
    description={Du findest sofort die interessante Stelle in einem Buch.\newline Mächtige Liturgie: Du findest die für dich interessanten Stellen aus 2/4/8 Büchern.\newline Probenschwierigkeit: 12\newline Vorbereitungszeit: 16 Aktionen\newline Ziel: Einzelobjekt\newline Reichweite: Berührung\newline Wirkungsdauer: augenblicklich\newline Kosten: 4 KaP\newline Fertigkeiten: List, Wissen\newline Erlernen: Hes, Phe 12; 20 EP}
}


\newglossaryentry{einBildfürdieEwigkeit_Talent}
{
    name={Ein Bild für die Ewigkeit},
    description={Du prägst dir die während der Wirkungsdauer erfahrenen Sinneseindrücke für immer ein.\newline Probenschwierigkeit:12\newline Vorbereitungszeit: 16 Aktionen\newline Ziel: selbst\newline Reichweite: Berührung\newline Wirkungsdauer: 4 Minuten\newline Kosten: 4 KaP\newline Fertigkeiten: Rausch, Wissen\newline Erlernen: Nan, Rah 14; 40 EP}
}


\newglossaryentry{entzugvonNandus'Gaben_Talent}
{
    name={Entzug von Nandus' Gaben},
    description={Du strafst dein Ziel mit Dummheit. Alle Proben auf KL und IN sind um –4 erschwert, Proben auf Fertigkeiten mit KL oder IN um –2.\newline Mächtige Liturgie: Der Malus steigt um –2/–1.\newline Probenschwierigkeit: 12\newline Vorbereitungszeit: 8 Aktionen\newline Ziel: Einzelperson\newline Reichweite: 8 Schritt\newline Wirkungsdauer: 8 Stunden\newline Kosten: 4 KaP\newline Fertigkeiten: List, Wissen\newline Erlernen: Hes, Nan 12; Phe 16; 20 EP}
}


\newglossaryentry{giftderErkenntnis_Talent}
{
    name={Gift der Erkenntnis},
    description={Du lässt dich von einer Schlange beißen. Solange das Gift wirkt, erhältst einen ungefähren Eindruck von den Gedanken deines Gegenübers. Weiß das Ziel, dass seine Gedanken gelesen werden, kann es dich mit einer Konterprobe (Willenskraft, 20) in die Irre führen.\newline Mächtige Liturgie: Du erhältst einen deutlichen/klaren/vollständigen Eindruck in die Gedanken deines Ziels.\newline Probenschwierigkeit: 12\newline Modifikationen: Tiefentelepathie (–16, 8 KaP; du erhältst einen Einblick in innerste Vorgänge.)\newline Vorbereitungszeit: 16 Aktionen\newline Ziel: selbst\newline Reichweite: Berührung\newline Wirkungsdauer: solange das Gift wirkt\newline Kosten: 4 KaP\newline Fertigkeiten: Wissen\newline Erlernen: Hes 18; 40 EP}
}


\newglossaryentry{grauesSiegel_Talent}
{
    name={Graues Siegel},
    description={Du verschlüsselst eine Nachricht, sodass sie nur noch von der von dir genannten Person gelesen werden kann. Die Verschlüsselung kann nur mit einer Freien Fertigkeit (z.B. Kryptograph) auf meisterlich oder mit einer Konterprobe (KL, 20) und 4 Stunden Zeitaufwand entschlüsselt werden.\newline Probenschwierigkeit: 12\newline Modifikationen: Personengruppe (–4; du kannst eine Personengruppe nennen.)\newline Vorbereitungszeit: 4 Minuten\newline Ziel: Einzelobjekt\newline Reichweite: Berührung\newline Wirkungsdauer: augenblicklich\newline Kosten: 4 KaP\newline Fertigkeiten: Nächtlicher Schatten, Wissen\newline Erlernen: Nan, Phe 14; Hes 18; 20 EP}
}


\newglossaryentry{meisterstück_Talent}
{
    name={Meisterstück},
    description={Während du dich in deine Handwerkskunst versenkst, wirst du mit Inspiration erfüllt. Du kannst ein Kunstwerk wie mit dem Vorteil Meisterwerk (S. 58) erschaffen. Besitzt du den Vorteil bereits, steigt die Qualität um eine weitere Stufe.\newline Probenschwierigkeit: Herstellungsschwierigkeit des Handwerksstücks\newline Vorbereitungszeit: 1 Tag\newline Ziel: selbst\newline Reichweite: Berührung\newline Wirkungsdauer: bis zur Fertigstellung\newline Kosten: 32 KaP\newline Fertigkeiten: Heiliges Handwerk, Wissen\newline Erlernen: Ang, Hes, Ing 16; 60 EP}
}


\newglossaryentry{schlangenstab_Talent}
{
    name={Schlangenstab},
    description={Dein Stab verwandelt sich in eine Smaragdnatter, die dich verteidigt. Die Natter verfügt über die Werte einer Schlange (S. 123) und die Vorteile Resistenz (profan) III und Schreckgestalt II (S. 98). Ihre Angriffe richten geweihten Schaden an.\newline Probenschwierigkeit: 12\newline Vorbereitungszeit: 1 Aktionen\newline Ziel: Einzelobjekt\newline Reichweite: Berührung\newline Wirkungsdauer: 1 Stunde\newline Kosten: 8 KaP\newline Fertigkeiten: Wissen\newline Erlernen: Hes 12; 40 EP}
}


\newglossaryentry{schrifttumfernerLande_Talent}
{
    name={Schrifttum ferner Lande},
    description={Du bist unerfahren in einer beliebigen Schrift.\newline Mächtige Liturgie: Je zwei Stufen steigern deine Schriftkenntnis auf erfahren/meisterlich.\newline Probenschwierigkeit: 12\newline Vorbereitungszeit: 16 Aktionen\newline Ziel: selbst\newline Reichweite: Berührung\newline Wirkungsdauer: 4 Minuten\newline Kosten: 8 KaP\newline Fertigkeiten: List, Wissen\newline Erlernen: Hes 14; Phe 18; 40 EP}
}


\newglossaryentry{sprechendeSymbole_Talent}
{
    name={Sprechende Symbole},
    description={Du erhältst einen groben Eindruck über die Bedeutung eines Symbols.\newline Mächtige Liturgie: Du erhältst einen ungefähren/guten/präzisen/vollständigen Eindruck.\newline Probenschwierigkeit: 12\newline Vorbereitungszeit: 4 Minuten\newline Ziel: selbst\newline Reichweite: Berührung\newline Wirkungsdauer: augenblicklich\newline Kosten: 4 KaP\newline Fertigkeiten: Wissen\newline Erlernen: Hes, Nan 12; 20 EP}
}


\newglossaryentry{sternenspur_Talent}
{
    name={Sternenspur},
    description={Du kannst mit einer einfachen Berührung einen leuchtenden Stern als Markierung auf einer Oberfläche hinterlassen, den nur du sehen kannst. Das funktioniert nur, wenn gerade weniger als 8 Sterne aktiv sind.\newline Mächtige Liturgie: Es können 4 weitere Sterne aktiv sein.\newline Probenschwierigkeit: 12\newline Vorbereitungszeit: 0 Aktionen\newline Ziel: selbst\newline Reichweite: Berührung\newline Wirkungsdauer: 1 Tag\newline Kosten: 4 KaP\newline Fertigkeiten: Nächtlicher Schatten, Wissen\newline Erlernen: Phe 14, Nan 18; 20 EP}
}


\newglossaryentry{urischarsordnenderBlick_Talent}
{
    name={Urischars ordnender Blick},
    description={Du siehst die verborgene Ordnung im Chaos: Du findest Bücher der gleichen Fachrichtung in einer Bibliothek, Gegenstände des gleichen Vorbesitzers beim Hehler, und das Muster in der Einbruchsliste der Stadtgarde. Entsprechende Proben sind um +4 erleichtert.\newline Mächtige Liturgie: Der Bonus steigt um +2.\newline Probenschwierigkeit: 12\newline Vorbereitungszeit: 4 Minuten\newline Ziel: selbst\newline Reichweite: 8 Schritt\newline Wirkungsdauer: augenblicklich\newline Kosten: 8 KaP\newline Fertigkeiten: Ordnung, Wissen\newline Erlernen: Pra 12, Nan 16; 20 EP}
}


\newglossaryentry{wundersameVerständigung_Talent}
{
    name={Wundersame Verständigung},
    description={Du bist unerfahren in einer beliebigen Sprache.\newline Mächtige Liturgie: Je zwei Stufen steigern deine Sprachkenntnis auf erfahren/meisterlich.\newline Probenschwierigkeit: 12\newline Modifikationen: Hoftag der Sprachen (Zone, Wirkungsdauer 4 Stunden, 16 KaP; die Liturgie wirkt auf alle Ziele in einem Radius von 8 Schritt)\newline Vorbereitungszeit: 16 Aktionen\newline Ziel: selbst\newline Reichweite: Berührung\newline Wirkungsdauer: 1 Stunde\newline Kosten: 8 KaP\newline Fertigkeiten: Fröhlicher Wanderer, List, Wissen\newline Erlernen: Phe 12; Aves, Nan 14; Hes 16; 40 EP}
}


\newglossaryentry{angroschsOpfergabe_Talent}
{
    name={Angroschs Opfergabe},
    description={Du kannst einen Gegenstand aus verarbeitetem Metall und Edelsteinen mit bis zu 8 Stein wieder mit dem Fels verschmelzen. Solche Gegenstände können nicht mehr herausgelöst werden.\newline Probenschwierigkeit: 12\newline Vorbereitungszeit: 16 Aktionen\newline Ziel: Einzelobjekt\newline Reichweite: Berührung\newline Wirkungsdauer: augenblicklich\newline Kosten: 2 KaP\newline Fertigkeiten: Heiliges Erz\newline Erlernen: Ang 12; 20 EP}
}


\newglossaryentry{eherneKraft_Talent}
{
    name={Eherne Kraft},
    description={Die Erde bebt. Alle Ziele in einem Radius von 4 Schritt werden niedergeschmettert, wenn eine Konterprobe (KK, 16) misslingt.\newline Mächtige Liturgie: Verdoppelt jeweils den Radius und richtet Schäden an loser Keramik/Möbeln/Holzbauten/Steinbauten an.\newline Probenschwierigkeit: 12\newline Modifikationen: Erdbeben (–8, 1 Stunde, 1 Meile, Wirkungsdauer 8 Initiativphasen, 32 KaP; Ingerimms Zorn fügt Gebäuden in einem Radius von 128 Schritt schwerste Schäden zu.)\newline Vorbereitungszeit: 4 Aktionen\newline Ziel: Zone\newline Reichweite: 4 Schritt\newline Wirkungsdauer: augenblicklich\newline Kosten: 8 KaP\newline Fertigkeiten: Heiliges Erz\newline Erlernen: Ang, Ing 18; 40 EP}
}


\newglossaryentry{geläutertseiErzundGoldgestein_Talent}
{
    name={Geläutert sei Erz und Goldgestein},
    description={Aus einem Erzklumpen mit bis zu 8 Stein Gewicht werden alle nutzbaren Metalle in reiner Form herausgepresst.\newline Mächtige Liturgie: Verdoppelt das Gewicht.\newline Probenschwierigkeit: 12\newline Vorbereitungszeit: 16 Aktionen\newline Ziel: Einzelobjekt\newline Reichweite: Berührung\newline Wirkungsdauer: augenblicklich\newline Kosten: 4 KaP\newline Fertigkeiten: Heiliges Erz\newline Erlernen: Ang 16; 20 EP}
}


\newglossaryentry{goldenerBlick_Talent}
{
    name={Goldener Blick},
    description={Du erspürst die Gegenwart und die genaue Lage von Bodenschätzen in einem Radius von 32 Schritt.\newline Mächtige Liturgie: Verdoppelt den Radius.\newline Probenschwierigkeit: 12\newline Vorbereitungszeit: 16 Aktionen\newline Ziel: selbst\newline Reichweite: Berührung\newline Wirkungsdauer: augenblicklich\newline Kosten: 8 KaP\newline Fertigkeiten: Heiliges Erz\newline Erlernen: Ang 16; Ing 18; 20 EP}
}


\newglossaryentry{ingerimmsZornverschoneuns_Talent}
{
    name={Ingerimms Zorn verschone uns},
    description={Der Stein des Ingerimm erscheint. Im Radius von 8 Meilen um diesen unbeweglichen Stein sind alle Lebewesen und Bauwerke vor den elementaren Kräften des Feuers und des Erzes (Brände, Steinschlag, Erdbeben, Vulkanausbrüche) geschützt.\newline Probenschwierigkeit: 12\newline Vorbereitungszeit: 4 Minuten\newline Ziel: Einzelobjekt\newline Reichweite: dereweit\newline Wirkungsdauer: 1 Woche\newline Kosten: 32 KaP\newline Fertigkeiten: Heiliges Erz\newline Erlernen: Ang, Ing 16; 20 EP}
}


\newglossaryentry{largorax‘Hammer_Talent}
{
    name={Largorax‘ Hammer},
    description={Du rufst Lagorax‘ Hammer herbei. Der Hammer erleichtert alle Schmieden-Proben an einem bestimmten Handwerksstück um +2. Der Hammer kann auch als Waffe gegen einen Drachen oder einen Gegenstand aus Erz oder Fels gerichtet werden und richtet 2W20+10 TP an. Er verschwindet nach einem Angriff.\newline Mächtige Liturgie: Erhöht die Erleichterung um +1 und den Schaden um +10.\newline Probenschwierigkeit: 12\newline Vorbereitungszeit: 4 Minuten\newline Ziel: Einzelobjekt\newline Reichweite: dereweit\newline Wirkungsdauer: 1 Woche\newline Kosten: 16 KaP\newline Fertigkeiten: Heiliges Erz, Heiliges Handwerk\newline Erlernen: Ang 18; 60 EP}
}


\newglossaryentry{lichtdesverborgenenPfades_Talent}
{
    name={Licht des verborgenen Pfades},
    description={An einer unterirdischen Weggabelung findest du die richtige Abzweigung zu deinem Ziel.\newline Probenschwierigkeit: 12\newline Vorbereitungszeit: 16 Aktionen\newline Ziel: selbst\newline Reichweite: Berührung\newline Wirkungsdauer: augenblicklich\newline Kosten: 2 KaP\newline Fertigkeiten: Heiliges Erz\newline Erlernen: Ang 12; Ing 14; 20 EP}
}


\newglossaryentry{sichererWegdurchdenFels_Talent}
{
    name={Sicherer Weg durch den Fels},
    description={Unter der Erde wird dein Ziel wie mit der Gabe Gefahreninstinkt vor natürlichen Gefahren gewarnt. Besitzt es die Gabe bereits, sind damit verbundene Proben um +4 erleichtert. Überleben-Proben sind um +4 erleichtert.\newline Mächtige Liturgie: Der Bonus steigt um +2.\newline Probenschwierigkeit: 12\newline Vorbereitungszeit: 4 Minuten\newline Ziel: Einzelperson\newline Reichweite: Berührung\newline Wirkungsdauer: 1 Tag\newline Kosten: 8 KaP\newline Fertigkeiten: Heiliges Erz\newline Erlernen: Ang, Ing 16; 20 EP}
}


\newglossaryentry{vertrauterdesFelsens_Talent}
{
    name={Vertrauter des Felsens},
    description={Du bist immun gegen Erzschaden und Schaden aus Steinschlag und Steinwaffen.\newline Probenschwierigkeit: 12\newline Modifikationen: Erweiterter Schutz (–4; der Schutz betrifft auch deine Ausrüstung.)\newline Vorbereitungszeit: 4 Aktionen\newline Ziel: selbst\newline Reichweite: Berührung\newline Wirkungsdauer: 1 Stunde\newline Kosten: 8 KaP\newline Fertigkeiten: Heiliges Erz\newline Erlernen: Ang 14; Ing 16; 20 EP}
}


\newglossaryentry{blickindieFlammen_Talent}
{
    name={Blick in die Flammen},
    description={Du erhältst grobe visionäre Einblicke in die Geschichte eines Brandes, der sich im Laufe des letzten Tages ereignet hat.\newline Mächtige Liturgie: Du erhältst einen ungefähren/guten/präzisen/vollständigen Eindruck. Spätestens dann erfährst du die Identität eines möglichen Brandstifters.\newline Probenschwierigkeit: 12\newline Modifikationen: Vergangener Brand (–4 pro Zeitstufe; der Brand hat sich im Laufe der letzten Woche/des letzten Monats/des letzten Jahres/seit Zwergengedenken ereignet.)\newline Vorbereitungszeit: 16 Aktionen\newline Ziel: Zone\newline Reichweite: Berührung\newline Wirkungsdauer: augenblicklich\newline Kosten: 8 KaP\newline Fertigkeiten: Heiliges Feuer\newline Erlernen: Ang 12; Ing 16; 20 EP}
}


\newglossaryentry{heiligeSchmiedeglut_Talent}
{
    name={Heilige Schmiedeglut},
    description={Ein Feuer wird so heiß, dass es zum Schmieden genutzt werden kann (Temperaturstufe Lava, S. 35). Mit dieser Liturgie und einigen transportablen Werkzeugen kannst du Bedingungen schaffen, die einer archaischen Schmiede entsprechen (S. 62).\newline Probenschwierigkeit: 12\newline Modifikationen: Allmacht der Lohe (–4, Einzelobjekt, 8 KaP; statt des Feuers erhitzt du direkt ein Metallstück von bis zu 8 Stein.)\newline Vorbereitungszeit: 4 Minuten\newline Ziel: Zone\newline Reichweite: 2 Schritt\newline Wirkungsdauer: 4 Stunden\newline Kosten: 4 KaP\newline Fertigkeiten: Heiliges Feuer\newline Erlernen: Ing 12; Ang 14; 40 EP}
}


\newglossaryentry{herrüberFeuerundGlut_Talent}
{
    name={Herr über Feuer und Glut},
    description={Du kannst ein bestehendes Feuer bis zur Größe eines großen Lagerfeuers formen. Die Verformung geschieht mit einer Geschwindigkeit von 4 Schritt pro Initiativephase und benötigt Konzentration.\newline Mächtige Liturgie: Du kannst Feuer bis zu einer Größe eines Scheiterhaufens/brennenden Hauses/einer brennenden Häusergruppe/eines brennenden Stadtviertels formen.\newline Probenschwierigkeit: 12\newline Modifikationen: Gebieter der Lava (–4; du kannst die Richtung von Lavaströmen bestimmen, solange die Lava dadurch nicht bergauf fließt.)\newline Vorbereitungszeit: 8 Aktionen\newline Ziel: Zone\newline Reichweite: 32 Schritt\newline Wirkungsdauer: 1 Stunde\newline Kosten: 8 KaP\newline Fertigkeiten: Heiliges Feuer\newline Erlernen: Ang, Ing 18; 40 EP\newline Anmerkung: Alternativ kann die Liturgie wie Antimagie gegen Feuerzauber wirken (S. 126). Die Randbedingungen entsprechen den dort angegebenen, außer dass die Kosten mit KaP bezahlt werden und die Liturgie um +4 erleichtert ist.}
}


\newglossaryentry{lodernderZorn_Talent}
{
    name={Lodernder Zorn},
    description={Stichflammen schießen aus einem mindestens fackelgroßen Feuer empor. Die Flammen richten 4W6 TP an und verursachen Nachbrennen (S. 98). Pro Schritt Entfernung fällt der niedrigste Würfel weg. Du selbst bist gegen den Schaden immun.\newline Mächtige Liturgie: Die TP steigen um 2W6.\newline Probenschwierigkeit: 12\newline Vorbereitungszeit: 4 Aktionen\newline Ziel: Zone\newline Reichweite: 32 Schritt\newline Wirkungsdauer: augenblicklich\newline Kosten: 16 KaP\newline Fertigkeiten: Heiliges Feuer\newline Erlernen: Ang 16; Ing 18; 60 EP\newline Anmerkung: Wenige Hochgeweihte kennen angeblich eine noch mächtigere Variante mit einem deutlich höheren Wirkungsbereich.}
}


\newglossaryentry{vertrauterderFlamme_Talent}
{
    name={Vertrauter der Flamme},
    description={Du bist immun gegen Feuerschaden und Hitze.\newline Modifikationen: Erweiterter Schutz (–4; der Schutz betrifft auch deine Ausrüstung.)\newline Vorbereitungszeit: 4 Aktionen\newline Ziel: selbst\newline Reichweite: Berührung\newline Wirkungsdauer: 1 Stunde\newline Kosten: 8 KaP\newline Fertigkeiten: Heiliges Feuer, Heim und Herd\newline Erlernen: Ang, Ing 12; Tra 18; 40 EP}
}


\newglossaryentry{waliburiasWehr_Talent}
{
    name={Waliburias Wehr},
    description={Bei jeder deiner Bewegungen stieben Funken und Rauch steigt auf. Für feindlich gesinnte Humanoide, Tiere, Feenwesen oder Mythenwesen giltst du als Schreckgestalt Stufe 2 (S. 98).\newline Probenschwierigkeit: 12\newline Vorbereitungszeit: 4 Aktionen\newline Ziel: selbst\newline Reichweite: Berührung\newline Wirkungsdauer: 16 Initiativphasen\newline Kosten: 8 KaP\newline Fertigkeiten: Guter Kampf, Heiliges Feuer\newline Erlernen: Ang 14; Ing 16; Kor 18; 40 EP}
}


\newglossaryentry{erneuerungdesGeborstenen_Talent}
{
    name={Erneuerung des Geborstenen},
    description={Ein einfacher Gegenstand wie eine Nahkampfwaffe oder ein Kutschenrad erhält sofort 8W6 Reparaturpunkte. Für jede Überschreitung der Härte wird eine Beschädigung repariert. Wirkt nicht auf völlig zerstörte Gegenstände.\newline Mächtige Liturgie: Erhöht die Reparaturpunkte um 16.\newline Probenschwierigkeit: 12\newline Modifikationen: Komplexer Gegenstand (–4; repariert einen komplexen Gegenstand wie eine Armbrust.)\newline Hochkomplexer Gegenstand (–8; repariert mechanische Meisterwerke wie ein Vinsalter Ei.)\newline Zerstört (–16, 16 KaP; wirkt nach Spielleiterentscheid auch auf zerstörte Gegenstände.)\newline Vorbereitungszeit: 16 Aktionen\newline Ziel: Einzelobjekt\newline Reichweite: Berührung\newline Wirkungsdauer: augenblicklich\newline Kosten: 8 KaP\newline Fertigkeiten: Heiliges Handwerk, Neubeginn\newline Erlernen: Ing 12; Ang 14; Tsa 16; 40 EP}
}


\newglossaryentry{göttlicheInspiration_Talent}
{
    name={Göttliche Inspiration},
    description={Wähle eine göttergefällige Freie Fertigkeit. Die Stufe dieser Freien Fertigkeit steigt um eins, oder du erhältst diese Fertigkeit auf unerfahren.\newline Tsa: Alle Fertigkeiten außer Sprachen und Schriften sind tsagefällig.\newline Rahja: Fertigkeiten mit Bezug zu Kunst, Weinbau und Pferdezucht.\newline Ingerimm \& Angrosch: Fertigkeiten des ehrbaren Handwerks.\newline Mächtige Liturgie: Je zwei Stufen erhöhen die Stufe der Freien Fertigkeit um 1.\newline Probenschwierigkeit: 12\newline Vorbereitungszeit: 1 Stunde\newline Ziel: selbst\newline Reichweite: Berührung\newline Wirkungsdauer: 1 Tag\newline Kosten: 8 KaP\newline Fertigkeiten: Heiliges Handwerk, Harmonie, Neubeginn\newline Erlernen: Tsa 12; Rah, Ing 16; Ang 18; 40 EP}
}


\newglossaryentry{handwerkssegen_Talent}
{
    name={Handwerkssegen},
    description={Der Segen erleichtert alle Proben zur Herstellung eines Handwerksstückes um +2.\newline Mächtige Liturgie: Der Bonus steigt um +1.\newline Probenschwierigkeit: 12\newline Vorbereitungszeit: 4 Minuten\newline Ziel: Einzelperson\newline Reichweite: Berührung\newline Wirkungsdauer: bis zur Fertigstellung\newline Kosten: 8 KaP\newline Fertigkeiten: Heiliges Handwerk\newline Erlernen: Ang, Ing 12; 60 EP}
}


\newglossaryentry{kunstverstand_Talent}
{
    name={Kunstverstand},
    description={Du erhältst einen groben Eindruck vom Verkaufswert des Gegenstandes.\newline Probenschwierigkeit: 12\newline Mächtige Liturgie: Du erhältst einen ungefähren/guten/genauen/detaillierten Eindruck.\newline Vorbereitungszeit: 1 Stunde\newline Ziel: Einzelobjekt\newline Reichweite: Berührung\newline Wirkungsdauer: augenblicklich\newline Kosten: 4 KaP\newline Fertigkeiten: Heiliges Handwerk, List\newline Erlernen: Ang 12; Ing, Phe 14; 20 EP}
}


\newglossaryentry{unterpfanddesHeiligenRhys_Talent}
{
    name={Unterpfand des Heiligen Rhys},
    description={Jeder Betrachter erkennt den gerechten Verkaufswert des gesegneten Gegenstandes. Um es zu einem anderen Wert zu erwerben oder zu kaufen, ist eine Konterprobe (Überreden, 16) nötig. Jeder Versuch, das Objekt magisch zu verändern oder zu zerstören, ist um –4 erschwert.\newline Mächtige Liturgie: Der Malus steigt um –2.\newline Probenschwierigkeit: 12\newline Vorbereitungszeit: 4 Minuten\newline Ziel: Einzelobjekt\newline Reichweite: Berührung\newline Wirkungsdauer: 1 Monat\newline Kosten: 8 KaP\newline Fertigkeiten: Heiliges Handwerk\newline Erlernen: Ing 18; 20 EP}
}


\newglossaryentry{fürbittendesHeiligenTherbûn_Talent}
{
    name={Fürbitten des Heiligen Therbûn},
    description={Du heilst einen Gesegneten von allen Krankheiten bis maximal Stufe 20.\newline Mächtige Liturgie: Die maximal aufgehobene Krankheitsstufe steigt um 4.\newline Probenschwierigkeit: 12\newline Vorbereitungszeit: 8 Stunden\newline Ziel: Einzelperson\newline Reichweite: Berührung\newline Wirkungsdauer: augenblicklich\newline Kosten: 8 KaP\newline Fertigkeiten: Heilung\newline Erlernen: Per 12; 20 EP\newline Anmerkung: Alternativ kann die Liturgie wie Antimagie gegen Krankheitszauber oder Krankheitsdämonen wirken (S. 126). Die Randbedingungen entsprechen den dort angegebenen, außer dass die Kosten mit KaP bezahlt werden und die Liturgie um +4 erleichtert ist.}
}


\newglossaryentry{großerGiftbann_Talent}
{
    name={Großer Giftbann},
    description={Du segnest eine Mahlzeit samt Getränken für bis zu 32 Personen. Die Mahlzeit wird von Giften und Krankheitsüberträgern gesäubert und sogar giftige Pflanzen können gegessen werden.\newline Mächtige Liturgie: Du segnest eine Mahlzeit für bis zu 16 weitere Personen.\newline Probenschwierigkeit: 12\newline Vorbereitungszeit: 4 Minuten\newline Ziel: Zone\newline Reichweite: Berührung\newline Wirkungsdauer: augenblicklich\newline Kosten: 8 KaP\newline Fertigkeiten: Heilung\newline Erlernen: Per 14; 20 EP}
}


\newglossaryentry{kleinerGiftbann_Talent}
{
    name={Kleiner Giftbann},
    description={Beendet die Wirkung eines Giftes bis maximal Stufe 20.\newline Mächtige Liturgie: Die maximal aufgehobene Giftstufe steigt um 4.\newline Probenschwierigkeit: 12\newline Vorbereitungszeit: 16 Aktionen\newline Ziel: Einzelperson\newline Reichweite: Berührung\newline Wirkungsdauer: augenblicklich\newline Kosten: 8 KaP\newline Fertigkeiten: Heilung\newline Erlernen: Per 16; 40 EP}
}


\newglossaryentry{lohnderUnverzagten_Talent}
{
    name={Lohn der Unverzagten},
    description={Du stärkst die Widerstandskraft des Gesegneten. Proben zur Abwehr von Krankheiten sind um +4 erleichtert.\newline Mächtige Liturgie: Der Bonus steigt um +2.\newline Probenschwierigkeit: 12\newline Vorbereitungszeit: 4 Minuten\newline Ziel: Einzelperson\newline Reichweite: Berührung\newline Wirkungsdauer: 1 Woche\newline Kosten: 4 KaP\newline Fertigkeiten: Heilung, Heim und Herd\newline Erlernen: Per 12; Tra 16; 20 EP}
}


\newglossaryentry{speisungderBedürftigen_Talent}
{
    name={Speisung der Bedürftigen},
    description={Du rufst einfaches und sättigendes Essen für 4 Menschen herbei.\newline Mächtige Liturgie: Das Essen sättigt 2 weitere Menschen.\newline Probenschwierigkeit: 12\newline Modifikationen: Heiliger Kessel (–8, Reichweite dereweit, 16 KaP; du rufst den Heiligen Kessel herbei. Die Nahrung in ihm reicht aus, eine große Menge von Bedürftigen zu speisen. Dann verschwindet der Kessel.)\newline Vorbereitungszeit: 4 Minuten\newline Ziel: Zone\newline Reichweite: Berührung\newline Wirkungsdauer: augenblicklich\newline Kosten: 4 KaP\newline Fertigkeiten: Heilung, Heim und Herd\newline Erlernen: Per, Tra 12; 20 EP}
}


\newglossaryentry{wundsegen_Talent}
{
    name={Wundsegen},
    description={Der Gesegnete erhält sofort 2W6+4 Heilpunkte, für jede Überschreitung der WS wird eine Wunde geheilt.\newline Mächtige Liturgie: Erhöht die Heilpunkte um 8.\newline Probenschwierigkeit: 12\newline Modifikationen: Blutung stoppen (4 Aktionen, 4 KaP; hebt Blutungen auf.)\newline Vorbereitungszeit: 16 Aktionen\newline Ziel: Einzelperson\newline Reichweite: Berührung\newline Wirkungsdauer: augenblicklich\newline Kosten: 8 KaP\newline Fertigkeiten: Friede, Heilung\newline Erlernen: Per 14; Tsa 16; 60 EP}
}


\newglossaryentry{dreifacherSaatsegen_Talent}
{
    name={Dreifacher Saatsegen},
    description={Das gesegnete Feld ist vor Ernteschäden durch schlechte Witterung geschützt. Jeder Versuch, die Ernte magisch zu zerstören, ist um –4 erschwert.\newline Mächtige Liturgie: Der Malus steigt um –2.\newline Probenschwierigkeit: 12\newline Vorbereitungszeit: 1 Stunde\newline Ziel: Zone\newline Reichweite: Berührung\newline Wirkungsdauer: 1 Monat\newline Kosten: 4 KaP\newline Fertigkeiten: Wachstum\newline Erlernen: Per 8; 0 EP}
}


\newglossaryentry{erneuerungdesLandes_Talent}
{
    name={Erneuerung des Landes},
    description={Dein Gebet schwächt dämonische Verseuchung in deiner Umgebung. Auf der Skala normal/leicht verseucht/deutlich verseucht/stark verseucht/kleines Unheiligtum/mittleres Unheiligtum/mächtiges Unheiligtum sinkt das Gebiet um eine Stufe.\newline Mächtige Liturgie: Die Verseuchung sinkt um eine weitere Stufe.\newline Probenschwierigkeit: 12\newline Vorbereitungszeit: 1 Stunde\newline Ziel: Zone\newline Reichweite: Berührung\newline Wirkungsdauer: 1 Jahr\newline Kosten: 16 KaP\newline Fertigkeiten: Wachstum\newline Erlernen: Per 18; 40 EP\newline Anmerkung: Verschiebst du die Skala auf normal, kehren die dämonischen Einflüsse nicht von selbst zurück – außer die Ursache der Verseuchung existiert weiterhin.}
}


\newglossaryentry{kälbchensegen_Talent}
{
    name={Kälbchensegen},
    description={Das neugeborene Tier ist immun gegen Krankheiten.\newline Probenschwierigkeit: 12\newline Vorbereitungszeit: 4 Minuten\newline Ziel: Tier\newline Reichweite: Berührung\newline Wirkungsdauer: 1 Monat\newline Kosten: 1 KaP\newline Fertigkeiten: Neubeginn, Wachstum\newline Erlernen: Per 12, Tsa 14; 0 EP}
}


\newglossaryentry{kraftdesLebens_Talent}
{
    name={Kraft des Lebens},
    description={Du segnest bis zu 4 Personen. Proben auf KO und KK sind um +2 erleichtert, Fertigkeitsproben mit einem dieser Attribute (außer Kampffertigkeiten) um +1.\newline Mächtige Liturgie: Der Bonus steigt um +2/+1.\newline Probenschwierigkeit: 12\newline Vorbereitungszeit: 16 Aktionen\newline Ziel: Zone\newline Reichweite: Berührung\newline Wirkungsdauer: 1 Stunde\newline Kosten: 8 KaP\newline Fertigkeiten: Wachstum\newline Erlernen: Per 16; 40 EP}
}


\newglossaryentry{parinorsVermächtnis_Talent}
{
    name={Parinors Vermächtnis},
    description={Die gesegnete Pflanze wächst doppelt so schnell, bis sie zu einem prächtigen Exemplar ihrer Art geworden ist.\newline Mächtige Liturgie: Die Pflanze wächst dreimal/viermal/fünfmal/sechsmal so schnell.\newline Probenschwierigkeit: 12\newline Modifikationen: Heilkraut (–4, 4 KaP; innerhalb einer Stunde wächst aus einem Samen eine durchschnittliche Heilpflanze.)\newline Vorbereitungszeit: 4 Minuten\newline Ziel: Pflanze\newline Reichweite: Berührung\newline Wirkungsdauer: permanent\newline Kosten: 8 KaP\newline Fertigkeiten: Wachstum\newline Erlernen: Per 16; 20 EP}
}


\newglossaryentry{perainesPflanzengespür_Talent}
{
    name={Peraines Pflanzengespür},
    description={Du erhältst einen groben Eindruck von der Wirkung der berührten Pflanze – etwa ob sie heilsam, essbar oder giftig ist.\newline Mächtige Liturgie: Du kannst eine weitere Pflanze untersuchen und erhältst einen ungefähren/guten/genauen/detaillierten Eindruck.\newline Probenschwierigkeit: 12\newline Vorbereitungszeit: 4 Minuten\newline Ziel: Pflanze\newline Reichweite: Berührung\newline Wirkungsdauer: augenblicklich\newline Kosten: 2 KaP\newline Fertigkeiten: Wachstum\newline Erlernen: Per 12; 20 EP}
}


\newglossaryentry{wohlverdienteRast_Talent}
{
    name={Wohlverdiente Rast},
    description={Du segnest bis zu 4 Personen. Die Gesegneten regenerieren 2 Punkte Erschöpfung.\newline Mächtige Magie: Regeneriert einen weiteren Punkt Erschöpfung.\newline Probenschwierigkeit: 12\newline Vorbereitungszeit: 16 Aktionen\newline Ziel: Zone\newline Reichweite: Berührung\newline Wirkungsdauer: augenblicklich\newline Kosten: 8 KaP\newline Fertigkeiten: Sichere Heimkehr, Wachstum\newline Erlernen: Per 12; Tra 14; 40 EP}
}


\newglossaryentry{phexensSternenwurf_Talent}
{
    name={Phexens Sternenwurf},
    description={Du rufst einen der Wurfsterne des Phex herbei. Dieser erleichtert die Fernkampfprobe um +4, richtet 2W20 TP an und verschwindet nach dem Wurf.\newline Mächtige Liturgie: Erhöht die TP um 1W20.\newline Probenschwierigkeit: 12\newline Vorbereitungszeit: 1 Aktion\newline Ziel: Einzelobjekt\newline Reichweite: dereweit\newline Wirkungsdauer: 4 Minuten\newline Kosten: 16 KaP\newline Fertigkeiten: Abu al'Mada\newline Erlernen: Phe 18; 60 EP}
}


\newglossaryentry{sternenstaub_Talent}
{
    name={Sternenstaub},
    description={Du wirfst Mondstaub in die Luft, der deine Gegner verwirrt. Alle Nah- und Fernkampfangriffe auf dich sind um –2 erschwert und Abzüge durch Dunkelheit sinken für dich um eine Stufe.\newline Mächtige Liturgie: Der Malus steigt um –1.\newline Probenschwierigkeit: 12\newline Vorbereitungszeit: 2 Aktionen\newline Ziel: selbst\newline Reichweite: Berührung\newline Wirkungsdauer: 4 Minuten\newline Kosten: 8 KaP\newline Fertigkeiten: Abu al'Mada\newline Erlernen: Phe 12; 40 EP}
}


\newglossaryentry{mondsilberzunge_Talent}
{
    name={Mondsilberzunge},
    description={Deine Zunge ist flink und die Worte fließen dir nur so von den Lippen. Überreden-Proben sind um +4 erleichtert.\newline Mächtige Liturgie: Der Bonus steigt um +2.\newline Probenschwierigkeit: 12\newline Vorbereitungszeit: 4 Aktionen\newline Ziel: selbst\newline Reichweite: Berührung\newline Wirkungsdauer: 1 Stunde\newline Kosten: 8 KaP\newline Fertigkeiten: Friede, Gutes Gold, List, Stiller Wanderer\newline Erlernen: Phe 12; Aves, Kor 16; Tsa 18; 40 EP}
}


\newglossaryentry{phexensAugenzwinkern_Talent}
{
    name={Phexens Augenzwinkern },
    description={Das Ziel kann sich kaum an dein Gesicht erinnern. Selbst wenn er darauf angesprochen oder verhört wird, sind alle KL-Proben zur Erinnerung um –4 erschwert. Die Liturgie wirkt nur, wenn deine wahre Identität unbekannt ist.\newline Mächtige Liturgie: Der Malus steigt um –2.\newline Probenschwierigkeit: 12\newline Vorbereitungszeit: 1 Aktion\newline Ziel: Einzelperson\newline Reichweite: 2 Schritt\newline Wirkungsdauer: augenblicklich\newline Kosten: 8 KaP\newline Fertigkeiten: List , Stiller Wanderer\newline Erlernen: Phe 12; Aves 16; 20 EP}
}


\newglossaryentry{sternenglanz_Talent}
{
    name={Sternenglanz},
    description={Du streust Mondstaub über das Objekt, das dadurch neu und wertvoll erscheint. Der Betrug lässt sich mit einer Konterprobe (Wachsamkeit oder passendes Handwerkstalent, 16) durchschauen.\newline Probenschwierigkeit: 12\newline Vorbereitungszeit: 4 Minuten\newline Ziel: Einzelobjekt\newline Reichweite: Berührung\newline Wirkungsdauer: 1 Stunde\newline Kosten: 8 KaP\newline Fertigkeiten: List\newline Erlernen: Phe 14; 40 EP}
}


\newglossaryentry{phexensElsterflug_Talent}
{
    name={Phexens Elsterflug},
    description={Das mit dieser Liturgie belegte Objekt von maximal 0,5 Stein Gewicht wird an Phexens Sternenhimmel entrückt. Es kehrt nach spätestens einem Jahr als Sternschnuppe zurück oder verbleibt am Sternenhimmel.\newline Mächtige Liturgie: Verdoppelt das maximale Gewicht.\newline Probenschwierigkeit: 12\newline Vorbereitungszeit: 4 Aktionen\newline Ziel: Einzelobjekt\newline Reichweite: Berührung\newline Wirkungsdauer: bis zu 1 Jahr\newline Kosten: 8 KaP\newline Fertigkeiten: Nächtlicher Schatten\newline Erlernen: Phe 14; 40 EP}
}


\newglossaryentry{phexensMeisterschlüssel_Talent}
{
    name={Phexens Meisterschlüssel},
    description={Du rufst Phexens Meisterschlüssel herbei, der fast jedes Schloss öffnet. Bei magisch gesicherten Schlössern wirkt die Liturgie als Konterprobe (8), bei deren Gelingen der Zauber aufgehoben wird.\newline Probenschwierigkeit: 12\newline Vorbereitungszeit: 4 Minuten\newline Ziel: Einzelobjekt\newline Reichweite: dereweit\newline Wirkungsdauer: 16 Initiativphasen\newline Kosten: 8 KaP\newline Fertigkeiten: Nächtlicher Schatten\newline Erlernen: Phe 18; 40 EP}
}


\newglossaryentry{phexensNebelleib_Talent}
{
    name={Phexens Nebelleib},
    description={Du verwandelst dich mitsamt Ausrüstung in Nebel, dem weder Waffen noch Magie etwas anhaben können. Du kannst dich mit GS 4 fortbewegen und sogar durch schmalste Öffnungen dringen.\newline Probenschwierigkeit: 12\newline Mächtige Liturgie: Erhöht die GS um +2.\newline Vorbereitungszeit: 4 Minuten\newline Ziel: selbst\newline Reichweite: Berührung\newline Wirkungsdauer: 8 Stunden\newline Kosten: 16 KaP\newline Fertigkeiten: Nächtlicher Schatten\newline Erlernen: Phe 14; 40 EP}
}


\newglossaryentry{sechsLebendesMungo_Talent}
{
    name={Sechs Leben des Mungo},
    description={Halbiert die effektive Höhe eines Sturzes oder Sprunges (kumulativ zur Akrobatik-Probe).\newline Probenschwierigkeit: 12\newline Vorbereitungszeit: 0 Aktionen\newline Ziel: selbst\newline Reichweite: Berührung\newline Wirkungsdauer: 16 Initiativphasen\newline Kosten: 8 KaP\newline Fertigkeiten: Nächtlicher Schatten\newline Erlernen: Phe 12; 40 EP}
}


\newglossaryentry{verborgenwiederNeumond_Talent}
{
    name={Verborgen wie der Neumond},
    description={Du passt dich deiner Umgebung an. Du erhältst den Vorteil Tarnung (S. 98) und deine Heimlichkeits-Proben sind um +4 erleichtert.\newline Mächtige Liturgie: Der Bonus steigt um +2.\newline Probenschwierigkeit: 12\newline Vorbereitungszeit: 2 Aktionen\newline Ziel: selbst\newline Reichweite: Berührung\newline Wirkungsdauer: 1 Stunde\newline Kosten: 8 KaP\newline Fertigkeiten: Nächtlicher Schatten\newline Erlernen: Phe 14; 40 EP}
}


\newglossaryentry{blendstrahlausAlveran_Talent}
{
    name={Blendstrahl aus Alveran},
    description={Ein gleißender Lichtblitz aus deiner Hand blendet dein Ziel. Er erleidet eine Erschwernis von –2 auf Proben und von –4 auf Zauber.\newline Mächtige Liturgie: Die Malusse steigen um –1/–2.\newline Probenschwierigkeit: 12\newline Vorbereitungszeit: 0 Aktionen\newline Ziel: Einzelperson\newline Reichweite: 16 Schritt\newline Wirkungsdauer: 8 Initiativphasen\newline Kosten: 4 KaP\newline Fertigkeiten: Licht\newline Erlernen: Pra 14; 20 EP}
}


\newglossaryentry{daradorsBannderSchatten_Talent}
{
    name={Daradors Bann der Schatten },
    description={Im Radius von 8 Schritt werden sämtliche Schatten, Dunkelheit und entsprechende Zauber aufgehoben. Wesenheiten mit dem Nachteil Lichtscheu erleiden einen Furcht-Effekt der Stufe 2, wenn ihre Beschwörungsschwierigkeit maximal 20 beträgt (beschworene Wesen) oder ihnen eine Konterprobe (MU, 16) misslingt (natürliche Wesen).\newline Mächtige Liturgie: Erhöht die maximale Beschwörungsschwierigkeit um 4.\newline Probenschwierigkeit: 12\newline Vorbereitungszeit: 4 Aktionen\newline Ziel: Zone\newline Reichweite: Berührung\newline Wirkungsdauer: 4 Minuten\newline Kosten: 8 KaP\newline Fertigkeiten: Licht\newline Erlernen: Pra 12; 20 EP}
}


\newglossaryentry{garafansgleißendeSchwingen_Talent}
{
    name={Garafans gleißende Schwingen},
    description={Du rufst einen Greifen (S. 101) herbei, der dir zur Seite steht. Was der Greif zu tun bereit ist und wie lange er bis zum Eintreffen benötigt, ist Spielleiterentscheid.\newline Probenschwierigkeit: 12\newline Vorbereitungszeit: 1 Stunde\newline Ziel: Einzelwesen\newline Reichweite: dereweit\newline Wirkungsdauer: augenblicklich\newline Kosten: 32 KaP\newline Fertigkeiten: Licht\newline Erlernen: Pra 18; 60 EP}
}


\newglossaryentry{goldeneRüstung_Talent}
{
    name={Goldene Rüstung},
    description={Du hüllst dich in eine Aureole gleißenden Lichts, das deine Gegner blendet. Sämtliche Nah- und Fernkampfangriffe und alle Zauber auf dich sind um –2 erschwert.\newline Mächtige Liturgie: Erhöht den Malus um –1.\newline Probenschwierigkeit: 12\newline Ziel: selbst\newline Reichweite: Berührung\newline Vorbereitungszeit: 2 Aktionen\newline Wirkungsdauer: 4 Minuten\newline Kosten: 8 KaP\newline Fertigkeiten: Licht\newline Erlernen: Pra 12; 40 EP}
}


\newglossaryentry{lichtdesHerrn_Talent}
{
    name={Licht des Herrn },
    description={Strahlendes Sonnenlicht erleuchtet deine Umgebung in einem Radius von 16 Schritt. Das Gebiet gilt als geweihter Boden. Die Liturgie gilt außerdem als Konterprobe (12) gegen Dunkelheitszauber, die bei Gelingen aufgehoben werden.\newline Mächtige Liturgie: Mit zwei Stufen gilt der Boden als heilig.\newline Probenschwierigkeit: 12\newline Vorbereitungszeit: 16 Aktionen\newline Ziel: selbst\newline Reichweite: Berührung\newline Wirkungsdauer: 1 Stunde\newline Kosten: 16 KaP\newline Fertigkeiten: Licht\newline Erlernen: Pra 14; 40 EP}
}


\newglossaryentry{ucurisGeleit_Talent}
{
    name={Ucuris Geleit},
    description={Ein Licht weist dir den Weg zum nächsten zwölfgöttlichen Geweihten oder Tempel.\newline Probenschwierigkeit: 12\newline Vorbereitungszeit: 16 Aktionen\newline Ziel: selbst\newline Reichweite: Berührung\newline Wirkungsdauer: 1 Stunde\newline Kosten: 4 KaP\newline Fertigkeiten: Licht\newline Erlernen: Pra 14; 20 EP}
}


\newglossaryentry{zerschmetternderBannstrahl_Talent}
{
    name={Zerschmetternder Bannstrahl},
    description={Ein gleißender Bannstrahl fährt aus dem Himmel herab und richtet an einem unheiligen Wesen oder Feenwesen 16W6 SP an und verbrennt ebenso viel Astralenergie.\newline Mächtige Liturgie: Richtet 16 SP zusätzlich an.\newline Probenschwierigkeit: 12\newline Vorbereitungszeit: 4 Aktionen\newline Ziel: Einzelwesen\newline Reichweite: 32 Schritt\newline Wirkungsdauer: augenblicklich\newline Kosten: 32 KaP\newline Fertigkeiten: Licht, Magiebann\newline Erlernen: Pra 18; 60 EP}
}


\newglossaryentry{praios‘Magiebann_Talent}
{
    name={Praios‘ Magiebann},
    description={In deiner Hand erscheint das Auge des Praios. Die Liturgie wirkt als Konterprobe (12) gegen alle Zauber im Radius von 16 Schritt, die bei gelungener Probe aufgehoben werden (permanente Zauber werden unterdrückt). Außerdem können keine neuen Zauber gewirkt oder Artefakte ausgelöst werden.\newline Probenschwierigkeit: 12\newline Vorbereitungszeit: 16 Aktionen\newline Ziel: Zone\newline Reichweite: Berührung\newline Wirkungsdauer: 1 Stunde\newline Kosten: 16 KaP\newline Fertigkeiten: Magiebann\newline Erlernen: Pra 18; 60 EP}
}


\newglossaryentry{praios‘Mahnung_Talent}
{
    name={Praios‘ Mahnung},
    description={Du nimmst dem Ziel eine von Praios‘ Gaben: Für den Rest der Wirkungsdauer ist das Ziel entweder unfähig zu sehen, sich zu orientieren oder Wahrheit von Lügen zu unterscheiden. Entsprechende Proben sind um –4 erschwert.\newline Mächtige Liturgie: Der Malus steigt um –2.\newline Probenschwierigkeit: 12\newline Vorbereitungszeit: 4 Minuten\newline Ziel: Einzelperson\newline Reichweite: Berührung\newline Wirkungsdauer: 1 Stunde\newline Kosten: 8 KaP\newline Fertigkeiten: Ordnung\newline Erlernen: Pra 14; 20 EP}
}


\newglossaryentry{willezurWahrheit_Talent}
{
    name={Wille zur Wahrheit},
    description={Alle Zuhörer im Radius von 16 Schritt halten sich an die Wahrheit, Versuche zu Lügen sind um –4 erschwert.\newline Mächtige Liturgie: Der Malus steigt um –2.\newline Probenschwierigkeit: 12\newline Vorbereitungszeit: 4 Minuten\newline Ziel: Zone\newline Reichweite: Berührung\newline Wirkungsdauer: 1 Stunde\newline Kosten: 16 KaP\newline Fertigkeiten: Ordnung\newline Erlernen: Pra 16; 40 EP}
}


\newglossaryentry{augefürdieSchönheit_Talent}
{
    name={Auge für die Schönheit},
    description={Der Gesegnete kann die Schönheit in sich selbst oder jemand anderem sehen. Dabei wird er nicht durch schlechte Eigenheiten oder andere Charakterschwächen und Vorurteile abgelenkt.\newline Probenschwierigkeit: 12\newline Vorbereitungszeit: 1 Aktion\newline Ziel: Einzelperson\newline Reichweite: Berührung\newline Wirkungsdauer: augenblicklich\newline Kosten: 1 KaP\newline Fertigkeiten: Harmonie\newline Erlernen: Rah 8; 0 EP}
}


\newglossaryentry{ewigeJugend_Talent}
{
    name={Ewige Jugend},
    description={Der Gesegnete behält Zeit seines Lebens ein jugendliches Aussehen. Frevelt der Gesegnete gegen die segenspendende Gottheit, setzt die natürliche Alterung wieder ein.\newline Probenschwierigkeit: 12\newline Modifikationen: Ewige Schönheit (–8, 32 KaP, Wirkungsdauer augenblicklich; verbessert das Aussehen des Gesegneten. Eine Eigenheit, die sich auf Hässlichkeit bezieht, verschwindet oder er erhält eine Eigenheit, die sich auf seine Schönheit bezieht.)\newline Vorbereitungszeit: 1 Tag\newline Ziel: Einzelperson\newline Reichweite: Berührung\newline Wirkungsdauer: bis die Bindung gelöst wird\newline Kosten: 16 KaP, davon 4 gKaP\newline Fertigkeiten: Harmonie, Neubeginn\newline Erlernen: Rah, Tsa 16; 20 EP}
}


\newglossaryentry{festderFreude_Talent}
{
    name={Fest der Freude},
    description={Du erbittest Rahjas Segen für ein Fest. Essen, Getränke und Unterhaltung übertreffen die Erwartungen und negative Eigenheiten (und nicht verregelte Charakterschwächen) können ignoriert werden. Aggressive Handlungen erfordern eine Konterprobe (Willenskraft, 20). Die Wirkung endet gemeinsam mit dem Fest.\newline Probenschwierigkeit: 12\newline Vorbereitungszeit: 4 Minuten\newline Ziel: Zone\newline Reichweite: Berührung\newline Wirkungsdauer: 1 Tag\newline Kosten: 16 KaP\newline Fertigkeiten: Harmonie\newline Erlernen: Rah 12; 20 EP}
}


\newglossaryentry{harmonischerRausch_Talent}
{
    name={Harmonischer Rausch },
    description={Du genießt mit dem Gesegneten gemeinsam den Rausch (egal ob mit Alkohol, Drogen, Sex, Tanz, einem Ausritt oder einem reinen Adrenalinrausch). Danach erfüllt euch die Harmonie Rahjas, durch die Furcht-Effekte um eine Stufe verringert sind.\newline Probenschwierigkeit: 12\newline Vorbereitungszeit: 16 Aktionen\newline Ziel: Einzelperson\newline Reichweite: Berührung\newline Wirkungsdauer: 1 Woche\newline Kosten: 8 KaP\newline Fertigkeiten: Harmonie, Rausch\newline Erlernen: Rah 12; 40 EP}
}


\newglossaryentry{levthansFesselung_Talent}
{
    name={Levthans Fesselung},
    description={Du rufst das heilige Levthansband herbei, das sich um das Ziel legt und es bindet. Das Ziel muss ein Daimonid, Dämon, Elementar, Feenwesen oder Untoter sein und die Beschwörungsschwierigkeit von maximal 24 aufweisen. Während der Fesselung ist das Ziel entrückt und zu keiner Handlung fähig, kann aber auch nicht zum Ziel von AT, Zaubern oder Liturgien werden.\newline Mächtige Liturgie: Erhöht die Beschwörungsschwierigkeit um 4.\newline Probenschwierigkeit: 12\newline Vorbereitungszeit: 4 Aktionen\newline Ziel: Einzelwesen\newline Reichweite: 4 Schritt\newline Wirkungsdauer: 1 Tag\newline Kosten: 16 KaP\newline Fertigkeiten: Harmonie\newline Erlernen: Rah 18; 60 EP\newline Anmerkung: Legenden zufolge soll es eine Variante der Liturgie geben, die sogar halbgöttliche Wesen binden kann. }
}


\newglossaryentry{rahjasWohlgefallen_Talent}
{
    name={Rahjas Wohlgefallen},
    description={Du rufst die Gunst der Göttin auf dich herab. Menschenkenntnis- und Betören-Proben sind um +4 erleichtert.\newline Mächtige Liturgie: Der Bonus steigt um +2.\newline Probenschwierigkeit: 12\newline Vorbereitungszeit: 4 Aktionen\newline Ziel: selbst\newline Reichweite: Berührung\newline Wirkungsdauer: 1 Stunde\newline Kosten: 8 KaP\newline Fertigkeiten: Fröhlicher Wanderer, Harmonie\newline Erlernen: Rah 12; Aves 16; 40 EP}
}


\newglossaryentry{schicksalsgemeinschaft_Talent}
{
    name={Schicksalsgemeinschaft},
    description={Du stärkst das gegenseitige Vertrauen zwischen bis zu 8 treuen Gefährten. Häufig wird dieser Segen auf eine Familie (Tra), ein Liebespaar (Rah) oder eine Reisegruppe (Aves) gewirkt. Einmal pro Woche kann einer der Gesegneten einen Schicksalspunkt an einen anderen weitergeben.\newline Mächtige Liturgie: Der Schicksalspunkt kann einmal pro Tag/4 Stunden/1 Stunde/jederzeit weitergegeben werden.\newline Probenschwierigkeit: 12\newline Modifikationen: Permanenz (–4, Wirkungsdauer bis die Bindung gelöst wird, Kosten 8 KaP, davon 2 gKaP)\newline Vorbereitungszeit: 16 Aktionen\newline Ziel: Zone\newline Reichweite: Berührung\newline Wirkungsdauer: 1 Woche\newline Kosten: 8 KaP\newline Fertigkeiten: Harmonie, Heim und Herd, Stiller Wanderer\newline Erlernen: Rah 14; Aves, Tra 16; 40 EP}
}


\newglossaryentry{sulvasGnade_Talent}
{
    name={Sulvas Gnade},
    description={Die Gnade der göttlichen Stute stärkt deine Beziehung zu einem Pferd. Erstens sind Proben im Umgang mit dem Pferd sind um +4 erleichtert, Freie Fertigkeiten (wie Abrichter) zählen als eine Stufe höher. Zweitens erhältst du den Vorteil Tierempathie (S. 30) für dieses Pferd und erleidest keine Erschöpfung durch den Einsatz dieser Gabe. Besitzt du die Gabe bereits, sind damit verbundene Proben um +4 erleichtert. Die dafür nötigen Proben kannst du auch auf Harmonie statt auf Überleben ablegen.\newline Mächtige Liturgie: Erhöht den Bonus um +2.\newline Probenschwierigkeit: 12\newline Vorbereitungszeit: 1 Stunde\newline Ziel: Tier\newline Reichweite: Berührung\newline Wirkungsdauer: bis die Bindung gelöst wird\newline Kosten: 8 KaP, davon 2 gKaP\newline Fertigkeiten: Harmonie\newline Erlernen: Rah 14; 40 EP}
}


\newglossaryentry{tharvunsSchwingen_Talent}
{
    name={Tharvuns Schwingen},
    description={Das gesegnete Pferd ist unaufhaltsam wie der göttliche Hengst. Seine GS erhöht sich um +4 und es verfügt über ein unendliches DH (S. 34). Alle Versuche, es in seiner Bewegung einzuschränken (etwa über Reiten, KK oder einen Corpofrigo) sind um –4 erschwert.\newline Mächtige Liturgie: Der Bonus steigt um +2, der Malus um –2.\newline Probenschwierigkeit: 12\newline Vorbereitungszeit: 1 Aktion\newline Ziel: Tier\newline Reichweite: Berührung\newline Wirkungsdauer: 1 Stunde\newline Kosten: 4 KaP\newline Fertigkeiten: Fröhlicher Wanderer, Harmonie\newline Erlernen: Aves, Rah 16; 20 EP}
}


\newglossaryentry{verbrüderungderFeinde_Talent}
{
    name={Verbrüderung der Feinde},
    description={Alle Lebewesen in einer Zone von 4 Schritt Radius verfallen in eine friedfertige Stimmung. Aggressive Handlungen erfordern eine Konterprobe (Willenskraft, 20). Erfordert Konzentration, erlaubt Aufrechterhalten.\newline Mächtige Liturgie: Verdoppelt den Radius.\newline Probenschwierigkeit: 12\newline Modifikationen: Heiliger Friedensschluss (–16, 1 Stunde, Wirkungsdauer 1 Woche, 32 KaP; du rufst das Joborner Friedenslicht (Rah) oder das Eidechsenauge (Tsa) herbei, das ganze Armeen in eine friedliche Stimmung versetzt und meist in einem großes Fest (Rah) oder eifrige Diskussionen über die Zukunft (Tsa) enden. Die Liturgie wirkt nicht, wenn eine der Seiten von dämonischen Kräften getrieben wird.)\newline Vorbereitungszeit: 4 Aktionen\newline Ziel: Zone\newline Reichweite: Berührung\newline Wirkungsdauer: 1 Stunde\newline Kosten: 8 KaP\newline Fertigkeiten: Friede, Harmonie\newline Erlernen: Tsa 12, Rah 14; 40 EP}
}


\newglossaryentry{entfesselnderRausch_Talent}
{
    name={Entfesselnder Rausch},
    description={Du genießt mit dem Gesegneten gemeinsam den Rausch (egal ob mit Alkohol, Drogen, Sex, Tanz, einem Ausritt oder einem reinen Adrenalinrausch). Dabei bekommt ihr einem tiefen Einblick in die Seele des Gegenübers: Ihr erfahrt von den Eigenheiten des jeweils anderen und Menschenkenntnis-Proben zwischen euch sind um +8 erleichtert, alle anderen Menschenkenntnis-Proben um +2.\newline Probenschwierigkeit: 12\newline Vorbereitungszeit: 16 Aktionen\newline Ziel: Einzelperson\newline Reichweite: Berührung\newline Wirkungsdauer: 1 Woche\newline Kosten: 4 KaP\newline Fertigkeiten: Rausch\newline Erlernen: Rah 12; 20 EP}
}


\newglossaryentry{göttlicheFreiheit_Talent}
{
    name={Göttliche Freiheit},
    description={Der Gesegnete wird von einer Fessel befreit, zum Beispiel von einer Eisenkette, von einer Sucht oder einem Einflusszauber. Körperliche Fesseln verschwinden dabei, geistige Fesseln werden während der Wirkungsdauer unterdrückt. Handelt sich es bei der Fessel um einen Zauber, wirkt die Liturgie als Konterprobe (8) gegen diesen Zauber. Bei einer gelungenen Probe wird der Zauber aufgehoben.\newline Probenschwierigkeit: 12\newline Vorbereitungszeit: 16 Aktionen\newline Ziel: Einzelwesen\newline Reichweite: Berührung\newline Wirkungsdauer: 1 Tag\newline Kosten: 8 KaP\newline Fertigkeiten: Fröhlicher Wanderer, Neubeginn, Rausch\newline Erlernen: Aves, Rah, Tsa 12; 40 EP}
}


\newglossaryentry{rahjasgeheiligterWein_Talent}
{
    name={Rahjas geheiligter Wein},
    description={Du rufst den Kelch der Rahja. In den Kelch gegossene Flüssigkeiten verwandeln sich in Wein, Wein erhält hingegen ein einzigartiges Aroma. Am Ende der Wirkungsdauer verschwindet der Kelch, die Verwandlung ist permanent.\newline Probenschwierigkeit: 12\newline Vorbereitungszeit: 16 Aktionen\newline Ziel: selbst\newline Reichweite: dereweit\newline Wirkungsdauer: 4 Minuten\newline Kosten: 2 KaP\newline Fertigkeiten: Rausch\newline Erlernen: Rah 12; 20 EP\newline Anmerkung: Erfahrene Rahjageweihte (oft Tempelvorsteher) können mit einer Variante dieser Liturgie den geweihten Tharf erschaffen, den Rahjageweihte der Göttin opfern (S. 84).}
}


\newglossaryentry{rahjalinasKuss_Talent}
{
    name={Rahjalinas Kuss},
    description={Du erhältst einen kurzen Einblick in den Traum deines Ziels. Über den Inhalt musst du anderen gegenüber schweigen.\newline Probenschwierigkeit: 12\newline Modifikationen: Traumreise (–4, 8 KaP; du kannst in den Traum des Schlafenden reisen.)\newline Große Traumreise (–8, 16 KaP; du kannst mit einigen Gefährten in den Traum reisen.)\newline Vorbereitungszeit: 16 Aktionen\newline Ziel: Einzelperson\newline Reichweite: Berührung\newline Wirkungsdauer: 1 Stunde\newline Kosten: 4 KaP\newline Fertigkeiten: Rausch\newline Erlernen: Rah 16; 20 EP}
}


\newglossaryentry{rahjasSinnlichkeit_Talent}
{
    name={Rahjas Sinnlichkeit},
    description={Durch den Segen der Göttin erhältst du Resistenz I (profan, magisch, geweiht, siehe S. 98). Gleichzeitig sind aber alle Sinneseindrücke verstärkt, weswegen du schon beim Erleiden von einer Wunde eine Wundschmerz-Probe ablegen musst.\newline Mächtige Liturgie: Je zwei Stufen erhöhen die Resistenz um eine Stufe.\newline Probenschwierigkeit: 12\newline Vorbereitungszeit: 16 Aktionen\newline Ziel: selbst\newline Reichweite: Berührung\newline Wirkungsdauer: 1 Stunde\newline Kosten: 16 KaP\newline Fertigkeiten: Rausch\newline Erlernen: Rah 18; 60 EP}
}


\newglossaryentry{reichungdesAmethyst_Talent}
{
    name={Reichung des Amethyst},
    description={Beendet die Wirkung eines Giftes bis maximal Stufe 20.\newline Mächtige Liturgie: Die maximal aufgehobene Giftstufe steigt um 4.\newline Probenschwierigkeit: 12\newline Modifikationen: Rausch beenden (1 KaP; du beendest einen Rauschzustand.)\newline Vorbereitungszeit: 16 Aktionen\newline Ziel: Einzelperson\newline Reichweite: Berührung\newline Wirkungsdauer: augenblicklich\newline Kosten: 8 KaP\newline Fertigkeiten: Rausch\newline Erlernen: Rah 12; 40 EP}
}


\newglossaryentry{segenderRotenSchwester_Talent}
{
    name={Segen der Roten Schwester},
    description={Der Gesegnete wird von einem glücklichen Rausch ergriffen, wenn ihm keine Konterprobe (Zähigkeit, 20) gelingt. Er erwacht nur, wenn er eine Wunde erleidet.\newline Probenschwierigkeit: 12\newline Vorbereitungszeit: 1 Aktion\newline Ziel: Einzelperson\newline Reichweite: Berührung\newline Wirkungsdauer: 1 Stunde\newline Kosten: 8 KaP\newline Fertigkeiten: Rausch\newline Erlernen: Rah 16; 40 EP}
}


\newglossaryentry{tanzderSchwerter_Talent}
{
    name={Tanz der Schwerter},
    description={Dein Kampfstil gleicht einem hypnotischen Tanz. Deine GS erhöht sich um +4 und alle AT und VT sind um +2 erleichtert.\newline Mächtige Liturgie: Die GS steigt um weitere +2.\newline Probenschwierigkeit: 12\newline Modifikationen: Fliegender Tanz (–8; du wirst nicht mehr zum Ziel von Passierschlägen.)\newline Vorbereitungszeit: 0 Aktionen\newline Ziel: selbst\newline Reichweite: Berührung\newline Wirkungsdauer: 16 Initiativphasen\newline Kosten: 4 KaP\newline Fertigkeiten: Rausch\newline Erlernen: Rah 18; 40 EP}
}


\newglossaryentry{ehrenhafterKampf_Talent}
{
    name={Ehrenhafter Kampf},
    description={Dein Gegner muss sich an die Gebote des ehrenhaften Zweikampfs halten und niemand kann in den Kampf eingreifen. Mit einer Konterprobe (Willenskraft, 20) kann die Wirkung ignoriert werden.\newline Probenschwierigkeit: 12\newline Modifikationen: Ehre der Schlacht (Zone, Wirkungsdauer 1 Stunde, Kosten 16 KaP; alle Kämpfer im Radius von 8 Schritt, denen die Konterprobe misslingt, verhalten sich rondragefällig. Mächtige Liturgie verdoppelt den Radius.)\newline Vorbereitungszeit: 2 Aktionen\newline Ziel: Einzelperson\newline Reichweite: 32 Schritt\newline Wirkungsdauer: 16 Initiativphasen\newline Kosten: 8 KaP\newline Fertigkeiten: Ehre, Heerführung\newline Erlernen: Ron 8; 40 EP}
}


\newglossaryentry{fürdieGöttin,fürRondra!_Talent}
{
    name={Für die Göttin, für Rondra!},
    description={Die Macht der Göttin erfüllt dich, während du einen Sturmangriff durchführst (S. 41). Der erste Sturmangriff während der Wirkungsdauer trifft alle Gegner in deiner Bahn, bis du durch mindestens 2 erfolgreiche Verteidigungen gestoppt wirst. Alle AT gelten als Angriffe zum Niederwerfen.\newline Mächtige Liturgie: Die AT ist um +1 erleichtert.\newline Probenschwierigkeit: 12\newline Vorbereitungszeit: 0 Aktionen\newline Ziel: selbst\newline Reichweite: Berührung\newline Wirkungsdauer: 16 Initiativphasen\newline Kosten: 8 KaP\newline Fertigkeiten: Ehre\newline Erlernen: Ron 14; 40 EP}
}


\newglossaryentry{schildderEhre_Talent}
{
    name={Schild der Ehre},
    description={Ungezielte Geschosse (wie in einem Pfeilhagel) treffen dich nicht, Fernkampfangriffe und Zauber auf dich sind um –4 erschwert.\newline Mächtige Magie: Erhöht den Malus um –2.\newline Probenschwierigkeit: 12\newline Vorbereitungszeit: 1 Aktion\newline Ziel: selbst\newline Reichweite: Berührung\newline Wirkungsdauer: 16 Initiativphasen\newline Kosten: 8 KaP\newline Fertigkeiten: Ehre\newline Erlernen: Ron 14; 40 EP}
}


\newglossaryentry{segenderHeiligenArdare_Talent}
{
    name={Segen der Heiligen Ardare},
    description={Du rufst das heilige Schwert Armalion herbei. Die Waffe richtet 2W20+10 SP an, hat einen WM von +2 und ist wendig.\newline Probenschwierigkeit: 12\newline Vorbereitungszeit: 4 Aktionen\newline Ziel: Einzelobjekt\newline Reichweite: dereweit\newline Wirkungsdauer: 16 Initiativphasen\newline Kosten: 32 KaP\newline Fertigkeiten: Ehre\newline Erlernen: Ron 20; 60 EP}
}


\newglossaryentry{bundderSchwerter_Talent}
{
    name={Bund der Schwerter },
    description={Du wählst bis zu 8 Mitstreiter. Bis zum Ende der Schlacht spürst du, in welcher Richtung und ungefährer Entfernung sie sich befinden. Du fühlst auch ihren Tod.\newline Mächtige Liturgie: Du spürst, wenn die Mitstreiter kampfunfähig werden/eine Wunde erleiden/sie auf übermächtige Gegner treffen.\newline Probenschwierigkeit: 12\newline Modifikationen: Permanenz (–4, Kosten 4 KaP, davon 2 gKaP, Wirkungsdauer bis die Bindung gelöst wird; der Bund wirkt immer, wenn einer der Mitglieder in einem Kampf gerät.)\newline Vorbereitungszeit: 16 Aktionen\newline Ziel: Zone\newline Reichweite: Berührung\newline Wirkungsdauer: 8 Stunden\newline Kosten: 4 KaP\newline Fertigkeiten: Heerführung\newline Erlernen: Ron 14; 20 EP}
}


\newglossaryentry{ritusderSchlachthilfe_Talent}
{
    name={Ritus der Schlachthilfe},
    description={Während du meditierst, kann deine Schwertseele einem Gefährten im Kampf beistehen. Sie erscheint als geisterhafte Walküre und kann 8 Angriffe oder Paraden führen. Die Schwertseele verwendet deine Kampfwerte und -fähigkeiten, ohne Abzüge durch Einschränkungen oder andere negative Effekte. Erfordert Konzentration.\newline Mächtige Liturgie: 4 weitere Angriffe oder Paraden sind möglich.\newline Probenschwierigkeit: 12\newline Vorbereitungszeit: 1 Aktion\newline Ziel: Einzelperson\newline Reichweite: 1 Meile\newline Wirkungsdauer: 16 Initiativphasen\newline Kosten: 8 KaP\newline Fertigkeiten: Heerführung\newline Erlernen: Ron 18; 40 EP\newline Anmerkung: Wenn ihr den Bund der Schwerter geschlossen habt, steigt die Reichweite auf aventurienweit.}
}


\newglossaryentry{rondrasHochzeit_Talent}
{
    name={Rondras Hochzeit},
    description={Im Radius von 1 Meile kannst du den Wind auf einer Skala von windstill/leichte Brise/steife Brise/Gewittersturm/gewittriger Orkan um zwei Stufen verändern und seine Richtung lenken.\newline Mächtige Liturgie: Du kannst den Wind um eine weitere Stufe verändern und den Radius verdoppeln.\newline Probenschwierigkeit: 12\newline Vorbereitungszeit: 4 Minuten\newline Ziel: Zone\newline Reichweite: Berührung\newline Wirkungsdauer: 8 Stunden\newline Kosten: 16 KaP\newline Fertigkeiten: Heerführung\newline Erlernen: Ron 18; 40 EP}
}


\newglossaryentry{segendesHeiligenHlûthar_Talent}
{
    name={Segen des Heiligen Hlûthar},
    description={Du wählst bis zu 8 Mitstreiter. Auf ihnen lastende Furcht-Effekte gelten als eine Stufe niedriger.\newline Probenschwierigkeit: 12\newline Modifikationen: Mut der Schlacht (–8, 16 KaP, Wirkungsdauer 4 Stunden; die Wirkung betrifft alle Mitglieder des Heeres, dem du angehörst.)\newline Vorbereitungszeit: 16 Aktionen\newline Ziel: Zone\newline Reichweite: Berührung\newline Wirkungsdauer: 1 Stunde\newline Kosten: 8 KaP\newline Fertigkeiten: Gutes Gold, Heerführung\newline Erlernen: Ron 14; Kor 16; 40 EP}
}


\newglossaryentry{weihesegenderWaffe_Talent}
{
    name={Weihesegen der Waffe},
    description={Du kannst vier Nahkampfwaffen weihen.\newline Mächtige Liturgie: Du weihst zwei weitere Waffen.\newline Probenschwierigkeit: 12\newline Modifikationen: Heeressegen (–8, 32 KaP; du segnest alle Waffen in einem Radius von 64 Schritt.)\newline Vorbereitungszeit: 16 Aktionen\newline Ziel: Zone\newline Reichweite: Berührung\newline Wirkungsdauer: 1 Tag\newline Kosten: 8 KaP\newline Fertigkeiten: Gutes Gold, Heerführung\newline Erlernen: Ron 12; Kor 14; 40 EP}
}


\newglossaryentry{zornderSturmherrin_Talent}
{
    name={Zorn der Sturmherrin},
    description={In einen passenden Ort deiner Nähe (ein Baum, eine Felsspitze) schlägt ein Blitz ein und ein gewaltiger Donnerknall ertönt. Ein Furcht-Effekt Stufe 2 befällt alle Feinde im Radius von 64 Schritt, denen eine Konterprobe (MU, 20) misslingt (natürliche Wesen) oder deren Beschwörungsschwierigkeit maximal 20 beträgt (beschworene Wesen). Die Liturgie kann auch bei heiterem Himmel oder unter der Erde eingesetzt werden, der Blitz schießt dann aus der Decke.\newline Mächtige Liturgie: Erhöht die maximale Beschwörungsschwierigkeit um 4. Je zwei Stufen erhöhen den Furcht-Effekt um eine Stufe.\newline Probenschwierigkeit: 12\newline Vorbereitungszeit: 4 Aktionen\newline Ziel: Zone\newline Reichweite: 32 Schritt\newline Wirkungsdauer: 16 Initiativphasen\newline Kosten: 16 KaP\newline Fertigkeiten: Heerführung\newline Erlernen: Ron 18; 40 EP}
}


\newglossaryentry{rondraswundersameRüstung_Talent}
{
    name={Rondras wundersame Rüstung},
    description={Du rufst die wundersame Rüstung, durch die dein RS um 2 steigt.\newline Mächtige Liturgie: Je zwei Stufen erhöhen den RS um +1.\newline Probenschwierigkeit: 12\newline Vorbereitungszeit: 4 Aktionen\newline Ziel: selbst\newline Reichweite: Berührung\newline Wirkungsdauer: 1 Stunde\newline Kosten: 8 KaP\newline Fertigkeiten: Schutz der Gläubigen\newline Erlernen: Ron 16; 40 EP}
}


\newglossaryentry{rondragabundsFührung_Talent}
{
    name={Rondragabunds Führung},
    description={Die Heilige Rondragabund führt die Hand des Gesegneten. Bei Verteidigungen kann er statt seines eigentlichen PW einen PW von 8 verwenden.\newline Mächtige Liturgie: Der Gesegnete kann einen PW von 10/12/14/16 verwenden.\newline Probenschwierigkeit: 12\newline Vorbereitungszeit: 1 Aktion\newline Ziel: Einzelperson\newline Reichweite: 4 Schritt\newline Wirkungsdauer: 16 Initiativphasen\newline Kosten: 8 KaP\newline Fertigkeiten: Schutz der Gläubigen\newline Erlernen: Ron 16; 40 EP}
}


\newglossaryentry{schwingendesSturms_Talent}
{
    name={Schwingen des Sturms},
    description={Du reitest auf den aufkommenden Sturmböen. Während der Wirkungsdauer kannst du zwei mal die Freie Aktion Sturmritt ausführen, in der du dich bis zu 8 Schritt bewegst und eine Attacke (als Teil der freien Aktion) ausführst. Falls diese Attacke ein Ausfall ist, ist das Manöver nicht erschwert. Die Bewegung mit dem Sturmritt löst keine Passierschläge aus und kann beim Spiel mit Bodenplänen auch durch Felder mit anderen Kämpfern führen.\newline Mächtige Liturgie: Du kannst die Freie Aktion ein zusätzliches Mal ausführen.\newline Probenschwierigkeit: 12\newline Vorbereitungszeit: 1 Aktion\newline Ziel: selbst\newline Reichweite: Berührung\newline Wirkungsdauer: 16 Initiativphasen\newline Kosten: 8 KaP\newline Fertigkeiten: Schutz der Gläubigen\newline Erlernen: Ron 16; 60 EP}
}


\newglossaryentry{segnungderstählernenStirn_Talent}
{
    name={Segnung der stählernen Stirn},
    description={Der Gesegnete fasst Mut. Auf ihm lastende Furcht-Effekte gelten als zwei Stufen niedriger.\newline Mächtige Liturgie: Je zwei Stufen senken Furcht-Effekte um eine weitere Stufe.\newline Probenschwierigkeit: 12\newline Vorbereitungszeit: 4 Aktionen\newline Ziel: Einzelperson\newline Reichweite: Berührung\newline Wirkungsdauer: 8 Stunden\newline Kosten: 8 KaP\newline Fertigkeiten: Schutz der Gläubigen\newline Erlernen: Ron 8; 40 EP}
}


\newglossaryentry{thalionmelsSchlachtgesang_Talent}
{
    name={Thalionmels Schlachtgesang},
    description={Du rufst Thalionmels Segen auf dich herab und erleidest keine Wundabzüge. Kannst du Wundabzüge ohnehin ignorieren, sind Proben gegen Wundschmerz um deine Wundabzüge erleichtert. Außerdem kannst du eine (weitere) VT zwischen zwei Initiativephasen als freie Reaktion ausführen. Erlaubt Aufrechterhalten.\newline Probenschwierigkeit: 12\newline Modifikationen: Thalionmels Opfer (Wirkungsdauer 1 Stunde; du bis zusätzlich immun gegen Entwaffnen-, Niederwerfen- oder Betäubungseffekte und bleibst kampffähig, bis du 16 Wunden erlitten hast. Am Ende der Wirkungsdauer ruft dich Rondra zu sich.)\newline Vorbereitungszeit: 4 Aktionen\newline Ziel: selbst\newline Reichweite: Berührung\newline Wirkungsdauer: 16 Initiativphasen\newline Kosten: 8 KaP\newline Fertigkeiten: Schutz der Gläubigen\newline Erlernen: Ron 16; 60 EP}
}


\newglossaryentry{wundersamesTeilendesMartyriums_Talent}
{
    name={Wundersames Teilen des Martyriums},
    description={Jede zweite Wunde, die dein Ziel erleidet, wird auf dich übertragen (beginnend mit der ersten).\newline Probenschwierigkeit: 12\newline Vorbereitungszeit: 0 Aktionen\newline Ziel: Einzelperson\newline Reichweite: 16 Schritt\newline Wirkungsdauer: 1 Stunde\newline Kosten: 8 KaP\newline Fertigkeiten: Schutz der Gläubigen\newline Erlernen: Ron 14; 20 EP}
}


\newglossaryentry{entzugvonTraviasGaben_Talent}
{
    name={Entzug von Travias Gaben},
    description={Du rufst Travias Strafe auf einen reuelosen Sünder herab. Wenn ihm eine Konterprobe (Willenskraft, 20) misslingt, sind Proben auf Beeinflussung, Gebräuche und Autorität um –4 erschwert und mögliche Gastgeber versagen ihm die Aufnahme.\newline Mächtige Liturgie: Der Malus steigt um –2.\newline Probenschwierigkeit: 12\newline Modifikationen: Permanenz (–4, Wirkungsdauer bis die Bindung gelöst wird, 8 KaP, davon 2 gKaP)\newline Vorbereitungszeit: 4 Minuten\newline Ziel: Einzelperson\newline Reichweite: 16 Schritt\newline Wirkungsdauer: 1 Woche\newline Kosten: 8 KaP\newline Fertigkeiten: Heim und Herd\newline Erlernen: Tra 12; 20 EP}
}


\newglossaryentry{hausfrieden_Talent}
{
    name={Hausfrieden},
    description={Alle Lebewesen im gesegneten Zuhause verfallen in eine friedfertige Stimmung. Aggressive Handlungen erfordern eine Konterprobe (Willenskraft, 20).\newline Probenschwierigkeit: 12\newline Vorbereitungszeit: 16 Aktionen\newline Ziel: Zone\newline Reichweite: Berührung\newline Wirkungsdauer: 4 Stunden\newline Kosten: 8 KaP\newline Fertigkeiten: Heim und Herd\newline Erlernen: Tra 8; 20 EP}
}


\newglossaryentry{hilfederGemeinschaft_Talent}
{
    name={Hilfe der Gemeinschaft},
    description={Du stärkst den Zusammenhalt in einer Familie, zwischen treuen Gefährten oder einer Pilgergruppe. Muss eine Gruppenprobe abgelegt werden, bei der ein einziger Misserfolg zum Scheitern führt, ist diese Probe um +4 erleichtert. Außerdem erleidet die Gruppe keinen Malus, wenn ein Versuch der Zusammenarbeit scheitert.\newline Mächtige Liturgie: Erhöht den Bonus um +2.\newline Probenschwierigkeit: 12\newline Modifikationen: Permanenz (–4, Wirkungsdauer bis die Bindung gelöst wird, Kosten 8 KaP, davon 2 gKaP)\newline Vorbereitungszeit: 16 Aktionen\newline Ziel: Zone\newline Reichweite: Berührung\newline Wirkungsdauer: 1 Woche\newline Kosten: 8 KaP\newline Fertigkeiten: Fröhlicher Wanderer, Heim und Herd\newline Erlernen: Aves, Tra 14; 40 EP}
}


\newglossaryentry{traviniansSegenderSchwelle_Talent}
{
    name={Travinians Segen der Schwelle},
    description={Du segnest die Schwelle. Der dahinter liegende Raum gilt als geweihter Boden und kann von unheiligen Wesenheiten mit einer Beschwörungsschwierigkeit von maximal 24 nicht betreten werden.\newline Mächtige Liturgie: Erhöht die Beschwörungsschwierigkeit um 4.\newline Probenschwierigkeit: 12\newline Vorbereitungszeit: 4 Aktionen\newline Ziel: Einzelobjekt\newline Reichweite: Berührung\newline Wirkungsdauer: 1 Stunde\newline Kosten: 4 KaP\newline Fertigkeiten: Heim und Herd\newline Erlernen: Tra 18; 20 EP}
}


\newglossaryentry{weihedesHeimsteins_Talent}
{
    name={Weihe des Heimsteins},
    description={Du segnest einen Heimstein, der in die Unterkunft eingebaut wird. Solange sich die Bewohner in der Unterkunft befinden und an die Gesetze Travias halten, können sie die Auswirkung negativer sozialer Eigenheiten ignorieren. Außerdem bleibt die Unterkunft vor den Naturgewalten verschont. Letzteres gilt nicht, wenn die Unterkunft in einem besonders gefährdeten Gebiet errichtet wurde.\newline Probenschwierigkeit: 12\newline Vorbereitungszeit: 8 Stunden\newline Ziel: Einzelobjekt\newline Reichweite: Berührung\newline Wirkungsdauer: bis die Unterkunft deutlich umgebaut wird\newline Kosten: 16 KaP\newline Fertigkeiten: Heim und Herd\newline Erlernen: Tra 8; 0 EP}
}


\newglossaryentry{freundlicheAufnahme_Talent}
{
    name={Freundliche Aufnahme},
    description={Du findest den direktesten Weg zum nächsten bewohnten Heim im Radius von 16 Meilen, dessen Bewohner dir mindestens Neutral gesonnen sind. Dieser Weg ist passierbar, aber nicht zwingend ungefährlich.\newline Mächtige Liturgie: Verdoppelt den Radius.\newline Probenschwierigkeit: 12\newline Vorbereitungszeit: 4 Minuten\newline Ziel: selbst\newline Reichweite: Berührung\newline Wirkungsdauer: augenblicklich\newline Kosten: 4 KaP\newline Fertigkeiten: Fröhlicher Wanderer, Sichere Heimkehr\newline Erlernen: Tra 12, Aves 14; 20 EP}
}


\newglossaryentry{gänsegeschnatter_Talent}
{
    name={Gänsegeschnatter},
    description={Im Laufe des Tages fliegt dir eine Wildgans (Tra) oder ein Zugvogel (Aves) zu, die dich fortan begleitet. Das Tier hat äußerst scharfe Sinne (Wachsamkeit 12) und schlägt lautstark Alarm, sobald sich jemand deinem Nachtlager in böser Absicht nähert. Es verlässt dich, wenn du es schlecht behandelst, oder die Wirkungsdauer endet. Die Wirkung endet auch, wenn das Tier stirbt (WS 2).\newline Mächtige Liturgie: Die Wachsamkeit des Tiers steigt um +4.\newline Probenschwierigkeit: 12\newline Modifikationen: Permanenz (–4, Wirkungsdauer bis die Bindung gelöst wird, 8 KaP, davon 2 gKaP)\newline Vorbereitungszeit: 4 Minuten\newline Ziel: selbst\newline Reichweite: Berührung\newline Wirkungsdauer: 1 Woche\newline Kosten: 8 KaP\newline Fertigkeiten: Fröhlicher Wanderer, Sichere Heimkehr\newline Erlernen: Aves 14; Tra 16; 40 EP}
}


\newglossaryentry{gebetderverborgenenHalle_Talent}
{
    name={Gebet der verborgenen Halle},
    description={Durch dein Gebet erscheint eine Tür in einen verborgenen Raum, in dem eine angenehme Temperatur herrscht und ausreichend Nahrung und Getränke für alle Schutzsuchenden bereit hält. Sobald die Tür geschlossen ist, kann der Raum von außen weder gefunden noch betreten werden. Bricht einer der Schutzsuchenden Travias Gebote, endet die Wirkung sofort. Wer sich am Ende der Wirkungsdauer im Raum befindet oder den Raum freiwillig verlässt, befindet sich am Ursprungsort.\newline Probenschwierigkeit: 12\newline Vorbereitungszeit: 4 Minuten\newline Ziel: Zone\newline Reichweite: Berührung\newline Wirkungsdauer: 1 Woche\newline Kosten: 16 KaP\newline Fertigkeiten: Sichere Heimkehr\newline Erlernen: Tra 18; 60 EP}
}


\newglossaryentry{gespürdesHeimsteins_Talent}
{
    name={Gespür des Heimsteins},
    description={Du erspürst die Richtung, in der sich ein von dir geweihter Heimstein befindet (siehe S. 199). Eine Überleben-Probe zur Orientierung ist um +4 erleichtert. Der Bonus erhöht sich auf +8, wenn sich ein Heimstein innerhalb eines Radius von 8 Meilen befindet.\newline Mächtige Liturgie: Verdoppelt den Radius für den höheren Bonus.\newline Probenschwierigkeit: 12\newline Vorbereitungszeit: 4 Aktionen\newline Ziel: Zone\newline Reichweite: dereweit\newline Wirkungsdauer: 16 Initiativphasen\newline Kosten: 4 KaP\newline Fertigkeiten: Sichere Heimkehr\newline Erlernen: Tra 16; 20 EP}
}


\newglossaryentry{reisesegen_Talent}
{
    name={Reisesegen},
    description={Der Segen Travias begleitet den Reisenden und warnt ihn vor Arglist und Tücke. Er kann Gefahren der Zivilisation (Überfälle, Diebe, betrügerische Führer) wie mit der Gabe Gefahreninstinkt erkennen. Besitzt er die Gabe bereits, sind damit verbundene Proben um +4 erleichtert. Wirkt nicht in dämonisch pervertiertem Gebiet.\newline Mächtige Liturgie: Gebräuche-Proben sind um +2 erleichtert.\newline Probenschwierigkeit: 12\newline Vorbereitungszeit: 4 Minuten\newline Ziel: Einzelperson\newline Reichweite: Berührung\newline Wirkungsdauer: 2 Tage\newline Kosten: 8 KaP\newline Fertigkeiten: Sichere Heimkehr, Stiller Wanderer\newline Erlernen: Aves, Tra 14; 40 EP}
}


\newglossaryentry{traviniansSegendesLagerfeuers_Talent}
{
    name={Travinians Segen des Lagerfeuers},
    description={Du segnest ein Lagerfeuer. Unheilige Wesenheiten in seinem Schein erleiden einen Furcht-Effekt der Stufe 2, wenn ihre Beschwörungsschwierigkeit maximal 20 beträgt.\newline Mächtige Liturgie: Erhöht die maximale Beschwörungsschwierigkeit um 4.\newline Probenschwierigkeit: 12\newline Modifikationen: Wärmendes Feuer (Wirkungsdauer 8 Stunden, 4 KaP; statt der gewöhnlichen Wirkung steigt die Temperatur um das Feuer zusätzlich um eine Stufe, bis maximal auf normal (siehe S. 35))\newline Vorbereitungszeit: 16 Aktionen\newline Ziel: Zone\newline Reichweite: 2 Schritt\newline Wirkungsdauer: 1 Stunde\newline Kosten: 8 KaP\newline Fertigkeiten: Sichere Heimkehr\newline Erlernen: Tra 14; 40 EP}
}


\newglossaryentry{auradesRegenbogens_Talent}
{
    name={Aura des Regenbogens},
    description={Das Licht des Regenbogens erleuchtet deine Umgebung im Radius von 8 Schritt. Das Gebiet gilt als geweihter Boden. Zusätzlich erleiden alle Chimären innerhalb der Zone einen Furcht-Effekt Stufe 2.\newline Mächtige Liturgie: Mit zwei Stufen gilt der Boden als heilig.\newline Probenschwierigkeit: 12\newline Vorbereitungszeit: 16 Aktionen\newline Ziel: selbst\newline Reichweite: Berührung\newline Wirkungsdauer: 1 Stunde\newline Kosten: 16 KaP\newline Fertigkeiten: Friede\newline Erlernen: Tsa 18; 40 EP}
}


\newglossaryentry{gabederEwigjungen_Talent}
{
    name={Gabe der Ewigjungen},
    description={In deiner Hand erscheint eine Frühlingsblume. Solange die Blume blüht, erhält ihr Träger einen Bonus von +1 auf Menschenkenntnis. Die Blume verblüht, wenn der Träger einer Person begegnet, die ihm feindlich gesonnen ist, er sich gegen Tsa versündigt oder am Ende der Wirkungsdauer.\newline Probenschwierigkeit: 12\newline Vorbereitungszeit: 1 Aktion\newline Ziel: Zone\newline Reichweite: Berührung\newline Wirkungsdauer: 1 Woche\newline Kosten: 4 KaP\newline Fertigkeiten: Friede\newline Erlernen: Tsa 14; 40 EP}
}


\newglossaryentry{kirschblütenregen_Talent}
{
    name={Kirschblütenregen},
    description={Kirschblüten regnen aus dem Himmel herab. Nah- und Fernkampfangriffe auf dich sind um –4 erschwert, solange du selbst nicht angreifst.\newline Mächtige Liturgie: Der Malus steigt um –1.\newline Probenschwierigkeit: 12\newline Vorbereitungszeit: 2 Aktionen\newline Ziel: selbst\newline Reichweite: Berührung\newline Wirkungsdauer: 4 Minuten\newline Kosten: 8 KaP\newline Fertigkeiten: Friede\newline Erlernen: Tsa 14; 40 EP}
}


\newglossaryentry{tsasLebensschutz_Talent}
{
    name={Tsas Lebensschutz},
    description={Verstirbt der Gesegnete während der Wirkung der Liturgie, erscheint ein überraschend stabiles  Ei an einem sicheren Ort in der Nähe. Nach einer Stunde schlüpft der Gesegnete aus dem Ei und erreicht innerhalb einer Woche seine ursprüngliche Größe (währenddessen Einschränkungen nach Spielleiterentscheid). Das Aussehen ähnelt seinem früheren Aussehen, kleinere Veränderungen wie eine bunte Haarsträhne sind aber wahrscheinlich. Die Wirkung der Liturgie kann nur einmal ausgelöst werden, dann endet ihre Wirkung.\newline Probenschwierigkeit: 12\newline Modifikationen: Permanenz (–8, Wirkungsdauer bis die Bindung gelöst wird, Kosten 16 KaP, davon 8 gKaP)\newline Vorbereitungszeit: 16 Aktionen\newline Ziel: Einzelperson\newline Reichweite: Berührung\newline Wirkungsdauer: 1 Tag\newline Kosten: 16 KaP\newline Fertigkeiten: Friede, Neubeginn\newline Erlernen: Tsa 16; 60 EP}
}


\newglossaryentry{willezumFrieden_Talent}
{
    name={Wille zum Frieden},
    description={Alle Zuhörer halten sich ans Gebot der Friedfertigkeit. Alle Versuche, Zwietracht zu sähen oder die Zuhörer aufzuwiegeln, sind um –4 erschwert.\newline Mächtige Liturgie: Der Malus steigt um –2.\newline Probenschwierigkeit: 12\newline Vorbereitungszeit: 4 Minuten\newline Ziel: Zone\newline Reichweite: Berührung\newline Wirkungsdauer: 1 Stunde\newline Kosten: 16 KaP\newline Fertigkeiten: Friede\newline Erlernen: Tsa 16; 40 EP}
}


\newglossaryentry{hautdesChamäleons_Talent}
{
    name={Haut des Chamäleons},
    description={Nach dem Willen Tsas ändern sich deine Haare, Gesichtszüge und Körperbau soweit, dass du nicht mehr als du selbst zu erkennen bist (Details sind Spielleiterentscheid). Du behältst dabei dein Geschlecht und bist immer noch demselben Kulturkreis zuzuordnen.\newline Mächtige Liturgie: Für je zwei Stufen kannst du auch den Kulturkreis/das Geschlecht verändern.\newline Probenschwierigkeit: 12\newline Modifikationen: Fremdverwandlung (–4, Einzelperson; verwandelt ein freiwilliges Ziel)\newline Vorbereitungszeit: 4 Aktionen\newline Ziel: selbst\newline Reichweite: Berührung\newline Wirkungsdauer: 1 Stunde\newline Kosten: 8 KaP\newline Fertigkeiten: Neubeginn\newline Erlernen: Tsa 18; 40 EP}
}


\newglossaryentry{salajanasSegen_Talent}
{
    name={Salajanas Segen},
    description={Du milderst den Wehenschmerz der Gesegneten.\newline Probenschwierigkeit: 12\newline Modifikationen: Salajanas Schutz (–8, Wirkungsdauer 1 Jahr, 8 KaP; die Gesegnete bleibt während der Schwangerschaft vor Krankheiten verschont)\newline Vorbereitungszeit: 16 Aktionen\newline Ziel: Einzelperson\newline Reichweite: Berührung\newline Wirkungsdauer: 1 Tag\newline Kosten: 4 KaP\newline Fertigkeiten: Neubeginn\newline Erlernen: Tsa 8; 0 EP}
}


\newglossaryentry{segensreicherNeuanfang_Talent}
{
    name={Segensreicher Neuanfang},
    description={Du segnest ein neues Unternehmen. Der Gesegnete erhält während der Wirkungsdauer die Eigenheit „Euphorisch für [Unternehmen]: Du widmest dich mit Elan der bevorstehenden Aufgabe und vergisst dabei alles um dich herum.“ Für den ersten Einsatz der Eigenheit bezahlt er keinen Schicksalspunkt. Für gesegnete NSC gelten die Vorschläge von S. 16.\newline Mächtige Liturgie: Für je zwei Stufen kostet ein weiterer Einsatz der Eigenheit keinen Schicksalspunkt.\newline Probenschwierigkeit: 12\newline Vorbereitungszeit: 16 Aktionen\newline Ziel: Einzelperson\newline Reichweite: Berührung\newline Wirkungsdauer: 1 Woche\newline Kosten: 8 KaP\newline Fertigkeiten: Neubeginn\newline Erlernen: Tsa 14; 40 EP}
}


\newglossaryentry{tsaslachendeGefolgschaft_Talent}
{
    name={Tsas lachende Gefolgschaft},
    description={Du rufst einen Kobold zur Hilfe. Befinden sich Kobolde in einem Radius von 1 Meile, eilt einer von ihnen herbei. Du kannst den Kobold um einen Gefallen bitten, aber er entscheidet selbst, ob er den Gefallen erfüllt.\newline Mächtige Liturgie: Verdoppelt den Radius.\newline Probenschwierigkeit: 12\newline Modifikationen: Namensruf (–4; du rufst einen dir bereits bekannten Kobold in einem Radius von 8 Meilen herbei.)\newline Vorbereitungszeit: 2 Aktionen\newline Ziel: Zone\newline Reichweite: Berührung\newline Wirkungsdauer: augenblicklich\newline Kosten: 4 KaP\newline Fertigkeiten: Neubeginn\newline Erlernen: Tsa 18; 20 EP}
}


\newglossaryentry{tsaswunderbareErneuerung_Talent}
{
    name={Tsas wunderbare Erneuerung},
    description={Ein verlorenes Körperteil wächst innerhalb einer Woche nach (währenddessen Einschränkungen nach Spielleiterentscheid). Der Verlust des Körperteils darf maximal 1 Stunde zurück liegen.\newline Mächtige Liturgie: Der Verlust darf maximal 1 Tag/1 Woche/1 Monat/1 Jahr zurückliegen.\newline Probenschwierigkeit: 12\newline Vorbereitungszeit: 1 Stunde\newline Ziel: Einzelperson\newline Reichweite: Berührung\newline Wirkungsdauer: augenblicklich\newline Kosten: 16 KaP\newline Fertigkeiten: Neubeginn\newline Erlernen: Tsa 16; 0 EP}
}


\newglossaryentry{tsasFruchtbarkeit_Talent}
{
    name={Tsas Fruchtbarkeit},
    description={Das gesegnete Lebewesen ist für die Wirkungsdauer deutlich fruchtbarer.\newline Probenschwierigkeit: 12\newline Vorbereitungszeit: 1 Stunde\newline Ziel: Einzelwesen\newline Reichweite: Berührung\newline Wirkungsdauer: 1 Woche\newline Kosten: 1 KaP\newline Fertigkeiten: Neubeginn\newline Erlernen: Tsa 12; 0 EP}
}


\newglossaryentry{unschuldigerGeist_Talent}
{
    name={Unschuldiger Geist},
    description={Du ermöglichst dem Gesegneten einen unvoreingenommenen Blick auf eine Entscheidung oder eine Person. Der Gesegnete verliert während der Wirkungsdauer alle früheren Erinnerungen, aber auch alle negativen Eigenheiten wie Vorurteile oder andere Charakterschwächen. Solche Entscheidungen erweisen sich meist als richtig – regeltechnisch kann er den Spielleiter diesbezüglich um einen Tipp bitten (wie mit dem Vorteil Eingebung, S. 59).\newline Mächtige Liturgie: Je zwei Stufen ermöglichen einen zusätzlichen Tipp.\newline Probenschwierigkeit: 12\newline Vorbereitungszeit: 1 Stunde\newline Ziel: Einzelperson\newline Reichweite: Berührung\newline Wirkungsdauer: 1 Stunde\newline Kosten: 8 KaP\newline Fertigkeiten: Neubeginn\newline Erlernen: Tsa 16; 40 EP}
}


\newglossaryentry{allerWeltFreund_Talent}
{
    name={Aller Welt Freund},
    description={Du rufst die Lapislazuliflöte. Solange du sie spielst, können dich Humanoide nur mit einer gelungenen Konterprobe (Willenskraft, 20) angreifen und Tiere gar nicht. Das Gebiet in 16 Schritt Radius gilt als geweihter Boden. Benötigt Konzentration, ermöglicht Aufrechterhalten.\newline Mächtige Liturgie: Mit zwei Stufen gilt der Boden als heilig.\newline Probenschwierigkeit: 12\newline Vorbereitungszeit: 16 Aktionen\newline Ziel: Einzelobjekt\newline Reichweite: dereweit\newline Wirkungsdauer: 1 Stunde\newline Kosten: 16 KaP\newline Fertigkeiten: Fröhlicher Wanderer\newline Erlernen: Aves 14; 60 EP}
}


\newglossaryentry{einFreundinderNot_Talent}
{
    name={Ein Freund in der Not},
    description={Durch Aves‘ Hilfe findest du den Weg zu einer hilfsbereiten Person im Radius von 1 Meile, die dich aus einer Notlage zu befreien versucht. Kommen mehrere Personen in Frage, führt der Weg zu der Person, die du schnell genug erreichen kannst und die dir am besten helfen kann.\newline Mächtige Magie: Verdoppelt den Radius.\newline Modifikationen: Aktiver Helfer (–4; der Helfer bewegt sich zusätzlich unbewusst in deine Richtung, sofern ihm daraus kein bedeutender Nachteil erwächst.)\newline Probenschwierigkeit: 12\newline Vorbereitungszeit: 4 Aktionen\newline Ziel: selbst\newline Reichweite: Berührung\newline Wirkungsdauer: 1 Tag\newline Kosten: 8 KaP\newline Fertigkeiten: Fröhlicher Wanderer\newline Erlernen: Aves 16; 40 EP\newline Anmerkung: Mit dieser Liturgie fanden schon viele Avesgeweihte Hilfe an den seltsamsten Orten – als hätte der Weber des Schicksals selbst seine Hände im Spiel. Wenn also eine geringe Chance auf Hilfe besteht, sollte der Spielleiter sie gewähren. An völlig menschenleeren Orten wie dem ewigen Eis des Nordens ist die Liturgie hingegen nutzlos. }
}


\newglossaryentry{überdieWolken_Talent}
{
    name={Über die Wolken},
    description={An deinem Rücken wachsen gewaltige Schwingen, mit denen du fliegen kannst. Dafür sind keine weiteren Proben nötig, nur besonders tollkühne Manöver erfordern eine GE- oder Akrobatik-Probe. Du bewegst dich in der Luft etwa so schnell wie am Boden und kannst etwa ein Viertel so viel tragen. Erlaubt Aufrechterhalten.\newline Mächtige Liturgie: Du bewegst dich zwei/drei/vier/fünfmal schneller als am Boden und kannst zwei/drei/vier/fünf Viertel so viel tragen.\newline Probenschwierigkeit: 12\newline Vorbereitungszeit: 16 Aktionen\newline Ziel: selbst\newline Reichweite: Berührung\newline Wirkungsdauer: 1 Stunde\newline Kosten: 16 KaP\newline Fertigkeiten: Fröhlicher Wanderer\newline Erlernen: Aves 18; 60 EP}
}


\newglossaryentry{blickdesKartographen_Talent}
{
    name={Blick des Kartographen},
    description={Du kannst dir alle Landmarken, denen du während der Wirkungsdauer auf deiner Reise begegnest, perfekt einprägen. Reist du noch einmal durch die eingeprägte Region, sind Proben zur Orientierung um +4 erleichtert. Die Erinnerung verschwindet, wenn du eine Karte der Region anfertigst oder die Gelegenheit dazu verstreichen lässt.\newline Probenschwierigkeit: 12\newline Vorbereitungszeit: 1 Stunde\newline Ziel: selbst\newline Reichweite: Berührung\newline Wirkungsdauer: 1 Woche\newline Kosten: 4 KaP\newline Fertigkeiten: Stiller Wanderer\newline Erlernen: Aves 12; 20 EP\newline Anmerkung: Eine mit diesem Wissen angefertigte Karte bringt einen Bonus von +2 (entsprechende Freie Fertigkeit auf erfahren) oder sogar +4 (Freie Fertigkeit auf meisterlich) auf Orientierungs-Proben in der Region.}
}


\newglossaryentry{versöhnlichesSchicksal_Talent}
{
    name={Versöhnliches Schicksal},
    description={Wenn eine mit glücklicher Fügung verbesserte Probe misslingt, erhältst du bei 1–2 auf dem W6 den Schicksalspunkt zurück.\newline Probenschwierigkeit: 12\newline Mächtige Liturgie: Erhöht die Chance um 1 auf dem W6.\newline Modifikationen: Zweite Chance (–8; die Liturgie wirkt auch, wenn eine mit einem Schicksalspunkt wiederholte Probe misslingt. Das gilt nur für die erste solche Wiederholung.)\newline Vorbereitungszeit: 1 Stunde\newline Ziel: selbst\newline Reichweite: Berührung\newline Wirkungsdauer: bis die Bindung gelöst wird\newline Kosten: 8 KaP, davon 2 gKaP\newline Fertigkeiten: Stiller Wanderer\newline Erlernen: Aves 16; 40 EP}
}


\newglossaryentry{blutfürBlut_Talent}
{
    name={Blut für Blut},
    description={Wenn du im Nahkampf eine Wunde erleidest, kannst du sofort einen Passierschlag gegen den Angreifer durchführen. Für je zwei (insgesamt) erlittene Wunden ist der Passierschlag um +1 erleichtert.\newline Probenschwierigkeit: 12\newline Vorbereitungszeit: 0 Aktionen\newline Ziel: selbst\newline Reichweite: Berührung\newline Wirkungsdauer: 8 Initiativphasen\newline Kosten: 8 KaP\newline Fertigkeiten: Guter Kampf\newline Erlernen: Kor 12; 60 EP}
}


\newglossaryentry{blutigerSchnitter_Talent}
{
    name={Blutiger Schnitter},
    description={Du rufst den Kampfrausch Kors auf dich herab und erleidest keine Wundabzüge. Kannst du Wundabzüge ohnehin ignorieren, sind AT und VT um den halben Wundabzug erleichtert. Für je zwei erlittene Wunden richten deine Angriffe +1 TP an. Erlaubt Aufrechterhalten.\newline Probenschwierigkeit: 12\newline Vorbereitungszeit: 4 Aktionen\newline Ziel: selbst\newline Reichweite: Berührung\newline Wirkungsdauer: 16 Initiativphasen\newline Kosten: 8 KaP\newline Fertigkeiten: Guter Kampf\newline Erlernen: Kor 14; 60 EP}
}


\newglossaryentry{dasschwarzeFelldurchdasroteBlut_Talent}
{
    name={Das schwarze Fell durch das rote Blut},
    description={Dein vergossenes Blut legt sich wie eine schützende Haut um deinen Körper. Dein RS steigt um 1 und für je zwei insgesamt erlittene Wunden steigt der RS um einen weiteren Punkt.\newline Mächtige Liturgie: Je zwei Stufen erhöhen den RS um +1. Dieser Bonus ist unabhängig von den erlittenen Wunden.\newline Probenschwierigkeit: 12\newline Vorbereitungszeit: 4 Aktionen\newline Ziel: selbst\newline Reichweite: Berührung\newline Wirkungsdauer: 1 Stunde\newline Kosten: 8 KaP\newline Fertigkeiten: Guter Kampf\newline Erlernen: Kor 14; 40 EP}
}


\newglossaryentry{immerwährenderKampf_Talent}
{
    name={Immerwährender Kampf},
    description={Die Lebenskraft deiner Feinde stärkt dich. Wenn du seit dem Beginn deiner letzten Initiativephase einen mindestens normal großen Gegner besiegt hast, regenerierst du eine Wunde, bei großen oder sehr großen Gegnern sogar zwei Wunden.\newline Mächtige Liturgie: Du kannst den Gegner seit der vor-/dritt-/viert-/fünftletzten Initiativephase besiegt haben. Allerdings kannst du die Liturgie pro besiegtem Gegner nur einmal nutzen.\newline Probenschwierigkeit: 12\newline Vorbereitungszeit: 0 Aktionen\newline Ziel: selbst\newline Reichweite: Berührung\newline Wirkungsdauer: augenblicklich\newline Kosten: 4 KaP\newline Fertigkeiten: Guter Kampf\newline Erlernen: Kor 16; 80 EP}
}


\newglossaryentry{neunStreicheineinem_Talent}
{
    name={Neun Streiche in einem},
    description={Du wirfst die Kraft des Gnadenlosen in deinen nächsten Angriff. Trifft dieser Angriff, richtest du für jeden bisherigen Treffer gegen das Ziel in diesem Kampf 1W6 TP zusätzlich an. Das Maximum liegt bei 8W6 TP.\newline Mächtige Liturgie: Der Angriff ist um +2 erleichtert.\newline Probenschwierigkeit: 12\newline Vorbereitungszeit: 0 Aktionen\newline Ziel: selbst\newline Reichweite: Berührung\newline Wirkungsdauer: augenblicklich\newline Kosten: 8 KaP\newline Fertigkeiten: Guter Kampf\newline Erlernen: Kor 18; 40 EP}
}


\newglossaryentry{fluchdesVerräters_Talent}
{
    name={Fluch des Verräters},
    description={Du rufst Kors Strafe auf einen reuelosen Sünder oder Deserteur herab. Wenn ihm eine Konterprobe (Willenskraft, 20) misslingt, erleidet er in jedem Kampf einen Furcht-Effekt Stufe 1.\newline Mächtige Liturgie: Je zwei Stufen erhöhen den Furcht-Effekt um eine Stufe.\newline Probenschwierigkeit: 12\newline Modifikationen: Permanenz (–4, Wirkungsdauer bis die Bindung gelöst wird, 8 KaP, davon 2 gKaP)\newline Vorbereitungszeit: 4 Minuten\newline Ziel: Einzelperson\newline Reichweite: 16 Schritt\newline Wirkungsdauer: 1 Woche\newline Kosten: 8 KaP\newline Fertigkeiten: Gutes Gold\newline Erlernen: Kor 14; 20 EP}
}


\newglossaryentry{orakeldesMantikors_Talent}
{
    name={Orakel des Mantikors},
    description={Du erfährst, ob ein Verschollener im Kampf gefallen ist. Sollte der Verschollene auf eine andere Art und Weise ums Leben gekommen sein, schlägt das Orakel nicht an.\newline Probenschwierigkeit: 12\newline Vorbereitungszeit: 4 Minuten\newline Ziel: selbst\newline Reichweite: Berührung\newline Wirkungsdauer: augenblicklich\newline Kosten: 4 KaP\newline Fertigkeiten: Gutes Gold\newline Erlernen: Kor 12; 0 EP}
}


\newglossaryentry{mirakel:Alchemie_Talent}
{
    name={Mirakel: Alchemie},
    description={Dein nächster Wurf auf Alchemie ist um +4 Punkte erleichtert.\newline Mächtige Liturgie: Erhöht die Erleichterung um +2.\newline Probenschwierigkeit: 12\newline Vorbereitungszeit: 0 Aktionen\newline Ziel: selbst\newline Reichweite: Berührung\newline Wirkungsdauer: 4 Minuten\newline Kosten: 4 KaP\newline Erlernen: 20 EP}
}


\newglossaryentry{mirakel:IN_Talent}
{
    name={Mirakel: IN},
    description={Dein nächster Wurf auf IN ist um +4 Punkte erleichtert.\newline Mächtige Liturgie: Erhöht die Erleichterung um +2.\newline Probenschwierigkeit: 12\newline Vorbereitungszeit: 0 Aktionen\newline Ziel: selbst\newline Reichweite: Berührung\newline Wirkungsdauer: 4 Minuten\newline Kosten: 4 KaP\newline Erlernen: 20 EP}
}


\newglossaryentry{mirakel:Mythenkunde_Talent}
{
    name={Mirakel: Mythenkunde},
    description={Dein nächster Wurf auf Mythenkunde ist um +4 Punkte erleichtert.\newline Mächtige Liturgie: Erhöht die Erleichterung um +2.\newline Probenschwierigkeit: 12\newline Vorbereitungszeit: 0 Aktionen\newline Ziel: selbst\newline Reichweite: Berührung\newline Wirkungsdauer: 4 Minuten\newline Kosten: 4 KaP\newline Erlernen: 20 EP}
}


\newglossaryentry{mirakel:MU_Talent}
{
    name={Mirakel: MU},
    description={Dein nächster Wurf auf MU ist um +4 Punkte erleichtert.\newline Mächtige Liturgie: Erhöht die Erleichterung um +2.\newline Probenschwierigkeit: 12\newline Vorbereitungszeit: 0 Aktionen\newline Ziel: selbst\newline Reichweite: Berührung\newline Wirkungsdauer: 4 Minuten\newline Kosten: 4 KaP\newline Erlernen: 20 EP}
}


\newglossaryentry{mirakel:Wahrnehmung_Talent}
{
    name={Mirakel: Wahrnehmung},
    description={Dein nächster Wurf auf Wahrnehmung ist um +4 Punkte erleichtert.\newline Mächtige Liturgie: Erhöht die Erleichterung um +2.\newline Probenschwierigkeit: 12\newline Vorbereitungszeit: 0 Aktionen\newline Ziel: selbst\newline Reichweite: Berührung\newline Wirkungsdauer: 4 Minuten\newline Kosten: 4 KaP\newline Erlernen: 20 EP}
}


\newglossaryentry{mirakel:CH_Talent}
{
    name={Mirakel: CH},
    description={Dein nächster Wurf auf CH ist um +4 Punkte erleichtert.\newline Mächtige Liturgie: Erhöht die Erleichterung um +2.\newline Probenschwierigkeit: 12\newline Vorbereitungszeit: 0 Aktionen\newline Ziel: selbst\newline Reichweite: Berührung\newline Wirkungsdauer: 4 Minuten\newline Kosten: 4 KaP\newline Erlernen: 20 EP}
}


\newglossaryentry{mirakel:Heilkunde_Talent}
{
    name={Mirakel: Heilkunde},
    description={Dein nächster Wurf auf Heilkunde ist um +4 Punkte erleichtert.\newline Mächtige Liturgie: Erhöht die Erleichterung um +2.\newline Probenschwierigkeit: 12\newline Vorbereitungszeit: 0 Aktionen\newline Ziel: selbst\newline Reichweite: Berührung\newline Wirkungsdauer: 4 Minuten\newline Kosten: 4 KaP\newline Erlernen: 20 EP}
}


\newglossaryentry{mirakel:KL_Talent}
{
    name={Mirakel: KL},
    description={Dein nächster Wurf auf KL ist um +4 Punkte erleichtert.\newline Mächtige Liturgie: Erhöht die Erleichterung um +2.\newline Probenschwierigkeit: 12\newline Vorbereitungszeit: 0 Aktionen\newline Ziel: selbst\newline Reichweite: Berührung\newline Wirkungsdauer: 4 Minuten\newline Kosten: 4 KaP\newline Erlernen: 20 EP}
}


\newglossaryentry{mirakel:GE_Talent}
{
    name={Mirakel: GE},
    description={Dein nächster Wurf auf GE ist um +4 Punkte erleichtert.\newline Mächtige Liturgie: Erhöht die Erleichterung um +2.\newline Probenschwierigkeit: 12\newline Vorbereitungszeit: 0 Aktionen\newline Ziel: selbst\newline Reichweite: Berührung\newline Wirkungsdauer: 4 Minuten\newline Kosten: 4 KaP\newline Erlernen: 20 EP}
}


\newglossaryentry{mirakel:KK_Talent}
{
    name={Mirakel: KK},
    description={Dein nächster Wurf auf KK ist um +4 Punkte erleichtert.\newline Mächtige Liturgie: Erhöht die Erleichterung um +2.\newline Probenschwierigkeit: 12\newline Vorbereitungszeit: 0 Aktionen\newline Ziel: selbst\newline Reichweite: Berührung\newline Wirkungsdauer: 4 Minuten\newline Kosten: 4 KaP\newline Erlernen: 20 EP}
}


\newglossaryentry{mirakel:Athletik_Talent}
{
    name={Mirakel: Athletik},
    description={Dein nächster Wurf auf Athletik ist um +4 Punkte erleichtert.\newline Mächtige Liturgie: Erhöht die Erleichterung um +2.\newline Probenschwierigkeit: 12\newline Vorbereitungszeit: 0 Aktionen\newline Ziel: selbst\newline Reichweite: Berührung\newline Wirkungsdauer: 4 Minuten\newline Kosten: 4 KaP\newline Erlernen: 20 EP}
}


\newglossaryentry{mirakel:KO_Talent}
{
    name={Mirakel: KO},
    description={Dein nächster Wurf auf KO ist um +4 Punkte erleichtert.\newline Mächtige Liturgie: Erhöht die Erleichterung um +2.\newline Probenschwierigkeit: 12\newline Vorbereitungszeit: 0 Aktionen\newline Ziel: selbst\newline Reichweite: Berührung\newline Wirkungsdauer: 4 Minuten\newline Kosten: 4 KaP\newline Erlernen: 20 EP}
}


\newglossaryentry{mirakel:Selbstbeherrschung_Talent}
{
    name={Mirakel: Selbstbeherrschung},
    description={Dein nächster Wurf auf Selbstbeherrschung ist um +4 Punkte erleichtert.\newline Mächtige Liturgie: Erhöht die Erleichterung um +2.\newline Probenschwierigkeit: 12\newline Vorbereitungszeit: 0 Aktionen\newline Ziel: selbst\newline Reichweite: Berührung\newline Wirkungsdauer: 4 Minuten\newline Kosten: 4 KaP\newline Erlernen: 20 EP}
}


\newglossaryentry{mirakel:Überleben_Talent}
{
    name={Mirakel: Überleben},
    description={Dein nächster Wurf auf Überleben ist um +4 Punkte erleichtert.\newline Mächtige Liturgie: Erhöht die Erleichterung um +2.\newline Probenschwierigkeit: 12\newline Vorbereitungszeit: 0 Aktionen\newline Ziel: selbst\newline Reichweite: Berührung\newline Wirkungsdauer: 4 Minuten\newline Kosten: 4 KaP\newline Erlernen: 20 EP}
}


\newglossaryentry{mirakel:Heimlichkeit_Talent}
{
    name={Mirakel: Heimlichkeit},
    description={Dein nächster Wurf auf Heimlichkeit ist um +4 Punkte erleichtert.\newline Mächtige Liturgie: Erhöht die Erleichterung um +2.\newline Probenschwierigkeit: 12\newline Vorbereitungszeit: 0 Aktionen\newline Ziel: selbst\newline Reichweite: Berührung\newline Wirkungsdauer: 4 Minuten\newline Kosten: 4 KaP\newline Erlernen: 20 EP}
}


\newglossaryentry{mirakel:Schusswaffen_Talent}
{
    name={Mirakel: Schusswaffen},
    description={Dein nächster Wurf auf Schusswaffen ist um +4 Punkte erleichtert.\newline Mächtige Liturgie: Erhöht die Erleichterung um +2.\newline Probenschwierigkeit: 12\newline Vorbereitungszeit: 0 Aktionen\newline Ziel: selbst\newline Reichweite: Berührung\newline Wirkungsdauer: 4 Minuten\newline Kosten: 4 KaP\newline Erlernen: 20 EP}
}


\newglossaryentry{mirakel:Wurfwaffen_Talent}
{
    name={Mirakel: Wurfwaffen},
    description={Dein nächster Wurf auf Wurfwaffen ist um +4 Punkte erleichtert.\newline Mächtige Liturgie: Erhöht die Erleichterung um +2.\newline Probenschwierigkeit: 12\newline Vorbereitungszeit: 0 Aktionen\newline Ziel: selbst\newline Reichweite: Berührung\newline Wirkungsdauer: 4 Minuten\newline Kosten: 4 KaP\newline Erlernen: 20 EP}
}


\newglossaryentry{mirakel:Magiekunde_Talent}
{
    name={Mirakel: Magiekunde},
    description={Dein nächster Wurf auf Magiekunde ist um +4 Punkte erleichtert.\newline Mächtige Liturgie: Erhöht die Erleichterung um +2.\newline Probenschwierigkeit: 12\newline Vorbereitungszeit: 0 Aktionen\newline Ziel: selbst\newline Reichweite: Berührung\newline Wirkungsdauer: 4 Minuten\newline Kosten: 4 KaP\newline Erlernen: 20 EP}
}


\newglossaryentry{mirakel:MR_Talent}
{
    name={Mirakel: MR},
    description={Dein nächster Wurf auf MR ist um +4 Punkte erleichtert.\newline Mächtige Liturgie: Erhöht die Erleichterung um +2.\newline Probenschwierigkeit: 12\newline Vorbereitungszeit: 0 Aktionen\newline Ziel: selbst\newline Reichweite: Berührung\newline Wirkungsdauer: 4 Minuten\newline Kosten: 4 KaP\newline Erlernen: 20 EP}
}


\newglossaryentry{mirakel:Autorität_Talent}
{
    name={Mirakel: Autorität},
    description={Dein nächster Wurf auf Autorität ist um +4 Punkte erleichtert.\newline Mächtige Liturgie: Erhöht die Erleichterung um +2.\newline Probenschwierigkeit: 12\newline Vorbereitungszeit: 0 Aktionen\newline Ziel: selbst\newline Reichweite: Berührung\newline Wirkungsdauer: 4 Minuten\newline Kosten: 4 KaP\newline Erlernen: 20 EP}
}


\newglossaryentry{mirakel:Beeinflussung_Talent}
{
    name={Mirakel: Beeinflussung},
    description={Dein nächster Wurf auf Beeinflussung ist um +4 Punkte erleichtert.\newline Mächtige Liturgie: Erhöht die Erleichterung um +2.\newline Probenschwierigkeit: 12\newline Vorbereitungszeit: 0 Aktionen\newline Ziel: selbst\newline Reichweite: Berührung\newline Wirkungsdauer: 4 Minuten\newline Kosten: 4 KaP\newline Erlernen: 20 EP}
}


\newglossaryentry{mirakel:Derekunde_Talent}
{
    name={Mirakel: Derekunde},
    description={Dein nächster Wurf auf Derekunde ist um +4 Punkte erleichtert.\newline Mächtige Liturgie: Erhöht die Erleichterung um +2.\newline Probenschwierigkeit: 12\newline Vorbereitungszeit: 0 Aktionen\newline Ziel: selbst\newline Reichweite: Berührung\newline Wirkungsdauer: 4 Minuten\newline Kosten: 4 KaP\newline Erlernen: 20 EP}
}


\newglossaryentry{mirakel:Hiebwaffen_Talent}
{
    name={Mirakel: Hiebwaffen},
    description={Dein nächster Wurf auf Hiebwaffen ist um +4 Punkte erleichtert.\newline Mächtige Liturgie: Erhöht die Erleichterung um +2.\newline Probenschwierigkeit: 12\newline Vorbereitungszeit: 0 Aktionen\newline Ziel: selbst\newline Reichweite: Berührung\newline Wirkungsdauer: 4 Minuten\newline Kosten: 4 KaP\newline Erlernen: 20 EP}
}


\newglossaryentry{mirakel:FF_Talent}
{
    name={Mirakel: FF},
    description={Dein nächster Wurf auf FF ist um +4 Punkte erleichtert.\newline Mächtige Liturgie: Erhöht die Erleichterung um +2.\newline Probenschwierigkeit: 12\newline Vorbereitungszeit: 0 Aktionen\newline Ziel: selbst\newline Reichweite: Berührung\newline Wirkungsdauer: 4 Minuten\newline Kosten: 4 KaP\newline Erlernen: 20 EP}
}


\newglossaryentry{mirakel:Handwerk_Talent}
{
    name={Mirakel: Handwerk},
    description={Dein nächster Wurf auf Handwerk ist um +4 Punkte erleichtert.\newline Mächtige Liturgie: Erhöht die Erleichterung um +2.\newline Probenschwierigkeit: 12\newline Vorbereitungszeit: 0 Aktionen\newline Ziel: selbst\newline Reichweite: Berührung\newline Wirkungsdauer: 4 Minuten\newline Kosten: 4 KaP\newline Erlernen: 20 EP}
}


\newglossaryentry{mirakel:Verschlagenheit_Talent}
{
    name={Mirakel: Verschlagenheit},
    description={Dein nächster Wurf auf Verschlagenheit ist um +4 Punkte erleichtert.\newline Mächtige Liturgie: Erhöht die Erleichterung um +2.\newline Probenschwierigkeit: 12\newline Vorbereitungszeit: 0 Aktionen\newline Ziel: selbst\newline Reichweite: Berührung\newline Wirkungsdauer: 4 Minuten\newline Kosten: 4 KaP\newline Erlernen: 20 EP}
}


\newglossaryentry{mirakel:Handgemenge_Talent}
{
    name={Mirakel: Handgemenge},
    description={Dein nächster Wurf auf Handgemenge ist um +4 Punkte erleichtert.\newline Mächtige Liturgie: Erhöht die Erleichterung um +2.\newline Probenschwierigkeit: 12\newline Vorbereitungszeit: 0 Aktionen\newline Ziel: selbst\newline Reichweite: Berührung\newline Wirkungsdauer: 4 Minuten\newline Kosten: 4 KaP\newline Erlernen: 20 EP}
}


\newglossaryentry{mirakel:Klingenwaffen_Talent}
{
    name={Mirakel: Klingenwaffen},
    description={Dein nächster Wurf auf Klingenwaffen ist um +4 Punkte erleichtert.\newline Mächtige Liturgie: Erhöht die Erleichterung um +2.\newline Probenschwierigkeit: 12\newline Vorbereitungszeit: 0 Aktionen\newline Ziel: selbst\newline Reichweite: Berührung\newline Wirkungsdauer: 4 Minuten\newline Kosten: 4 KaP\newline Erlernen: 20 EP}
}


\newglossaryentry{mirakel:Stangenwaffen_Talent}
{
    name={Mirakel: Stangenwaffen},
    description={Dein nächster Wurf auf Stangenwaffen ist um +4 Punkte erleichtert.\newline Mächtige Liturgie: Erhöht die Erleichterung um +2.\newline Probenschwierigkeit: 12\newline Vorbereitungszeit: 0 Aktionen\newline Ziel: selbst\newline Reichweite: Berührung\newline Wirkungsdauer: 4 Minuten\newline Kosten: 4 KaP\newline Erlernen: 20 EP}
}


\newglossaryentry{mirakel:Gebräuche_Talent}
{
    name={Mirakel: Gebräuche},
    description={Dein nächster Wurf auf Gebräuche ist um +4 Punkte erleichtert.\newline Mächtige Liturgie: Erhöht die Erleichterung um +2.\newline Probenschwierigkeit: 12\newline Vorbereitungszeit: 0 Aktionen\newline Ziel: selbst\newline Reichweite: Berührung\newline Wirkungsdauer: 4 Minuten\newline Kosten: 4 KaP\newline Erlernen: 20 EP}
}


\newglossaryentry{mirakel:ErsteFertigkeitnachWahl_Talent}
{
    name={Mirakel: Erste Fertigkeit nach Wahl},
    description={Dein nächster Wurf auf die erste Fertigkeit nach Wahl ist um +4 Punkte erleichtert.\newline Mächtige Liturgie: Erhöht die Erleichterung um +2.\newline Probenschwierigkeit: 12\newline Vorbereitungszeit: 0 Aktionen\newline Ziel: selbst\newline Reichweite: Berührung\newline Wirkungsdauer: 4 Minuten\newline Kosten: 4 KaP\newline Erlernen: 20 EP}
}


\newglossaryentry{mirakel:ZweiteFertigkeitnachWahl_Talent}
{
    name={Mirakel: Zweite Fertigkeit nach Wahl},
    description={Dein nächster Wurf auf die zweite Fertigkeit nach Wahl ist um +4 Punkte erleichtert.\newline Mächtige Liturgie: Erhöht die Erleichterung um +2.\newline Probenschwierigkeit: 12\newline Vorbereitungszeit: 0 Aktionen\newline Ziel: selbst\newline Reichweite: Berührung\newline Wirkungsdauer: 4 Minuten\newline Kosten: 4 KaP\newline Erlernen: 20 EP}
}


\newglossaryentry{dämonischeStärkung:Alchemie_Talent}
{
    name={Dämonische Stärkung: Alchemie},
    description={Dein nächster Wurf auf Alchemie ist um +4 Punkte erleichtert.\newline Mächtige Anrufung: Erhöht die Erleichterung um +2.\newline Probenschwierigkeit: 12\newline Vorbereitungszeit: 0 Aktionen\newline Ziel: selbst\newline Reichweite: Berührung\newline Wirkungsdauer: 4 Minuten\newline Kosten: 4 GuP\newline Erlernen: 20 EP}
}


\newglossaryentry{dämonischeStärkung:IN_Talent}
{
    name={Dämonische Stärkung: IN},
    description={Dein nächster Wurf auf IN ist um +4 Punkte erleichtert.\newline Mächtige Anrufung: Erhöht die Erleichterung um +2.\newline Probenschwierigkeit: 12\newline Vorbereitungszeit: 0 Aktionen\newline Ziel: selbst\newline Reichweite: Berührung\newline Wirkungsdauer: 4 Minuten\newline Kosten: 4 GuP\newline Erlernen: 20 EP}
}


\newglossaryentry{dämonischeStärkung:Mythenkunde_Talent}
{
    name={Dämonische Stärkung: Mythenkunde},
    description={Dein nächster Wurf auf Mythenkunde ist um +4 Punkte erleichtert.\newline Mächtige Anrufung: Erhöht die Erleichterung um +2.\newline Probenschwierigkeit: 12\newline Vorbereitungszeit: 0 Aktionen\newline Ziel: selbst\newline Reichweite: Berührung\newline Wirkungsdauer: 4 Minuten\newline Kosten: 4 GuP\newline Erlernen: 20 EP}
}


\newglossaryentry{dämonischeStärkung:MU_Talent}
{
    name={Dämonische Stärkung: MU},
    description={Dein nächster Wurf auf MU ist um +4 Punkte erleichtert.\newline Mächtige Anrufung: Erhöht die Erleichterung um +2.\newline Probenschwierigkeit: 12\newline Vorbereitungszeit: 0 Aktionen\newline Ziel: selbst\newline Reichweite: Berührung\newline Wirkungsdauer: 4 Minuten\newline Kosten: 4 GuP\newline Erlernen: 20 EP}
}


\newglossaryentry{dämonischeStärkung:Wahrnehmung_Talent}
{
    name={Dämonische Stärkung: Wahrnehmung},
    description={Dein nächster Wurf auf Wahrnehmung ist um +4 Punkte erleichtert.\newline Mächtige Anrufung: Erhöht die Erleichterung um +2.\newline Probenschwierigkeit: 12\newline Vorbereitungszeit: 0 Aktionen\newline Ziel: selbst\newline Reichweite: Berührung\newline Wirkungsdauer: 4 Minuten\newline Kosten: 4 GuP\newline Erlernen: 20 EP}
}


\newglossaryentry{dämonischeStärkung:CH_Talent}
{
    name={Dämonische Stärkung: CH},
    description={Dein nächster Wurf auf CH ist um +4 Punkte erleichtert.\newline Mächtige Anrufung: Erhöht die Erleichterung um +2.\newline Probenschwierigkeit: 12\newline Vorbereitungszeit: 0 Aktionen\newline Ziel: selbst\newline Reichweite: Berührung\newline Wirkungsdauer: 4 Minuten\newline Kosten: 4 GuP\newline Erlernen: 20 EP}
}


\newglossaryentry{dämonischeStärkung:Heilkunde_Talent}
{
    name={Dämonische Stärkung: Heilkunde},
    description={Dein nächster Wurf auf Heilkunde ist um +4 Punkte erleichtert.\newline Mächtige Anrufung: Erhöht die Erleichterung um +2.\newline Probenschwierigkeit: 12\newline Vorbereitungszeit: 0 Aktionen\newline Ziel: selbst\newline Reichweite: Berührung\newline Wirkungsdauer: 4 Minuten\newline Kosten: 4 GuP\newline Erlernen: 20 EP}
}


\newglossaryentry{dämonischeStärkung:KL_Talent}
{
    name={Dämonische Stärkung: KL},
    description={Dein nächster Wurf auf KL ist um +4 Punkte erleichtert.\newline Mächtige Anrufung: Erhöht die Erleichterung um +2.\newline Probenschwierigkeit: 12\newline Vorbereitungszeit: 0 Aktionen\newline Ziel: selbst\newline Reichweite: Berührung\newline Wirkungsdauer: 4 Minuten\newline Kosten: 4 GuP\newline Erlernen: 20 EP}
}


\newglossaryentry{dämonischeStärkung:GE_Talent}
{
    name={Dämonische Stärkung: GE},
    description={Dein nächster Wurf auf GE ist um +4 Punkte erleichtert.\newline Mächtige Anrufung: Erhöht die Erleichterung um +2.\newline Probenschwierigkeit: 12\newline Vorbereitungszeit: 0 Aktionen\newline Ziel: selbst\newline Reichweite: Berührung\newline Wirkungsdauer: 4 Minuten\newline Kosten: 4 GuP\newline Erlernen: 20 EP}
}


\newglossaryentry{dämonischeStärkung:KK_Talent}
{
    name={Dämonische Stärkung: KK},
    description={Dein nächster Wurf auf KK ist um +4 Punkte erleichtert.\newline Mächtige Anrufung: Erhöht die Erleichterung um +2.\newline Probenschwierigkeit: 12\newline Vorbereitungszeit: 0 Aktionen\newline Ziel: selbst\newline Reichweite: Berührung\newline Wirkungsdauer: 4 Minuten\newline Kosten: 4 GuP\newline Erlernen: 20 EP}
}


\newglossaryentry{dämonischeStärkung:Athletik_Talent}
{
    name={Dämonische Stärkung: Athletik},
    description={Dein nächster Wurf auf Athletik ist um +4 Punkte erleichtert.\newline Mächtige Anrufung: Erhöht die Erleichterung um +2.\newline Probenschwierigkeit: 12\newline Vorbereitungszeit: 0 Aktionen\newline Ziel: selbst\newline Reichweite: Berührung\newline Wirkungsdauer: 4 Minuten\newline Kosten: 4 GuP\newline Erlernen: 20 EP}
}


\newglossaryentry{dämonischeStärkung:KO_Talent}
{
    name={Dämonische Stärkung: KO},
    description={Dein nächster Wurf auf KO ist um +4 Punkte erleichtert.\newline Mächtige Anrufung: Erhöht die Erleichterung um +2.\newline Probenschwierigkeit: 12\newline Vorbereitungszeit: 0 Aktionen\newline Ziel: selbst\newline Reichweite: Berührung\newline Wirkungsdauer: 4 Minuten\newline Kosten: 4 GuP\newline Erlernen: 20 EP}
}


\newglossaryentry{dämonischeStärkung:Selbstbeherrschung_Talent}
{
    name={Dämonische Stärkung: Selbstbeherrschung},
    description={Dein nächster Wurf auf Selbstbeherrschung ist um +4 Punkte erleichtert.\newline Mächtige Anrufung: Erhöht die Erleichterung um +2.\newline Probenschwierigkeit: 12\newline Vorbereitungszeit: 0 Aktionen\newline Ziel: selbst\newline Reichweite: Berührung\newline Wirkungsdauer: 4 Minuten\newline Kosten: 4 GuP\newline Erlernen: 20 EP}
}


\newglossaryentry{dämonischeStärkung:Überleben_Talent}
{
    name={Dämonische Stärkung: Überleben},
    description={Dein nächster Wurf auf Überleben ist um +4 Punkte erleichtert.\newline Mächtige Anrufung: Erhöht die Erleichterung um +2.\newline Probenschwierigkeit: 12\newline Vorbereitungszeit: 0 Aktionen\newline Ziel: selbst\newline Reichweite: Berührung\newline Wirkungsdauer: 4 Minuten\newline Kosten: 4 GuP\newline Erlernen: 20 EP}
}


\newglossaryentry{dämonischeStärkung:Heimlichkeit_Talent}
{
    name={Dämonische Stärkung: Heimlichkeit},
    description={Dein nächster Wurf auf Heimlichkeit ist um +4 Punkte erleichtert.\newline Mächtige Anrufung: Erhöht die Erleichterung um +2.\newline Probenschwierigkeit: 12\newline Vorbereitungszeit: 0 Aktionen\newline Ziel: selbst\newline Reichweite: Berührung\newline Wirkungsdauer: 4 Minuten\newline Kosten: 4 GuP\newline Erlernen: 20 EP}
}


\newglossaryentry{dämonischeStärkung:Schusswaffen_Talent}
{
    name={Dämonische Stärkung: Schusswaffen},
    description={Dein nächster Wurf auf Schusswaffen ist um +4 Punkte erleichtert.\newline Mächtige Anrufung: Erhöht die Erleichterung um +2.\newline Probenschwierigkeit: 12\newline Vorbereitungszeit: 0 Aktionen\newline Ziel: selbst\newline Reichweite: Berührung\newline Wirkungsdauer: 4 Minuten\newline Kosten: 4 GuP\newline Erlernen: 20 EP}
}


\newglossaryentry{dämonischeStärkung:Wurfwaffen_Talent}
{
    name={Dämonische Stärkung: Wurfwaffen},
    description={Dein nächster Wurf auf Wurfwaffen ist um +4 Punkte erleichtert.\newline Mächtige Anrufung: Erhöht die Erleichterung um +2.\newline Probenschwierigkeit: 12\newline Vorbereitungszeit: 0 Aktionen\newline Ziel: selbst\newline Reichweite: Berührung\newline Wirkungsdauer: 4 Minuten\newline Kosten: 4 GuP\newline Erlernen: 20 EP}
}


\newglossaryentry{dämonischeStärkung:Magiekunde_Talent}
{
    name={Dämonische Stärkung: Magiekunde},
    description={Dein nächster Wurf auf Magiekunde ist um +4 Punkte erleichtert.\newline Mächtige Anrufung: Erhöht die Erleichterung um +2.\newline Probenschwierigkeit: 12\newline Vorbereitungszeit: 0 Aktionen\newline Ziel: selbst\newline Reichweite: Berührung\newline Wirkungsdauer: 4 Minuten\newline Kosten: 4 GuP\newline Erlernen: 20 EP}
}


\newglossaryentry{dämonischeStärkung:MR_Talent}
{
    name={Dämonische Stärkung: MR},
    description={Dein nächster Wurf auf MR ist um +4 Punkte erleichtert.\newline Mächtige Anrufung: Erhöht die Erleichterung um +2.\newline Probenschwierigkeit: 12\newline Vorbereitungszeit: 0 Aktionen\newline Ziel: selbst\newline Reichweite: Berührung\newline Wirkungsdauer: 4 Minuten\newline Kosten: 4 GuP\newline Erlernen: 20 EP}
}


\newglossaryentry{dämonischeStärkung:Autorität_Talent}
{
    name={Dämonische Stärkung: Autorität},
    description={Dein nächster Wurf auf Autorität ist um +4 Punkte erleichtert.\newline Mächtige Anrufung: Erhöht die Erleichterung um +2.\newline Probenschwierigkeit: 12\newline Vorbereitungszeit: 0 Aktionen\newline Ziel: selbst\newline Reichweite: Berührung\newline Wirkungsdauer: 4 Minuten\newline Kosten: 4 GuP\newline Erlernen: 20 EP}
}


\newglossaryentry{dämonischeStärkung:Beeinflussung_Talent}
{
    name={Dämonische Stärkung: Beeinflussung},
    description={Dein nächster Wurf auf Beeinflussung ist um +4 Punkte erleichtert.\newline Mächtige Anrufung: Erhöht die Erleichterung um +2.\newline Probenschwierigkeit: 12\newline Vorbereitungszeit: 0 Aktionen\newline Ziel: selbst\newline Reichweite: Berührung\newline Wirkungsdauer: 4 Minuten\newline Kosten: 4 GuP\newline Erlernen: 20 EP}
}


\newglossaryentry{dämonischeStärkung:Derekunde_Talent}
{
    name={Dämonische Stärkung: Derekunde},
    description={Dein nächster Wurf auf Derekunde ist um +4 Punkte erleichtert.\newline Mächtige Anrufung: Erhöht die Erleichterung um +2.\newline Probenschwierigkeit: 12\newline Vorbereitungszeit: 0 Aktionen\newline Ziel: selbst\newline Reichweite: Berührung\newline Wirkungsdauer: 4 Minuten\newline Kosten: 4 GuP\newline Erlernen: 20 EP}
}


\newglossaryentry{dämonischeStärkung:Hiebwaffen_Talent}
{
    name={Dämonische Stärkung: Hiebwaffen},
    description={Dein nächster Wurf auf Hiebwaffen ist um +4 Punkte erleichtert.\newline Mächtige Anrufung: Erhöht die Erleichterung um +2.\newline Probenschwierigkeit: 12\newline Vorbereitungszeit: 0 Aktionen\newline Ziel: selbst\newline Reichweite: Berührung\newline Wirkungsdauer: 4 Minuten\newline Kosten: 4 GuP\newline Erlernen: 20 EP}
}


\newglossaryentry{dämonischeStärkung:FF_Talent}
{
    name={Dämonische Stärkung: FF},
    description={Dein nächster Wurf auf FF ist um +4 Punkte erleichtert.\newline Mächtige Anrufung: Erhöht die Erleichterung um +2.\newline Probenschwierigkeit: 12\newline Vorbereitungszeit: 0 Aktionen\newline Ziel: selbst\newline Reichweite: Berührung\newline Wirkungsdauer: 4 Minuten\newline Kosten: 4 GuP\newline Erlernen: 20 EP}
}


\newglossaryentry{dämonischeStärkung:Handwerk_Talent}
{
    name={Dämonische Stärkung: Handwerk},
    description={Dein nächster Wurf auf Handwerk ist um +4 Punkte erleichtert.\newline Mächtige Anrufung: Erhöht die Erleichterung um +2.\newline Probenschwierigkeit: 12\newline Vorbereitungszeit: 0 Aktionen\newline Ziel: selbst\newline Reichweite: Berührung\newline Wirkungsdauer: 4 Minuten\newline Kosten: 4 GuP\newline Erlernen: 20 EP}
}


\newglossaryentry{dämonischeStärkung:Verschlagenheit_Talent}
{
    name={Dämonische Stärkung: Verschlagenheit},
    description={Dein nächster Wurf auf Verschlagenheit ist um +4 Punkte erleichtert.\newline Mächtige Anrufung: Erhöht die Erleichterung um +2.\newline Probenschwierigkeit: 12\newline Vorbereitungszeit: 0 Aktionen\newline Ziel: selbst\newline Reichweite: Berührung\newline Wirkungsdauer: 4 Minuten\newline Kosten: 4 GuP\newline Erlernen: 20 EP}
}


\newglossaryentry{dämonischeStärkung:Handgemenge_Talent}
{
    name={Dämonische Stärkung: Handgemenge},
    description={Dein nächster Wurf auf Handgemenge ist um +4 Punkte erleichtert.\newline Mächtige Anrufung: Erhöht die Erleichterung um +2.\newline Probenschwierigkeit: 12\newline Vorbereitungszeit: 0 Aktionen\newline Ziel: selbst\newline Reichweite: Berührung\newline Wirkungsdauer: 4 Minuten\newline Kosten: 4 GuP\newline Erlernen: 20 EP}
}


\newglossaryentry{dämonischeStärkung:Klingenwaffen_Talent}
{
    name={Dämonische Stärkung: Klingenwaffen},
    description={Dein nächster Wurf auf Klingenwaffen ist um +4 Punkte erleichtert.\newline Mächtige Anrufung: Erhöht die Erleichterung um +2.\newline Probenschwierigkeit: 12\newline Vorbereitungszeit: 0 Aktionen\newline Ziel: selbst\newline Reichweite: Berührung\newline Wirkungsdauer: 4 Minuten\newline Kosten: 4 GuP\newline Erlernen: 20 EP}
}


\newglossaryentry{dämonischeStärkung:Stangenwaffen_Talent}
{
    name={Dämonische Stärkung: Stangenwaffen},
    description={Dein nächster Wurf auf Stangenwaffen ist um +4 Punkte erleichtert.\newline Mächtige Anrufung: Erhöht die Erleichterung um +2.\newline Probenschwierigkeit: 12\newline Vorbereitungszeit: 0 Aktionen\newline Ziel: selbst\newline Reichweite: Berührung\newline Wirkungsdauer: 4 Minuten\newline Kosten: 4 GuP\newline Erlernen: 20 EP}
}


\newglossaryentry{dämonischeStärkung:Gebräuche_Talent}
{
    name={Dämonische Stärkung: Gebräuche},
    description={Dein nächster Wurf auf Gebräuche ist um +4 Punkte erleichtert.\newline Mächtige Anrufung: Erhöht die Erleichterung um +2.\newline Probenschwierigkeit: 12\newline Vorbereitungszeit: 0 Aktionen\newline Ziel: selbst\newline Reichweite: Berührung\newline Wirkungsdauer: 4 Minuten\newline Kosten: 4 GuP\newline Erlernen: 20 EP}
}


\newglossaryentry{dämonischeStärkung:ErsteFertigkeitnachWahl_Talent}
{
    name={Dämonische Stärkung: Erste Fertigkeit nach Wahl},
    description={Dein nächster Wurf auf die erste Fertigkeit nach Wahl ist um +4 Punkte erleichtert.\newline Mächtige Anrufung: Erhöht die Erleichterung um +2.\newline Probenschwierigkeit: 12\newline Vorbereitungszeit: 0 Aktionen\newline Ziel: selbst\newline Reichweite: Berührung\newline Wirkungsdauer: 4 Minuten\newline Kosten: 4 GuP\newline Erlernen: 20 EP}
}


\newglossaryentry{dämonischeStärkung:ZweiteFertigkeitnachWahl_Talent}
{
    name={Dämonische Stärkung: Zweite Fertigkeit nach Wahl},
    description={Dein nächster Wurf auf die zweite Fertigkeit nach Wahl ist um +4 Punkte erleichtert.\newline Mächtige Anrufung: Erhöht die Erleichterung um +2.\newline Probenschwierigkeit: 12\newline Vorbereitungszeit: 0 Aktionen\newline Ziel: selbst\newline Reichweite: Berührung\newline Wirkungsdauer: 4 Minuten\newline Kosten: 4 GuP\newline Erlernen: 20 EP}
}


\newglossaryentry{austrocknen_Talent}
{
    name={Austrocknen},
    description={Das Opfer trocknet langsam aus. Jede Stunde erleidet es 1 Wunde. Stirbt es während der Wirkungsdauer, erhebt es sich als Wasserleiche.\newline Probenschwierigkeit: Magieresistenz\newline Vorbereitungszeit: 4 Aktionen\newline Ziel: Einzelperson\newline Reichweite: 8 Schritt\newline Wirkungsdauer: 8 Stunden\newline Kosten: 16 GuP\newline Fertigkeiten: Dämonische Hilfe (Cpt)\newline Erlernen: Cpt 16; 60 EP}
}


\newglossaryentry{ertränken_Talent}
{
    name={Ertränken},
    description={Brackwasser füllt die Lunge des Opfers und verursacht 2W6 SP und Ertränken (S. 98), aber keine Wundschmerzeffekte.\newline Mächtige Anrufung: Erhöht die SP um +4.\newline Probenschwierigkeit: 12\newline Vorbereitungszeit: 0 Aktionen\newline Ziel: Einzelperson\newline Reichweite: 4 Schritt\newline Wirkungsdauer: augenblicklich\newline Kosten: 4 GuP\newline Fertigkeiten: Dämonische Hilfe (Cpt)\newline Erlernen: Cpt 12; 40 EP}
}


\newglossaryentry{fischgift_Talent}
{
    name={Fischgift},
    description={Vergiftet das Wasser in einem Radius von 16 Schritt, sodass kleinere Fische und Meeresfrüchte sterben und sogar Wale eine Vergiftung erleiden.\newline Mächtige Anrufung: Verdoppelt den Radius.\newline Probenschwierigkeit: 12\newline Vorbereitungszeit: 16 Aktionen\newline Ziel: Zone\newline Reichweite: Berührung\newline Wirkungsdauer: augenblicklich\newline Kosten: 2 GuP\newline Fertigkeiten: Dämonische Hilfe (Cpt)\newline Erlernen: Cpt 8; 20 EP}
}


\newglossaryentry{gebieterderGezeiten_Talent}
{
    name={Gebieter der Gezeiten},
    description={Ebbe oder Flut können in einem Radius von 2 Meilen zurückgehalten, beschleunigt oder verstärkt werden.\newline Mächtige Anrufung: Verdoppelt den Radius\newline Probenschwierigkeit: 12\newline Vorbereitungszeit: 4 Minuten\newline Ziel: Zone\newline Reichweite: 4 Meilen\newline Wirkungsdauer: 4 Stunden\newline Kosten: 8 GuP\newline Fertigkeiten: Dämonische Hilfe (Cpt)\newline Erlernen: Cpt 16; 20 EP}
}


\newglossaryentry{herrschaftüberDämonen(passiv)_Talent}
{
    name={Herrschaft über Dämonen (passiv)},
    description={Erleichtert die Beschwörungsproben für Dämonen aus der Domäne Charyptoroths um +2, Kontrollproben um +4.\newline Fertigkeiten: Dämonische Hilfe (Cpt)\newline Erlernen: Cpt 18; 60 EP}
}


\newglossaryentry{herrschaftüberWasserelementare_Talent}
{
    name={Herrschaft über Wasserelementare},
    description={Während der Wirkungsdauer kannst du einmal einen Dienst von einem Wasserelementar fordern. Wenn die Beherrschungsprobe (S. 81), die auf MU gewürfelt wird, gelingt, erfüllt das Elementar den Dienst des Paktierers anstatt des ursprünglichen Dienstes.\newline Probenschwierigkeit: 12\newline Vorbereitungszeit: 2 Aktionen\newline Ziel: selbst\newline Reichweite: Berührung\newline Wirkungsdauer: 16 Initiativephasen\newline Kosten: halbe Basiskosten der Beschwörung in GuP\newline Fertigkeiten: Dämonische Hilfe (Cpt)\newline Erlernen: Cpt 16; 40 EP}
}


\newglossaryentry{jägerausderTiefe_Talent}
{
    name={Jäger aus der Tiefe},
    description={Verleiht unter Wasser den Vorteil Tarnung (S. 98) und Pirschen-Proben sind um +4 erleichtert.\newline Mächtige Anrufung: Erleichtert Pirschen-Proben unter Wasser um +2.\newline Probenschwierigkeit: 12\newline Vorbereitungszeit: 1 Aktion\newline Ziel: selbst\newline Reichweite: Berührung\newline Wirkungsdauer: 1 Stunde\newline Kosten: 4 GuP\newline Fertigkeiten: Dämonische Hilfe (Cpt)\newline Erlernen: Cpt 12; 20 EP}
}


\newglossaryentry{schiffssymbiose_Talent}
{
    name={Schiffssymbiose},
    description={Der Paktierer übt besondere Macht über ein Schiff aus und kann Gegenstände wie mit einem Motoricus (S. 151) bewegen, das Deck mit rutschigem Brackwasser überziehen und ähnliche Effekte hervorrufen.\newline Probenschwierigkeit: 12\newline Vorbereitungszeit: 1 Stunde\newline Ziel: Zone\newline Reichweite: Berührung\newline Wirkungsdauer: bis die Bindung gelöst wird\newline Kosten: 16 GuP, davon 4 gGuP\newline Fertigkeiten: Dämonische Hilfe (Cpt)\newline Erlernen: Cpt 18; 40 EP}
}


\newglossaryentry{sichererTritt(passiv)_Talent}
{
    name={Sicherer Tritt (passiv)},
    description={An Bord eines Schiffes sinken die Abzüge durch schwierigen Untergrund um eine Stufe.\newline Fertigkeiten: Dämonische Hilfe (Cpt)\newline Erlernen: Cpt 14; 40 EP}
}


\newglossaryentry{wasseratmung(passiv)_Talent}
{
    name={Wasseratmung (passiv)},
    description={Der Paktierer kann unter Wasser atmen, sieht unter Wasser wie durch Luft und wird vom Druck der Tiefe nicht beeinflusst.\newline Fertigkeiten: Dämonische Hilfe (Cpt)\newline Erlernen: Cpt 14; 60 EP}
}


\newglossaryentry{dämonischeWaffe_Talent}
{
    name={Dämonische Waffe},
    description={Du segnest eine Waffe. Diese gilt als dämonisch.\newline Probenschwierigkeit: 12\newline Vorbereitungszeit: 1 Stunde\newline Ziel: Einzelobjekt\newline Reichweite: Berührung\newline Wirkungsdauer: bis die Bindung gelöst wird\newline Kosten: 8 GuP, davon 2 gGuP\newline Fertigkeiten: Dämonische Hilfe (Cpt)\newline Erlernen: Cpt 18; 60 EP}
}


\newglossaryentry{irrlichtertanz_Talent}
{
    name={Irrlichtertanz},
    description={Bei jeder deiner Bewegungen stieben Funken und Rauch steigt auf. Für feindlich gesinnte Humanoide, Tiere, Feenwesen oder Mythenwesen giltst du als Schreckgestalt Stufe 2 (S. 98).\newline Probenschwierigkeit: 12\newline Vorbereitungszeit: 4 Aktionen\newline Ziel: selbst\newline Reichweite: Berührung\newline Wirkungsdauer: 16 Initiativphasen\newline Kosten: 8 GuP\newline Fertigkeiten: Dämonische Hilfe (Cpt)\newline Erlernen: Cpt 14; 40 EP}
}


\newglossaryentry{meisterderMaritimen_Talent}
{
    name={Meister der Maritimen},
    description={Das  Ziel befolgt einen Charyptorothgefälligen Befehl, wenn ihm keine Konterprobe (Willenskraft, 20) gelingt.\newline Probenschwierigkeit: 12\newline Vorbereitungszeit: 4 Aktionen\newline Ziel: Einzelperson (nur maritime Humanoide)\newline Reichweite: 32 Schritt\newline Wirkungsdauer: 1 Tag\newline Kosten: 8 GuP\newline Fertigkeiten: Dämonische Hilfe (Cpt)\newline Erlernen: Cpt 16; 40 EP}
}


\newglossaryentry{wasserbrücke_Talent}
{
    name={Wasserbrücke},
    description={Du kannst auf der Wasseroberfläche gehen, als würdest du von einer Welle getragen. Du erleidest keine Abzüge durch ungünstige Position und wirst nicht durch Wellen und Strömungen beeinflusst. Erlaubt Aufrechterhalten.\newline Probenschwierigkeit: 12\newline Vorbereitungszeit: 8 Aktionen\newline Ziel: selbst\newline Reichweite: Berührung\newline Wirkungsdauer: 1 Stunde\newline Kosten: 4 GuP\newline Fertigkeiten: Dämonische Hilfe (Cpt)\newline Erlernen: Cpt 14; 20 EP}
}


\newglossaryentry{wasserleicheerheben_Talent}
{
    name={Wasserleiche erheben},
    description={Du erhebst eine Leiche als Wasserleiche (mehr zu Beschwörungen siehe S. 81), die in zwei INI-Phasen einsatzfähig ist.\newline Probenschwierigkeit: 12\newline Modifikationen: Schnelle Erhebung (–4; die Wasserleiche ist sofort bereit.)\newline Vorbereitungszeit: frei wählbar\newline Ziel: Material für eine einzelne Wasserleiche\newline Wirkungsdauer: augenblicklich\newline Kosten: 4 GuP\newline Fertigkeiten: Dämonische Hilfe (Cpt)\newline Erlernen: Cpt 14; 40 EP}
}


\newglossaryentry{amrychothsTanz_Talent}
{
    name={Amrychoths Tanz},
    description={Im Radius von einer Meile kannst du den Wind auf einer Skala von windstill/leichte Brise/steife Brise/Sturm/Orkan um zwei Stufen verändern und seine Richtung lenken.\newline Mächtige Anrufung: Verändert den Wind um eine weitere Stufe und verdoppelt den Radius.\newline Probenschwierigkeit: 12\newline Modifikationen: Windhauch (0 Aktionen, Wirkungsdauer augenblicklich, 1 GuP; ein kurzer Windstoß weht giftige Dämpfe fort oder löscht ein kleines Feuer.)\newline Vorbereitungszeit: 4 Minuten\newline Ziel: Zone\newline Reichweite: Berührung\newline Wirkungsdauer: 8 Stunden\newline Kosten: 16 GuP\newline Fertigkeiten: Dämonische Hilfe (Cpt)\newline Erlernen: Cpt 12; 40 EP}
}


\newglossaryentry{krakenhaut_Talent}
{
    name={Krakenhaut},
    description={Du verwandelst dich in einen Krakenmolch, Hai oder Rochen (beim Erlernen wählen), wobei du deine geistigen Fähigkeiten behältst. Deine körperlichen Fähigkeiten entsprechen denen des Tiers und du kannst in Tiergestalt keine Anrufungen wirken. Erlaubt Aufrechterhalten.\newline Mächtige Anrufung: Das Tier ist ein überdurchschnittlicher/außergewöhnlicher/herausragender/einzigartiger Vertreter seiner Art. Die Werte des Tieres sind nach Spielleiterentscheid erhöht.\newline Probenschwierigkeit: 12\newline Vorbereitungszeit: 16 Aktionen\newline Ziel: selbst\newline Reichweite: Berührung\newline Wirkungsdauer: 4 Stunden\newline Kosten: 8 GuP\newline Fertigkeiten: Dämonische Hilfe (Cpt)\newline Erlernen: Cpt 16; 40 EP\newline Anmerkungen: Nur herausragende Paktierere können sich in Tiere verwandeln, die nicht zu den Größenklassen klein oder mittel gehören. Solche Verwandlungen sind prinzipiell um –8 erschwert. Die Lernkosten des Zaubers orientieren sich an der Tierart, wobei fliegende, giftige, sehr starke usw. Tiere teurer sind.\newline Sephrasto: Trage das entsprechende Tier in das Kommentarfeld ein.}
}


\newglossaryentry{krakenruf_Talent}
{
    name={Krakenruf},
    description={Du rufst ein Seeungeheuer wie eine Krake oder eine Seeschlage aus der Umgebung herbei, das sich so schnell wie möglich nähert, und kannst es um einen Gefallen bitten. Ist kein Ungeheuer in Reichweite, passiert nichts.\newline Mächtige Anrufung: Du rufst 2/4/8/16 Tiere.\newline Probenschwierigkeit: 12\newline Vorbereitungszeit: 2 Aktionen\newline Ziel: Zone\newline Reichweite: 4 Meilen\newline Wirkungsdauer: 1 Stunde\newline Kosten: 16 GuP\newline Fertigkeiten: Dämonische Hilfe (Cpt)\newline Erlernen: Cpt 16; 60 EP}
}


\newglossaryentry{mholurenhaut_Talent}
{
    name={Mholurenhaut},
    description={Der RS deines Zieles steigt um 1.\newline Mächtige Anrufung: Je zwei Stufen verleihen +1 RS.\newline Probenschwierigkeit: 12\newline Modifikationen: Körperschild (–4, 1 GuP; die Mholurenhaut schützt nur eine Trefferzone.)\newline Vorbereitungszeit: 1 Aktion\newline Ziel: Einzelperson\newline Reichweite: Berührung\newline Wirkungsdauer: 4 Minuten\newline Kosten: 2 GuP\newline Fertigkeiten: Dämonische Hilfe (Cpt)\newline Erlernen: Cpt 14; 40 EP}
}


\newglossaryentry{handgemenge_Fertigkeit}
{
    name={Handgemenge},
    description={Auf Handgemenge würfelst du bei allen waffenlosen Nahkampftechniken, im Umgang mit Handgemengewaffen wie Messern, Dolchen, Schlagstöcken und bei der Verwendung von Schilden.}
}


\newglossaryentry{hiebwaffen_Fertigkeit}
{
    name={Hiebwaffen},
    description={Mit der Fertigkeit Hiebwaffen führst du alle Arten von stumpfen und scharfen Hiebwaffen, wie Keulen, Äxte und Hämmer. }
}


\newglossaryentry{klingenwaffen_Fertigkeit}
{
    name={Klingenwaffen},
    description={Im Kampf mit allen Varianten von Schwertern, Säbeln und Fechtwaffen werden Proben auf die Fertigkeit Klingenwaffen abgelegt. Dolche gehören nicht zu den Klingenwaffen - sie werden mit der Fertigkeit Handgemenge genutzt.}
}


\newglossaryentry{stangenwaffen_Fertigkeit}
{
    name={Stangenwaffen},
    description={Alle Waffen mit einem langen Schaft gelten als Stangenwaffen. Zu den klassischen Stangenwaffen zählen Speere, Hellebarden und der Schnitter. Zusätzlich werden Kampfstäbe und die Lanzen berittener Kämpfer werden mit diesem Talent genutzt.}
}


\newglossaryentry{schusswaffen_Fertigkeit}
{
    name={Schusswaffen},
    description={Mit der Fertigkeit Schusswaffen bedienst du alle Waffen, mit denen Geschosse auf den Gegner abgefeuert werden. Dazu gehören so verbreitete Schusswaffen wie der Bogen oder die Armbrust, aber auch das Blasrohr.}
}


\newglossaryentry{wurfwaffen_Fertigkeit}
{
    name={Wurfwaffen},
    description={Als Wurfwaffen gelten sämtliche geworfenen Geschosse wie Diskusse oder Wurfspeere. Zusätzlich umfassen sie kurze Wurfwaffen, zu denen Dolche, Wurfsterne und improvisierte Geschosse wie einen Stein oder eine Flasche gehören.}
}


\newglossaryentry{athletik_Fertigkeit}
{
    name={Athletik},
    description={Athletik umfasst alle Aktivitäten, bei denen der Charakter seinen gesamten Körper kurz- oder längerfristig koordiniert einsetzen muss.}
}


\newglossaryentry{heimlichkeit_Fertigkeit}
{
    name={Heimlichkeit},
    description={Heimlichkeit ermöglicht es deinem Charakter, ungehört und ungesehen zu bleiben. Die Probe wird häufig vergleichend gegen die Wachsamkeit deines Kontrahenten gewürfelt.}
}


\newglossaryentry{selbstbeherrschung_Fertigkeit}
{
    name={Selbstbeherrschung},
    description={Selbstbeherrschung erlaubt es dir, widrige Umstände, körperliche Schmerzen oder andere Ablenkungen zu ignorieren und dich auf das einzig Wichtige zu konzentrieren.}
}


\newglossaryentry{wahrnehmung_Fertigkeit}
{
    name={Wahrnehmung},
    description={Wahrnehmung ist die Fähigkeit, selbst kleinste Sinneseindrücke wahrzunehmen und richtig zu interpretieren.}
}


\newglossaryentry{autorität_Fertigkeit}
{
    name={Autorität},
    description={Autorität erlaubt es deinem Charakter, sich Respekt zu verschaffen und seinen Willen durchzusetzen.}
}


\newglossaryentry{beeinflussung_Fertigkeit}
{
    name={Beeinflussung},
    description={Hierunter fallen sämtliche Techniken, um das Gegenüber zu einer gewünschten Handlung zu bewegen. Dabei werden Betören und Überreden oft als vergleichende Proben gegen die Menschenkenntnis des Opfers gewürfelt.}
}


\newglossaryentry{gebräuche_Fertigkeit}
{
    name={Gebräuche},
    description={Gebräuche stellt das Wissen um fremde und vertraute Kulturen sowie die dort herrschenden gesellschaftlichen Normen dar. Wie heißt der Markgrafen und wie sprichst du ihn an? In welchem Viertel findest du eine vertrauenswürdige Herberge, verschwiegene Schläger oder die neuesten Gerüchte? Wie funktioniert das Kamelspiel und auf welches Getränk könntest du deinen Gegner einladen? Herrscht in der nächsten Stadt ein Waffenverbot und wie streng wird es ausgelegt? Gebräuche beantwortet all diese Fragen und verhindert, dass du unangenehm auffällst.}
}


\newglossaryentry{derekunde_Fertigkeit}
{
    name={Derekunde},
    description={Darunter fällt das theoretische Wissen über Natur, Tiere und Pflanzen, sowie um die Geographie.}
}


\newglossaryentry{mythenkunde_Fertigkeit}
{
    name={Mythenkunde},
    description={Mythenkunde ist das Wissen über die Götter und ihre Diener, sowohl die eigene Religion, als auch – weniger genau – fremde Religionen betreffend. Zusätzlich gehört dazu die Kenntnis der Geschichte sowie bekannter Sagen und Legenden, was meist schwer voneinander zu trennen ist.}
}


\newglossaryentry{magiekunde_Fertigkeit}
{
    name={Magiekunde},
    description={Die astrale Kraft durchzieht die ganze Welt und wird seit Jahrtausenden von Zauberern genutzt. Magiekunde ist das theoretische Wissen über diese Kraft und ihre Verwendung.}
}


\newglossaryentry{überleben_Fertigkeit}
{
    name={Überleben},
    description={Hierunter fallen alle Fähigkeiten, die zum Überleben in der Wildnis erforderlich sind. Du kannst dich orientieren oder einen Lagerplatz, Nahrung und Wasser finden. Außerdem bist du mit typischen Gefahren der Wildnis vertraut und kannst tierische und menschliche Fährten deuten und ihnen folgen.\newline Dieses Wissen ist rein praktisch; exotische Tiere, Monster oder seltene Heilpflanzen fallen nicht in diesen Bereich.}
}


\newglossaryentry{verschlagenheit_Fertigkeit}
{
    name={Verschlagenheit},
    description={Verschlagenheit ist das Handwerk von zwielichtigen Gestalten, die Diebstahl, Falschsspiel und Einbruch nicht scheuen.}
}


\newglossaryentry{alchemie_Fertigkeit}
{
    name={Alchemie},
    description={Alchemie ist die Wissenschaft von der Umwandlung der Stoffe. Mit ihr kannst du nützliche Tränke herstellen oder unbekannte Gebräue analysieren; für beides ist aber zumindest ein einfacher Analysekoffer notwendig.}
}


\newglossaryentry{heilkunde_Fertigkeit}
{
    name={Heilkunde},
    description={Ein Heilkundiger bekämpft tödliche Blutungen, heimtückische Gifte sowie langwierige Krankheiten und hilft seinen Gefährten so, wieder auf die Beine zu kommen. }
}


\newglossaryentry{handwerk_Fertigkeit}
{
    name={Handwerk},
    description={Mit Handwerk schmiedest du Waffen und Rüstungen, fertigst filigrane Mechaniken und stabile Truhen.}
}


\newglossaryentry{antimagie_Übernatürliche-Fertigkeit}
{
    name={Antimagie},
    description={Mit Antimagie kannst du Zauber während der Vorbereitung stören, bereits gewirkte Zauber aufheben und beschworene Wesenheiten bannen. Gerade Magier der Weißen Gilde gelten als Experten der Antimagie.\newline Die meisten antimagischen Zauber richten sich gegen alle Zauber einer bestimmten übernatürlichen Fertigkeit (siehe Zauber) und verfügen über vier Modifikationen:\newline 1. Gegenzauber\newline Der Gegenzauber stört einen anderen Zauberer während dessen Zaubervorbereitung. Er wirkt als Konterprobe (12), bei deren Gelingen der Zielzauber sofort scheitert.\newline Probenschwierigkeit: 12\newline Vorbereitungszeit: 0 Aktionen\newline Ziel: Zauber in Vorbereitung\newline Reichweite: 16 Schritt\newline Wirkungsdauer: augenblicklich\newline Kosten: 4 AsP\newline \newline 2. Magie unterdrücken\newline Diese Variante unterdrückt zukünftige Zauber. In einer Zone von 16 Schritt Radius sind alle Zauber der Fertigkeit um –8 erschwert.\newline Mächtige Magie: Der Malus steigt um –4.\newline Probenschwierigkeit: 12\newline Vorbereitungszeit: 16 Aktionen\newline Ziel: Zone\newline Reichweite: 8 Schritt\newline Wirkungsdauer: 1 Stunde\newline Kosten: 8 AsP\newline 3. Zauber aufheben\newline Hiermit hebst du einen bereits gewirkten Zauber auf, in den keine gAsP geflossen sind. Zauber aufheben gilt als Konterprobe (12), bei deren Gelingen der Zielzauber sofort aufgehoben wird.\newline Probenschwierigkeit: 12\newline Vorbereitungszeit: 8 Aktionen\newline Ziel: Zauber\newline Reichweite: 8 Schritt\newline Wirkungsdauer: augenblicklich\newline Kosten: halbe Basiskosten des Zaubers\newline 4. Wesenheit bannen\newline Hiermit bannst du beschworene Wesenheiten, in die keine gAsP geflossen sind (wie nichtgebundene Elementare und Dämonen).\newline Probenschwierigkeit: Beschwörungsschwierigkeit des zu bannenden Wesens\newline Vorbereitungszeit: 8 Aktionen\newline Ziel: beschworenes Wesen\newline Reichweite: 8 Schritt\newline Wirkungsdauer: augenblicklich\newline Kosten: halbe Basiskosten der Beschwörung\newline Anmerkung: Wesenheiten und Zauber mit gAsP werden mit dem Destructibo gebannt.\newline Eine dieser vier Modifikationen kombinierst du mit einem antimagischen Zauber. Zum Beispiel kannst du mit dem Zauber Einfluss bannen und der Modifikation Zauber aufheben einen Imperavi aufheben, oder mit dem Elementarbann und Wesenheit bannen einen Dschinn austreiben.}
}


\newglossaryentry{dämonisch_Übernatürliche-Fertigkeit}
{
    name={Dämonisch},
    description={Diese finstere Zauberei ruft Dämonen in die dritte Sphäre, erhebt Untote, erschafft Chimären und Golems oder nutzt die Macht der Niederhöllen für andere zerstörerische Effekte.}
}


\newglossaryentry{eigenschaften_Übernatürliche-Fertigkeit}
{
    name={Eigenschaften},
    description={Mit Eigenschaftsmagie kannst du deine Fähigkeiten und die deiner Gefährten verbessern oder Feinde behindern. }
}


\newglossaryentry{einfluss_Übernatürliche-Fertigkeit}
{
    name={Einfluss},
    description={Als Einflusszauberer kannst du die Gedanken, Sinne und Gefühle deiner Mitmenschen manipulieren. Manche Einflusszauber (z.B. Bannbaladin) wirken äußerst subtil und sind während der Wirkung nicht zu bemerken. Im Zweifelsfall kann das Ziel nach dem Ende der Wirkung eine Konterprobe (KL, 16) ablegen, um sich die Verzauberung bewusst zu machen.}
}


\newglossaryentry{eis_Übernatürliche-Fertigkeit}
{
    name={Eis},
    description={Neben der Kälte wird Eis auch mit Stillstand, Tod, Gefühlskälte und Vernunft assoziiert. Diese Charakterzüge treffen auch auf die elementaren Wesenheiten des Eises zu.}
}


\newglossaryentry{erz_Übernatürliche-Fertigkeit}
{
    name={Erz},
    description={Härte und Beständigkeit gelten als Eigenschaften des Erzes, die mit Erzzauberei verstärkt werden können und von Erzelementaren verkörpert werden. }
}


\newglossaryentry{feuer_Übernatürliche-Fertigkeit}
{
    name={Feuer},
    description={Mit Feuerzauberei kannst du Gegenstände erwärmen oder verbrennen und die lodernden Wesenheiten des Feuers herbeirufen. }
}


\newglossaryentry{hellsicht_Übernatürliche-Fertigkeit}
{
    name={Hellsicht},
    description={Die Hellsicht kann deine Sinne verstärken oder erweitern, sodass du magische Kräfte, Wärme oder Gefühle wahrnehmen kannst. Hellseher sind auch Experten für die Analyse von Zaubern und Artefakten. }
}


\newglossaryentry{humus_Übernatürliche-Fertigkeit}
{
    name={Humus},
    description={Mit Humusmagie kannst du die Kraft der lebenden Natur nutzen, um Wunden zu heilen, Pflanzen wachsen zu lassen oder Humuselementare herbeizurufen. }
}


\newglossaryentry{illusion_Übernatürliche-Fertigkeit}
{
    name={Illusion},
    description={Illusionen sind Trugbilder, mit denen Scharlatane ihr Publikum in den Bann ziehen. Aber auch manche Dämonen erschaffen Illusionen, um ihre Opfer zu täuschen. }
}


\newglossaryentry{kraft_Übernatürliche-Fertigkeit}
{
    name={Kraft},
    description={Mit dieser Zauberei kannst die Zauberkraft selbst beherrschen, sie verändern und mit ihr Artefakte und magische Waffen erschaffen oder durch den Limbus reisen.}
}


\newglossaryentry{luft_Übernatürliche-Fertigkeit}
{
    name={Luft},
    description={Luftzauberer können Stürme erzeugen und beruhigen, das Wetter beeinflussen und die unsteten Luftelementare herbeirufen. }
}


\newglossaryentry{temporal_Übernatürliche-Fertigkeit}
{
    name={Temporal},
    description={Die mysteriöse Temporalmagie greift in den Lauf der Zeit ein und ermöglicht Legenden zufolge immerwährende Zauber oder sogar die Unsterblichkeit. Doch wer die Zeitmagie missbraucht, zieht den Zorn Satinavs auf sich.}
}


\newglossaryentry{umwelt_Übernatürliche-Fertigkeit}
{
    name={Umwelt},
    description={Mit Umweltzaubern kannst du deine direkte Umgebung - also Wetter, Helligkeit oder Schwerkraft - verändern oder mittels Telekinese Gegenstände bewegen. }
}


\newglossaryentry{verständigung_Übernatürliche-Fertigkeit}
{
    name={Verständigung},
    description={Mit Verständigung kannst du Gefühle und Gedanken übertragen oder Tiere, Geister oder magische Wesen zur Hilfe rufen. }
}


\newglossaryentry{verwandlung_Übernatürliche-Fertigkeit}
{
    name={Verwandlung},
    description={Verwandler können sich selbst, andere Menschen und sogar Gegenstände verwandeln. }
}


\newglossaryentry{wasser_Übernatürliche-Fertigkeit}
{
    name={Wasser},
    description={Als Wasserzauberer beherrschst du das unergründlichste aller Elemente und kannst die wechselhaften und unberechenbaren Elementare des Wassers herbeirufen.}
}


\newglossaryentry{dolchzauber_Übernatürliche-Fertigkeit}
{
    name={Dolchzauber},
    description={Die geheimen Rituale der Druiden beruhen auf ihrem verzauberten Dolch aus Obsidian, während Geoden eine goldene Sichel als Ritualinstrument bevorzugen. Alle Talente gelten als Objektrituale.}
}


\newglossaryentry{elfenlieder_Übernatürliche-Fertigkeit}
{
    name={Elfenlieder},
    description={Die auf einem Iama vorgetragenen Elfenlieder gehören zu den ältesten Spielarten der Zauberei, der du begegnen kannst. Da viele Lieder zweistimmig gesungen werden, steht die Fertigkeit nur Elfen und manchen Halbelfen zur Verfügung. Elfenlieder gelten als Objektrituale. }
}


\newglossaryentry{gabendesBlutgeists_Übernatürliche-Fertigkeit}
{
    name={Gaben des Blutgeists},
    description={Ein Anach-Nûr isst das Herz eines frisch erlegten Tieres, um dessen Blutgeist in sich aufzunehmen und sich durch dessen Kräfte zur stärken oder sogar zu verwandeln. Doch ein unaufmerksamer Moment und der Blutgeist gewinnt die Oberhand. Dann wird der Anach-Nûr zu einer rasenden Bestie...\newline   Die Tiere findest du auf Seite 160f.}
}


\newglossaryentry{gabendesOdun_Übernatürliche-Fertigkeit}
{
    name={Gaben des Odun},
    description={Durro-dûn spüren eine intensive Verbindung zu einem Tiergeist, dem Odun. Mit der Kraft des Odun stärken sie sich im Kampf oder kommen ihrer animalischen Seite näher.\newline   Die Tiere findest du auf Seite 160f.}
}


\newglossaryentry{geisterderStärkung_Übernatürliche-Fertigkeit}
{
    name={Geister der Stärkung},
    description={Geschickte Schamanen lassen wohlgesonnene Geister in Gegenstände und Personen einfahren und stärken sie so. Solche Schamanen sind oft ein Segen für den ganzen Stamm, den sie in vielerlei Situationen unterstützen können.}
}


\newglossaryentry{geisterdesZorns_Übernatürliche-Fertigkeit}
{
    name={Geister des Zorns},
    description={Manche Schamanen sind finstere Gesellen, die zornige Geister auf ihre Feinde hetzen. Solche Schamanen sind oft bei Freund und Feind gefürchtet.}
}


\newglossaryentry{geisterrufen_Übernatürliche-Fertigkeit}
{
    name={Geister rufen},
    description={Einige Schamanen sehen sich hauptsächlich als Vermittler zwischen der Welt der Menschen und dem Geisterreich. Sie können dessen Bewohner herbeirufen und sie um Hilfe bitten.}
}


\newglossaryentry{geistervertreiben_Übernatürliche-Fertigkeit}
{
    name={Geister vertreiben},
    description={Wann immer der Stamm von finsteren Geistern - also übernatürlichen Wesen, Wunden oder Krankheiten - bedroht ist, schreitet der Schamane ein, um die Geister wieder zu vertreiben.}
}


\newglossaryentry{hexenflüche_Übernatürliche-Fertigkeit}
{
    name={Hexenflüche},
    description={Als Hexe hast du zwei Möglichkeiten, deine Flüche zu überbringen. Entweder du schleuderst sie direkt lautstark auf das Opfer, wodurch du häufig vom Bonus der Hexischen Tradition II profitieren kannst, oder du lässt den Fluch unauffällig von deinem Vertrauten überbringen. Dafür benötigst du allerdings ein Körperteil (z.B. Haar) deines Opfers. In beiden Fällen legst du danach die Probe auf eine passende übernatürliche Fertigkeit ab. Jeder Hexenfluch kann auch permanent gesprochen werden. Dann halten ein Viertel der AsP als gAsP den Fluch aufrecht. Neben Antimagie oder göttlichem Wirken kann ein permanenter Fluch immer auch durch das Erfüllen einer Bedingung beendet werden. Diese Bedingung muss realistisch erfüllbar sein und du musst sie dem Verfluchten mitteilen (direkter Fluch) oder er erfährt sie in seinen Träumen (überbrachter Fluch).\newline   Neben den als Talente aufgelisteten Beispielen sollen die Töchter Satuarias zahlreiche weitere Flüche kennen, die Unfruchtbarkeit und Krankheit über Mensch und Tier bringen und Ernten zerstören können. Sogar von einem Todesfluch wird gemunkelt, der sein Opfer langsam, aber unaufhaltsam auszehrt.\newline   Jede Hexe kann auf einem Besen oder einem anderen Holzgegenstand reiten. Dazu erhält sie bei der halbjährlichen Hexennacht die Hexensalbe, mit der das bestrichene Objekt flugfähig ist. Die Aktivierung des Besens vor jedem Flug kostet 1 AsP. Tollkühne Flugmanöver erfordern Proben auf das Talent Akrobatik. Misslungene Proben bedeuten, dass die Hexe die Kontrolle über den Besen verliert und schlimmstenfalls unsanft notlanden muss. Bei einem Patzer kann der Hexe der Absturz drohen, sie könnte aber auch den Zorn eines Drachen geweckt haben.}
}


\newglossaryentry{keulenrituale_Übernatürliche-Fertigkeit}
{
    name={Keulenrituale},
    description={Die Knochenkeule kann von Geistern beseelt sein oder ihnen schaden. Knochenkeulen werden unter Schamanen vom Lehrer an den Schüler weitergegeben. Alle Keulenrituale gelten als Objektritual.}
}


\newglossaryentry{kristallmagie_Übernatürliche-Fertigkeit}
{
    name={Kristallmagie},
    description={Die Magie eines Kristallomanten steht und fällt mit seinen gebundenen Kristallen, da ohne den passenden Kristall eine wichtige Bedingung der kristallomantischen Tradition nicht erfüllt ist (S. 76). Jeder Kristall ist dabei einer Fertigkeit zugeordnet, die der Kristallomant fortan ohne Einschränkungen oder sogar mit bedeutenden Vorteilen nutzen kann. Nur die Kristallmagie selbst benötigt keinen Kristall. Die Kristalle und andere nützliche Gegenstände kann der Kristallomant in seinem magischen Schuppenbeutel transportieren, der etwa 4 Liter fasst. Alle Talente gelten als Objektrituale, die passiven Effekte des Schuppenbeutels erfordern aber keine ständige Berührung.}
}


\newglossaryentry{kugelzauber_Übernatürliche-Fertigkeit}
{
    name={Kugelzauber},
    description={Die Kristallkugel eines Scharlatans oder Gildenmagiers dient als Fokus für Illusionen, aber auch als Spionagewerkzeug. Alle Talente gelten als Objektrituale.}
}


\newglossaryentry{ringrituale_Übernatürliche-Fertigkeit}
{
    name={Ringrituale},
    description={Die Geoden sprechen ihre Rituale in einen Schlagenhalsreif aus Gold. Gleichzeitig ist der Schlangenreif ein besonders einladender Ort für Geister, wodurch es zu erwünschten und unerwünschten Besessenheiten kommen kann. Alle Talente gelten als Objektrituale.\newline   Bis der Geode die Seele seines verstorbenen Zwillingsbruders aufgespürt hat, ist der Schlangenreif ein Magnet für Geister. Das kannst du nutzen, um wohlwollende Elementargeister an den Schlangenreif zu binden, allerdings kann es auch zu unangenehmen oder sogar gefährlichen Besessenheiten kommen. Der Schlangenreif ist also Vorteil und Nachteil gleichermaßen - und eignet sich damit perfekt als eine Eigenheit für deinen Geoden.}
}


\newglossaryentry{schalenzauber_Übernatürliche-Fertigkeit}
{
    name={Schalenzauber},
    description={Dieses Ritualinstrument aus Silber oder Mondsilber wird von Alchemisten und einigen Gildenmagiern benutzt. Hexen verwenden stattdessen einen Kessel. Die Talente gelten als Objektrituale, die passiven Effekte der Schale erfordern jedoch keine ständige Berührung.}
}


\newglossaryentry{stabzauber_Übernatürliche-Fertigkeit}
{
    name={Stabzauber},
    description={Der Zauberstab eines Gildenmagiers ist sicherlich das bekannteste Ritualinstrument. Er unterstützt den Magier als Fackel, Seil oder magischer Rammbock, kann aber auch Astralenergie speichern. Alle Talente gelten als Objektrituale. }
}


\newglossaryentry{trommelrituale_Übernatürliche-Fertigkeit}
{
    name={Trommelrituale},
    description={Die Derwische rufen mit ihren Trommelritualen ihren Gott Rashtullah an und stärken so ihre rechtgläubigen Mitstreiter.}
}


\newglossaryentry{vertrautenmagie_Übernatürliche-Fertigkeit}
{
    name={Vertrautenmagie},
    description={Hexen und Geoden (wir nennen sie der Einfachheit halber Bindungspartner) schaffen mit der Bindung des Vertrauten ein magisches Band zwischen sich und dem Vertrautentier. Dabei erlangt das Vertrautentier magische Fähigkeiten, die es fortan einsetzen kann.\newline Doch auch wenn das Vertrautentier diese Fähigkeiten einsetzt - der Einfachheit halber behandeln wir die Vertrautenmagie so, als würdest du den Zauber wirken. Du entscheidest, wann ein Zauber eingesetzt wird, legst die Probe auf deinen PW in Vertrautenmagie ab und bezahlst die Kosten – nur Reichweite und Ziel gehen vom Vertrautentier aus.\newline Die Spielwerte typischer Vertrautentiere findest du im Bestiarium. Da Vertrautentiere aber besonders prächtige Vertreter ihrer Gattung sind, erhältst du 4 Vertrautenpunkte pro Stufe der Geodischen oder Hexischen Tradition. Pro Punkt darfst du einen Spielwert des Vertrautentiers (ausgenommen WS, WS* und RW) um 1 Punkt erhöhen, solange das Ergebnis dem gesunden Menschenverstand nicht widerspricht.}
}


\newglossaryentry{zaubermelodien_Übernatürliche-Fertigkeit}
{
    name={Zaubermelodien},
    description={Der Zauberbarde webt mit seinen Melodien ein subtiles Netz der Magie, das für Außenstehende oft gar nicht als Zauber erkennbar ist.}
}


\newglossaryentry{zaubertänze_Übernatürliche-Fertigkeit}
{
    name={Zaubertänze},
    description={Die Zaubertänze der Tulamiden werden nur selten als Magie wahrgenommen und gehören doch zu einer der ältesten menschlichen Zaubertraditionen Aventuriens.}
}


\newglossaryentry{zwölfgöttlicherRitus_Übernatürliche-Fertigkeit}
{
    name={Zwölfgöttlicher Ritus},
    description={Durch die Zusammenarbeit der Kirchen hat sich ein Grundstock allgemein bekannter Liturgien ergeben, die aus dem Leben des Geweihten nicht mehr wegzudenken sind.\newline Fast alle Geweihten der Zwölf beherrschen die zwölf Segnungen: Eid-, Feuer-, Geburts-, Glücks-, Grab-, Harmonie-, Heilungs-, Märtyrer-, Schutz-, Speise-, Trank- und Weisheitssegen.}
}


\newglossaryentry{schlaf_Übernatürliche-Fertigkeit}
{
    name={Schlaf},
    description={Boron schenkt den Menschen Schlaf und schickt ihnen Träume und Visionen, die zu erkunden sich viele Borongeweihte zur Aufgabe gemacht haben. Gerade im Al’Anfanischen Kult wird dabei auch mit Rauschkräutern nachgeholfen. }
}


\newglossaryentry{tod_Übernatürliche-Fertigkeit}
{
    name={Tod},
    description={Der Alltag vieler Borongeweihter besteht aus der Sterbebegleitung und der Pflege der Boronsanger. Doch in diesen düsteren Zeiten tauchen auch Untote immer häufiger auf, die von Borongeweihten und den Golgariten entschlossen bekämpft werden.}
}


\newglossaryentry{vergessen_Übernatürliche-Fertigkeit}
{
    name={Vergessen},
    description={Vergessen ist eine weitere Gabe Borons, die insbesondere die Noioniten ihren geisteskranken Schützlingen zukommen lassen wollen. }
}


\newglossaryentry{flüsseundQuellen_Übernatürliche-Fertigkeit}
{
    name={Flüsse und Quellen},
    description={Der Legende nach sind die Flüsse die süßen Tränen Efferds, dementsprechend sind auch viele Geweihte des Efferd im Binnenland tätig.}
}


\newglossaryentry{seefahrt_Übernatürliche-Fertigkeit}
{
    name={Seefahrt},
    description={Pragmatische Efferdgeweihte sind aus der aventurischen Seefahrt nicht wegzudenken. Sie dienen als Bordgeweihte und schützen Schiffe vor den Gefahren der blutigen See. }
}


\newglossaryentry{windundWogen_Übernatürliche-Fertigkeit}
{
    name={Wind und Wogen},
    description={Die Mystiker unter den Efferdgeweihten schwimmen mit den Delphinen, studieren Wind und Wetter und versuchen aus dem Wellenspiel oder dem Vogelflug den Willen des Unergründlichen abzuleiten.}
}


\newglossaryentry{jagd_Übernatürliche-Fertigkeit}
{
    name={Jagd},
    description={Die Hüter der Jagd begleiten Jagdgesellschaften und achten über die Einhaltung der waidmännischen Regeln.}
}


\newglossaryentry{wildnis_Übernatürliche-Fertigkeit}
{
    name={Wildnis},
    description={Die Waldläufer des Nordens streifen oft monatelang abseits aller Wege durch die Wildnis, ohne ihr Ziel aus den Augen zu verlieren.}
}


\newglossaryentry{winter_Übernatürliche-Fertigkeit}
{
    name={Winter},
    description={Firungeweihte Einsiedler des Nordens harren in lebensfeindlicher Kälte aus und trotzen Schneestürmen, um sich ihrem Herrn näher zu fühlen.}
}


\newglossaryentry{magie_Übernatürliche-Fertigkeit}
{
    name={Magie},
    description={Hesinde gilt als Schutzpatronin der Magie und nicht wenige Geweihte, gerade unter den Draconitern, gelten als Experten in diesem Gebiet.}
}


\newglossaryentry{wissen_Übernatürliche-Fertigkeit}
{
    name={Wissen},
    description={Hesinde ist die Quelle des Wissens, das von den Geweihten gesammelt und je nach Strömung freimütig geteilt oder eifersüchtig gehütet wird. }
}


\newglossaryentry{heiligesErz_Übernatürliche-Fertigkeit}
{
    name={Heiliges Erz},
    description={Ingerimm ist der Herr der Erzes und der unterirdischen Schätze. Er bewahrt die Gläubigen unter der Erde, während Frevler seinen alles erschütternden Zorn spüren. }
}


\newglossaryentry{heiligesFeuer_Übernatürliche-Fertigkeit}
{
    name={Heiliges Feuer},
    description={Nordaventurische Feueranbeter und zwergische Hüter der Wacht nutzen die lebensspendende und erstörerische Kraft des Feuers, um gegen eisige Kälte oder unheiliges Drachenfeuer zu bestehen.}
}


\newglossaryentry{heiligesHandwerk_Übernatürliche-Fertigkeit}
{
    name={Heiliges Handwerk},
    description={Die meisten menschlichen Ingerimm-Geweihten und die zwergischen Hüter der Esse verstehen sich als Handwerker, die dem vollkommenen Handwerksstück nacheifern. }
}


\newglossaryentry{heilung_Übernatürliche-Fertigkeit}
{
    name={Heilung},
    description={Perainegeweihte gelten als hervorragende Ärzte, die sich in den Städten oder auf Schlachtfeldern unermüdlich um Kranke und Verwundete kümmern. }
}


\newglossaryentry{wachstum_Übernatürliche-Fertigkeit}
{
    name={Wachstum},
    description={Viele ländliche Perainegeweihte beteiligen sich an der bäuerlichen Arbeit und segnen Felder und Jungtiere. }
}


\newglossaryentry{abual'Mada_Übernatürliche-Fertigkeit}
{
    name={Abu al'Mada},
    description={In den Tulamidenlanden gehört zum Glauben an Feqz seit jeher auch sein Beiname als Herr der Magie und Vertreiber der Echsen.}
}


\newglossaryentry{list_Übernatürliche-Fertigkeit}
{
    name={List},
    description={Phex ist ein Gott der Opportunisten, Betrüger und listenreichen Händler. Seine Diener sind darauf angewiesen, schnell zu denken - und am besten um drei Ecken mehr als das Gegenüber.}
}


\newglossaryentry{nächtlicherSchatten_Übernatürliche-Fertigkeit}
{
    name={Nächtlicher Schatten},
    description={Phex ist der Herrscher der Nacht und viele seiner Diener handeln unter dem Mantel der Dunkelheit wenig gesetzestreu. Doch die sternenerfüllte Nacht zieht auch viele phexgeweihte Mystiker an. }
}


\newglossaryentry{licht_Übernatürliche-Fertigkeit}
{
    name={Licht},
    description={Das Licht Praios' ist vor allem von den Mystikern in der Kirche hoch verehrt. }
}


\newglossaryentry{magiebann_Übernatürliche-Fertigkeit}
{
    name={Magiebann},
    description={Die Antimagie und der Kampf gegen Dämonen haben gerade in den letzten Jahren einen erhöhten Stellenwert innerhalb der Praioskirche erhalten.}
}


\newglossaryentry{ordnung_Übernatürliche-Fertigkeit}
{
    name={Ordnung},
    description={Die Praiospriester bewahren das Recht und die Gesetze, die göttergewollte Ordnung und die Wahrheit. }
}


\newglossaryentry{harmonie_Übernatürliche-Fertigkeit}
{
    name={Harmonie},
    description={Im güldenländisch geprägten Kult West- und Nordaventuriens steht die Harmonie der Göttin im Vordergrund. Die Geweihten widmen sich der Ausrichtung von Festen, der Pferdezucht, dem Weinbau und den schönen Künsten. }
}


\newglossaryentry{rausch_Übernatürliche-Fertigkeit}
{
    name={Rausch},
    description={Im tulamidisch geprägten Kult des Ostens suchen die Geweihten die göttliche Ekstase durch Leidenschaft, Drogen oder akrobatische Tänze - denn nur in der Ekstase wird die Seele frei von Hass, Neid und Angst.}
}


\newglossaryentry{ehre_Übernatürliche-Fertigkeit}
{
    name={Ehre},
    description={Ehre ist der traditionellste und bekannteste Aspekt der Rondra. Neben den Geboten der Ritterlichkeit umfasst sie auch sehr spezielle Regeln, wann und wie gekämpft werden darf.}
}


\newglossaryentry{heerführung_Übernatürliche-Fertigkeit}
{
    name={Heerführung},
    description={Fast jedes größere Heer wird von einem rondrageweihten Feldkaplan begleitet, der über die Einhaltung des Kriegsrechts wacht und die Kämpfer gegen finstere Mächte stärkt. }
}


\newglossaryentry{schutzderGläubigen_Übernatürliche-Fertigkeit}
{
    name={Schutz der Gläubigen},
    description={Der Schutz der Gläubigen ist die Bestimmung der Rondrakirche. Ein Geweihter hat an im Kampf gegen die Finsternis an vorderster Stelle zu stehen.}
}


\newglossaryentry{heimundHerd_Übernatürliche-Fertigkeit}
{
    name={Heim und Herd},
    description={Viele Traviageweihte segnen Wohnstätten, schließen Ehen und achten auf die Einhaltung von guten Sitten und Gastfreundschaft. Die Badilakaner widmen sich hingegen der Aufgabe, das Leid der Armen und Bedürftigen zu lindern.}
}


\newglossaryentry{sichereHeimkehr_Übernatürliche-Fertigkeit}
{
    name={Sichere Heimkehr},
    description={Manche Geweihte der Travia begeben sich als Wildgänse selbst auf Wanderschaft, um auch den Heimatlosen die Vorzüge eines geregelten Lebens nahe zu bringen. Auch Thorwaler Traviageweihte segnen Reisende auf der großen Fahrt, sodass sie wohlbehalten zurückkehren mögen. }
}


\newglossaryentry{friede_Übernatürliche-Fertigkeit}
{
    name={Friede},
    description={Tsageweihte bewahren die Schöpfung und das Leben. Besonders die Friedensfreunde und die Anhänger des Dreischwesternordens suchen stets nach friedlichen Lösungen für Konflikte und anstatt Waffen zu tragen, heilen sie die von ihnen geschlagenen Wunden. }
}


\newglossaryentry{neubeginn_Übernatürliche-Fertigkeit}
{
    name={Neubeginn},
    description={Tsageweihte begrüßen Veränderung und Freiheit, egal ob bei der Geburt eines Kindes oder beim Ausüben immer neuer Steckenpferde. Manche Geweihte richten sich dabei sogar gegen althergebrachte Traditionen: sie gelten als gefährliche Freidenker und Rebellen. }
}


\newglossaryentry{fröhlicherWanderer_Übernatürliche-Fertigkeit}
{
    name={Fröhlicher Wanderer},
    description={Viele Avesgeweihte fühlen sich der lebenslustigen Rahja nahe und ziehen in bunten Gewändern in die Welt hinaus, um Farbe in das Leben der Menschen - und natürlich ihrer Reisegefährten - zu bringen. }
}


\newglossaryentry{stillerWanderer_Übernatürliche-Fertigkeit}
{
    name={Stiller Wanderer},
    description={Andere Avesgeweihte wandern auf den Spuren von dessen göttlichen Vaters Phex' und ziehen unerkannt durch die Welt. Sie vertrauen auf die Macht des Schicksals und die Tatsache, dass oft auch kleine Taten den Lauf der Welt ändern können.}
}


\newglossaryentry{guterKampf_Übernatürliche-Fertigkeit}
{
    name={Guter Kampf},
    description={Die Mystiker der Korkirche suchen den Willen ihres Gottes im Kampf. Ihr Weg führt sie dorthin, wo die Schlacht am heftigsten tobt und wo sie ihr Blut für den Geifernden Schnitter vergießen.}
}


\newglossaryentry{gutesGold_Übernatürliche-Fertigkeit}
{
    name={Gutes Gold},
    description={Die Pragmatiker unter den Korgeweihten stehen Gladiatoren zur Seite oder führen Söldnerbanner an. Sie sorgen für vertragsgetreue Bezahlung und die Einhaltung der Söldnerehre.}
}


\newglossaryentry{dämonischeHilfe(Cpt)_Übernatürliche-Fertigkeit}
{
    name={Dämonische Hilfe (Cpt)},
    description={Charyptoroth ist die Herrin der nachtblauen Tiefen, aus denen verseuchtes Wasser und Meeresungeheuer strömen.}
}


\newglossaryentry{eisenmantel_Rüstung}
{
    name={Eisenmantel},
    description={\textbf{RsBeine}: 2 \textbf{RsLArm}: 2 \textbf{RsRArm}: 2 \textbf{RsBauch}: 3 \textbf{RsBrust}: 3 \textbf{RsKopf}: 0\newline Stoff- oder Lederrüstung, in die - von außen nicht sichtbar - Metallplättchen genäht wurden.}
}


\newglossaryentry{fünflagenharnisch_Rüstung}
{
    name={Fünflagenharnisch},
    description={\textbf{RsBeine}: 1 \textbf{RsLArm}: 0 \textbf{RsRArm}: 0 \textbf{RsBauch}: 3 \textbf{RsBrust}: 3 \textbf{RsKopf}: 0\newline Lederweste, in die - von außen nicht sichtbar - Metallplättchen genäht wurden.}
}


\newglossaryentry{gestechrüstung_Rüstung}
{
    name={Gestechrüstung},
    description={\textbf{RsBeine}: 5 \textbf{RsLArm}: 5 \textbf{RsRArm}: 5 \textbf{RsBauch}: 5 \textbf{RsBrust}: 5 \textbf{RsKopf}: 5\newline Sehr schwere Komplettrüstung, kann nicht ergänzt werden.}
}


\newglossaryentry{gladiatorenschulter_Rüstung}
{
    name={Gladiatorenschulter},
    description={\textbf{RsBeine}: 0 \textbf{RsLArm}: 0 \textbf{RsRArm}: 2 \textbf{RsBauch}: 0 \textbf{RsBrust}: 2 \textbf{RsKopf}: 0\newline Kann nicht zweimal getragen werden.}
}


\newglossaryentry{hartholzharnisch_Rüstung}
{
    name={Hartholzharnisch},
    description={\textbf{RsBeine}: 1 \textbf{RsLArm}: 1 \textbf{RsRArm}: 1 \textbf{RsBauch}: 3 \textbf{RsBrust}: 3 \textbf{RsKopf}: 0\newline Die Rüstung besteht aus maraskanischem Hartholz und ist feuchtigkeitsempfindlich. Benötigt wattierte Unterkleidung.}
}


\newglossaryentry{horasischerReiterharnisch_Rüstung}
{
    name={Horasischer Reiterharnisch},
    description={\textbf{RsBeine}: 3 \textbf{RsLArm}: 4 \textbf{RsRArm}: 4 \textbf{RsBauch}: 5 \textbf{RsBrust}: 5 \textbf{RsKopf}: 3\newline Komplettrüstung, kann nicht ergänzt werden.}
}


\newglossaryentry{iryanrüstung_Rüstung}
{
    name={Iryanrüstung},
    description={\textbf{RsBeine}: 1 \textbf{RsLArm}: 0 \textbf{RsRArm}: 0 \textbf{RsBauch}: 2 \textbf{RsBrust}: 2 \textbf{RsKopf}: 0\newline Die Rüstung ist feuerresistent.}
}


\newglossaryentry{kettenmantel_Rüstung}
{
    name={Kettenmantel},
    description={\textbf{RsBeine}: 3 \textbf{RsLArm}: 3 \textbf{RsRArm}: 3 \textbf{RsBauch}: 3 \textbf{RsBrust}: 3 \textbf{RsKopf}: 0\newline }
}


\newglossaryentry{kettenweste_Rüstung}
{
    name={Kettenweste},
    description={\textbf{RsBeine}: 0 \textbf{RsLArm}: 0 \textbf{RsRArm}: 0 \textbf{RsBauch}: 3 \textbf{RsBrust}: 3 \textbf{RsKopf}: 0\newline }
}


\newglossaryentry{kürass_Rüstung}
{
    name={Kürass},
    description={\textbf{RsBeine}: 0 \textbf{RsLArm}: 0 \textbf{RsRArm}: 0 \textbf{RsBauch}: 2 \textbf{RsBrust}: 3 \textbf{RsKopf}: 0\newline Benötigt keine wattierte Unterkleidung.}
}


\newglossaryentry{lederharnisch_Rüstung}
{
    name={Lederharnisch},
    description={\textbf{RsBeine}: 0 \textbf{RsLArm}: 0 \textbf{RsRArm}: 0 \textbf{RsBauch}: 2 \textbf{RsBrust}: 2 \textbf{RsKopf}: 0\newline }
}


\newglossaryentry{mammutonpanzer_Rüstung}
{
    name={Mammutonpanzer},
    description={\textbf{RsBeine}: 2 \textbf{RsLArm}: 2 \textbf{RsRArm}: 2 \textbf{RsBauch}: 3 \textbf{RsBrust}: 3 \textbf{RsKopf}: 0\newline Lederrüstung mit Scheiben von Mammutstoßzähnen.}
}


\newglossaryentry{ringelpanzer_Rüstung}
{
    name={Ringelpanzer},
    description={\textbf{RsBeine}: 1 \textbf{RsLArm}: 3 \textbf{RsRArm}: 3 \textbf{RsBauch}: 3 \textbf{RsBrust}: 3 \textbf{RsKopf}: 0\newline Benötigt keine wattierte Unterkleidung.}
}


\newglossaryentry{spiegelpanzer_Rüstung}
{
    name={Spiegelpanzer},
    description={\textbf{RsBeine}: 2 \textbf{RsLArm}: 3 \textbf{RsRArm}: 3 \textbf{RsBauch}: 4 \textbf{RsBrust}: 4 \textbf{RsKopf}: 0\newline }
}


\newglossaryentry{tuchrüstung_Rüstung}
{
    name={Tuchrüstung},
    description={\textbf{RsBeine}: 0 \textbf{RsLArm}: 0 \textbf{RsRArm}: 0 \textbf{RsBauch}: 2 \textbf{RsBrust}: 2 \textbf{RsKopf}: 0\newline }
}


\newglossaryentry{amazonenrüstung_Rüstung}
{
    name={Amazonenrüstung},
    description={\textbf{RsBeine}: 3 \textbf{RsLArm}: 2 \textbf{RsRArm}: 2 \textbf{RsBauch}: 4 \textbf{RsBrust}: 4 \textbf{RsKopf}: 3\newline Komplettrüstung aus Bronzekürass, Arm- und Beinschienen, Streifenschurz und visierlosem Helm. Kann nicht ergänzt werden.}
}


\newglossaryentry{armschienen(Bronze)_Rüstung}
{
    name={Armschienen (Bronze)},
    description={\textbf{RsBeine}: 0 \textbf{RsLArm}: 2 \textbf{RsRArm}: 2 \textbf{RsBauch}: 0 \textbf{RsBrust}: 0 \textbf{RsKopf}: 0\newline Rüstungsergänzung für die Arme.}
}


\newglossaryentry{armschienen(Leder)_Rüstung}
{
    name={Armschienen (Leder)},
    description={\textbf{RsBeine}: 0 \textbf{RsLArm}: 1 \textbf{RsRArm}: 1 \textbf{RsBauch}: 0 \textbf{RsBrust}: 0 \textbf{RsKopf}: 0\newline Rüstungsergänzung für die Arme.}
}


\newglossaryentry{armschienen(Stahl)_Rüstung}
{
    name={Armschienen (Stahl)},
    description={\textbf{RsBeine}: 0 \textbf{RsLArm}: 3 \textbf{RsRArm}: 3 \textbf{RsBauch}: 0 \textbf{RsBrust}: 0 \textbf{RsKopf}: 0\newline Rüstungsergänzung für die Arme.}
}


\newglossaryentry{beinschienen(Bronze)_Rüstung}
{
    name={Beinschienen (Bronze)},
    description={\textbf{RsBeine}: 2 \textbf{RsLArm}: 0 \textbf{RsRArm}: 0 \textbf{RsBauch}: 0 \textbf{RsBrust}: 0 \textbf{RsKopf}: 0\newline Rüstungsergänzung für die Beine.}
}


\newglossaryentry{beinschienen(Leder)_Rüstung}
{
    name={Beinschienen (Leder)},
    description={\textbf{RsBeine}: 1 \textbf{RsLArm}: 0 \textbf{RsRArm}: 0 \textbf{RsBauch}: 0 \textbf{RsBrust}: 0 \textbf{RsKopf}: 0\newline Rüstungsergänzung für die Beine.}
}


\newglossaryentry{beinschienen(Stahl)_Rüstung}
{
    name={Beinschienen (Stahl)},
    description={\textbf{RsBeine}: 3 \textbf{RsLArm}: 0 \textbf{RsRArm}: 0 \textbf{RsBauch}: 0 \textbf{RsBrust}: 0 \textbf{RsKopf}: 0\newline Rüstungsergänzung für die Beine.}
}


\newglossaryentry{beintaschen_Rüstung}
{
    name={Beintaschen},
    description={\textbf{RsBeine}: 2 \textbf{RsLArm}: 0 \textbf{RsRArm}: 0 \textbf{RsBauch}: 1 \textbf{RsBrust}: 0 \textbf{RsKopf}: 0\newline Rüstungsergänzung für die Beine.}
}


\newglossaryentry{hoheStiefel_Rüstung}
{
    name={Hohe Stiefel},
    description={\textbf{RsBeine}: 1 \textbf{RsLArm}: 0 \textbf{RsRArm}: 0 \textbf{RsBauch}: 0 \textbf{RsBrust}: 0 \textbf{RsKopf}: 0\newline Rüstungsergänzung für die Füße.}
}


\newglossaryentry{kettenbeinlinge_Rüstung}
{
    name={Kettenbeinlinge},
    description={\textbf{RsBeine}: 2 \textbf{RsLArm}: 0 \textbf{RsRArm}: 0 \textbf{RsBauch}: 0 \textbf{RsBrust}: 0 \textbf{RsKopf}: 0\newline Rüstungsergänzung für die Beine.}
}


\newglossaryentry{kettenhandschuhe_Rüstung}
{
    name={Kettenhandschuhe},
    description={\textbf{RsBeine}: 0 \textbf{RsLArm}: 1 \textbf{RsRArm}: 1 \textbf{RsBauch}: 0 \textbf{RsBrust}: 0 \textbf{RsKopf}: 0\newline Rüstungsergänzung für die Hände.}
}


\newglossaryentry{kettenkragen_Rüstung}
{
    name={Kettenkragen},
    description={\textbf{RsBeine}: 0 \textbf{RsLArm}: 0 \textbf{RsRArm}: 0 \textbf{RsBauch}: 0 \textbf{RsBrust}: 0 \textbf{RsKopf}: 1\newline Rüstungsergänzung für den Hals.}
}


\newglossaryentry{lederhose_Rüstung}
{
    name={Lederhose},
    description={\textbf{RsBeine}: 1 \textbf{RsLArm}: 0 \textbf{RsRArm}: 0 \textbf{RsBauch}: 1 \textbf{RsBrust}: 0 \textbf{RsKopf}: 0\newline Rüstungsergänzung für die Beine.}
}


\newglossaryentry{lederweste_Rüstung}
{
    name={Lederweste},
    description={\textbf{RsBeine}: 0 \textbf{RsLArm}: 0 \textbf{RsRArm}: 0 \textbf{RsBauch}: 1 \textbf{RsBrust}: 1 \textbf{RsKopf}: 0\newline }
}


\newglossaryentry{löwenmähne_Rüstung}
{
    name={Löwenmähne},
    description={\textbf{RsBeine}: 0 \textbf{RsLArm}: 1 \textbf{RsRArm}: 1 \textbf{RsBauch}: 0 \textbf{RsBrust}: 1 \textbf{RsKopf}: 1\newline Rüstungsergänzung für den Hals.}
}


\newglossaryentry{mattenrücken_Rüstung}
{
    name={Mattenrücken},
    description={\textbf{RsBeine}: 0 \textbf{RsLArm}: 0 \textbf{RsRArm}: 0 \textbf{RsBauch}: 0 \textbf{RsBrust}: 1 \textbf{RsKopf}: 1\newline Rüstungsergänzung der Achaz für den Rücken. Wird üblicherweise ergänzt mit einer Brustplatte (Leder).}
}


\newglossaryentry{panzerbeine_Rüstung}
{
    name={Panzerbeine},
    description={\textbf{RsBeine}: 3 \textbf{RsLArm}: 0 \textbf{RsRArm}: 0 \textbf{RsBauch}: 0 \textbf{RsBrust}: 0 \textbf{RsKopf}: 0\newline Rüstungsergänzung für die Beine.}
}


\newglossaryentry{panzerhandschuhe_Rüstung}
{
    name={Panzerhandschuhe},
    description={\textbf{RsBeine}: 0 \textbf{RsLArm}: 2 \textbf{RsRArm}: 2 \textbf{RsBauch}: 0 \textbf{RsBrust}: 0 \textbf{RsKopf}: 0\newline Rüstungsergänzung für die Hände.}
}


\newglossaryentry{panzerschuh_Rüstung}
{
    name={Panzerschuh},
    description={\textbf{RsBeine}: 1 \textbf{RsLArm}: 0 \textbf{RsRArm}: 0 \textbf{RsBauch}: 0 \textbf{RsBrust}: 0 \textbf{RsKopf}: 0\newline Rüstungsergänzung für die Füße.}
}


\newglossaryentry{pelzweste_Rüstung}
{
    name={Pelzweste},
    description={\textbf{RsBeine}: 0 \textbf{RsLArm}: 0 \textbf{RsRArm}: 0 \textbf{RsBauch}: 1 \textbf{RsBrust}: 1 \textbf{RsKopf}: 0\newline }
}


\newglossaryentry{plattenarme_Rüstung}
{
    name={Plattenarme},
    description={\textbf{RsBeine}: 0 \textbf{RsLArm}: 3 \textbf{RsRArm}: 3 \textbf{RsBauch}: 0 \textbf{RsBrust}: 0 \textbf{RsKopf}: 0\newline Rüstungsergänzung für die Arme.}
}


\newglossaryentry{plattenschultern_Rüstung}
{
    name={Plattenschultern},
    description={\textbf{RsBeine}: 0 \textbf{RsLArm}: 1 \textbf{RsRArm}: 1 \textbf{RsBauch}: 0 \textbf{RsBrust}: 1 \textbf{RsKopf}: 0\newline Rüstungsergänzung für die Schultern.}
}


\newglossaryentry{schürze_Rüstung}
{
    name={Schürze},
    description={\textbf{RsBeine}: 2 \textbf{RsLArm}: 0 \textbf{RsRArm}: 0 \textbf{RsBauch}: 1 \textbf{RsBrust}: 0 \textbf{RsKopf}: 0\newline Rüstungsergänzung für die Beine.}
}


\newglossaryentry{wattiertesUnterzeug(Kettenteile)_Rüstung}
{
    name={Wattiertes Unterzeug (Kettenteile)},
    description={\textbf{RsBeine}: 1 \textbf{RsLArm}: 2 \textbf{RsRArm}: 2 \textbf{RsBauch}: 1 \textbf{RsBrust}: 2 \textbf{RsKopf}: 0\newline Wird üblicherweise unter Metallrüstungen gegen Abschürfungen getragen.}
}


\newglossaryentry{brustplatte(ZRS)_Rüstung}
{
    name={Brustplatte (ZRS)},
    description={\textbf{RsBeine}: 0 \textbf{RsLArm}: 0 \textbf{RsRArm}: 0 \textbf{RsBauch}: 1 \textbf{RsBrust}: 1 \textbf{RsKopf}: 0\newline }
}


\newglossaryentry{brustplatte_Rüstung}
{
    name={Brustplatte},
    description={\textbf{RsBeine}: 1 \textbf{RsLArm}: 1 \textbf{RsRArm}: 1 \textbf{RsBauch}: 1 \textbf{RsBrust}: 1 \textbf{RsKopf}: 1\newline }
}


\newglossaryentry{brustschalen(ZRS)_Rüstung}
{
    name={Brustschalen (ZRS)},
    description={\textbf{RsBeine}: 0 \textbf{RsLArm}: 0 \textbf{RsRArm}: 0 \textbf{RsBauch}: 0 \textbf{RsBrust}: 1 \textbf{RsKopf}: 0\newline Kann nicht mit anderen Torsorüstungen kombiniert werden.}
}


\newglossaryentry{garetherPlatte(ZRS)_Rüstung}
{
    name={Garether Platte (ZRS)},
    description={\textbf{RsBeine}: 3 \textbf{RsLArm}: 4 \textbf{RsRArm}: 4 \textbf{RsBauch}: 5 \textbf{RsBrust}: 5 \textbf{RsKopf}: 0\newline Komplettrüstung, kann bis auf Hals, Kopf und Füße nicht ergänzt werden. Wird üblicherweise mit Schaller und Bart ergänzt.}
}


\newglossaryentry{baburinerHut_Rüstung}
{
    name={Baburiner Hut},
    description={\textbf{RsBeine}: 0 \textbf{RsLArm}: 0 \textbf{RsRArm}: 0 \textbf{RsBauch}: 0 \textbf{RsBrust}: 0 \textbf{RsKopf}: 4\newline Helm. Üblicherweise gepolstert.}
}


\newglossaryentry{bart_Rüstung}
{
    name={Bart},
    description={\textbf{RsBeine}: 0 \textbf{RsLArm}: 0 \textbf{RsRArm}: 0 \textbf{RsBauch}: 0 \textbf{RsBrust}: 0 \textbf{RsKopf}: 1\newline Rüstungsergänzung für den Hals. Wird üblicherweise mit der Schaller kombiniert.}
}


\newglossaryentry{drachenhelm_Rüstung}
{
    name={Drachenhelm},
    description={\textbf{RsBeine}: 0 \textbf{RsLArm}: 0 \textbf{RsRArm}: 0 \textbf{RsBauch}: 0 \textbf{RsBrust}: 0 \textbf{RsKopf}: 3\newline Helm. Üblicherweise gepolstert.}
}


\newglossaryentry{halsberge_Rüstung}
{
    name={Halsberge},
    description={\textbf{RsBeine}: 0 \textbf{RsLArm}: 0 \textbf{RsRArm}: 0 \textbf{RsBauch}: 0 \textbf{RsBrust}: 0 \textbf{RsKopf}: 1\newline Rüstungsergänzung für den Hals. Wird üblicherweise mit der Schaller kombiniert. Variante des Barts.}
}


\newglossaryentry{kettenhaubemitGesichtsschutz_Rüstung}
{
    name={Kettenhaube mit Gesichtsschutz},
    description={\textbf{RsBeine}: 0 \textbf{RsLArm}: 0 \textbf{RsRArm}: 0 \textbf{RsBauch}: 0 \textbf{RsBrust}: 0 \textbf{RsKopf}: 3\newline Helm.}
}


\newglossaryentry{lederhelm(verstärkt)_Rüstung}
{
    name={Lederhelm (verstärkt)},
    description={\textbf{RsBeine}: 0 \textbf{RsLArm}: 0 \textbf{RsRArm}: 0 \textbf{RsBauch}: 0 \textbf{RsBrust}: 0 \textbf{RsKopf}: 3\newline Helm.}
}


\newglossaryentry{morion_Rüstung}
{
    name={Morion},
    description={\textbf{RsBeine}: 0 \textbf{RsLArm}: 0 \textbf{RsRArm}: 0 \textbf{RsBauch}: 0 \textbf{RsBrust}: 0 \textbf{RsKopf}: 3\newline Helm.}
}


\newglossaryentry{schaller_Rüstung}
{
    name={Schaller},
    description={\textbf{RsBeine}: 0 \textbf{RsLArm}: 0 \textbf{RsRArm}: 0 \textbf{RsBauch}: 0 \textbf{RsBrust}: 0 \textbf{RsKopf}: 3\newline Helm. Wird üblicherweise mit einem Bart ergänzt.}
}


\newglossaryentry{stechhelm_Rüstung}
{
    name={Stechhelm},
    description={\textbf{RsBeine}: 0 \textbf{RsLArm}: 0 \textbf{RsRArm}: 0 \textbf{RsBauch}: 0 \textbf{RsBrust}: 0 \textbf{RsKopf}: 5\newline Helm (Teil der Gestechrüstung).}
}


\newglossaryentry{sturmhaube_Rüstung}
{
    name={Sturmhaube},
    description={\textbf{RsBeine}: 0 \textbf{RsLArm}: 0 \textbf{RsRArm}: 0 \textbf{RsBauch}: 0 \textbf{RsBrust}: 0 \textbf{RsKopf}: 3\newline Helm.}
}


\newglossaryentry{topfhelm_Rüstung}
{
    name={Topfhelm},
    description={\textbf{RsBeine}: 0 \textbf{RsLArm}: 0 \textbf{RsRArm}: 0 \textbf{RsBauch}: 0 \textbf{RsBrust}: 0 \textbf{RsKopf}: 4\newline Helm. Wird üblicherweise mit einer Kettenhaube ergänzt.}
}


\newglossaryentry{visierhelm_Rüstung}
{
    name={Visierhelm},
    description={\textbf{RsBeine}: 0 \textbf{RsLArm}: 0 \textbf{RsRArm}: 0 \textbf{RsBauch}: 0 \textbf{RsBrust}: 0 \textbf{RsKopf}: 5\newline Helm. Variante des Stechhelms für Fußkämpfer.}
}


\newglossaryentry{wattierteKappe_Rüstung}
{
    name={Wattierte Kappe},
    description={\textbf{RsBeine}: 0 \textbf{RsLArm}: 0 \textbf{RsRArm}: 0 \textbf{RsBauch}: 0 \textbf{RsBrust}: 0 \textbf{RsKopf}: 1\newline Helm. Wird üblicherweise unter ungefütterten Metallhelmen getragen.}
}


\newglossaryentry{kettenhaube_Rüstung}
{
    name={Kettenhaube},
    description={\textbf{RsBeine}: 0 \textbf{RsLArm}: 0 \textbf{RsRArm}: 0 \textbf{RsBauch}: 0 \textbf{RsBrust}: 0 \textbf{RsKopf}: 2\newline Helm. Wird üblicherweise unter einem Topfhelm oder Tellerhelm getragen.}
}


\newglossaryentry{lederhelm_Rüstung}
{
    name={Lederhelm},
    description={\textbf{RsBeine}: 0 \textbf{RsLArm}: 0 \textbf{RsRArm}: 0 \textbf{RsBauch}: 0 \textbf{RsBrust}: 0 \textbf{RsKopf}: 2\newline Helm.}
}


\newglossaryentry{tellerhelm_Rüstung}
{
    name={Tellerhelm},
    description={\textbf{RsBeine}: 0 \textbf{RsLArm}: 0 \textbf{RsRArm}: 0 \textbf{RsBauch}: 0 \textbf{RsBrust}: 0 \textbf{RsKopf}: 2\newline Helm.}
}


\newglossaryentry{streifenschurz_Rüstung}
{
    name={Streifenschurz},
    description={\textbf{RsBeine}: 1 \textbf{RsLArm}: 0 \textbf{RsRArm}: 0 \textbf{RsBauch}: 1 \textbf{RsBrust}: 0 \textbf{RsKopf}: 0\newline Rüstungsergänzung für die Leistengegend. Wird insbesondere kombiniert mit Kettenwesten, Bronzeharnischen oder Lederrüstungen.}
}


\newglossaryentry{kettenzeug_Rüstung}
{
    name={Kettenzeug},
    description={\textbf{RsBeine}: 1 \textbf{RsLArm}: 1 \textbf{RsRArm}: 1 \textbf{RsBauch}: 1 \textbf{RsBrust}: 1 \textbf{RsKopf}: 1\newline Rüstungsergänzung, bestehend aus Kettenbeinlingen, Kettenkragen, Kettenhandschuhen und einem Ketten- oder Stahlhelm.}
}


\newglossaryentry{plattenzeug_Rüstung}
{
    name={Plattenzeug},
    description={\textbf{RsBeine}: 2 \textbf{RsLArm}: 2 \textbf{RsRArm}: 2 \textbf{RsBauch}: 2 \textbf{RsBrust}: 2 \textbf{RsKopf}: 2\newline Rüstungsergänzung, bestehend aus Plattenschultern, Plattenarmen, Beintaschen, Panzerhandschuhen und einem Stahlhelm.}
}


\newglossaryentry{lederzeug_Rüstung}
{
    name={Lederzeug},
    description={\textbf{RsBeine}: 1 \textbf{RsLArm}: 1 \textbf{RsRArm}: 1 \textbf{RsBauch}: 1 \textbf{RsBrust}: 1 \textbf{RsKopf}: 1\newline Rüstungsergänzung, bestehend aus ledernen Arm- und Beinschienen, Streifenschurz und einem Lederhelm.}
}


\newglossaryentry{bronzezeug_Rüstung}
{
    name={Bronzezeug},
    description={\textbf{RsBeine}: 1 \textbf{RsLArm}: 1 \textbf{RsRArm}: 1 \textbf{RsBauch}: 1 \textbf{RsBrust}: 1 \textbf{RsKopf}: 1\newline Rüstungsergänzung, bestehend aus bronzenen Arm- und Beinschienen, Streifenschurz und einem Bronzehelm.}
}


\newglossaryentry{brustschalen_Rüstung}
{
    name={Brustschalen},
    description={\textbf{RsBeine}: 1 \textbf{RsLArm}: 1 \textbf{RsRArm}: 1 \textbf{RsBauch}: 1 \textbf{RsBrust}: 1 \textbf{RsKopf}: 1\newline }
}


\newglossaryentry{bronzeharnisch_Rüstung}
{
    name={Bronzeharnisch},
    description={\textbf{RsBeine}: 0 \textbf{RsLArm}: 0 \textbf{RsRArm}: 0 \textbf{RsBauch}: 2 \textbf{RsBrust}: 3 \textbf{RsKopf}: 0\newline Wird üblicherweise mit Bronzezeug ergänzt.}
}


\newglossaryentry{bronzeharnisch(ZRS)_Rüstung}
{
    name={Bronzeharnisch (ZRS)},
    description={\textbf{RsBeine}: 0 \textbf{RsLArm}: 0 \textbf{RsRArm}: 0 \textbf{RsBauch}: 2 \textbf{RsBrust}: 3 \textbf{RsKopf}: 0\newline Wird üblicherweise mit Streifenschurz ergänzt.}
}


\newglossaryentry{kuslikerLamellar_Rüstung}
{
    name={Kusliker Lamellar},
    description={\textbf{RsBeine}: 1 \textbf{RsLArm}: 1 \textbf{RsRArm}: 1 \textbf{RsBauch}: 3 \textbf{RsBrust}: 3 \textbf{RsKopf}: 0\newline Wird üblicherweise mit Bronzezeug ergänzt.}
}


\newglossaryentry{kuslikerLamellar(ZRS)_Rüstung}
{
    name={Kusliker Lamellar (ZRS)},
    description={\textbf{RsBeine}: 1 \textbf{RsLArm}: 1 \textbf{RsRArm}: 1 \textbf{RsBauch}: 3 \textbf{RsBrust}: 3 \textbf{RsKopf}: 0\newline Wird üblicherweise mit Bronzehelm, -Arm und -Beinschienen ergänzt.}
}


\newglossaryentry{leichtePlatte_Rüstung}
{
    name={Leichte Platte},
    description={\textbf{RsBeine}: 2 \textbf{RsLArm}: 0 \textbf{RsRArm}: 0 \textbf{RsBauch}: 4 \textbf{RsBrust}: 4 \textbf{RsKopf}: 0\newline Wird üblicherweise mit Kettenzeug oder Plattenzeug ergänzt.}
}


\newglossaryentry{krötenhaut_Rüstung}
{
    name={Krötenhaut},
    description={\textbf{RsBeine}: 0 \textbf{RsLArm}: 1 \textbf{RsRArm}: 1 \textbf{RsBauch}: 1 \textbf{RsBrust}: 2 \textbf{RsKopf}: 0\newline Wird üblicherweise mit Lederzeug ergänzt.}
}


\newglossaryentry{wattiertesUnterzeug(ZRS)_Rüstung}
{
    name={Wattiertes Unterzeug (ZRS)},
    description={\textbf{RsBeine}: 1 \textbf{RsLArm}: 1 \textbf{RsRArm}: 1 \textbf{RsBauch}: 1 \textbf{RsBrust}: 1 \textbf{RsKopf}: 0\newline Wird üblicherweise unter Metallrüstungen gegen Abschürfungen getragen.}
}


\newglossaryentry{gambeson(ZRS)_Rüstung}
{
    name={Gambeson (ZRS)},
    description={\textbf{RsBeine}: 1 \textbf{RsLArm}: 1 \textbf{RsRArm}: 1 \textbf{RsBauch}: 2 \textbf{RsBrust}: 2 \textbf{RsKopf}: 0\newline Kann unter vielen Torsorüstungen anstelle von wattierter Unterkleidung getragen werden, wenn diese entsprechend größer ausfallen.}
}


\newglossaryentry{fellumhang_Rüstung}
{
    name={Fellumhang},
    description={\textbf{RsBeine}: 1 \textbf{RsLArm}: 1 \textbf{RsRArm}: 1 \textbf{RsBauch}: 0 \textbf{RsBrust}: 1 \textbf{RsKopf}: 0\newline Rüstungsergänzung. Erhöht die Temperaturstufe um +1 (S. 35).}
}


\newglossaryentry{fuhrmannsmantel_Rüstung}
{
    name={Fuhrmannsmantel},
    description={\textbf{RsBeine}: 1 \textbf{RsLArm}: 1 \textbf{RsRArm}: 1 \textbf{RsBauch}: 0 \textbf{RsBrust}: 1 \textbf{RsKopf}: 0\newline Rüstungsergänzung. Erhöht die Temperaturstufe um +1 (S. 35).}
}


\newglossaryentry{wattiertesUnterzeug_Rüstung}
{
    name={Wattiertes Unterzeug},
    description={\textbf{RsBeine}: 1 \textbf{RsLArm}: 1 \textbf{RsRArm}: 1 \textbf{RsBauch}: 1 \textbf{RsBrust}: 1 \textbf{RsKopf}: 0\newline Wird üblicherweise unter Metallrüstungen gegen Abschürfungen getragen. Wirkt dann allerdings mit RS/BE 0.}
}


\newglossaryentry{gambeson_Rüstung}
{
    name={Gambeson},
    description={\textbf{RsBeine}: 1 \textbf{RsLArm}: 1 \textbf{RsRArm}: 1 \textbf{RsBauch}: 2 \textbf{RsBrust}: 2 \textbf{RsKopf}: 0\newline Kann unter vielen Torsorüstungen anstelle von wattierter Unterkleidung getragen werden, wenn diese entsprechend größer ausfallen. Wirkt dann allerdings mit RS/BE 0.}
}


\newglossaryentry{garetherPlatte_Rüstung}
{
    name={Garether Platte},
    description={\textbf{RsBeine}: 3 \textbf{RsLArm}: 3 \textbf{RsRArm}: 3 \textbf{RsBauch}: 3 \textbf{RsBrust}: 3 \textbf{RsKopf}: 3\newline Wird üblicherweise mit Plattenzeug ergänzt. Da dieses bereits teilweise in der Rüstung integriert ist, gibt es allerdings nur RS/BE 1.}
}


\newglossaryentry{kettenhemd(lang)(ZRS)_Rüstung}
{
    name={Kettenhemd (lang) (ZRS)},
    description={\textbf{RsBeine}: 2 \textbf{RsLArm}: 3 \textbf{RsRArm}: 3 \textbf{RsBauch}: 3 \textbf{RsBrust}: 3 \textbf{RsKopf}: 0\newline }
}


\newglossaryentry{kettenhemd(kurz)(ZRS)_Rüstung}
{
    name={Kettenhemd (kurz) (ZRS)},
    description={\textbf{RsBeine}: 1 \textbf{RsLArm}: 2 \textbf{RsRArm}: 2 \textbf{RsBauch}: 3 \textbf{RsBrust}: 3 \textbf{RsKopf}: 0\newline }
}


\newglossaryentry{kettenhemd(kurz)_Rüstung}
{
    name={Kettenhemd (kurz)},
    description={\textbf{RsBeine}: 1 \textbf{RsLArm}: 1 \textbf{RsRArm}: 1 \textbf{RsBauch}: 1 \textbf{RsBrust}: 1 \textbf{RsKopf}: 1\newline Wird üblicherweise mit Leder- oder Kettenzeug ergänzt.}
}


\newglossaryentry{kettenhemd(lang)_Rüstung}
{
    name={Kettenhemd (lang)},
    description={\textbf{RsBeine}: 2 \textbf{RsLArm}: 2 \textbf{RsRArm}: 2 \textbf{RsBauch}: 2 \textbf{RsBrust}: 2 \textbf{RsKopf}: 0\newline Wird üblicherweise mit Leder- oder Kettenzeug ergänzt.}
}


\newglossaryentry{schuppenpanzer(lang)_Rüstung}
{
    name={Schuppenpanzer (lang)},
    description={\textbf{RsBeine}: 4 \textbf{RsLArm}: 3 \textbf{RsRArm}: 3 \textbf{RsBauch}: 4 \textbf{RsBrust}: 4 \textbf{RsKopf}: 0\newline Benötigt keine wattierte Unterkleidung.}
}


\newglossaryentry{brustplatte(Leder)_Rüstung}
{
    name={Brustplatte (Leder)},
    description={\textbf{RsBeine}: 0 \textbf{RsLArm}: 0 \textbf{RsRArm}: 0 \textbf{RsBauch}: 1 \textbf{RsBrust}: 1 \textbf{RsKopf}: 0\newline }
}


\newglossaryentry{schuppenpanzer_Rüstung}
{
    name={Schuppenpanzer},
    description={\textbf{RsBeine}: 3 \textbf{RsLArm}: 3 \textbf{RsRArm}: 3 \textbf{RsBauch}: 4 \textbf{RsBrust}: 4 \textbf{RsKopf}: 0\newline Benötigt keine wattierte Unterkleidung.}
}


\newglossaryentry{brigantina_Rüstung}
{
    name={Brigantina},
    description={\textbf{RsBeine}: 0 \textbf{RsLArm}: 1 \textbf{RsRArm}: 1 \textbf{RsBauch}: 3 \textbf{RsBrust}: 3 \textbf{RsKopf}: 0\newline Stoff- oder Lederrüstung, in die - von außen nicht sichtbar - Metallplättchen genäht wurden.}
}


\newglossaryentry{brabakerRingmantel_Rüstung}
{
    name={Brabaker Ringmantel},
    description={\textbf{RsBeine}: 2 \textbf{RsLArm}: 2 \textbf{RsRArm}: 2 \textbf{RsBauch}: 2 \textbf{RsBrust}: 2 \textbf{RsKopf}: 0\newline Benötigt keine wattierte Unterkleidung.}
}


\newglossaryentry{anaurak_Rüstung}
{
    name={Anaurak},
    description={\textbf{RsBeine}: 2 \textbf{RsLArm}: 2 \textbf{RsRArm}: 2 \textbf{RsBauch}: 2 \textbf{RsBrust}: 2 \textbf{RsKopf}: 1\newline Erhöht die Temperaturstufe um +2 (S. 35). Kann nicht ergänzt werden.}
}



\newglossaryentry{brabakbengel_Waffe}
{
    name={Brabakbengel},
    description={\textbf{TP}: 2W6+2 \textbf{RW}: 1 \textbf{WM}: -1 \textbf{Härte}: 8        \textbf{Fertigkeit}: Hiebwaffen \textbf{Talent}: Einhandhiebwaffen \textbf{{Eigenschaften}}: Kopflastig, Rüstungsbrechend}
}



\newglossaryentry{haumesser_Waffe}
{
    name={Haumesser},
    description={\textbf{TP}: 2W6+1 \textbf{RW}: 1 \textbf{WM}: -1 \textbf{Härte}: 9        \textbf{Fertigkeit}: Hiebwaffen \textbf{Talent}: Einhandhiebwaffen \textbf{{Eigenschaften}}: }
}



\newglossaryentry{keule_Waffe}
{
    name={Keule},
    description={\textbf{TP}: 2W6+0 \textbf{RW}: 1 \textbf{WM}: -1 \textbf{Härte}: 8        \textbf{Fertigkeit}: Hiebwaffen \textbf{Talent}: Einhandhiebwaffen \textbf{{Eigenschaften}}: Stumpf, Kopflastig}
}



\newglossaryentry{knüppel_Waffe}
{
    name={Knüppel},
    description={\textbf{TP}: 1W6+2 \textbf{RW}: 1 \textbf{WM}: -1 \textbf{Härte}: 5        \textbf{Fertigkeit}: Hiebwaffen \textbf{Talent}: Einhandhiebwaffen \textbf{{Eigenschaften}}: Stumpf, Kopflastig}
}



\newglossaryentry{lindwurmschläger_Waffe}
{
    name={Lindwurmschläger},
    description={\textbf{TP}: 2W6+2 \textbf{RW}: 1 \textbf{WM}: 0 \textbf{Härte}: 10        \textbf{Fertigkeit}: Hiebwaffen \textbf{Talent}: Einhandhiebwaffen \textbf{{Eigenschaften}}: }
}



\newglossaryentry{morgenstern_Waffe}
{
    name={Morgenstern},
    description={\textbf{TP}: 3W6+0 \textbf{RW}: 1 \textbf{WM}: -1 \textbf{Härte}: 9        \textbf{Fertigkeit}: Hiebwaffen \textbf{Talent}: Einhandhiebwaffen \textbf{{Eigenschaften}}: Kopflastig, Unberechenbar}
}



\newglossaryentry{ochsenherde_Waffe}
{
    name={Ochsenherde},
    description={\textbf{TP}: 4W6+1 \textbf{RW}: 1 \textbf{WM}: -3 \textbf{Härte}: 8        \textbf{Fertigkeit}: Hiebwaffen \textbf{Talent}: Einhandhiebwaffen \textbf{{Eigenschaften}}: Kopflastig, Schwer (8), Unberechenbar}
}



\newglossaryentry{ogerschelle_Waffe}
{
    name={Ogerschelle},
    description={\textbf{TP}: 3W6+2 \textbf{RW}: 1 \textbf{WM}: -2 \textbf{Härte}: 9        \textbf{Fertigkeit}: Hiebwaffen \textbf{Talent}: Einhandhiebwaffen \textbf{{Eigenschaften}}: Kopflastig, Schwer (4), Unberechenbar}
}



\newglossaryentry{rabenschnabel_Waffe}
{
    name={Rabenschnabel},
    description={\textbf{TP}: 2W6+1 \textbf{RW}: 1 \textbf{WM}: 0 \textbf{Härte}: 9        \textbf{Fertigkeit}: Hiebwaffen \textbf{Talent}: Einhandhiebwaffen \textbf{{Eigenschaften}}: Rüstungsbrechend}
}



\newglossaryentry{skraja_Waffe}
{
    name={Skraja},
    description={\textbf{TP}: 2W6+1 \textbf{RW}: 0 \textbf{WM}: 0 \textbf{Härte}: 10        \textbf{Fertigkeit}: Hiebwaffen \textbf{Talent}: Einhandhiebwaffen \textbf{{Eigenschaften}}: Rüstungsbrechend}
}



\newglossaryentry{sonnenszepter_Waffe}
{
    name={Sonnenszepter},
    description={\textbf{TP}: 2W6+0 \textbf{RW}: 1 \textbf{WM}: -1 \textbf{Härte}: 10        \textbf{Fertigkeit}: Hiebwaffen \textbf{Talent}: Einhandhiebwaffen \textbf{{Eigenschaften}}: }
}



\newglossaryentry{streitaxt_Waffe}
{
    name={Streitaxt},
    description={\textbf{TP}: 2W6+1 \textbf{RW}: 1 \textbf{WM}: 0 \textbf{Härte}: 8        \textbf{Fertigkeit}: Hiebwaffen \textbf{Talent}: Einhandhiebwaffen \textbf{{Eigenschaften}}: Kopflastig}
}



\newglossaryentry{streitkolben_Waffe}
{
    name={Streitkolben},
    description={\textbf{TP}: 2W6+0 \textbf{RW}: 1 \textbf{WM}: 0 \textbf{Härte}: 8        \textbf{Fertigkeit}: Hiebwaffen \textbf{Talent}: Einhandhiebwaffen \textbf{{Eigenschaften}}: Stumpf, Kopflastig, Rüstungsbrechend}
}



\newglossaryentry{barbarenstreitaxt_Waffe}
{
    name={Barbarenstreitaxt},
    description={\textbf{TP}: 4W6+1 \textbf{RW}: 1 \textbf{WM}: -2 \textbf{Härte}: 7        \textbf{Fertigkeit}: Hiebwaffen \textbf{Talent}: Zweihandhiebwaffen \textbf{{Eigenschaften}}: Kopflastig, Schwer (4), Zweihändig}
}



\newglossaryentry{echsischeAxt_Waffe}
{
    name={Echsische Axt},
    description={\textbf{TP}: 2W6+2 \textbf{RW}: 2 \textbf{WM}: 0 \textbf{Härte}: 8        \textbf{Fertigkeit}: Hiebwaffen \textbf{Talent}: Zweihandhiebwaffen \textbf{{Eigenschaften}}: Rüstungsbrechend, Wendig, Zweihändig}
}



\newglossaryentry{felsspalter_Waffe}
{
    name={Felsspalter},
    description={\textbf{TP}: 3W6+2 \textbf{RW}: 1 \textbf{WM}: -1 \textbf{Härte}: 11        \textbf{Fertigkeit}: Hiebwaffen \textbf{Talent}: Zweihandhiebwaffen \textbf{{Eigenschaften}}: Kopflastig, Schwer (4), Zweihändig}
}



\newglossaryentry{kriegsflegel_Waffe}
{
    name={Kriegsflegel},
    description={\textbf{TP}: 3W6+0 \textbf{RW}: 2 \textbf{WM}: -1 \textbf{Härte}: 6        \textbf{Fertigkeit}: Hiebwaffen \textbf{Talent}: Zweihandhiebwaffen \textbf{{Eigenschaften}}: Kopflastig, Zweihändig}
}



\newglossaryentry{kriegshammer_Waffe}
{
    name={Kriegshammer},
    description={\textbf{TP}: 3W6+1 \textbf{RW}: 1 \textbf{WM}: -1 \textbf{Härte}: 9        \textbf{Fertigkeit}: Hiebwaffen \textbf{Talent}: Zweihandhiebwaffen \textbf{{Eigenschaften}}: Kopflastig, Schwer (4), Rüstungsbrechend, Stumpf, Zweihändig}
}



\newglossaryentry{langaxt_Waffe}
{
    name={Langaxt},
    description={\textbf{TP}: 3W6+1 \textbf{RW}: 1 \textbf{WM}: -1 \textbf{Härte}: 9        \textbf{Fertigkeit}: Hiebwaffen \textbf{Talent}: Zweihandhiebwaffen \textbf{{Eigenschaften}}: Kopflastig, Schwer (4), Rüstungsbrechend, Zweihändig}
}



\newglossaryentry{orknase_Waffe}
{
    name={Orknase},
    description={\textbf{TP}: 3W6+0 \textbf{RW}: 1 \textbf{WM}: 0 \textbf{Härte}: 8        \textbf{Fertigkeit}: Hiebwaffen \textbf{Talent}: Zweihandhiebwaffen \textbf{{Eigenschaften}}: Kopflastig, Schwer (4), Zweihändig}
}



\newglossaryentry{warunkerHammer_Waffe}
{
    name={Warunker Hammer},
    description={\textbf{TP}: 2W6+2 \textbf{RW}: 2 \textbf{WM}: 0 \textbf{Härte}: 8        \textbf{Fertigkeit}: Hiebwaffen \textbf{Talent}: Zweihandhiebwaffen \textbf{{Eigenschaften}}: Kopflastig, Rüstungsbrechend, Zweihändig}
}



\newglossaryentry{amazonensäbel_Waffe}
{
    name={Amazonensäbel},
    description={\textbf{TP}: 2W6+1 \textbf{RW}: 1 \textbf{WM}: 0 \textbf{Härte}: 10        \textbf{Fertigkeit}: Klingenwaffen \textbf{Talent}: Einhandklingenwaffen \textbf{{Eigenschaften}}: Wendig}
}



\newglossaryentry{barbarenschwert_Waffe}
{
    name={Barbarenschwert},
    description={\textbf{TP}: 3W6+0 \textbf{RW}: 1 \textbf{WM}: -1 \textbf{Härte}: 6        \textbf{Fertigkeit}: Klingenwaffen \textbf{Talent}: Einhandklingenwaffen \textbf{{Eigenschaften}}: Kopflastig, Schwer (4)}
}



\newglossaryentry{breitschwert_Waffe}
{
    name={Breitschwert},
    description={\textbf{TP}: 2W6+2 \textbf{RW}: 1 \textbf{WM}: -1 \textbf{Härte}: 11        \textbf{Fertigkeit}: Klingenwaffen \textbf{Talent}: Einhandklingenwaffen \textbf{{Eigenschaften}}: Kopflastig}
}



\newglossaryentry{degen_Waffe}
{
    name={Degen},
    description={\textbf{TP}: 2W6+0 \textbf{RW}: 1 \textbf{WM}: 1 \textbf{Härte}: 6        \textbf{Fertigkeit}: Klingenwaffen \textbf{Talent}: Einhandklingenwaffen \textbf{{Eigenschaften}}: Wendig}
}



\newglossaryentry{entermesser_Waffe}
{
    name={Entermesser},
    description={\textbf{TP}: 2W6+1 \textbf{RW}: 1 \textbf{WM}: 0 \textbf{Härte}: 10        \textbf{Fertigkeit}: Klingenwaffen \textbf{Talent}: Einhandklingenwaffen \textbf{{Eigenschaften}}: }
}



\newglossaryentry{khunchomer_Waffe}
{
    name={Khunchomer},
    description={\textbf{TP}: 2W6+1 \textbf{RW}: 1 \textbf{WM}: 0 \textbf{Härte}: 9        \textbf{Fertigkeit}: Klingenwaffen \textbf{Talent}: Einhandklingenwaffen \textbf{{Eigenschaften}}: Kopflastig}
}



\newglossaryentry{kurzschwert_Waffe}
{
    name={Kurzschwert},
    description={\textbf{TP}: 2W6+0 \textbf{RW}: 0 \textbf{WM}: 0 \textbf{Härte}: 10        \textbf{Fertigkeit}: Klingenwaffen \textbf{Talent}: Einhandklingenwaffen \textbf{{Eigenschaften}}: Wendig}
}



\newglossaryentry{nachtwind_Waffe}
{
    name={Nachtwind},
    description={\textbf{TP}: 2W6+2 \textbf{RW}: 1 \textbf{WM}: 0 \textbf{Härte}: 8        \textbf{Fertigkeit}: Klingenwaffen \textbf{Talent}: Einhandklingenwaffen \textbf{{Eigenschaften}}: Schwer (4), Wendig}
}



\newglossaryentry{panzerstecher_Waffe}
{
    name={Panzerstecher},
    description={\textbf{TP}: 2W6+2 \textbf{RW}: 1 \textbf{WM}: -1 \textbf{Härte}: 11        \textbf{Fertigkeit}: Klingenwaffen \textbf{Talent}: Einhandklingenwaffen \textbf{{Eigenschaften}}: Rüstungsbrechend}
}



\newglossaryentry{rapier_Waffe}
{
    name={Rapier},
    description={\textbf{TP}: 2W6+1 \textbf{RW}: 1 \textbf{WM}: 1 \textbf{Härte}: 7        \textbf{Fertigkeit}: Klingenwaffen \textbf{Talent}: Einhandklingenwaffen \textbf{{Eigenschaften}}: }
}



\newglossaryentry{säbel_Waffe}
{
    name={Säbel},
    description={\textbf{TP}: 2W6+2 \textbf{RW}: 1 \textbf{WM}: 0 \textbf{Härte}: 10        \textbf{Fertigkeit}: Klingenwaffen \textbf{Talent}: Einhandklingenwaffen \textbf{{Eigenschaften}}: }
}



\newglossaryentry{schwert_Waffe}
{
    name={Schwert},
    description={\textbf{TP}: 2W6+2 \textbf{RW}: 1 \textbf{WM}: 0 \textbf{Härte}: 10        \textbf{Fertigkeit}: Klingenwaffen \textbf{Talent}: Einhandklingenwaffen \textbf{{Eigenschaften}}: }
}



\newglossaryentry{sklaventod_Waffe}
{
    name={Sklaventod},
    description={\textbf{TP}: 2W6+3 \textbf{RW}: 1 \textbf{WM}: -1 \textbf{Härte}: 8        \textbf{Fertigkeit}: Klingenwaffen \textbf{Talent}: Einhandklingenwaffen \textbf{{Eigenschaften}}: Kopflastig}
}



\newglossaryentry{turnierschwert_Waffe}
{
    name={Turnierschwert},
    description={\textbf{TP}: 1W6+1 \textbf{RW}: 1 \textbf{WM}: 0 \textbf{Härte}: 7        \textbf{Fertigkeit}: Klingenwaffen \textbf{Talent}: Einhandklingenwaffen \textbf{{Eigenschaften}}: Stumpf}
}



\newglossaryentry{wolfsmesser_Waffe}
{
    name={Wolfsmesser},
    description={\textbf{TP}: 2W6+1 \textbf{RW}: 1 \textbf{WM}: 0 \textbf{Härte}: 10        \textbf{Fertigkeit}: Klingenwaffen \textbf{Talent}: Einhandklingenwaffen \textbf{{Eigenschaften}}: Wendig}
}



\newglossaryentry{andergaster_Waffe}
{
    name={Andergaster},
    description={\textbf{TP}: 3W6+3 \textbf{RW}: 2 \textbf{WM}: -2 \textbf{Härte}: 9        \textbf{Fertigkeit}: Klingenwaffen \textbf{Talent}: Zweihandklingenwaffen \textbf{{Eigenschaften}}: Kopflastig, Schwer (4), Zweihändig}
}



\newglossaryentry{anderthalbhänder_Waffe}
{
    name={Anderthalbhänder},
    description={\textbf{TP}: 2W6+2 \textbf{RW}: 2 \textbf{WM}: 1 \textbf{Härte}: 10        \textbf{Fertigkeit}: Klingenwaffen \textbf{Talent}: Zweihandklingenwaffen \textbf{{Eigenschaften}}: Zweihändig}
}



\newglossaryentry{boronssichel_Waffe}
{
    name={Boronssichel},
    description={\textbf{TP}: 3W6+5 \textbf{RW}: 2 \textbf{WM}: -2 \textbf{Härte}: 8        \textbf{Fertigkeit}: Klingenwaffen \textbf{Talent}: Zweihandklingenwaffen \textbf{{Eigenschaften}}: Schwer (4), Zweihändig}
}



\newglossaryentry{doppelkhunchomer_Waffe}
{
    name={Doppelkhunchomer},
    description={\textbf{TP}: 2W6+3 \textbf{RW}: 2 \textbf{WM}: 0 \textbf{Härte}: 8        \textbf{Fertigkeit}: Klingenwaffen \textbf{Talent}: Zweihandklingenwaffen \textbf{{Eigenschaften}}: Kopflastig, Schwer (4), Zweihändig}
}



\newglossaryentry{großerSklaventod_Waffe}
{
    name={Großer Sklaventod},
    description={\textbf{TP}: 3W6+1 \textbf{RW}: 2 \textbf{WM}: -1 \textbf{Härte}: 8        \textbf{Fertigkeit}: Klingenwaffen \textbf{Talent}: Zweihandklingenwaffen \textbf{{Eigenschaften}}: Kopflastig, Schwer (4), Zweihändig}
}



\newglossaryentry{rondrakamm_Waffe}
{
    name={Rondrakamm},
    description={\textbf{TP}: 2W6+4 \textbf{RW}: 2 \textbf{WM}: 0 \textbf{Härte}: 10        \textbf{Fertigkeit}: Klingenwaffen \textbf{Talent}: Zweihandklingenwaffen \textbf{{Eigenschaften}}: Zweihändig}
}



\newglossaryentry{tuzakmesser_Waffe}
{
    name={Tuzakmesser},
    description={\textbf{TP}: 2W6+3 \textbf{RW}: 2 \textbf{WM}: 0 \textbf{Härte}: 10        \textbf{Fertigkeit}: Klingenwaffen \textbf{Talent}: Zweihandklingenwaffen \textbf{{Eigenschaften}}: Wendig, Zweihändig}
}



\newglossaryentry{zweihänder_Waffe}
{
    name={Zweihänder},
    description={\textbf{TP}: 3W6+1 \textbf{RW}: 2 \textbf{WM}: 0 \textbf{Härte}: 8        \textbf{Fertigkeit}: Klingenwaffen \textbf{Talent}: Zweihandklingenwaffen \textbf{{Eigenschaften}}: Schwer (4), Zweihändig}
}



\newglossaryentry{dschadra(Infanteriewaffen)_Waffe}
{
    name={Dschadra (Infanteriewaffen)},
    description={\textbf{TP}: 2W6+3 \textbf{RW}: 2 \textbf{WM}: -1 \textbf{Härte}: 6        \textbf{Fertigkeit}: Stangenwaffen \textbf{Talent}: Infanteriewaffen und Speere \textbf{{Eigenschaften}}: }
}



\newglossaryentry{efferdbart(Infanteriewaffen)_Waffe}
{
    name={Efferdbart (Infanteriewaffen)},
    description={\textbf{TP}: 2W6+0 \textbf{RW}: 2 \textbf{WM}: 0 \textbf{Härte}: 6        \textbf{Fertigkeit}: Stangenwaffen \textbf{Talent}: Infanteriewaffen und Speere \textbf{{Eigenschaften}}: }
}



\newglossaryentry{hellebarde_Waffe}
{
    name={Hellebarde},
    description={\textbf{TP}: 2W6+2 \textbf{RW}: 2 \textbf{WM}: 0 \textbf{Härte}: 7        \textbf{Fertigkeit}: Stangenwaffen \textbf{Talent}: Infanteriewaffen und Speere \textbf{{Eigenschaften}}: Kopflastig, Rüstungsbrechend, Zweihändig}
}



\newglossaryentry{holzspeer(Infanteriewaffen)_Waffe}
{
    name={Holzspeer (Infanteriewaffen)},
    description={\textbf{TP}: 2W6+-1 \textbf{RW}: 2 \textbf{WM}: -1 \textbf{Härte}: 4        \textbf{Fertigkeit}: Stangenwaffen \textbf{Talent}: Infanteriewaffen und Speere \textbf{{Eigenschaften}}: Wendig}
}



\newglossaryentry{jagdspieß_Waffe}
{
    name={Jagdspieß},
    description={\textbf{TP}: 3W6+0 \textbf{RW}: 2 \textbf{WM}: 0 \textbf{Härte}: 7        \textbf{Fertigkeit}: Stangenwaffen \textbf{Talent}: Infanteriewaffen und Speere \textbf{{Eigenschaften}}: Wendig, Zweihändig}
}



\newglossaryentry{kampfstab_Waffe}
{
    name={Kampfstab},
    description={\textbf{TP}: 1W6+1 \textbf{RW}: 2 \textbf{WM}: 1 \textbf{Härte}: 4        \textbf{Fertigkeit}: Stangenwaffen \textbf{Talent}: Infanteriewaffen und Speere \textbf{{Eigenschaften}}: Wendig, Stumpf, Zweihändig}
}



\newglossaryentry{korspieß_Waffe}
{
    name={Korspieß},
    description={\textbf{TP}: 2W6+3 \textbf{RW}: 2 \textbf{WM}: 0 \textbf{Härte}: 8        \textbf{Fertigkeit}: Stangenwaffen \textbf{Talent}: Infanteriewaffen und Speere \textbf{{Eigenschaften}}: Kopflastig, Zweihändig}
}



\newglossaryentry{magierstab_Waffe}
{
    name={Magierstab},
    description={\textbf{TP}: 1W6+1 \textbf{RW}: 2 \textbf{WM}: 0 \textbf{Härte}: 1        \textbf{Fertigkeit}: Stangenwaffen \textbf{Talent}: Infanteriewaffen und Speere \textbf{{Eigenschaften}}: Stumpf, Unzerstörbar, Zweihändig}
}



\newglossaryentry{pailos_Waffe}
{
    name={Pailos},
    description={\textbf{TP}: 3W6+3 \textbf{RW}: 2 \textbf{WM}: -2 \textbf{Härte}: 9        \textbf{Fertigkeit}: Stangenwaffen \textbf{Talent}: Infanteriewaffen und Speere \textbf{{Eigenschaften}}: Kopflastig, Schwer (4), Zweihändig}
}



\newglossaryentry{partisane_Waffe}
{
    name={Partisane},
    description={\textbf{TP}: 3W6+1 \textbf{RW}: 2 \textbf{WM}: -1 \textbf{Härte}: 7        \textbf{Fertigkeit}: Stangenwaffen \textbf{Talent}: Infanteriewaffen und Speere \textbf{{Eigenschaften}}: Kopflastig, Schwer (4), Zweihändig}
}



\newglossaryentry{schnitter_Waffe}
{
    name={Schnitter},
    description={\textbf{TP}: 2W6+3 \textbf{RW}: 2 \textbf{WM}: 0 \textbf{Härte}: 8        \textbf{Fertigkeit}: Stangenwaffen \textbf{Talent}: Infanteriewaffen und Speere \textbf{{Eigenschaften}}: Wendig, Zweihändig}
}



\newglossaryentry{speer(Infanteriewaffen)_Waffe}
{
    name={Speer (Infanteriewaffen)},
    description={\textbf{TP}: 2W6+1 \textbf{RW}: 2 \textbf{WM}: 0 \textbf{Härte}: 6        \textbf{Fertigkeit}: Stangenwaffen \textbf{Talent}: Infanteriewaffen und Speere \textbf{{Eigenschaften}}: }
}



\newglossaryentry{stoßspeer_Waffe}
{
    name={Stoßspeer},
    description={\textbf{TP}: 3W6+3 \textbf{RW}: 2 \textbf{WM}: -1 \textbf{Härte}: 7        \textbf{Fertigkeit}: Stangenwaffen \textbf{Talent}: Infanteriewaffen und Speere \textbf{{Eigenschaften}}: Zweihändig}
}



\newglossaryentry{sturmsense_Waffe}
{
    name={Sturmsense},
    description={\textbf{TP}: 2W6+3 \textbf{RW}: 2 \textbf{WM}: -2 \textbf{Härte}: 6        \textbf{Fertigkeit}: Stangenwaffen \textbf{Talent}: Infanteriewaffen und Speere \textbf{{Eigenschaften}}: Zweihändig}
}



\newglossaryentry{wurmspieß_Waffe}
{
    name={Wurmspieß},
    description={\textbf{TP}: 3W6+2 \textbf{RW}: 2 \textbf{WM}: -1 \textbf{Härte}: 8        \textbf{Fertigkeit}: Stangenwaffen \textbf{Talent}: Infanteriewaffen und Speere \textbf{{Eigenschaften}}: Rüstungsbrechend, Zweihändig}
}



\newglossaryentry{zweililien_Waffe}
{
    name={Zweililien},
    description={\textbf{TP}: 2W6+2 \textbf{RW}: 1 \textbf{WM}: 1 \textbf{Härte}: 5        \textbf{Fertigkeit}: Stangenwaffen \textbf{Talent}: Infanteriewaffen und Speere \textbf{{Eigenschaften}}: Wendig, Zweihändig}
}



\newglossaryentry{dschadra(Lanzenreiten)_Waffe}
{
    name={Dschadra (Lanzenreiten)},
    description={\textbf{TP}: 2W6+1 \textbf{RW}: 2 \textbf{WM}: -1 \textbf{Härte}: 6        \textbf{Fertigkeit}: Stangenwaffen \textbf{Talent}: Lanzenreiten \textbf{{Eigenschaften}}: }
}



\newglossaryentry{kriegslanze_Waffe}
{
    name={Kriegslanze},
    description={\textbf{TP}: 3W6+1 \textbf{RW}: 2 \textbf{WM}: -1 \textbf{Härte}: 5        \textbf{Fertigkeit}: Stangenwaffen \textbf{Talent}: Lanzenreiten \textbf{{Eigenschaften}}: }
}



\newglossaryentry{basiliskenzunge_Waffe}
{
    name={Basiliskenzunge},
    description={\textbf{TP}: 1W6+1 \textbf{RW}: 0 \textbf{WM}: 0 \textbf{Härte}: 7        \textbf{Fertigkeit}: Handgemenge \textbf{Talent}: Handgemengewaffen \textbf{{Eigenschaften}}: }
}



\newglossaryentry{bock(Handgemengewaffen)_Waffe}
{
    name={Bock (Handgemengewaffen)},
    description={\textbf{TP}: 1W6+2 \textbf{RW}: 0 \textbf{WM}: 0 \textbf{Härte}: 11        \textbf{Fertigkeit}: Handgemenge \textbf{Talent}: Handgemengewaffen \textbf{{Eigenschaften}}: Parierwaffe, Schild}
}



\newglossaryentry{borndorn(Handgemengewaffen)_Waffe}
{
    name={Borndorn (Handgemengewaffen)},
    description={\textbf{TP}: 1W6+2 \textbf{RW}: 0 \textbf{WM}: 0 \textbf{Härte}: 10        \textbf{Fertigkeit}: Handgemenge \textbf{Talent}: Handgemengewaffen \textbf{{Eigenschaften}}: }
}



\newglossaryentry{buckler(Handgemengewaffen)_Waffe}
{
    name={Buckler (Handgemengewaffen)},
    description={\textbf{TP}: 1W6+1 \textbf{RW}: 0 \textbf{WM}: 0 \textbf{Härte}: 11        \textbf{Fertigkeit}: Handgemenge \textbf{Talent}: Handgemengewaffen \textbf{{Eigenschaften}}: Parierwaffe, Schild, Stumpf}
}



\newglossaryentry{dolch(Handgemengewaffen)_Waffe}
{
    name={Dolch (Handgemengewaffen)},
    description={\textbf{TP}: 1W6+1 \textbf{RW}: 0 \textbf{WM}: 0 \textbf{Härte}: 9        \textbf{Fertigkeit}: Handgemenge \textbf{Talent}: Handgemengewaffen \textbf{{Eigenschaften}}: }
}



\newglossaryentry{drachenzahn_Waffe}
{
    name={Drachenzahn},
    description={\textbf{TP}: 1W6+2 \textbf{RW}: 0 \textbf{WM}: 0 \textbf{Härte}: 11        \textbf{Fertigkeit}: Handgemenge \textbf{Talent}: Handgemengewaffen \textbf{{Eigenschaften}}: }
}



\newglossaryentry{hakendolch_Waffe}
{
    name={Hakendolch},
    description={\textbf{TP}: 1W6+1 \textbf{RW}: 0 \textbf{WM}: 0 \textbf{Härte}: 12        \textbf{Fertigkeit}: Handgemenge \textbf{Talent}: Handgemengewaffen \textbf{{Eigenschaften}}: Parierwaffe}
}



\newglossaryentry{kriegsfächer_Waffe}
{
    name={Kriegsfächer},
    description={\textbf{TP}: 1W6+2 \textbf{RW}: 0 \textbf{WM}: 0 \textbf{Härte}: 7        \textbf{Fertigkeit}: Handgemenge \textbf{Talent}: Handgemengewaffen \textbf{{Eigenschaften}}: Parierwaffe}
}



\newglossaryentry{langdolch_Waffe}
{
    name={Langdolch},
    description={\textbf{TP}: 1W6+2 \textbf{RW}: 0 \textbf{WM}: 0 \textbf{Härte}: 10        \textbf{Fertigkeit}: Handgemenge \textbf{Talent}: Handgemengewaffen \textbf{{Eigenschaften}}: Parierwaffe}
}



\newglossaryentry{linkhand_Waffe}
{
    name={Linkhand},
    description={\textbf{TP}: 1W6+1 \textbf{RW}: 0 \textbf{WM}: 0 \textbf{Härte}: 10        \textbf{Fertigkeit}: Handgemenge \textbf{Talent}: Handgemengewaffen \textbf{{Eigenschaften}}: Parierwaffe, Wendig}
}



\newglossaryentry{mengbilar_Waffe}
{
    name={Mengbilar},
    description={\textbf{TP}: 1W6+1 \textbf{RW}: 0 \textbf{WM}: -1 \textbf{Härte}: 5        \textbf{Fertigkeit}: Handgemenge \textbf{Talent}: Handgemengewaffen \textbf{{Eigenschaften}}: Zerbrechlich}
}



\newglossaryentry{messer_Waffe}
{
    name={Messer},
    description={\textbf{TP}: 1W6+1 \textbf{RW}: 0 \textbf{WM}: -1 \textbf{Härte}: 6        \textbf{Fertigkeit}: Handgemenge \textbf{Talent}: Handgemengewaffen \textbf{{Eigenschaften}}: }
}



\newglossaryentry{panzerarm_Waffe}
{
    name={Panzerarm},
    description={\textbf{TP}: 1W6+2 \textbf{RW}: 0 \textbf{WM}: -1 \textbf{Härte}: 12        \textbf{Fertigkeit}: Handgemenge \textbf{Talent}: Handgemengewaffen \textbf{{Eigenschaften}}: }
}



\newglossaryentry{scheibendolch_Waffe}
{
    name={Scheibendolch},
    description={\textbf{TP}: 1W6+1 \textbf{RW}: 0 \textbf{WM}: 0 \textbf{Härte}: 9        \textbf{Fertigkeit}: Handgemenge \textbf{Talent}: Handgemengewaffen \textbf{{Eigenschaften}}: Rüstungsbrechend}
}



\newglossaryentry{schlagring_Waffe}
{
    name={Schlagring},
    description={\textbf{TP}: 1W6+1 \textbf{RW}: 0 \textbf{WM}: 0 \textbf{Härte}: 9        \textbf{Fertigkeit}: Handgemenge \textbf{Talent}: Handgemengewaffen \textbf{{Eigenschaften}}: Stumpf}
}



\newglossaryentry{veteranenhand_Waffe}
{
    name={Veteranenhand},
    description={\textbf{TP}: 1W6+2 \textbf{RW}: 0 \textbf{WM}: 0 \textbf{Härte}: 7        \textbf{Fertigkeit}: Handgemenge \textbf{Talent}: Handgemengewaffen \textbf{{Eigenschaften}}: }
}



\newglossaryentry{bock(Schilde)_Waffe}
{
    name={Bock (Schilde)},
    description={\textbf{TP}: 1W6+2 \textbf{RW}: 0 \textbf{WM}: 0 \textbf{Härte}: 11        \textbf{Fertigkeit}: Handgemenge \textbf{Talent}: Schilde \textbf{{Eigenschaften}}: Parierwaffe, Schild}
}



\newglossaryentry{buckler(Schilde)_Waffe}
{
    name={Buckler (Schilde)},
    description={\textbf{TP}: 1W6+1 \textbf{RW}: 0 \textbf{WM}: 0 \textbf{Härte}: 11        \textbf{Fertigkeit}: Handgemenge \textbf{Talent}: Schilde \textbf{{Eigenschaften}}: Parierwaffe, Schild, Stumpf}
}



\newglossaryentry{großschild_Waffe}
{
    name={Großschild},
    description={\textbf{TP}: 1W6+-1 \textbf{RW}: 0 \textbf{WM}: 2 \textbf{Härte}: 7        \textbf{Fertigkeit}: Handgemenge \textbf{Talent}: Schilde \textbf{{Eigenschaften}}: Schild, Stumpf}
}



\newglossaryentry{holzschild_Waffe}
{
    name={Holzschild},
    description={\textbf{TP}: 1W6+0 \textbf{RW}: 0 \textbf{WM}: 1 \textbf{Härte}: 8        \textbf{Fertigkeit}: Handgemenge \textbf{Talent}: Schilde \textbf{{Eigenschaften}}: Schild, Stumpf}
}



\newglossaryentry{lederschild_Waffe}
{
    name={Lederschild},
    description={\textbf{TP}: 1W6+-1 \textbf{RW}: 0 \textbf{WM}: 1 \textbf{Härte}: 5        \textbf{Fertigkeit}: Handgemenge \textbf{Talent}: Schilde \textbf{{Eigenschaften}}: Schild, Stumpf}
}



\newglossaryentry{hand_Waffe}
{
    name={Hand},
    description={\textbf{TP}: 1W6+0 \textbf{RW}: 0 \textbf{WM}: 0 \textbf{Härte}: 1        \textbf{Fertigkeit}: Handgemenge \textbf{Talent}: Unbewaffnet \textbf{{Eigenschaften}}: Parierwaffe, Stumpf, Zerbrechlich, kein Malus als Nebenwaffe}
}



\newglossaryentry{fuß_Waffe}
{
    name={Fuß},
    description={\textbf{TP}: 1W6+0 \textbf{RW}: 0 \textbf{WM}: 0 \textbf{Härte}: 1        \textbf{Fertigkeit}: Handgemenge \textbf{Talent}: Unbewaffnet \textbf{{Eigenschaften}}: Stumpf, Zerbrechlich, kein Malus als Nebenwaffe}
}



\newglossaryentry{diskus_Waffe}
{
    name={Diskus},
    description={\textbf{TP}: 2W6+0 \textbf{RW}: 16 \textbf{LZ}: 0 \textbf{Härte}: 7        \textbf{Fertigkeit}: Wurfwaffen \textbf{Talent}: Diskusse \textbf{{Eigenschaften}}: }
}



\newglossaryentry{jagddiskus_Waffe}
{
    name={Jagddiskus},
    description={\textbf{TP}: 2W6+0 \textbf{RW}: 16 \textbf{LZ}: 0 \textbf{Härte}: 7        \textbf{Fertigkeit}: Wurfwaffen \textbf{Talent}: Diskusse \textbf{{Eigenschaften}}: Stumpf}
}



\newglossaryentry{kampfdiskus_Waffe}
{
    name={Kampfdiskus},
    description={\textbf{TP}: 2W6+2 \textbf{RW}: 16 \textbf{LZ}: 0 \textbf{Härte}: 8        \textbf{Fertigkeit}: Wurfwaffen \textbf{Talent}: Diskusse \textbf{{Eigenschaften}}: Schwer (4)}
}



\newglossaryentry{borndorn(KurzeWurfwaffen)_Waffe}
{
    name={Borndorn (Kurze Wurfwaffen)},
    description={\textbf{TP}: 1W6+2 \textbf{RW}: 4 \textbf{LZ}: 0 \textbf{Härte}: 9        \textbf{Fertigkeit}: Wurfwaffen \textbf{Talent}: Kurze Wurfwaffen \textbf{{Eigenschaften}}: }
}



\newglossaryentry{dolch(KurzeWurfwaffen)_Waffe}
{
    name={Dolch (Kurze Wurfwaffen)},
    description={\textbf{TP}: 1W6+0 \textbf{RW}: 4 \textbf{LZ}: 0 \textbf{Härte}: 8        \textbf{Fertigkeit}: Wurfwaffen \textbf{Talent}: Kurze Wurfwaffen \textbf{{Eigenschaften}}: }
}



\newglossaryentry{schneidzahn_Waffe}
{
    name={Schneidzahn},
    description={\textbf{TP}: 2W6+1 \textbf{RW}: 8 \textbf{LZ}: 0 \textbf{Härte}: 8        \textbf{Fertigkeit}: Wurfwaffen \textbf{Talent}: Kurze Wurfwaffen \textbf{{Eigenschaften}}: }
}



\newglossaryentry{wurfbeil_Waffe}
{
    name={Wurfbeil},
    description={\textbf{TP}: 2W6+0 \textbf{RW}: 8 \textbf{LZ}: 0 \textbf{Härte}: 8        \textbf{Fertigkeit}: Wurfwaffen \textbf{Talent}: Kurze Wurfwaffen \textbf{{Eigenschaften}}: }
}



\newglossaryentry{wurfdolch,-scheibe_Waffe}
{
    name={Wurfdolch, -scheibe},
    description={\textbf{TP}: 1W6+1 \textbf{RW}: 4 \textbf{LZ}: 0 \textbf{Härte}: 10        \textbf{Fertigkeit}: Wurfwaffen \textbf{Talent}: Kurze Wurfwaffen \textbf{{Eigenschaften}}: }
}



\newglossaryentry{wurfkeule_Waffe}
{
    name={Wurfkeule},
    description={\textbf{TP}: 1W6+2 \textbf{RW}: 8 \textbf{LZ}: 0 \textbf{Härte}: 7        \textbf{Fertigkeit}: Wurfwaffen \textbf{Talent}: Kurze Wurfwaffen \textbf{{Eigenschaften}}: Stumpf}
}



\newglossaryentry{wurfmesser_Waffe}
{
    name={Wurfmesser},
    description={\textbf{TP}: 1W6+0 \textbf{RW}: 4 \textbf{LZ}: 0 \textbf{Härte}: 8        \textbf{Fertigkeit}: Wurfwaffen \textbf{Talent}: Kurze Wurfwaffen \textbf{{Eigenschaften}}: }
}



\newglossaryentry{dschadra(Wurfspeere)_Waffe}
{
    name={Dschadra (Wurfspeere)},
    description={\textbf{TP}: 2W6+2 \textbf{RW}: 8 \textbf{LZ}: 0 \textbf{Härte}: 5        \textbf{Fertigkeit}: Wurfwaffen \textbf{Talent}: Wurfspeere \textbf{{Eigenschaften}}: }
}



\newglossaryentry{efferdbart(Wurfspeere)_Waffe}
{
    name={Efferdbart (Wurfspeere)},
    description={\textbf{TP}: 2W6+0 \textbf{RW}: 8 \textbf{LZ}: 0 \textbf{Härte}: 5        \textbf{Fertigkeit}: Wurfwaffen \textbf{Talent}: Wurfspeere \textbf{{Eigenschaften}}: Niederwerfen}
}



\newglossaryentry{holzspeer(Wurfspeere)_Waffe}
{
    name={Holzspeer (Wurfspeere)},
    description={\textbf{TP}: 1W6+2 \textbf{RW}: 8 \textbf{LZ}: 0 \textbf{Härte}: 4        \textbf{Fertigkeit}: Wurfwaffen \textbf{Talent}: Wurfspeere \textbf{{Eigenschaften}}: }
}



\newglossaryentry{speer(Wurfspeere)_Waffe}
{
    name={Speer (Wurfspeere)},
    description={\textbf{TP}: 2W6+0 \textbf{RW}: 8 \textbf{LZ}: 0 \textbf{Härte}: 5        \textbf{Fertigkeit}: Wurfwaffen \textbf{Talent}: Wurfspeere \textbf{{Eigenschaften}}: }
}



\newglossaryentry{speerschleuder_Waffe}
{
    name={Speerschleuder},
    description={\textbf{TP}: 2W6+1 \textbf{RW}: 16 \textbf{LZ}: 1 \textbf{Härte}: 5        \textbf{Fertigkeit}: Wurfwaffen \textbf{Talent}: Wurfspeere \textbf{{Eigenschaften}}: }
}



\newglossaryentry{wurfspeer_Waffe}
{
    name={Wurfspeer},
    description={\textbf{TP}: 2W6+3 \textbf{RW}: 8 \textbf{LZ}: 0 \textbf{Härte}: 5        \textbf{Fertigkeit}: Wurfwaffen \textbf{Talent}: Wurfspeere \textbf{{Eigenschaften}}: }
}



\newglossaryentry{fledermaus_Waffe}
{
    name={Fledermaus},
    description={\textbf{TP}: 0W6+0 \textbf{RW}: 8 \textbf{LZ}: 0 \textbf{Härte}: 3        \textbf{Fertigkeit}: Wurfwaffen \textbf{Talent}: Schleudern \textbf{{Eigenschaften}}: Umklammern (-2; 12)}
}



\newglossaryentry{kettenkugel_Waffe}
{
    name={Kettenkugel},
    description={\textbf{TP}: 2W6+2 \textbf{RW}: 4 \textbf{LZ}: 0 \textbf{Härte}: 7        \textbf{Fertigkeit}: Wurfwaffen \textbf{Talent}: Schleudern \textbf{{Eigenschaften}}: Niederwerfen}
}



\newglossaryentry{lasso_Waffe}
{
    name={Lasso},
    description={\textbf{TP}: 0W6+0 \textbf{RW}: 4 \textbf{LZ}: 0 \textbf{Härte}: 2        \textbf{Fertigkeit}: Wurfwaffen \textbf{Talent}: Schleudern \textbf{{Eigenschaften}}: Umklammern (-2; 12)}
}



\newglossaryentry{schleuder_Waffe}
{
    name={Schleuder},
    description={\textbf{TP}: 1W6+2 \textbf{RW}: 16 \textbf{LZ}: 0 \textbf{Härte}: 3        \textbf{Fertigkeit}: Wurfwaffen \textbf{Talent}: Schleudern \textbf{{Eigenschaften}}: }
}



\newglossaryentry{stabschleuder_Waffe}
{
    name={Stabschleuder},
    description={\textbf{TP}: 2W6+0 \textbf{RW}: 32 \textbf{LZ}: 1 \textbf{Härte}: 4        \textbf{Fertigkeit}: Wurfwaffen \textbf{Talent}: Schleudern \textbf{{Eigenschaften}}: Nicht für Reiter}
}



\newglossaryentry{wurfnetz_Waffe}
{
    name={Wurfnetz},
    description={\textbf{TP}: 0W6+0 \textbf{RW}: 2 \textbf{LZ}: 0 \textbf{Härte}: 5        \textbf{Fertigkeit}: Wurfwaffen \textbf{Talent}: Schleudern \textbf{{Eigenschaften}}: Umklammern (-4; 16)}
}



\newglossaryentry{schweresWurfnetz_Waffe}
{
    name={Schweres Wurfnetz},
    description={\textbf{TP}: 0W6+0 \textbf{RW}: 2 \textbf{LZ}: 0 \textbf{Härte}: 6        \textbf{Fertigkeit}: Wurfwaffen \textbf{Talent}: Schleudern \textbf{{Eigenschaften}}: Umklammern (-8; 16), Zweihändig}
}



\newglossaryentry{arbalette_Waffe}
{
    name={Arbalette},
    description={\textbf{TP}: 3W6+2 \textbf{RW}: 32 \textbf{LZ}: 8 \textbf{Härte}: 5        \textbf{Fertigkeit}: Schusswaffen \textbf{Talent}: Armbrüste \textbf{{Eigenschaften}}: Niederwerfen (-4), Zweihändig}
}



\newglossaryentry{arbalone_Waffe}
{
    name={Arbalone},
    description={\textbf{TP}: 3W6+7 \textbf{RW}: 64 \textbf{LZ}: 16 \textbf{Härte}: 6        \textbf{Fertigkeit}: Schusswaffen \textbf{Talent}: Armbrüste \textbf{{Eigenschaften}}: Niederwerfen (-8), nicht für Reiter, Stationär}
}



\newglossaryentry{balestra_Waffe}
{
    name={Balestra},
    description={\textbf{TP}: 3W6+0 \textbf{RW}: 32 \textbf{LZ}: 4 \textbf{Härte}: 4        \textbf{Fertigkeit}: Schusswaffen \textbf{Talent}: Armbrüste \textbf{{Eigenschaften}}: Niederwerfen, Zweihändig}
}



\newglossaryentry{balestrina_Waffe}
{
    name={Balestrina},
    description={\textbf{TP}: 2W6+1 \textbf{RW}: 8 \textbf{LZ}: 1 \textbf{Härte}: 4        \textbf{Fertigkeit}: Schusswaffen \textbf{Talent}: Armbrüste \textbf{{Eigenschaften}}: }
}



\newglossaryentry{balläster_Waffe}
{
    name={Balläster},
    description={\textbf{TP}: 2W6+3 \textbf{RW}: 32 \textbf{LZ}: 2 \textbf{Härte}: 4        \textbf{Fertigkeit}: Schusswaffen \textbf{Talent}: Armbrüste \textbf{{Eigenschaften}}: Zweihändig}
}



\newglossaryentry{eisenwalder_Waffe}
{
    name={Eisenwalder},
    description={\textbf{TP}: 2W6+2 \textbf{RW}: 8 \textbf{LZ}: 1 \textbf{Härte}: 4        \textbf{Fertigkeit}: Schusswaffen \textbf{Talent}: Armbrüste \textbf{{Eigenschaften}}: Magazin (10 Bolzen) laden 8 Akt., Zweihändig}
}



\newglossaryentry{leichteArmbrust_Waffe}
{
    name={Leichte Armbrust},
    description={\textbf{TP}: 3W6+1 \textbf{RW}: 32 \textbf{LZ}: 4 \textbf{Härte}: 4        \textbf{Fertigkeit}: Schusswaffen \textbf{Talent}: Armbrüste \textbf{{Eigenschaften}}: Zweihändig}
}



\newglossaryentry{windenarmbrust_Waffe}
{
    name={Windenarmbrust},
    description={\textbf{TP}: 3W6+4 \textbf{RW}: 64 \textbf{LZ}: 8 \textbf{Härte}: 5        \textbf{Fertigkeit}: Schusswaffen \textbf{Talent}: Armbrüste \textbf{{Eigenschaften}}: Zweihändig}
}



\newglossaryentry{blasrohr_Waffe}
{
    name={Blasrohr},
    description={\textbf{TP}: 1W6+-1 \textbf{RW}: 8 \textbf{LZ}: 1 \textbf{Härte}: 2        \textbf{Fertigkeit}: Schusswaffen \textbf{Talent}: Blasrohre \textbf{{Eigenschaften}}: }
}



\newglossaryentry{elfenbogen_Waffe}
{
    name={Elfenbogen},
    description={\textbf{TP}: 2W6+3 \textbf{RW}: 64 \textbf{LZ}: 1 \textbf{Härte}: 3        \textbf{Fertigkeit}: Schusswaffen \textbf{Talent}: Bögen \textbf{{Eigenschaften}}: Zweihändig}
}



\newglossaryentry{kompositbogen_Waffe}
{
    name={Kompositbogen},
    description={\textbf{TP}: 2W6+3 \textbf{RW}: 32 \textbf{LZ}: 1 \textbf{Härte}: 3        \textbf{Fertigkeit}: Schusswaffen \textbf{Talent}: Bögen \textbf{{Eigenschaften}}: Zweihändig}
}



\newglossaryentry{kriegsbogen_Waffe}
{
    name={Kriegsbogen},
    description={\textbf{TP}: 2W6+5 \textbf{RW}: 32 \textbf{LZ}: 1 \textbf{Härte}: 3        \textbf{Fertigkeit}: Schusswaffen \textbf{Talent}: Bögen \textbf{{Eigenschaften}}: Schwer (8), nicht für Reiter, Zweihändig}
}



\newglossaryentry{kurzbogen_Waffe}
{
    name={Kurzbogen},
    description={\textbf{TP}: 2W6+1 \textbf{RW}: 16 \textbf{LZ}: 0 \textbf{Härte}: 3        \textbf{Fertigkeit}: Schusswaffen \textbf{Talent}: Bögen \textbf{{Eigenschaften}}: Zweihändig}
}



\newglossaryentry{langbogen_Waffe}
{
    name={Langbogen},
    description={\textbf{TP}: 2W6+3 \textbf{RW}: 64 \textbf{LZ}: 1 \textbf{Härte}: 3        \textbf{Fertigkeit}: Schusswaffen \textbf{Talent}: Bögen \textbf{{Eigenschaften}}: Schwer (4), nicht für Reiter, Zweihändig}
}



\newglossaryentry{orkischerReiterbogen_Waffe}
{
    name={Orkischer Reiterbogen},
    description={\textbf{TP}: 2W6+3 \textbf{RW}: 32 \textbf{LZ}: 1 \textbf{Härte}: 4        \textbf{Fertigkeit}: Schusswaffen \textbf{Talent}: Bögen \textbf{{Eigenschaften}}: Schwer (4), Zweihändig}
}



\newglossaryentry{reitpferd_Waffe}
{
    name={Reitpferd},
    description={\textbf{TP}: 2W6+0 \textbf{RW}: 1 \textbf{WM}: -1 \textbf{Härte}: 13        \textbf{Fertigkeit}: Athletik \textbf{Talent}: Reiten \textbf{{Eigenschaften}}: Niederwerfen, Reittier}
}



\newglossaryentry{reitpony_Waffe}
{
    name={Reitpony},
    description={\textbf{TP}: 2W6+0 \textbf{RW}: 1 \textbf{WM}: 0 \textbf{Härte}: 11        \textbf{Fertigkeit}: Athletik \textbf{Talent}: Reiten \textbf{{Eigenschaften}}: Niederwerfen, Reittier}
}



\newglossaryentry{kriegspferd_Waffe}
{
    name={Kriegspferd},
    description={\textbf{TP}: 2W6+2 \textbf{RW}: 1 \textbf{WM}: 1 \textbf{Härte}: 15        \textbf{Fertigkeit}: Athletik \textbf{Talent}: Reiten \textbf{{Eigenschaften}}: Niederwerfen, Reittier}
}



\newglossaryentry{tralloperRiese_Waffe}
{
    name={Tralloper Riese},
    description={\textbf{TP}: 2W6+4 \textbf{RW}: 1 \textbf{WM}: 0 \textbf{Härte}: 16        \textbf{Fertigkeit}: Athletik \textbf{Talent}: Reiten \textbf{{Eigenschaften}}: Niederwerfen, Reittier}
}



\newglossaryentry{wildschwein_Waffe}
{
    name={Wildschwein},
    description={\textbf{TP}: 2W6+4 \textbf{RW}: 0 \textbf{WM}: 0 \textbf{Härte}: 11        \textbf{Fertigkeit}: Athletik \textbf{Talent}: Reiten \textbf{{Eigenschaften}}: Niederwerfen, Reittier}
}



\newglossaryentry{kriegswildschwein_Waffe}
{
    name={Kriegswildschwein},
    description={\textbf{TP}: 2W6+6 \textbf{RW}: 0 \textbf{WM}: 1 \textbf{Härte}: 13        \textbf{Fertigkeit}: Athletik \textbf{Talent}: Reiten \textbf{{Eigenschaften}}: Niederwerfen, Reittier}
}


\newglossaryentry{ausweichen_Manöver}
{
    name={Ausweichen},
    description={Der Verteidiger entgeht dem Angriff und seinen Auswirkungen vollständig. Ausweichen ist die sinnvollste waffenlose Abwehr gegen Bewaffnete (siehe Waffeneigenschaft Zerbrechlich auf S. 47).}
}


\newglossaryentry{binden_Manöver}
{
    name={Binden},
    description={Bis zum Ende der nächsten eigenen Initiativephase sind alle Verteidigungen des Gegners um -X (maximal 8) erschwert.}
}


\newglossaryentry{entfernungverändern_Manöver}
{
    name={Entfernung verändern},
    description={Du kannst dich aus dem Nahkampf lösen, ohne Passierschläge zu riskieren.\newline Besonderheiten: Beim Spiel mit Bodenplänen erhält das Ziel in dieser Runde keine Passierschläge durch deine Bewegung.}
}


\newglossaryentry{entwaffnen_Manöver}
{
    name={Entwaffnen},
    description={Dem Ziel wird eine Waffe deiner Wahl entrissen, die danach zwischen den Kämpfern am Boden liegt. Das Aufheben der Waffe benötigt eine Aktion Konflikt; falls ein anderer Kämpfer das in einer Reaktion verhindern möchte, ist zusätzlich eine vergleichende GE-Probe (I) notwendig. Das Manöver senkt den Schaden des Angriffes auf 0. Sollte der Angriff ohnehin keinen Schaden anrichten – etwa durch andere Manöver – ist das Manöver nicht möglich.}
}


\newglossaryentry{gezielterSchlag_Manöver}
{
    name={Gezielter Schlag},
    description={Du bestimmst, an welcher Trefferzone (S. 33) du den Gegner triffst. Nur beim Spiel mit Trefferzonen.}
}


\newglossaryentry{umreißen_Manöver}
{
    name={Umreißen},
    description={Dein Ziel stürzt und liegt am Boden. Das Manöver senkt den Schaden des Angriffes auf 0. Sollte der Angriff ohnehin keinen Schaden anrichten, ist das Manöver nicht möglich.}
}


\newglossaryentry{wuchtschlag_Manöver}
{
    name={Wuchtschlag},
    description={Der Angriff richtet X (maximal 8) Trefferpunkte mehr an.}
}


\newglossaryentry{auflaufenlassen_Manöver}
{
    name={Auflaufen lassen},
    description={Der Verteidiger fügt dem Angreifer seinen Waffenschaden plus die GS des Angreifers zu.\newline Voraussetzungen: größere Reichweite; Gegner kombiniert die Aktionen Bewegung und Konflikt; die Bewegung muss vor dem Angriff ausgeführt werden.}
}


\newglossaryentry{rüstungsbrecher_Manöver}
{
    name={Rüstungsbrecher},
    description={Der Angriff ignoriert die gegnerische Rüstung und richtet SP statt TP an.\newline Voraussetzung: Waffeneigenschaft Rüstungsbrechend\newline Anmerkung: Das Manöver lohnt sich, wenn der Gegner schwere Rüstung trägt oder du mehrere Wunden anrichten kannst.}
}


\newglossaryentry{schildspalter_Manöver}
{
    name={Schildspalter},
    description={Du fügst dem Schild deines Gegners deinen Waffenschaden zu.\newline Voraussetzung: Gegner führt Schild.}
}


\newglossaryentry{stumpferSchlag_Manöver}
{
    name={Stumpfer Schlag},
    description={Der Angriff verursacht Erschöpfung statt Wunden. Wird das Ziel kampfunfähig, erleidet es keine Blutung.\newline Voraussetzung: Waffeneigenschaft Stumpf.}
}


\newglossaryentry{umklammern_Manöver}
{
    name={Umklammern},
    description={Handlungen des Umklammerten sind um –X erschwert und seine GS sinkt auf 0, solange die Umklammerung anhält. Bei mehrfachen Umklammerungen summieren sich die Malusse. Du kannst dein Ziel in einer Freien Aktion loslassen. Der Umklammerte kann eine Aktion Konflikt aufwenden und eine vergleichenden GE-oder KK-Probe (I) gegen die GE oder KK des Umklammernden ablegen. Gelingt die Probe, lässt der Umklammernde sofort los. Das Manöver senkt den Schaden des Angriffes auf 0. Sollte der Angriff ohnehin keinen Schaden anrichten, ist das Manöver nicht möglich.\newline Voraussetzungen: Angriff mit beiden Händen im waffenlosen Kampf; Ziel darf nicht einer höheren Größenklasse angehören}
}


\newglossaryentry{ausfall_Manöver}
{
    name={Ausfall},
    description={Das Ziel muss in einer Freien Reaktion bis zu 2 Schritt zurückweichen, du folgst in einer Freien Aktion. Möchte es in seiner nächsten Initiativephase einen anderen Gegner als dich angreifen, darfst du einen Passierschlag ausführen.}
}


\newglossaryentry{befreiungsschlag_Manöver}
{
    name={Befreiungsschlag},
    description={Du führst eine Attacke aus, die sich (samt anderen Manövern) gegen alle Ziele in einem Winkel von 180° vor dir richtet. Jedem Ziel steht eine eigene Verteidigung zu.\newline Voraussetzungen: Kampf im Kraftvollen Stil; kein Befreiungsschlag seit letztem Beginn einer eigenen Initiativephase}
}


\newglossaryentry{doppelangriff_Manöver}
{
    name={Doppelangriff},
    description={Du führst mit beiden Händen je eine Attacke –4 aus. Beide Angriffe sind unabhängig voneinander; sie werden separat ausgewürfelt, können sich gegen verschiedene Ziele richten und mit unterschiedlichen Manövern versehen werden.\newline Voraussetzungen: Kampf im Beidhändigen Stil; kein Doppelangriff seit letztem Beginn einer eigenen Initiativephase}
}


\newglossaryentry{hammerschlag_Manöver}
{
    name={Hammerschlag},
    description={Der Angriff richtet doppelten Waffenschaden an. Zusätzliche TP aus Kommandos, anderen Manövern usw. werden nicht verdoppelt.}
}


\newglossaryentry{klingentanz_Manöver}
{
    name={Klingentanz},
    description={Wenn der Angriff gelingt, darfst du sofort noch einmal angreifen. Dieser Folgeangriff ist nicht erschwert.\newline Besonderheit: Bei einer Kombination von Klingentanz und Befreiungsschlag muss jedes Ziel getroffen werden, damit der Klingentanz gelingt.\newline Voraussetzung: Kein Klingentanz seit letztem Beginn einer eigenen Initiativephase}
}


\newglossaryentry{niederwerfen_Manöver}
{
    name={Niederwerfen},
    description={Dein Ziel stürzt und liegt am Boden.}
}


\newglossaryentry{riposte_Manöver}
{
    name={Riposte},
    description={Du fügst dem Angreifer den Waffenschaden deiner Waffe zu. Die Riposte ist kombinierbar mit Attackemanövern, die in diesem Fall die Verteidigung zusätzlich erschweren. Zum Beispiel wäre eine Kombination aus Riposte und Hammerschlag um –12 erschwert, würde aber doppelten Schaden anrichten.\newline Voraussetzung: Kampf im Parrierwaffen-Stil; keine Riposte seit letztem Beginn einer eigenen Initiativephase}
}


\newglossaryentry{schildwall_Manöver}
{
    name={Schildwall},
    description={Du kannst einen Angriff auf einen benachbarten Verbündeten abwehren. Das Manöver erfolgt vor der Verteidigung des Verbündeten.\newline Voraussetzungen: Kampf im Schildkampf-Stil; kein Schildwall seit letztem Beginn einer eigenen Initiativephase}
}


\newglossaryentry{sturmangriff_Manöver}
{
    name={Sturmangriff},
    description={Die Kombination der Aktionen Bewegung und Konflikt ist nicht erschwert. Außerdem richtet die nächste Attacke GS Trefferpunkte zusätzlich an.\newline Voraussetzungen: Vorteil Sturmangriff oder Reiterkampf I und Kampf in diesem Stil; die Bewegung muss vor dem Angriff ausgeführt werden.\newline Anmerkung: Sturzflugangriffe von fliegenden Gegnern folgen denselben Regeln. Bei besonders kleinen Gegnern oder kurzem Anlauf sollte der Spielleiter den Schaden verringern.}
}


\newglossaryentry{todesstoß_Manöver}
{
    name={Todesstoß},
    description={Der Angriff richtet zwei zusätzliche Wunden an, auch wenn der Schaden zu gering ist, um Wunden anzurichten.}
}


\newglossaryentry{überrennen_Manöver}
{
    name={Überrennen},
    description={Das Reittier stürmt in die gegnerische Formation. Seine Attacke trifft (samt anderen Manövern) alle Gegner in seiner Bahn, bis es durch mindestens 2 erfolgreiche Verteidigungen gestoppt wird. Außerdem richtet die Attacke GS Trefferpunkte zusätzlich an. Die Kombination der Aktionen Bewegung und Konflikt ist nicht erschwert.\newline Voraussetzungen: Kampf im Reiterkampfstil; GS Schritt Anlauf; Ziele sind in einer kleineren Größenklasse als das Reittier}
}


\newglossaryentry{unterlaufen_Manöver}
{
    name={Unterlaufen},
    description={Wenn die Verteidigung gelingt, darfst du in der nächsten eigenen Initiativephase eine Freie Aktion nutzen, um deinen Gegner (ein weiteres Mal) anzugreifen.\newline Voraussetzungen: Kampf im Schnellen Stil; kein Unterlaufen seit letztem Beginn einer eigenen Initiativephase}
}


\newglossaryentry{gezielterSchuss_Manöver}
{
    name={Gezielter Schuss},
    description={Du bestimmst, an welcher Trefferzone (S. 33) du den Gegner triffst. Nur beim Spiel mit Trefferzonen.}
}


\newglossaryentry{reichweiteerhöhen_Manöver}
{
    name={Reichweite erhöhen},
    description={Die Reichweite deines Fernkampfangriffes verdoppelt sich. Kann bis zu zweimal pro Angriff eingesetzt werden.}
}


\newglossaryentry{scharfschuss_Manöver}
{
    name={Scharfschuss},
    description={Der Angriff richtet X (maximal 8) Trefferpunkte mehr an.}
}


\newglossaryentry{zielen_Manöver}
{
    name={Zielen},
    description={Erhöht die Vorbereitungszeit um 1 Aktion.}
}


\newglossaryentry{meisterschuss_Manöver}
{
    name={Meisterschuss},
    description={Der Angriff richtet zwei zusätzliche Wunden an - auch wenn der Schaden zu gering ist, um eine Wunde anzurichten.}
}


\newglossaryentry{rüstungsbrecher(FK)_Manöver}
{
    name={Rüstungsbrecher (FK)},
    description={Der Angriff ignoriert die gegnerische Rüstung und richtet SP statt TP an.\newline Voraussetzung: Pfeile mit Waffeneigenschaft Rüstungsbrechend.}
}


\newglossaryentry{schnellschuss_Manöver}
{
    name={Schnellschuss},
    description={Verkürzt die Vorbereitungszeit um die Hälfte. Eine Vorbereitungszeit von 1 Aktion wird zu 0 Aktionen. Die Aktion Konflikt ist von dieser Modifikation nicht betroffen. Kann bis zu zweimal pro FK eingesetzt werden.}
}


\newglossaryentry{mächtigeMagie_Manöver}
{
    name={Mächtige Magie},
    description={Verstärkt die Wirkung des Zaubers (siehe Zauber). Zusätzlich steigt die Schwierigkeit von Konterproben (S. 124) gegen den Zauber um +4 Punkte.}
}


\newglossaryentry{mehrereZiele(M)_Manöver}
{
    name={Mehrere Ziele (M)},
    description={Der Zauber wirkt auf mehrere Ziele in Reichweite. Die Kosten werden für jedes Ziel einzeln bezahlt, Modifikationen wie Kosten sparen für jedes Ziel einzeln abgerechnet. Richtet sich der Zauber gegen die MR, steht jedem unfreiwilligen Ziel eine eigene MR-Probe zu, um dem Zauber zu widerstehen.}
}


\newglossaryentry{reichweiteerhöhen(M)_Manöver}
{
    name={Reichweite erhöhen (M)},
    description={Die Reichweite des Zaubers verdoppelt sich. Die Reichweite Berührung wird zu 2 Schritt.}
}


\newglossaryentry{vorbereitungverkürzen(M)_Manöver}
{
    name={Vorbereitung verkürzen (M)},
    description={Die Vorbereitungszeit des Zaubers halbiert sich. Eine Vorbereitungszeit von 1 Aktion wird zu 0 Aktionen. Die Aktion Konflikt ist von dieser Modifikation nicht betroffen.}
}


\newglossaryentry{wirkungsdauerverlängern(M)_Manöver}
{
    name={Wirkungsdauer verlängern (M)},
    description={Die Wirkungsdauer des Zaubers verdoppelt sich.}
}


\newglossaryentry{zaubertechnikignorieren_Manöver}
{
    name={Zaubertechnik ignorieren},
    description={Du ignorierst eine der bei der Tradition angeführten Bedingungen. Ignorierst du die Bedingung Sicht, musst du das Ziel auf eine andere Art und Weise wahrnehmen.}
}


\newglossaryentry{erzwingen_Manöver}
{
    name={Erzwingen},
    description={Die Kosten des Zaubers steigen um die Hälfte der Basiskosten, dafür ist der Zauber um 4 Punkte erleichtert. Kann nur einmal pro Zauber eingesetzt werden.\newline Voraussetzungen: Tradition der Anach-nûrim, Borbaradianer, Druiden, Hexen oder Schelme (jeweils) III und Verwendung dieser Tradition}
}


\newglossaryentry{kostensparen(M)_Manöver}
{
    name={Kosten sparen (M)},
    description={Die AsP-Kosten des Zaubers sinken um ein Viertel der Basiskosten. Die Kosten können dadurch nicht unter die Hälfte der Basiskosten sinken.}
}


\newglossaryentry{zeitlassen_Manöver}
{
    name={Zeit lassen},
    description={Verdoppelt die Vorbereitungszeit, aber erleichtert den Zauber um +2. Eine Vorbereitungszeit von 0 Aktionen wird zu 1 Aktion. Kann einmal pro Zauber eingesetzt werden und wirkt nicht auf Zauber mit frei wählbarer Vorbereitungszeit.\newline Voraussetzung: Tradition der Alchemisten, Elfen, Geoden, Gildenmagier, Kristallomanten, Scharlatane oder Zauberbarden (jeweils) III und Verwendung dieser Tradition.}
}


\newglossaryentry{opferung(Aves)_Manöver}
{
    name={Opferung (Aves)},
    description={Ein rituelles Opfer erleichtert die Liturgie um +4. Der Opfergegenstand wird dabei verbraucht, zerstört oder verschwindet. Avesgeweihte opfern die Feder eines Zugvogels, dessen Weg der Geweihte auf seinen Reisen kreuzt.\newline Anmerkung: Opfergegenstände sollten nicht so häufig sein, dass jede zweite Liturgie mit dieser Modifikation gewirkt wird. Umgekehrt sollten sie nicht so selten sein, dass die Modifikation nie eingesetzt werden kann. Als Richtwert gilt eine Opferung alle zwei Spielabende.}
}


\newglossaryentry{opferung(Efferd)_Manöver}
{
    name={Opferung (Efferd)},
    description={Ein rituelles Opfer erleichtert die Liturgie um +4. Der Opfergegenstand wird dabei verbraucht, zerstört oder verschwindet. Efferdgeweihte opfern Perlen, Aquamarine oder Gwen-Petryl-Steine. Solche Opfergaben werden von Tempeln in begrenzten Mengen weitergegeben.\newline Anmerkung: Opfergegenstände sollten nicht so häufig sein, dass jede zweite Liturgie mit dieser Modifikation gewirkt wird. Umgekehrt sollten sie nicht so selten sein, dass die Modifikation nie eingesetzt werden kann. Als Richtwert gilt eine Opferung alle zwei Spielabende.}
}


\newglossaryentry{opferung(Firun)_Manöver}
{
    name={Opferung (Firun)},
    description={Ein rituelles Opfer erleichtert die Liturgie um +4. Der Opfergegenstand wird dabei verbraucht, zerstört oder verschwindet. Firungeweihte opfern besondere Beute oder Trophäen.\newline Anmerkung: Opfergegenstände sollten nicht so häufig sein, dass jede zweite Liturgie mit dieser Modifikation gewirkt wird. Umgekehrt sollten sie nicht so selten sein, dass die Modifikation nie eingesetzt werden kann. Als Richtwert gilt eine Opferung alle zwei Spielabende.}
}


\newglossaryentry{opferung(Ingerimm)_Manöver}
{
    name={Opferung (Ingerimm)},
    description={Ein rituelles Opfer erleichtert die Liturgie um +4. Der Opfergegenstand wird dabei verbraucht, zerstört oder verschwindet. Ingerimmgeweihte opfern ihrem Gott kunstvolle Handwerksgegenstände, die im Opferfeuer vergehen.\newline Anmerkung: Opfergegenstände sollten nicht so häufig sein, dass jede zweite Liturgie mit dieser Modifikation gewirkt wird. Umgekehrt sollten sie nicht so selten sein, dass die Modifikation nie eingesetzt werden kann. Als Richtwert gilt eine Opferung alle zwei Spielabende.}
}


\newglossaryentry{opferung(Nandus)_Manöver}
{
    name={Opferung (Nandus)},
    description={Ein rituelles Opfer erleichtert die Liturgie um +4. Der Opfergegenstand wird dabei verbraucht, zerstört oder verschwindet. Nandusgeweihte opfern falsches oder veraltetes Wissen, zum Beispiel ein Buch oder gesammelte Pamphlete.\newline Anmerkung: Opfergegenstände sollten nicht so häufig sein, dass jede zweite Liturgie mit dieser Modifikation gewirkt wird. Umgekehrt sollten sie nicht so selten sein, dass die Modifikation nie eingesetzt werden kann. Als Richtwert gilt eine Opferung alle zwei Spielabende.}
}


\newglossaryentry{opferung(Phex)_Manöver}
{
    name={Opferung (Phex)},
    description={Ein rituelles Opfer erleichtert die Liturgie um +4. Der Opfergegenstand wird dabei verbraucht, zerstört oder verschwindet. Phexgeweihte opfern einen wertvollen und möglichst phexgefällig erworbenen Gegenstand, der einen spürbaren Teil ihres Vermögens ausmacht.\newline Anmerkung: Opfergegenstände sollten nicht so häufig sein, dass jede zweite Liturgie mit dieser Modifikation gewirkt wird. Umgekehrt sollten sie nicht so selten sein, dass die Modifikation nie eingesetzt werden kann. Als Richtwert gilt eine Opferung alle zwei Spielabende.}
}


\newglossaryentry{opferung(Rahja)_Manöver}
{
    name={Opferung (Rahja)},
    description={Ein rituelles Opfer erleichtert die Liturgie um +4. Der Opfergegenstand wird dabei verbraucht, zerstört oder verschwindet. Rahjageweihte opfern ihrer Göttin den heiligen Wein Tharf. Tharf wird von Tempeln in begrenzten Mengen weitergegeben.\newline Anmerkung: Opfergegenstände sollten nicht so häufig sein, dass jede zweite Liturgie mit dieser Modifikation gewirkt wird. Umgekehrt sollten sie nicht so selten sein, dass die Modifikation nie eingesetzt werden kann. Als Richtwert gilt eine Opferung alle zwei Spielabende.}
}


\newglossaryentry{opferung(Rondra)_Manöver}
{
    name={Opferung (Rondra)},
    description={Ein rituelles Opfer erleichtert die Liturgie um +4. Der Opfergegenstand wird dabei verbraucht, zerstört oder verschwindet. Rondrageweihte opfern ihrer Göttin ihr Blut und fügen sich dabei eine Wunde zu, deren Auswirkungen bei der Probe für die Liturgie ignoriert werden.\newline Anmerkung: Opfergegenstände sollten nicht so häufig sein, dass jede zweite Liturgie mit dieser Modifikation gewirkt wird. Umgekehrt sollten sie nicht so selten sein, dass die Modifikation nie eingesetzt werden kann. Als Richtwert gilt eine Opferung alle zwei Spielabende.}
}


\newglossaryentry{opferung(Swafnir)_Manöver}
{
    name={Opferung (Swafnir)},
    description={Ein rituelles Opfer erleichtert die Liturgie um +4. Der Opfergegenstand wird dabei verbraucht, zerstört oder verschwindet. Swafnirgeweihte opfern ihrem Gott ihr Blut und fügen sich dabei eine Wunde zu, deren Auswirkungen bei der Probe für die Liturgie ignoriert werden.\newline Anmerkung: Opfergegenstände sollten nicht so häufig sein, dass jede zweite Liturgie mit dieser Modifikation gewirkt wird. Umgekehrt sollten sie nicht so selten sein, dass die Modifikation nie eingesetzt werden kann. Als Richtwert gilt eine Opferung alle zwei Spielabende.}
}


\newglossaryentry{opferung(Kor)_Manöver}
{
    name={Opferung (Kor)},
    description={Ein rituelles Opfer erleichtert die Liturgie um +4. Der Opfergegenstand wird dabei verbraucht, zerstört oder verschwindet. Korgeweihte opfern ihrem Gott ihr Blut und fügen sich dabei eine Wunde zu, deren Auswirkungen bei der Probe für die Liturgie ignoriert werden.\newline Anmerkung: Opfergegenstände sollten nicht so häufig sein, dass jede zweite Liturgie mit dieser Modifikation gewirkt wird. Umgekehrt sollten sie nicht so selten sein, dass die Modifikation nie eingesetzt werden kann. Als Richtwert gilt eine Opferung alle zwei Spielabende.}
}


\newglossaryentry{opferung(Derwische)_Manöver}
{
    name={Opferung (Derwische)},
    description={Ein Opfer erleichtert den Zauber um +4. Derwische legen ihre ganze Kraft in den Zauber und erleiden danach 2 Punkte Erschöpfung.\newline Voraussetzung: Verwendung der Tradition der Derwische}
}


\newglossaryentry{opferung(Zaubertänzer)_Manöver}
{
    name={Opferung (Zaubertänzer)},
    description={Ein Opfer erleichtert den Zauber um +4. Zaubertänzer legen ihre ganze Kraft in den Zauber und erleiden danach 2 Punkte Erschöpfung.\newline Voraussetzung: Verwendung der Tradition der Zaubertänzer III}
}


\newglossaryentry{opferung(Durro-dûn)_Manöver}
{
    name={Opferung (Durro-dûn)},
    description={Ein Opfer erleichtert den Zauber um +4. Die Tierkrieger opfern den Geistern ihr Blut und fügen sich dabei eine Wunde zu, deren Auswirkungen bei der Probe für den Zauber ignoriert werden.\newline Voraussetzung: Verwendung der Tradition der Durro-dûn}
}


\newglossaryentry{zeremonie(M)_Manöver}
{
    name={Zeremonie (M)},
    description={Du kannst die Vorbereitungszeit freiwillig um 1 Minute/Stunde/Tag/Woche/Monat/Jahr erhöhen, wodurch der Zauber um +4/6/8/10/12/14 erleichtert ist. Die Vorbereitungszeit muss dadurch mindestens verdoppelt werden.\newline Voraussetzung: Verwendung der Tradition der Schamanen.}
}


\newglossaryentry{mächtigeLiturgie_Manöver}
{
    name={Mächtige Liturgie},
    description={Verstärkt die Wirkung der Liturgie (siehe Liturgie). Zusätzlich steigt die Schwierigkeit von Konterproben (S. 124) gegen die Liturgie um +4 Punkte.}
}


\newglossaryentry{mehrereZiele(L)_Manöver}
{
    name={Mehrere Ziele (L)},
    description={Die Liturgie wirkt auf mehrere Ziele in Reichweite. Die Kosten werden für jedes Ziel einzeln bezahlt, Modifikationen wie Kosten sparen für jedes Ziel einzeln abgerechnet.}
}


\newglossaryentry{liturgischeTechnikignorieren_Manöver}
{
    name={Liturgische Technik ignorieren},
    description={Du ignorierst eine der bei der Tradition angeführten Bedingungen. Ignorierst du die Bedingung Sicht, musst du das Ziel auf eine andere Art und Weise wahrnehmen.}
}


\newglossaryentry{reichweiteerhöhen(L)_Manöver}
{
    name={Reichweite erhöhen (L)},
    description={Die Reichweite der Liturgie verdoppelt sich. Die Reichweite Berührung wird zu 2 Schritt.}
}


\newglossaryentry{vorbereitungverkürzen(L)_Manöver}
{
    name={Vorbereitung verkürzen (L)},
    description={Die Vorbereitungszeit der Liturgie halbiert sich. Eine Vorbereitungszeit von 1 Aktion wird zu 0 Aktionen. Die Aktion Konflikt ist von dieser Modifikation nicht betroffen.}
}


\newglossaryentry{wirkungsdauerverlängern(L)_Manöver}
{
    name={Wirkungsdauer verlängern (L)},
    description={Die Wirkungsdauer der Liturgie verdoppelt sich.}
}


\newglossaryentry{kostensparen(L)_Manöver}
{
    name={Kosten sparen (L)},
    description={Die Kosten der Liturgie sinken um ein Viertel der Basiskosten. Die Kosten können dadurch nicht unter die Hälfte der Basiskosten sinken.}
}


\newglossaryentry{zeremonie(L)_Manöver}
{
    name={Zeremonie (L)},
    description={Du kannst die Vorbereitungszeit freiwillig um 1 Minute/Stunde/Tag/Woche/Monat/Jahr erhöhen, wodurch die Liturgie um +4/6/8/10/12/14 erleichtert ist. Die Vorbereitungszeit muss dadurch mindestens verdoppelt werden.}
}


\newglossaryentry{mächtigeAnrufung_Manöver}
{
    name={Mächtige Anrufung},
    description={Verstärkt die Wirkung der Anrufung. Zusätzlich steigt die Schwierigkeit von Konterproben (S. 124) gegen die Anrufung um +4 Punkte.}
}


\newglossaryentry{mehrereZiele(D)_Manöver}
{
    name={Mehrere Ziele (D)},
    description={Die Anrufung wirkt auf mehrere Ziele in Reichweite. Die Kosten werden für jedes Ziel einzeln bezahlt.}
}


\newglossaryentry{anrufungstechnikignorieren_Manöver}
{
    name={Anrufungstechnik ignorieren},
    description={Du ignorierst eine der bei der Tradition angeführten Bedingungen. Ignorierst du die Bedingung Sicht, musst du das Ziel auf eine andere Art und Weise wahrnehmen.}
}


\newglossaryentry{reichweiteerhöhen(D)_Manöver}
{
    name={Reichweite erhöhen (D)},
    description={Die Reichweite der Anrufung verdoppelt sich. Die Reichweite Berührung wird zu 2 Schritt.}
}


\newglossaryentry{vorbereitungverkürzen(D)_Manöver}
{
    name={Vorbereitung verkürzen (D)},
    description={Die Vorbereitungszeit der Anrufung halbiert sich. Eine Vorbereitungszeit von 1 Aktion wird zu 0 Aktionen. Die Aktion Konflikt ist von dieser Modifikation nicht betroffen.}
}


\newglossaryentry{wirkungsdauerverlängern(D)_Manöver}
{
    name={Wirkungsdauer verlängern (D)},
    description={Die Wirkungsdauer der Anrufung verdoppelt sich.}
}


\newglossaryentry{opferung(Paktierer)_Manöver}
{
    name={Opferung (Paktierer)},
    description={Das rituelle Opfer eines intelligenten Wesens erleichtert die Anrufung um +4, ein Geweihter der Gegengottheit verdoppelt den Bonus sogar.}
}


\newglossaryentry{antimagie-Gegenzauber_Manöver}
{
    name={Antimagie - Gegenzauber},
    description={Der Gegenzauber stört einen anderen Zauberer während dessen Zaubervorbereitung. Er wirkt als Konterprobe (12), bei deren Gelingen der Zielzauber sofort scheitert.\newline Probenschwierigkeit: 12\newline Vorbereitungszeit: 0 Aktionen\newline Ziel: Zauber in Vorbereitung\newline Reichweite: 16 Schritt\newline Wirkungsdauer: augenblicklich\newline Kosten: 4 AsP}
}


\newglossaryentry{antimagie-Magieunterdrücken_Manöver}
{
    name={Antimagie - Magie unterdrücken},
    description={Diese Variante unterdrückt zukünftige Zauber. In einer Zone von 16 Schritt Radius sind alle Zauber der Fertigkeit um –8 erschwert.\newline Mächtige Magie: Der Malus steigt um –4.\newline Probenschwierigkeit: 12\newline Vorbereitungszeit: 16 Aktionen\newline Ziel: Zone\newline Reichweite: 8 Schritt\newline Wirkungsdauer: 1 Stunde\newline Kosten: 8 AsP}
}


\newglossaryentry{antimagie-Zauberaufheben_Manöver}
{
    name={Antimagie - Zauber aufheben},
    description={Hiermit hebst du einen bereits gewirkten Zauber auf, in den keine gAsP geflossen sind. Zauber aufheben gilt als Konterprobe (12), bei deren Gelingen der Zielzauber sofort aufgehoben wird.\newline Probenschwierigkeit: 12\newline Vorbereitungszeit: 8 Aktionen\newline Ziel: Zauber\newline Reichweite: 8 Schritt\newline Wirkungsdauer: augenblicklich\newline Kosten: halbe Basiskosten des Zaubers}
}


\newglossaryentry{antimagie-Wesenheitbannen_Manöver}
{
    name={Antimagie - Wesenheit bannen},
    description={Hiermit bannst du beschworene Wesenheiten, in die keine gAsP geflossen sind (wie nicht gebundene Elementare und Dämonen).\newline Probenschwierigkeit: Beschwörungsschwierigkeit des zu bannenden Wesens\newline Vorbereitungszeit: 8 Aktionen\newline Ziel: beschworenes Wesen\newline Reichweite: 8 Schritt\newline Wirkungsdauer: augenblicklich\newline Kosten: halbe Basiskosten der Beschwörung\newline Anmerkung: Wesenheiten und Zauber mit gAsP werden mit dem Destructibo gebannt.}
}


\newglossaryentry{hexenbesenreiten_Manöver}
{
    name={Hexenbesen reiten},
    description={Jede Hexe kann auf einem Besen oder einem anderen Holzgegenstand reiten. Dazu erhält sie bei der halbjährlichen Hexennacht die Hexensalbe, mit der das bestrichene Objekt flugfähig ist. Die Aktivierung des Besens vor jedem Flug kostet 1 AsP. Tollkühne Flugmanöver erfordern Proben auf das Talent Akrobatik. Misslungene Proben bedeuten, dass die Hexe die Kontrolle über den Besen verliert und schlimmstenfalls unsanft notlanden muss - schwere Verletzungen nicht ausgeschlossen. Bei einem Patzer kann der Hexe der Absturz drohen, sie könnte aber auch den Zorn eines Drachen geweckt haben.}
}


\newglossaryentry{hexenflüche_Manöver}
{
    name={Hexenflüche},
    description={Als Hexe hast du zwei Möglichkeiten, deine Flüche zu überbringen. Entweder du schleuderst sie direkt lautstark auf das Opfer, wodurch du häufig vom Bonus der Hexischen Tradition II profitieren kannst, oder du lässt den Fluch unauffällig von deinem Vertrauten überbringen. Dafür benötigst du allerdings ein Körperteil (z.B. Haar) deines Opfers. In beiden Fällen legst du danach die Probe auf eine passende übernatürliche Fertigkeit ab. Jeder Hexenfluch kann auch permanent gesprochen werden. Dann halten ein Viertel der AsP als gAsP den Fluch aufrecht. Neben Antimagie oder göttlichem Wirken kann ein permanenter Fluch immer auch durch das Erfüllen einer Bedingung beendet werden. Diese Bedingung muss realistisch erfüllbar sein und du musst sie dem Verfluchten mitteilen (direkter Fluch) oder er erfährt sie in seinen Träumen (überbrachter Fluch).}
}


\newglossaryentry{eigenschaftAufrechterhalten_Manöver}
{
    name={Eigenschaft Aufrechterhalten},
    description={Bevor die Wirkung eines Zaubers oder einer Liturgie endet, kannst du sie um die Basiswirkungsdauer verlängern. Dazu musst du nur die Hälfte der Basiskosten bezahlen.}
}


\newglossaryentry{eigenschaftBallistisch_Manöver}
{
    name={Eigenschaft Ballistisch},
    description={Genau wie ein Fernkampfangriff benötigen ballistische Zauber eine direkte Sichtlinie zu ihrem Ziel und sie können wie ein Fernkampfangriff abgewehrt werden (S. 47).}
}


\newglossaryentry{eigenschaftIllusion(Sinn)_Manöver}
{
    name={Eigenschaft Illusion (Sinn)},
    description={Der Zauber erzeugt ein Trugbild, das die angegebenen Sinne täuscht. Die Illusion kann nicht mit stofflicher Materie interagieren. Ein illusorisches Goldstück kann nicht berührt werden, die Illusion einer Wand lässt keine Gegenstände abprallen und der Hieb eines illusionären Schwertes verursacht keinen Schaden. Illusionen wechselwirken aber mit Licht. Sie werfen Schatten und können selbst als Lichtquelle dienen. Fast alle Illusionen haben Fehler. Diese Fehler können mit einer Konterprobe (Wachsamkeit, 16) entdeckt werden, womit die Illusion durchschaut wird und keine Wirkung mehr hat.}
}


\newglossaryentry{eigenschaftKonterprobe_Manöver}
{
    name={Eigenschaft Konterprobe},
    description={Manchen Zaubern kannst du auch mit profanen Fertigkeiten widerstehen, zum Beispiel wenn du dich mit deiner Willenskraft gegen die Wirkung des Friedenslieds stemmst. In diesem Fall kannst du eine Konterprobe ablegen. Die Auswirkungen und die Schwierigkeit einer Konterprobe findest du beim jeweiligen Zauber, die Schwierigkeit einer Konterprobe steigt aber in jedem Fall um +4 pro Mächtige Magie/Liturgie. Das gilt auch, wenn Mächtige Magie keine anderen Auswirkungen hat und deswegen beim Zauber gar nicht angeführt ist.}
}


\newglossaryentry{eigenschaftKonzentration_Manöver}
{
    name={Eigenschaft Konzentration},
    description={Der Zauber fordert deine volle Aufmerksamkeit. Du musst während der gesamten Wirkungsdauer die Aktion Konzentration (S. 37) wählen. Kannst oder willst du das nicht, endet der Zauber.}
}


\newglossaryentry{eigenschaftObjektritual_Manöver}
{
    name={Eigenschaft Objektritual},
    description={Objektrituale sind Talente, die sich immer auf ein bestimmtes Objekt – oder das Vertrautentier – beziehen, das mit einem Bindungsritual an dich gebunden wird. Danach kannst du (und nur du) die besonderen Fähigkeiten des Ritualobjekts nutzen. Manche Objektrituale erlauben auch passive Wirkungen, wobei du das Objekt meist berühren musst. Die Bindung erlischt, wenn du sie in einer zeitraubenden Zeremonie freiwillig löst oder wenn das Objekt entzaubert wird, was aber nur durch mächtigste Antimagie oder liturgisches Wirken möglich ist. Zuletzt gilt, dass du niemals zwei Objekte der gleichen Kategorie (zwei Zauberstäbe, Vertrautentiere) an dich binden kannst.}
}


\newglossaryentry{elementarerSekundäreffekt_Manöver}
{
    name={Elementarer Sekundäreffekt},
    description={Sekundäreffekte treten sein, sobald ein Angriff auf einen Schlag mindestens 2 Wunden verursacht. Dann muss das Ziel eine Attributsprobe (20, I) ablegen, +4 für jede zusätzliche Wunde. Entsteht der Effekt ohne, dass Schaden angerichtet wird, dann wird statt der Attributsprobe eine Konterprobe (S. 124) abgelegt.\newline - Erfrieren: Misslingt dem Opfer eine KO-Probe (20, I), erleidet es einen Punkt Erschöpfung. \newline - Ertränken: Misslingt dem Opfer eine GE-Probe (20, I), ist es bis zu seiner übernächsten Initiativephase handlungsunfähig.\newline - Fesseln: Misslingt dem Opfer eine GE-Probe (20, I), kann es sich für 4 Initiativephasen nicht von Ort und Stelle bewegen.\newline - Nachbrennen: Misslingt dem Opfer eine KO-Probe (20, I), geht es in Flammen auf und erleidet nach 4 Initiativephasen einmalig eine Wunde, wenn es nicht gelöscht wird.\newline - Niederschmettern: Das Opfer muss eine KK-Probe (20, I) ablegen, um nicht zu stürzen.\newline - Zurückstoßen: Misslingt dem Opfer eine KK-Probe (20, I), wird es 4 Schritt zurückgeworfen.}
}


\newglossaryentry{verbotenePforten_Manöver}
{
    name={Verbotene Pforten},
    description={Mit dem Vorteil Verbotene Pforten (S. 73) oder der Tradition der Borbaradianer (S. 74) kannst du deine Zauber mit deiner Lebenskraft speisen. Wenn du nicht über ausreichend AsP für einen Zauber verfügst, kannst du dir selbst Wunden zufügen. Jede dieser Wunden stellt WS+4 AsP für deinen Zauber zur Verfügung, überschüssige AsP verfallen. Verwendest du sogar beide oben genannten Vorteile, kannst du sogar WS+8 AsP pro Wunde nutzen. Die eigene Lebenskraft kann jedoch niemals genutzt werden, um sich selbst zu heilen. Die Verbotenen Pforten können übrigens keine gAsP zur Verfügung stellen – gAsP müssen immer vom Zaubernden selbst stammen.}
}


\newglossaryentry{freieAktion_Manöver}
{
    name={Freie Aktion},
    description={Zusätzlich kannst du in deiner Initiativephase beliebig viele verschiedene Freie Aktionen ausführen. Du kannst in einer Freien Aktion bis zu 2 Schritt gehen, einen kurzen Satz rufen, dich umdrehen oder in den Händen gehaltene Gegenstände fallen lassen. Es ist völlig egal, ob du Freie Aktionen vor-, nach oder zwischen zwei Aktionen nutzt. Allerdings darf niemals eine andere (Freie) Aktion unterbrochen werden und du darfst dieselbe Freie Aktion nicht zweimal ausführen.}
}


\newglossaryentry{reaktion_Manöver}
{
    name={Reaktion},
    description={Reaktionen kannst du aufwenden, um auf Ereignisse außerhalb deiner Initiativephase zu reagieren. Typische Reaktionen sind Passierschläge (S. 39), vor allem aber Verteidigungen (VT) als Antwort auf einen Nahkampfangriff. Jede Reaktion nach der ersten erleidet einen kumulativen Malus von –4. Dieser Malus verschwindet am Beginn deiner nächsten Initiativephase. Ausgenommen von diesem Malus sind Verteidigungen gegen Gegner in einer kleineren Größenklasse; sie gelten als Freie Reaktionen.\newline     Die Reihenfolge, in der du deine Reaktionen ausführst, kann eine Rolle spielen. Deswegen kannst du auch freiwillig Reaktionen mit einer höheren Erschwernis ausführen, um dir die nicht erschwerte Reaktion freizuhalten.}
}


\newglossaryentry{freieReaktion_Manöver}
{
    name={Freie Reaktion},
    description={Manchmal möchtest du vor einem Lichtblitz die Augen schließen, einem Zauber deine Magieresistenz entgegenstellen oder dich fallen lassen – solche Handlungen können „nebenbei“ ausgeführt werden und gelten als Freie Reaktionen. Freie Reaktionen können beliebig oft durchgeführt werden; sie sind niemals durch vorherige Reaktionen erschwert (wenn sie eine Probe erfordern) und erschweren auch keine Folgereaktionen.}
}


\newglossaryentry{wundschmerz_Manöver}
{
    name={Wundschmerz},
    description={Wenn dein Charakter auf einen Schlag zwei Einschränkungen erleidet, muss er sofort eine KO-Probe (20, I) ablegen. Für jede zusätzliche Einschränkung steigt die Schwierigkeit um weitere +4. Misslingt die Probe, wird er vom Schmerz überwältigt und ist betäubt. Er kann erst in seiner übernächsten Initiativephase wieder Aktionen und Reaktionen ausführen.}
}


\newglossaryentry{kampfunfähigkeit_Manöver}
{
    name={Kampfunfähigkeit},
    description={Sobald dein Charakter vier Einschränkungen erlitten hat, droht er nach jeder weiteren Einschränkung kampfunfähig zu werden. Ein kampfunfähiger Charakter stürzt, kann nicht mehr aufstehen und keine Aktionen oder Reaktionen ausführen. Der Spielleiter entscheidet, ob der Charakter bei Bewusstsein ist.\newline     Du kannst eine Zähigkeits-Probe (12, I) ablegen, um weiter handlungsfähig zu bleiben. Der Wundabzug gilt natürlich auch bei dieser Probe. Misslingt sie, erleidet dein Charakter durch diese Anstrengung noch einen zusätzlichen Punkt Erschöpfung und wird dennoch kampfunfähig. Der Einfachheit halber kann der Spielleiter davon ausgehen, dass die meisten NSC dieses Risiko nicht eingehen. Die Kampfunfähigkeit endet nach einer Stunde oder sobald eine Einschränkung geheilt wurde – allerdings nur, wenn keine schädlichen Effekte wie Gifte, Krankheiten oder Durst auf deinen Charakter wirken.}
}


\newglossaryentry{blutungen_Manöver}
{
    name={Blutungen},
    description={Blutungen drohen dir, wenn du durch Waffengewalt oder ähnliche akute Verletzungen kampfunfähig wirst, oder wenn du bereits kampfunfähig bist und eine weitere solche Verletzung erleidest. Dann muss dir eine KO-Probe (12, I) gelingen. Misslingt die Probe, beginnt dein Charakter zu bluten: Du musst nach jeweils WS Initiativephasen eine KO-Probe (12, I) ablegen, bei deren Misslingen du eine weitere Wunde erleidest. Eine Blutung endet bei erfolgreicher erster Hilfe (S. 33). Bei acht Einschränkungen ist auch für den zähesten Charakter die Grenze erreicht. Eine weitere Einschränkung bedeutet unweigerlich den Tod}
}


\newglossaryentry{mirakel[FertigkeitoderAttribut]_Manöver}
{
    name={Mirakel [Fertigkeit oder Attribut]},
    description={Dein nächster Wurf auf [Fertigkeit oder Attribut] ist um +4 Punkte erleichtert. Diese Probe muss sich auf eine kurze Handlung von maximal einigen Minuten beziehen – du kannst also nicht mit einem Mirakel auf Autorität einen Feldzug in die schwarzen Lande planen.\newline Mächtige Liturgie: Erhöht die Erleichterung um +2.\newline Probenschwierigkeit: 12\newline Vorbereitungszeit: 0 Aktionen\newline Ziel: selbst\newline Reichweite: Berührung\newline Wirkungsdauer: 4 Minuten\newline Kosten: 4 KaP\newline Anmerkung: Jedes Mirakel für eine bestimmte Fertigkeit oder ein Attribut gilt als einzelne Liturgie und muss separat erlernt werden.}
}


\newglossaryentry{dämonischeStärkung[FertigkeitoderAttribut]_Manöver}
{
    name={Dämonische Stärkung [Fertigkeit oder Attribut]},
    description={Dein nächster Wurf auf [Fertigkeit oder Attribut] ist um +4 Punkte erleichtert. Diese Probe muss sich auf eine kurze Handlung von maximal einigen Minuten beziehen.\newline Mächtige Liturgie: Erhöht die Erleichterung um +2.\newline Probenschwierigkeit: 12\newline Vorbereitungszeit: 0 Aktionen\newline Ziel: selbst\newline Reichweite: Berührung\newline Wirkungsdauer: 4 Minuten\newline Kosten: 4 GuP\newline Anmerkung: Jede Dämonische Stärkung für eine bestimmte Fertigkeit oder ein Attribut gilt als einzelne Anrufung und muss separat erlernt werden.}
}


\newglossaryentry{beschwörungen-ZusätzlicheFähigkeiten_Manöver}
{
    name={Beschwörungen - Zusätzliche Fähigkeiten},
    description={Du kannst dir die Beschwörungsprobe freiwillig erschweren, um dem beschworenen Wesen zusätzliche Fähigkeiten (Auswirkungen siehe S. 97) zu verleihen:\newline - Starke Offensive (-2; +2 AT, +1 TP)\newline - Starke Defensive (-2; +2 VT, +1 WS)\newline - Zaubermacht (-2; +2 MR, +2 auf übernat. Fertigkeiten)\newline - Resistenz (-4; Erhöht die Resistenz gegen profane, magische oder geweihte Waffen um +1 Stufe.)\newline - Regeneration (-4; Erhöht Regeneration um +1 Stufe.)\newline - Schreckgestalt (-4; Erhöht die Schreckgestalt um +1 Stufe.)\newline - Zusatzangriff (-4; Erhöht Zusätzliche Attacke um +1 Stufe.)\newline - Spezialfähigkeit (siehe S. 97) (-X; Den Modifikator für ätzende Waffen, Feuerauren o.ä. bestimmt der Spielleiter.)}
}


\newglossaryentry{rededuell_Manöver}
{
    name={Rededuell},
    description={Es werden Argumente ausgetauscht und die Aktionen des eigenen Charakters beschrieben. Sobald das Argument abgeschlossen ist (oder umgekehrt), werfen beide Seiten eine Probe (I).\newline     Abwarten: Manchmal passt keines der Talente zur eigenen Aktion, etwa wenn du abwartest, während dein Gegenüber seine Argumente vorträgt oder dir droht. Dann können die Proben stattdessen auch auf ein anderes Talent/Attribut abgelegt werden: Willenskraft gegen Betören, Menschenkenntnis gegen Überreden, MU gegen Einschüchtern oder KL gegen Rhetorik.\newline     Ungewohnte Umgebung: bei Statusunterschieden, mangelnden Sprachkenntnissen oder fremden Umgengsformen steigt die Patzerchance steigt auf 2 oder sogar 3 auf dem W20 und solche Würfe gelten immer als misslungen.\newline     Kenne deinen Feind: Zu Beginn legt der Spielleiter fest, welche der Talente angemessen, unpassend oder sogar unsinnig sind, ohne dies mitzuteilen. Verwenden die Spieler ein unpassendes/unsinniges Talent, erhält das Gegenüber eine Erleichterung von +8/+16. Wenn du in einem Rededuell direkt auf eine Schwäche des Gegenübers eingehst, ist dessen Probe um –8 erschwert. Du kannst eine bestimmte Schwäche nur einmal pro Rededuell ausnutzen.\newline     Setze dir realistische Ziele: Der Spielleiter kann bei Vorhaben, die den moralischen Vorstellungen des Gegenübers widersprechen oder die gefährlich sind um +8 erleichtern, bei Lebensgefahr sogar um +16.}
}


\newglossaryentry{artefaktherstellung_Manöver}
{
    name={Artefaktherstellung},
    description={Ladungsbasierte Artefakte lösen den Spruch ein- oder mehrmals aus, bis alle Ladungen verbraucht sind. Bei einfachen ladungsbasierten Artefakten verfliegt dann die astrale Kraftt.\newline     Der Probenmodifikator richtet sich nach der Funktionsweise des Artefakts, durch Material und Sympathetik modifiziert von +4 (eine Halskette aus Amulettmetall und sympathetischen Edelsteinen für ein Bannbaladin-Artefakt) bis –4 (ein Schild aus Stahl zum selben Zweck). Hierbei musst du auch den Auslöser festlegen. Simple Auslöser wie eine Berührung, einfache Gesten oder Schlüsselwörter modifizieren die Arcanovi-Probe nicht. Bei komplexen Auslösern wie Hitze, Kälte oder einem bestimmten Zeitpunkt ist der Arcanovi um –4 erschwert. Der Auslöser kann sich nur auf Umstände beziehen, die das Artefakt oder den Träger selbst betreffen. „Wenn mich jemand belügt“ ist beispielsweise nicht möglich. \newline     Bei ladungsbasierten Artefakten müssen die wirkenden Sprüche so oft gezaubert werden, wie sie später ausgelöst werden sollen. Misslingt dir einer dieser Zauber, ist das Artefakt missglückt. Du musst alle Zauber mit denselben spontanen Modifikationen wirken – über genau diese Modifikationen verfügt der Zauber auch beim Auslösen. Vergiss also nicht, dir die spontanen Modifikationen und bei Zaubern gegen die MR auch den durchschnittlichen Erfolgswert zu notieren!\newline     Ein Viertel der Basiskosten der wirkenden Sprüche, aber mindestens 1 AsP, strömen als gAsP in das Artefakt, bis das Artefakt seine Ladungen verbraucht hat, zerstört wurde oder du die Bindung löst (S. 70). Der Arcanovi kostet dich so viele AsP, wie die Basiskosten der wirkenden Sprüche betragen.\newline     Normalerweise ist zum Auslösen des Artefakts eine Aktion Bereit machen erforderlich. Der Zauber kommt dann nach seiner halben Vorbereitungszeit mit dem beim Einspeisen erzielten Erfolgswert zur Wirkung, wobei dem Opfer eventuell noch eine MR-Probe zusteht.}
}


\newglossaryentry{regeneration_Manöver}
{
    name={Regeneration},
    description={Du Regenerierst eine Wunde pro durchgeschlafener Nacht von min. 6 Stunden oder einen Punkt Erschöpfung pro Stunde. Wenn sich dadurch Lücken in der Statusleiste ergeben, wandern die Einschränkungen dahinter nach links. Zauberer regenerieren außerdem 4 AsP, Geweihte 1 KaP.\newline Voraussetzungen: Du musst ruhen oder besser noch schlafen und es dürfen keine schädlichen Effekte wie Gifte, Krankheiten oder Durst auf dir wirken.}
}


\newglossaryentry{bereitmachen(einfach)_Manöver}
{
    name={Bereit machen (einfach)},
    description={Du ziehst eine Waffe (1 Aktion), kramst einen Heiltrank hervor (je nach Aufbewahrungsort 2-4 Aktionen) oder führst andere Handlungen aus, die nicht deine volle Aufmerksamkeit benötigen.}
}


\newglossaryentry{konflikt(einfach)_Manöver}
{
    name={Konflikt (einfach)},
    description={Du kannst deinen Gegner angreifen, mächtige Zauber weben oder dein Gegenüber einschüchtern. Fast jede Aktion, die eine vergleichende Probe beinhaltet, ist ein Konflikt - insbesondere das Aufstehen: Ein Kontrahent kann versuchen, das mit einer Reaktion zu verhindern, dann muss dir zusätzlich eine vergleichende GE-Probe (I) gelingen. Um nur von der liegenden in eine kniende oder von der knienden in eine stehende Position zu kommen, ist keine Probe nötig – wohl aber die Aktion Konflikt.}
}


\newglossaryentry{volleDefensive(voll)_Manöver}
{
    name={Volle Defensive (voll)},
    description={Du konzentrierst dich voll auf deine Verteidigung. Alle Verteidigungen bis zu deiner nächsten Initiativephase sind um +4 erleichtert.}
}


\newglossaryentry{volleOffensive(voll)_Manöver}
{
    name={Volle Offensive (voll)},
    description={Du führst einen tollkühnen Angriff aus. Alle Nahkampfangriffe in deiner Aktion sind um +4 erleichtert, alle Verteidigungen bis zu deiner nächsten Initiativephase um -8 erschwert.}
}


\newglossaryentry{bewegung(einfach)_Manöver}
{
    name={Bewegung (einfach)},
    description={Du läufst, reitest oder schwingst an einem Seil. In einer normalen Kampfsituation kannst du so GS Schritt zurücklegen. Auf unsicherem Untergrund sinkt dieser Wert auf die Hälfte, in liegender oder kniender Position auf ein Viertel. Unter folgenden Bedingungen kannst du dich aber auch weiter bewegen: Geradeaus vorwärts doppelt so weit; geradeaus vorwärts und zusätzlich ohne Gepäck, Rüstung und sperrige Waffen viermal so weit.}
}


\newglossaryentry{konzentration(voll)_Manöver}
{
    name={Konzentration (voll)},
    description={Du führst eine Handlung aus, die volle Konzentration erfordert. Dazu gehört das Vorbereiten eines Zaubers oder Fernkampfangriffes sowie das Entschärfen einer Falle. Du kannst bis zu deiner nächsten Initiativephase keine freien Aktionen oder Reaktionen ausführen und musst bei Störungen eine Willenskraft-Probe (16, I) ablegen. Erleidest du Schaden, steigt die Schwierigkeit der Probe um +4 Punkte pro soeben erlittener Wunde. Bei Misslingen verfallen alle bereits aufgewendeten Aktionen Konzentration.}
}


\newglossaryentry{verzögern(voll)_Manöver}
{
    name={Verzögern (voll)},
    description={Du wartest ab und handelst erst, wenn ein bestimmtes Ereignis eintritt. Dazu bestimmst du das Ereignis und die genaue Handlung, die du ausführen möchtest. Diese Handlung kann unmittelbar vor oder nach dem Ereignis stattfinden, muss aber eine einfache Aktion sein (Proben in dieser Aktion sind um –4 erschwert, weil auch hier zwei Aktionen genutzt werden).}
}


\newglossaryentry{heilkunde_Manöver}
{
    name={Heilkunde},
    description={Mit erster Hilfe kann eine Blutung gestoppt werden oder die Auswirkung von Giften und Krankheiten gemildert werden (siehe S. 35). Dafür benötigst du so viele Aktionen Konzentration (S. 37), wie die Patientin vor Beginn der Heilung Wunden erlitten hat, mindestens aber 4 Initiativephasen. Außerdem musst du über Verbandsmaterial (Wunden) oder passende Heilkräuter (Gifte und Krankheiten) verfügen. Die Schwierigkeit beträgt 16 bzw. die Gift- oder Krankheitsstufe.\newline     Darauf folgen heilungsfördernde Maßnahmen, wodurch der Patient sofort eine Wunde regeneriert. Pro Patient und Tag darf nur eine Probe abgelegt werden. Eine solche Probe dauert eine halbe Stunde. Außerdem benötigst du heilende Salben oder ähnliche Hilfsmittel, Verbands material und sauberes Wasser. Die Schwierigkeit der Probe beträgt 16 + den Betrag der aktuellen Wundabzüge des Patienten.}
}


\newglossaryentry{beschwörungen_Manöver}
{
    name={Beschwörungen},
    description={Das Wesen wird mit einer Beschwörungsprobe gerufen und anschließend mit einer Beherrschungsprobe zu einem Dienst bewegt. Für 1 Minute/Stunde/Tag/Woche/Monat/Jahr, die du in Vorbereitung investierst, sind alle diese Proben um +2/4/6/8/10/12 erleichtert.\newline     Zauber, Kosten und Schwierigkeit der Beschwörungsprobe hängen vom Wesen ab (s. u.). Nach erfolgreicher Beschwörung kannst du in einer Aktion Konzentration mit einer Kontrollprobe (Attribut nach Wesen) einen Dienst verlangen. Die Basisschwierigkeit entspricht der Beschwörungsprobe, dazu kommt der Bonus durch gute Vorbereitung und ein Modifikator nach Schwierigkeit des Dienstes. Wir unterscheiden einfache (+4), gewöhnliche (0), schwierige (–4) und anmaßende Dienste (–8).\newline     Misslingt die Beherrschungsprobe knapp, erfüllt das Wesen den Dienst nur teilweise oder interpretiert ihn um. Deutlich misslungene Beherrschungsproben bedeuten, dass das Wesen frei ist und sich seinem Naturell entsprechend verhält. Kurzfristig herbeigerufene Wesen verschwinden anschließend, während du bei permanent erschaffenen Wesen nach 1 Stunde eine weitere Beherrschungsprobe ablegen kannst.\newline     Gelingt die Beherrschungsprobe, führt das Wesen den Dienst sofort aus. Anschließend verschwinden herbeigerufene Wesen, während du erschaffenen Wesen mit weiteren Beherrschungsproben weitere Dienste abringen kannst.\newline     Ein besonderer Dienst ist die Bindung, mit der du ein herbeigerufenes Wesen in/an einen Gegenstand binden kannst, sodass es dir während dieser Zeit einen zusätzlichen Dienst erfüllen kann. Als Faustregel gilt eine Bindung für 1 Woche als gewöhnlicher Dienst. In jedem Fall fließt jedoch ein Viertel der Basiskosten der Beschwörung als gAsP in die Bindung, um diese aufrecht zu erhalten.}
}


\newglossaryentry{einschüchternimKampf_Manöver}
{
    name={Einschüchtern im Kampf},
    description={Um deinen Gegner einzuschüchtern, musst du eine Aktion Konflikt aufwenden. Eine misslungene Probe kann nicht wiederholt werden. Gelingt die Probe, ist dein Gegner von einem Furcht-Effekt Stufe 1 betroffen. Hohe Qualität (S. 7) erhöht den Furcht-Effekt um eine Stufe oder verdoppelt die Zahl der eingeschüchterten Gegner. Dann legen die Gegner ihre MU-Probe als Gruppenprobe ab: Wenn die Mutigste flieht, fliehen alle. Der Effekt endet am Ende des Kampfes, bei Kampf unfähigkeit des Einschüchternden oder wenn der eingeschüchterte Gegner weit genug geflohen ist.}
}


\newglossaryentry{kommandos_Manöver}
{
    name={Kommandos},
    description={Du kannst bis zu 4 Mitstreitern in Hörweite den Bonus eines Kommandovorteils (s.u.) für diesen Kampf verleihen. Hohe Qualität (S. 7) vervierfacht die Zahl der Mitstreiter. Jeder Kämpfer kann nur von einem Anführen-Effekt gleichzeitig profitieren.\newline     Der Effekt endet am Ende des Kampfes, bei Kampfunfähigkeit des Anführers oder wenn dieser neue Befehle ruft. Der Anführer selbst profitiert nicht von Kommandos.}
}


\newglossaryentry{allgemeineFernkampfregeln_Manöver}
{
    name={Allgemeine Fernkampfregeln},
    description={Fernkampfangriffe (FK) führst du in drei Schritten aus:\newline - Du wählst deine Manöver und ein Ziel in Sicht- und Reichweite (RW).\newline - Du wendest so viele Aktionen Konzentration auf, wie bei deiner Fernkampfwaffe unter Ladezeit (LZ) angegeben (S. 53).\newline - In einer Aktion Konflikt würfelst du eine Probe (12, I) auf den passenden Fertigkeitswert.\newline Modifikatoren:\newline - Größenklasse: sehr groß (Elefant) +8, groß (Pferd, Oger) +4, mittel (Mensch, Zwerg) +0, klein (Wolf, Reh) -4, sehr klein (Fasan, Hase) -8, winzig (Maus) -12\newline - Umgebung: Dämmerung -4, Mondlicht -8, Sternenlicht -16 | Wind -4, Sturm -8\newline - Bewegung: schnell (laufender Mensch) -4, sehr schnell (Pferd) -8, extrem schnell (Vogel) -12\newline - Sonstiges: halbe Deckung -4, Dreivierteldeckung -8, berittener Schütze -4\newline Schüsse ins Kampfgetümmel: die Patzerchance steigt auf 2/4 auf dem W20, wenn in offenem Feld/auf beengtem Raum gekämpft wird. Diese Würfe lassen die Probe auf jeden Fall misslingen und bedeuten, dass du ein verbündetes Ziel vor dir triffst.\newline Verteidigung gegen Fernkampfangriffe: Wenn dein Ziel dich beim Zielen beobachtet, kann es versuchen, dem Fernkampfangriff zu entgehen. Dazu muss es eine Reaktion aufwenden und eine Schild-VT (28, I) oder eine Akrobatik-Probe (28, I) ablegen. Befindest du dich im Kontrollbereich deines Ziels, genügt ihm eine Freie Reaktion, um deine Waffe beiseite zu schlagen.}
}


\newglossaryentry{beschwörungen-Chimären_Manöver}
{
    name={Beschwörungen - Chimären},
    description={Probenschwierigkeiten und Kosten (1/4 der Basis als gAsP):\newline - 12 (schwache Ch., z.B. Hundeblume), 8 AsP\newline - 16 (nützliche Ch., z.B. Widderhyäne), 16 AsP\newline - 20 (starke Ch., z.B. Mantikor), 24 AsP\newline - 24 (mächtige Ch., z.B. Drachenchim.), 32 AsP\newline Beschwörungsprobe Mod.: +Bonus durch Vorbereitung, –Kosten für Zusatzfähigkeiten\newline Kontrollprobe Mod.: +Bonus durch Vorbereitung, +4 bis –8 je nach Dienst}
}


\newglossaryentry{beschwörungen-Dämonen_Manöver}
{
    name={Beschwörungen - Dämonen},
    description={Probenschwierigkeiten und Kosten:\newline - 16 (mindere D. z.B. Gotongi), 16 AsP\newline - 20 (niedere D. z.B. Zant), 24 AsP\newline - 24 (weniggehörnte D., z.B. Ulchuchu), 32 AsP\newline - 28 (gehörnte D., z.B. Shruuf), 48 AsP\newline - 32 (vielgehörnte D., z.B. Duglum), 64 AsP\newline - 36 (mächtige D., z.B. Yo’Na‘Hoh), 80 AsP\newline Beschwörungsprobe Mod.: +Bonus durch Vorbereitung, –Kosten für Zusatzfähigkeiten, +4 beim Einsatz von Blutmagie\newline Kontrollprobe Mod.: +Bonus durch Vorbereitung, +4 bis –8 je nach Dienst, –4 beim Einsatz von Blutmagie\newline Chaotische Dämonen (optional): Dämonen sind Wesen des Chaos und oft weiß nicht einmal der Beschwörer, welche Eigenschaften das von ihm gerufene Wesen haben wird. Der Spielleiter kann dem Dämon versteckt zusätzliche Fähigkeiten verleihen entsprechend der Zufallstabelle auf S. 98.\newline Anmerkung: Bei misslungenen Dämonenbeschwörungen kann es zu gefährlichen chaotischen Nebeneffekten kommen. }
}


\newglossaryentry{beschwörungen-Elementare_Manöver}
{
    name={Beschwörungen - Elementare},
    description={Probenschwierigkeiten und Kosten:\newline - 16 (Diener, z.B. Kräuterkopf), 16 AsP\newline - 24 (Dschinne, z.B. Sausewind), 32 AsP\newline - 32 (Meister, z.B. Hefashar), 64 AsP\newline Beschwörungsprobe Mod.: +Bonus durch Vorbereitung, –Kosten für Zusatzfähigkeiten\newline Kontrollprobe Mod.: +Bonus durch Vorbereitung, +4 bis –8 je nach Dienst, –4 beim Einsatz von Blutmagie}
}


\newglossaryentry{beschwörungen-Golems_Manöver}
{
    name={Beschwörungen - Golems},
    description={Probenschwierigkeiten und Kosten (1/4 der Basis als gAsP):\newline - 16 (schwache G., z.B. lauf. Truhe), 8 AsP\newline - 20 (nützliche G., z.B. Mephir), 16 AsP\newline - 24 (starke G., z.B. Homunculus), 24 AsP\newline - 28 (mächtige G., z.B. Ogerstatue), 32 AsP\newline Beschwörungsprobe Mod.: +Bonus durch Vorbereitung, –Kosten für Zusatzfähigkeiten\newline Kontrollprobe Mod.: +Bonus durch Vorbereitung, +4 bis –8 je nach Dienst}
}


\newglossaryentry{beschwörungen-Untote_Manöver}
{
    name={Beschwörungen - Untote},
    description={Probenschwierigkeiten und Kosten:\newline - 12 (schwache U., z.B. Skelett), 8 AsP\newline - 16 (nützliche U., z.B. Knochenritter), 16 AsP\newline - 20 (starke U., z.B. Kriegermumie), 24 AsP\newline - 24 (mächtige U., z.B. Kriegsherr), 32 AsP\newline Beschwörungsprobe Mod.: +Bonus durch Vorbereitung, –Kosten für Zusatzfähigkeiten\newline Kontrollprobe Mod.: +Bonus durch Vorbereitung, +4 bis –8 je nach Dienst\newline Anmerkung: Mit dem Totes handle! können nur permanente Untote erschaffen werden, bei den Kosten sind hier 1/4 der Basiskosten gAsp}
}


\newglossaryentry{kopflastig_Waffeneigenschaft}
{
    name={Kopflastig},
    description={Die Waffe kann zusätzliche Kraft besonders effizient in Schaden umwandeln. Der Schadensbonus durch hohe KK zählt für sie doppelt.}
}


\newglossaryentry{parierwaffe_Waffeneigenschaft}
{
    name={Parierwaffe},
    description={Das Führen einer solchen Waffe ist Voraussetzung für den Parierwaffenstil.}
}


\newglossaryentry{reittier_Waffeneigenschaft}
{
    name={Reittier},
    description={Ein Reittier ist Voraussetzung für den Reiterkampfstil. Im Reiterkampf bewegst du dich mit der GS des Reittiers und verwendest Reiten als Gegenprobe gegen Niederwerfen und Umreißen-Attacken (S. 40), bei deren Misslingen du aus dem Sattel geworfen wirst und 2W6 SP erleidest.\newline     Im Kampf bilden Kavallerist und Reittier eine Einheit. Letzteres wird deswegen wie eine Waffe behandelt, die mit dem Talent Reiten eingesetzt wird. Du kannst dich daher bei jeder Attacke und jeder Verteidigung entscheiden, ob du das Pferd oder eine andere Waffe nutzt. Reiten-Attacken sind um +4 erleichtert, wenn das Reittier in einer höheren Größenklasse als das Ziel ist – wie zum Beispiel ein Pferd gegenüber einem Infanteristen. Bei einer Reiten-Verteidigung nimmt das Tier keinen Schaden.\newline     Lanzenreiten-Attacken sind nur für berittene Kämpfer aus der Bewegung heraus möglich. Sie verursachen Niederwerfen und sind gegen Infanteristen um +4 erleichtert. Allerdings kannst du dich mit dem Talent Lanzenreiten nicht verteidigen.}
}


\newglossaryentry{rüstungsbrechend_Waffeneigenschaft}
{
    name={Rüstungsbrechend},
    description={Ein schmaler Stoßdorn oder andere Waffenteile können Rüstungen durchschlagen. Ermöglicht das Manöver Rüstungsbrecher (S. 40).}
}


\newglossaryentry{schild_Waffeneigenschaft}
{
    name={Schild},
    description={Ermöglicht den Kampf im Schildkampfstil. Verteidigungen mit einem Schild sind um +4 erleichtert, wenn der Angreifer in einer höheren Größenklasse ist.}
}


\newglossaryentry{schwer_Waffeneigenschaft}
{
    name={Schwer},
    description={Du brauchst mindestens die angegebene Körperkraft, um die Waffe ohne Einschränkungen führen zu können. Darunter erleidest du einen Malus von -2.}
}


\newglossaryentry{stumpf_Waffeneigenschaft}
{
    name={Stumpf},
    description={Stumpfe Waffen können den Gegner mit dem Manöver Stumpfer Schlag (S. 40) bewusstlos schlagen.}
}


\newglossaryentry{unberechenbar_Waffeneigenschaft}
{
    name={Unberechenbar},
    description={Die Waffe verursacht auch bei einer 2 einen Patzer. Dafür kann sie auch um Schilde herumschlagen, was Angriffe gegen Schildträger um +4 erleichtert.}
}


\newglossaryentry{wendig_Waffeneigenschaft}
{
    name={Wendig},
    description={Die Waffe ist ungewöhnlich wendig, du kannst den ersten Passierschlag zwischen zwei Initiativephasen als freie Reaktion ausführen.}
}


\newglossaryentry{zerbrechlich_Waffeneigenschaft}
{
    name={Zerbrechlich},
    description={Die Waffe erleidet den vollen Waffenschaden, wenn sie eine nicht-zerbrechliche Waffe erfolgreich verteidigt oder von einer solchen erfolgreich verteidigt wird. Ausweichen vermeidet diesen Schaden.}
}


\newglossaryentry{zweihändig_Waffeneigenschaft}
{
    name={Zweihändig},
    description={Die Waffe benötigt die Führung mit beiden Händen. Bei einhändiger Führung erleidest du einen Malus von -2 und die Waffe richtet -4 TP weniger an. Zweihändige Fernkampfwaffen können nicht einhändig benutzt werden.}
}


\newglossaryentry{niederwerfen_Waffeneigenschaft}
{
    name={Niederwerfen},
    description={Schon gewöhnliche Angriffe haben den Effekt des Manövers Niederwerfen. In Klammern befinden sich - falls relevant - der Modifikator der Gegenprobe und die Ansage des Manövers.}
}


\newglossaryentry{unzerstörbar_Waffeneigenschaft}
{
    name={Unzerstörbar},
    description={}
}


\newglossaryentry{keinMalusalsNebenwaffe_Waffeneigenschaft}
{
    name={Kein Malus als Nebenwaffe},
    description={}
}


\newglossaryentry{nichtfürReiter_Waffeneigenschaft}
{
    name={Nicht für Reiter},
    description={}
}


\newglossaryentry{umklammern_Waffeneigenschaft}
{
    name={Umklammern},
    description={Schon gewöhnliche Angriffe haben den Effekt des Manövers Umklammern. In Klammern befinden sich - falls relevant - der Modifikator der Gegenprobe und die Ansage des Manövers.}
}


\newglossaryentry{stationär_Waffeneigenschaft}
{
    name={Stationär},
    description={}
}


\newglossaryentry{magazin_Waffeneigenschaft}
{
    name={Magazin},
    description={}
}


\newglossaryentry{chrmk}
{
    name={Chrmk},
    description={Sprache}
}


\newglossaryentry{chuchas}
{
    name={Chuchas},
    description={Sprache}
}


\newglossaryentry{isdira-undAsdharia-Zeichen}
{
    name={Isdira- und Asdharia-Zeichen},
    description={Sprache}
}


\newglossaryentry{drakhard-Zinken}
{
    name={Drakhard-Zinken},
    description={Sprache}
}


\newglossaryentry{gimaril}
{
    name={Gimaril},
    description={Sprache}
}


\newglossaryentry{imperialeZeichen}
{
    name={Imperiale Zeichen},
    description={Sprache}
}


\newglossaryentry{kuslikerZeichen}
{
    name={Kusliker Zeichen},
    description={Sprache}
}


\newglossaryentry{nanduria-Zeichen}
{
    name={Nanduria-Zeichen},
    description={Sprache}
}


\newglossaryentry{mahrischeGlyphen}
{
    name={Mahrische Glyphen},
    description={Sprache}
}


\newglossaryentry{trollischeRaumbilderschrift}
{
    name={Trollische Raumbilderschrift},
    description={Sprache}
}


\newglossaryentry{gjalskerRunen}
{
    name={Gjalsker Runen},
    description={Sprache}
}


\newglossaryentry{hjaldingscheRunen}
{
    name={Hjaldingsche Runen},
    description={Sprache}
}


\newglossaryentry{thorwalscheRunen}
{
    name={Thorwalsche Runen},
    description={Sprache}
}


\newglossaryentry{altesAlaani}
{
    name={Altes Alaani},
    description={Sprache}
}


\newglossaryentry{amulashtra-Glyphen}
{
    name={Amulashtra-Glyphen},
    description={Sprache}
}


\newglossaryentry{geheiligteGlyphenvonUnau}
{
    name={Geheiligte Glyphen von Unau},
    description={Sprache}
}


\newglossaryentry{kemi-Symbole}
{
    name={Kemi-Symbole},
    description={Sprache}
}


\newglossaryentry{tulamidya-Zeichen}
{
    name={Tulamidya-Zeichen},
    description={Sprache}
}


\newglossaryentry{ur-Tulamidya-Zeichen}
{
    name={Ur-Tulamidya-Zeichen},
    description={Sprache}
}


\newglossaryentry{angram-Bilderschrift}
{
    name={Angram-Bilderschrift},
    description={Sprache}
}


\newglossaryentry{rogolan-Runen}
{
    name={Rogolan-Runen},
    description={Sprache}
}


\newglossaryentry{arkanil}
{
    name={Arkanil},
    description={Sprache}
}


\newglossaryentry{drakned-Glyphen}
{
    name={Drakned-Glyphen},
    description={Sprache}
}


\newglossaryentry{zhayad-Zeichen}
{
    name={Zhayad-Zeichen},
    description={Sprache}
}


\newglossaryentry{krakonisch}
{
    name={Krakonisch},
    description={Sprache}
}


\newglossaryentry{rssahh}
{
    name={Rssahh},
    description={Sprache}
}


\newglossaryentry{asdharia}
{
    name={Asdharia},
    description={Sprache}
}


\newglossaryentry{isdira}
{
    name={Isdira},
    description={Sprache}
}


\newglossaryentry{aureliani}
{
    name={Aureliani},
    description={Sprache}
}


\newglossaryentry{bosparano}
{
    name={Bosparano},
    description={Sprache}
}


\newglossaryentry{bukanisch}
{
    name={Bukanisch},
    description={Sprache}
}


\newglossaryentry{garethi}
{
    name={Garethi},
    description={Sprache}
}


\newglossaryentry{zyklopäisch}
{
    name={Zyklopäisch},
    description={Sprache}
}


\newglossaryentry{mahrisch}
{
    name={Mahrisch},
    description={Sprache}
}


\newglossaryentry{neckergesang}
{
    name={Neckergesang},
    description={Sprache}
}


\newglossaryentry{rissoal}
{
    name={Rissoal},
    description={Sprache}
}


\newglossaryentry{z’Lit}
{
    name={Z’Lit},
    description={Sprache}
}


\newglossaryentry{oloarkh}
{
    name={Oloarkh},
    description={Sprache}
}


\newglossaryentry{ologhaijan}
{
    name={Ologhaijan},
    description={Sprache}
}


\newglossaryentry{golp}
{
    name={Golp},
    description={Sprache}
}


\newglossaryentry{trollisch}
{
    name={Trollisch},
    description={Sprache}
}


\newglossaryentry{yetan}
{
    name={Yetan},
    description={Sprache}
}


\newglossaryentry{fjarningsch}
{
    name={Fjarningsch},
    description={Sprache}
}


\newglossaryentry{gjalskisch}
{
    name={Gjalskisch},
    description={Sprache}
}


\newglossaryentry{hjaldingsch}
{
    name={Hjaldingsch},
    description={Sprache}
}


\newglossaryentry{thorwalsch}
{
    name={Thorwalsch},
    description={Sprache}
}


\newglossaryentry{alaani}
{
    name={Alaani},
    description={Sprache}
}


\newglossaryentry{ferkina}
{
    name={Ferkina},
    description={Sprache}
}


\newglossaryentry{kemi}
{
    name={Kemi},
    description={Sprache}
}


\newglossaryentry{rabensprache}
{
    name={Rabensprache},
    description={Sprache}
}


\newglossaryentry{ruuz}
{
    name={Ruuz},
    description={Sprache}
}


\newglossaryentry{tulamidya}
{
    name={Tulamidya},
    description={Sprache}
}


\newglossaryentry{ur-Tulamidya}
{
    name={Ur-Tulamidya},
    description={Sprache}
}


\newglossaryentry{zelemja}
{
    name={Zelemja},
    description={Sprache}
}


\newglossaryentry{zhulchammaqra}
{
    name={Zhulchammaqra},
    description={Sprache}
}


\newglossaryentry{angram}
{
    name={Angram},
    description={Sprache}
}


\newglossaryentry{rogolan}
{
    name={Rogolan},
    description={Sprache}
}


\newglossaryentry{tiefzwergisch}
{
    name={Tiefzwergisch},
    description={Sprache}
}


\newglossaryentry{amuurak}
{
    name={Amuurak},
    description={Sprache}
}


\newglossaryentry{atak(+Schrift)}
{
    name={Atak (+Schrift)},
    description={Sprache}
}


\newglossaryentry{drachisch}
{
    name={Drachisch},
    description={Sprache}
}


\newglossaryentry{füchsisch(+Schrift)}
{
    name={Füchsisch (+Schrift)},
    description={Sprache}
}


\newglossaryentry{goblinisch}
{
    name={Goblinisch},
    description={Sprache}
}


\newglossaryentry{grolmisch}
{
    name={Grolmisch},
    description={Sprache}
}


\newglossaryentry{koboldisch}
{
    name={Koboldisch},
    description={Sprache}
}


\newglossaryentry{mohisch}
{
    name={Mohisch},
    description={Sprache}
}


\newglossaryentry{molochisch}
{
    name={Molochisch},
    description={Sprache}
}


\newglossaryentry{nujuka}
{
    name={Nujuka},
    description={Sprache}
}


\newglossaryentry{ogrisch}
{
    name={Ogrisch},
    description={Sprache}
}


\newglossaryentry{riesen-Sprache}
{
    name={Riesen-Sprache},
    description={Sprache}
}


\newglossaryentry{zhayad}
{
    name={Zhayad},
    description={Sprache}
}


\newglossaryentry{zyklopisch}
{
    name={Zyklopisch},
    description={Sprache}
}

